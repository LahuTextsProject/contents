\thispagestyle{empty}
\renewcommand{\thefootnote}{\arabic{footnote}}
\setcounter{footnote}{0}

\section*{Foreword}
\sectionmark{Foreword}
\addcontentsline{toc}{chapter}{Foreword}
The Computational Resource for South Asian Languages (CoRSAL)
Occasional Publications is published by the University of North Texas
(UNT) under the Aquiline Books imprint. CoRSAL Occasional Publications
provides a venue for sharing further analysis of audio and video
language documentation materials archived at the CoRSAL archive.
These publications are freely available for download or viewing
through the UNT Digital Library and the author has the option to print
copies for distribution to interested readers.

This second volume in the CoRSAL Occasional Publications series is
authored by James A. Matisoff, Professor Emeritus of Linguistics at
the University of California, Berkeley. Professor Matisoff is a
renowned expert on the languages of the Tibeto-Burman area.  His
influential work on Lahu, which has been widely read and often cited,
includes numerous articles, a descriptive grammar of Lahu, a
Lahu-English dictionary and English-Lahu Lexicon.  We are honored to
publish this Lahu text collection which brings to us an unprecedented
variety and depth of analysis presented in the interlinear gloss
format.  As the title states, it also provides a window into a world
that few of us could otherwise witness.  This text collection,
especially as complemented by the audio recordings in CoRSAL, will be
of lasting interest to historical, comparative, and typological
linguists, as well as speakers connecting or reconnecting with
cultural and linguistic traditions.

We aim to publish at least two volumes per year. Digital versions of
CoRSAL Occasional Publications will be housed in the University of
North Texas Digital Library.\footnote{https://digital.library.unt.edu/explore/collections/UNTEE/}  Please
contact Shobhana Chelliah (or corsalunt@gmail.com) for information on
publishing with CoRSAL Occasional Publications.  The CoRSAL archive
exists in large part due to the digital infrastructure and support of
Mark Phillips, Associate Dean for the UNT Digital Libraries.  As well,
this series would not be possible without the continued encouragement
and support of Kevin Hawkins, Assistant Dean for Scholarly
Communication at University of North Texas Libraries.  As director of
the archive and series editor, I rely fully on the information science
expertise of Mary Burke who plays many critical roles in the archive
and publication.

\vspace{4cm}
Shobhana Chelliah\\
New Delhi