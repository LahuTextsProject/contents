\vspace{0.25em}

\renewcommand{\thefootnote}{\arabic{footnote}}
\setcounter{footnote}{0}

\section*{Acknowledgments}
\sectionmark{Acknowledgments}
\addcontentsline{toc}{chapter}{Acknowledgments}
It is a pleasure to thank the many people who have made the publication
of these texts possible. First and foremost I shall always be grateful
to the Lahu villagers in Northern Thailand who welcomed me and taught me
their language in such imaginative ways during my fieldtrips in 1965--66,
1970, and 1977.

The texts I recorded during those trips were transcribed and partially
translated back in the 1960's and '70's, but processing them for
publication had to await the computer revolution. Fortunately there
was a talented cadre of computer-seasoned graduate students and
post-docs at Berkeley who were more than equal to the task. Our goals
were twofold: to produce a print version (a ``camera-ready'' PDF) of
the texts, as well as an online version (a set of static HTML pages
including links to the recordings and other metadata). Once the
transcriptions of the texts had been double-checked, they were
imported into the FLeX interlinearization program, a laborious process
carried out over a period of several years (2013--2016), at first with
the help of Virginia Dawson and Tyler Lau.  (Following the axiom that
``If you give a man a fish you help him for one day, but if you teach
a man how to fish you help him for life'', Tyler whipped up a user's
manual for FLeX and taught me how to use it.)

Meanwhile, Dr. J.B. Lowe, who had digitized the texts back in 2006,
proceeded to develop the Python scripts that would render the FLeX XML
exports into the LaTeX program, that could then generate the
camera-ready copy. (In this effort, J.B. was assisted by Chundra
Cathcart.) The Python scripts also provided a means to match the texts
with their free translations (which are separate documents not included
in FLeX), to sequence the texts into sections and chapters according to
the master catalog, and to inject the necessary LaTeX code to produce
the front and back matter. My wholehearted gratitude to Dr. Lowe and the
other members of the ``team'' for their inspired work!

Finally, starting in the fall of 2017, I benefitted from Berkeley's
Undergraduate Research Apprentice Program (URAP), which made it possible
for me to hire Charles Zhang, a gifted freshman with a joint Linguistics
and Mathematics major, to complete the process of converting the
texts and translations into LaTeX.

I'd like to thank my friend and colleague in Lahu studies, Dr. Anthony
R. Walker, for permission to republish several of his Red Lahu religious
texts.

I am most grateful to the National Science Foundation for awarding me
a grant to finish up this project (Award \#BCS1252226), which
originally ran from 9/1/13 to 2/29/16, but which was continued via a
``no-cost extension'' until August 31, 2017. I particularly want to
thank Prof.\ Joan M. Maling, program director for NSF's Social,
Behavioral, and Economic Sciences division, for her understanding and
patience during this long drawn-out process!

My sincere thanks to Paula Floro, Manager of the Berkeley Linguistics
Department, who skillfully handled the budgetary details of the NSF
grant.

Last but not least in love, my thanks to my wife of 56 years, Susan
Kimball Matisoff, for her support every step of the way.