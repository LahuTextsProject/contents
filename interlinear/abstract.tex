\vspace{0.25em}

\renewcommand{\thefootnote}{\arabic{footnote}}
\setcounter{footnote}{0}

\section*{Abstract}
\sectionmark{Abstract}
\addcontentsline{toc}{chapter}{Abstract}

The Lahu texts in this collection were recorded in villages in
Northern Thailand in the 1960's. They reflect a variety of genres and
subject matter, and are of considerable historical and cultural
interest. Each text is transcribed verbatim in Lahu, with a
word-by-word ``interlinear'' translation, followed by a free English
translation accompanied by footnotes.

The 137 texts are divided into three large groups: (1) Lahu daily life
in the 1960's; (2) the Lahu imagination; and (3) Lahu spiritual and
musical life. The texts are followed by a glossary containing all the
Lahu words that appear in the texts.

Three different orthographies for Lahu are in use; one devised by
Baptist missionaries in China, Burma, and Thailand; a pinyin-inspired
system now used by some Lahu in China; and the author's own phonemic
transcription. The electronic publication format permits the user to
select the orthography of his/her choice.