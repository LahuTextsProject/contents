\makeatletter
\newcommand\lahutableofcontents{%
  \chapter*{%
    \contentsname \ (in phonemic transcription)
    \@mkboth{\MakeUppercase\contentsname}
            {\MakeUppercase\contentsname}%
  }%
  \@starttoc{lahutable}%
}
\newcommand{\addlahutoc}[2]{%
  \addcontentsline{lahutable}{#1}{\protect\numberline{\csname the#1\endcsname}\hspace {1em}#2}%
}

\newcommand\chinesetableofcontents{%
  \chapter*{%
    \contentsname
    \@mkboth{\MakeUppercase\contentsname}
            {\MakeUppercase\contentsname}%
  }%
  \@starttoc{chinesetable}%
}
\newcommand{\addchinesetoc}[2]{%
  \addcontentsline{chinesetable}{#1}{\protect\numberline{\csname the#1\endcsname}#2}%
}

\newcommand\baptisttableofcontents{%
  \chapter*{%
    \contentsname
    \@mkboth{\MakeUppercase\contentsname}
            {\MakeUppercase\contentsname}%
  }%
  \@starttoc{baptisttable}%
}

\newcommand{\nop}[1]{}

\newcommand{\addbaptisttoc}[2]{%
  \let\oldphantom\phantom
  \let\phantom\nop
  %% FIXME: we have to make \phantom a nop for now. this might make the
  %% transcription look weird.
  \addcontentsline{baptisttable}{#1}{\protect\numberline{\csname the#1\endcsname}#2}%
  \let\phantom\oldphantom
}
\makeatother

%%% Local Variables:
%%% mode: latex
%%% End:
