\vspace{0.25em}

\renewcommand{\thefootnote}{\arabic{footnote}}
\setcounter{footnote}{0}

\chapter*{Introduction}
\addcontentsline{toc}{chapter}{Introduction}

Lahu, a member of the Central Loloish group of TB languages, spoken by about half a million people in SW China, Shan State of Burma, Northern Thailand, NW Laos, and N. Vietnam,  is perhaps the most richly documented of all the minority languages in the great Tibeto-Burman family (ca. 250-300 languages). 

The Grammar of Lahu (1973, reprinted 1982; The Dictionary of Lahu (1988); English-Lahu Lexicon (2006).  SUPPLY COMPLETE REFS., incl. number of pages.


\section{History of previous work on the texts}

I recorded these texts on a rather primitive reel-to-reel tape recorder in 1965-66, in several Black and Red Lahu villages, both Christian and animist, in Northern Thailand, transcribing them in longhand in some 20 field notebooks. At the conclusion of this fieldtrip, when I was teaching at Columbia in the late 1960’s, I began the task of translating the texts, but was only able to complete 34 items (out of 170) before the press of other duties made me put the job on hold.

Nevertheless, these texts, both translated and untranslated, formed the basis for my Lahu grammar (1973/1982), as well as a good part of the material for my Lahu dictionary (1988). 

There matters stood with the texts until 2009, when I made the acquaintance of a Lahu graduate student in linguistics at Payap University in Chiang Mai, Thailand, named Aaron Maung Maung Tun, who impressed me both by his linguistic acumen and his zeal to make the results of my work more accessible to the Lahu people themselves. I was ultimately able to invite Maung Maung to spend two and a half months as a Visiting Scholar at Berkeley in the spring of 2011.   During his stay he entered virtually my entire corpus of texts into the computer, a great achievement. In addition, he began the arduous task of providing interlinear glosses and form-class designations for each morpheme. This interlinearization was done using the FLEX program developed by the Summer Institute of Linguistics (SIL), which, despite its many virtues, is rather buggy and delicate to use. By the time Maung Maung had to leave Berkeley, the interlinear glossing was about one-third completed.

\section{Cultural and historical importance of the texts}

To my knowledge these 170 texts constitute the largest corpus of recorded natural speech for any minority language of the Tibeto-Burman family. Most of them were recorded in Black Lahu Christian villages, but a number of them are from animist villages, and/or in the Red Lahu or Yellow Lahu dialects. 

A few words about the circumstances of their elicitation are in order. Once I had made it clear that I was interested in hearing Lahu spoken in all kinds of situational contexts, the villagers would talk over various subjects -- building a house, slashing and burning a swidden, boar-hunting, the New Year’s celebrations, gathering crabs, killing a pig for a feast, the institution of the village headman, etc. Then they would actually conduct rehearsals, assigning specific roles to various people, until they felt they could discuss the matter smoothly. When they were satisfied, they would signal me to turn on the tape-recorder, and proceed enthusiastically to act out little playlets on the desired topic, often embellished with sound effects for gunshots, animal noises, etc. (Sometimes I would turn on the recorder before they realized it, to get even more natural conversation. ) Besides this kind of multi-speaker texts, I also collected many items from individuals (stories, songs, sermons).

As a result of this process, my corpus comprises a wide variety of genres and subject matter, including:

\begin{itemize}
\item Descriptions of daily activities
\item Fables and other stories with a moral
\item Political discussions
\item Village life and its hardships
\item Traditional love poetry
\item Traditional songs
\item Jokes  and anecdotes
\item Trickster tales
\item ”Just-so stories”: explanations for why things are as they are
\item Pre-Christian myths
\item Animist prayers
\item Christian sermons, hymns, Bible readings
\item Candid conversations
\end{itemize}

In general these texts, recorded for the most part almost half a century ago, reflect a vanished world: a time when the forests of Burma and Thailand teemed with game, when “slash-and-burn” agricultural techniques were universal in the hills of SE Asia, when animist villages were still resisting the imposition of missionary Christianity, and religious syncretism was the order of the day; a time before young people left the villages to seek work in Thai towns, before Thai loanwords had made too many inroads into the Lahu vocabulary, before store-bought clothes replaced homespun, before electricity, when people still used pine-torches to light their way at night.

