
{86 Headman's Communication at a Meeting in Church}

{Cà-bí (Huey Tat)}

{1. You women who have come, if among the menfolk in your household, there
are people who haven't come, please tell all of them about it, OK?}

{2. Among all of us [2] who make up this community, we're always overloaded
with all kinds of work, aren't we.}

{3. All the men, all the women, we always have work loaded upon us.}

{4. All kinds [of work].}

{5. These days when we're all so busy, even though there is work to be done
in the village, you brethren haven't had a chance to know about it, right?}

{6. But now I'm going to tell you all [about it].}

{7. A while ago a preacher [3] was coming to visit us, so we were told ``Go
and welcome him.''}

{8. We had heard about this in advance, you see.}

{9. We were to go and welcome him.}

{10. We were supposed to help him with all of his belongings.}

{11. So last Tuesday, I went .[4]}

{12. And on Saturday, Mehle's father [5] went and saw him off again.}

{13. So I've heard that we paid out ten baht for the expenses we had to
help him with.}

{14. That's right, isn't it, ten baht?}

{Somebody 15. Yes.}

{Headman 16. We spent ten baht. [6]}

{17. So what shall we do about that money?}

{18. We ought to think about the question of this money, all the members
of the committee. [7]}

{19. And our whole community, all of us, ought to listen to each other's
opinions and think this over.}

{20. And I still have other things to tell you, brethren.}

{21. Soon, when church is over, [8] if we have things to discuss in the
committee room, we can discuss them in the committee room.}

{22. If there's nothing to discuss in the committee room, soon when church
is over everybody please gather at my house for a while, OK?}

{23. There are matters I'd like to discuss. Let's consider them. There are
several matters.}

{24. Brethren, please don't think ``[I] don't want to go, since today is
Sunday and I don't want to do a bad thing.'' [9]}

{25. If you're doing something good, it's good to do it even on Sunday.}

{26. In fact, if it wasn't today, if today wasn't Sunday, we wouldn't be
seeing each other at all, would we!}

{27. Since we're so busy, everybody is off to his own place, his own field,
his own mountain.}

{28. Unless it's Saturday or Sunday, we don't get to see each other at all.}

{29. For that reason if there are people who won't come, everybody tell
them to!}

{30. You really must come [10], brethren, every house [in the village].}

{31. Unless somebody is actually sick.}

{[2] }{\textit{n}}{ɔ̀}{\textit{-ŋà-h}}{ɨ:
lit. ``you pl. and me''. This is the way to express ``inclusive 1st p. plural.''}

{[3] This was a Karen pastor from a distant village.}

{[4] To meet him and bring him to our village.}

{[5] Villagers are frequently referred to by their children's names, a practice
called }{\textit{teknonymy}}{ by anthropologists.}

{[6] At this time (mid-1960s) a baht was worth about 5 cents (20 baht to
the dollar). Ten baht was a considerable amount of money in Huey Tat in those days.}

{[7] A committee to handle important matters in Christian villages, of which
the headman was an }{\textit{ex officio}}{ member.}

{[8] }{\textit{c}}{ɔ̂}{\textit{-y}}{ɛ̀}{\textit{
yà}}{ʔ}{\textit{ pə̀ tê yân thâ:}}{ lit. ``when [we've]
finished going down [from] church''.}

{[9] By breaking the Sabbath.}

{[10] }{\textit{mâ là ve mâ ph}}{ɛ̀ʔ}{\textit{:}}{
lit. ``cannot not come.''}


