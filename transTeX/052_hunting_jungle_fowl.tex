\setcounter{footnote}{0}

1. When the time comes to trap wild fowl\footnote{\textbf{hɛ́-g̈âʔ}: the first syllable is a sandhi form of \textbf{hɛ} `swidden; mountain ricefield'. This animal [\textit{Gallus gallus}].is called \textit{kàj pàa} in Thai, and \textbf{tô} \textbf{cɛ́ʔ} in Burmese (Written Burmese \textbf{tâw-krak})..} in the mountains, we Lahu just love
to trap them.

2. [We] really want to do it!

3. Early in the morning we carry a [decoy] jungle-fowl and climb up into the mountains,
and when we hear the sound of jungle-fowl calling we're really happy.

4. When we hear the sound, we release [the decoy], and when [the other birds] come
they fight viciously [with the decoy], and when we get a good look at this we carefully
cock our rifles and fire them off, and if we see them die we're very happy.

5. We quickly pick them up and stuff them in our shoulder-bags, and the hunters
who killed them\footnote{\textbf{ɔ̀-šɛ̄-phâ}: lit. ``owner'', here used in the figurative sense of `one who is responsible for an action'.} come close [to receive our congratulations].

6. This is one thing that makes the Lahu very happy.

7. This last couple of days when we went to trap jungle fowl, [the fallen leaves]
were very dry and crinkly, so when I saw a jungle chicken

and tried to shoot it, even though I had my rifle cocked right I slipped and couldn't
get the shot off.

8. This morning I'm not in a very good mood.

9. I can't even enjoy my food.

