
\textbf{69 Picking Tea}

[Tape VI, Side 2]

1: Càbo: Well, let's talk a little bit about picking tea.  A-kí, have you tried
your hand at picking tea yet this year?

2: A-kí: Well, I've only managed to go picking one day so far.  Since I'm so busy
hoeing my rice-field.  But my tea\textit{is} \{doing fine/flourishing\}\footnote{cà ve `to flourish, be doing well, be luxuriant (of growing things).'}!  I
just haven't managed to pick it yet.  I'll just finish hoeing my field first.

3: Càbo: So how is it, Thû-yì? Have you done any tea-picking yet?

4: Thû-Yì: Oh, I've picked two days already.

5: Càbo: Is it good and thick?\footnote{cà dàʔ à lâ.  cà `grow thickly'.}

6: TY: Well, a little bit of it is doing nicely.  Some of it has gotten rough\footnote{šâʔ ve `be rough, hard; to lose tenderness'.}
though.

7: Càbo: Oh, then have you already tried carrying it down to sell at the tea-market\footnote{A four-verb concatenation: ca pû hɔ̂ ni [go and carry sell tea].  Actually a tea-factory (owned by the šathê-ló)}
``place where you sell tea''?  Way over there [at the tea-factory].

8: Thúyì: I've already tried going there.

9: Càbo: This year how much do they say you get for one kilo?

10: Thúyì: They say about a baht and a half.

11: CB: A baht and a half.  Wow, last year we only got 20 satang.\footnote{one baht=100 satang}  This year
we get a baht and a half!

12: TY Those were the leaves, now it's [we're talking about] the shoots.\footnote{Ordinary tea-leaves went for 20 stg/kilo, while the tender shoots go for 11/2 baht/kilo}

13: CB: This year you're going to pick the shoots too, are you!  You know [ôo],
when you do something like plucking the shoots the plant nearly dies!

14: TY: They told us to do that.  [``What they said was, do it this way'']  Pick
the shoots this year, then next year, you get to pick the leaves again.\footnote{The same principle as pruning a tree}

15: CB: Next year you pick the leaves again, they say.

16: TY: Yeah.

17: CB: Well, well!  That's all right, all right then, in that case!  Pā-ɛ́,
this year how many kilo did you get when you picked tea?\footnote{Pa-eh: Pā-ɛ́h (short for Yâ-pā-ɛ́ ``Sonny, Little Boy'') was my chief consultant during my 1977 fieldtrip}

18: Pa-eh: Only five kilo.

19: CB: Only five kilo.  What happened, the tea wasn't growing thick, in your orchard?

20:[Whisper]: Because he's lazy!

21: CB: Because he's lazy.  Alas, that's how it goes with lazy people.  They can't
earn a proper living/food to eat.

22: Pa-eh: Because Thûyì keeps telling me to go visit Thai girls---because that's
all\textit{he} does/he just keeps visiting Thai girls.

23: TY: [We]\textit{have} to go visit Thai girls.  After you work a little---You
just want to see a woman---we don't have wives yet!\footnote{Fragmentary sentence: kÁ a-cí te qo, yâ-mî kàʔ a-cí-nɛ̀-Á ɔ̄ [ɔ̀-mî-ma] mâ cɔ̀ šē ve ɔ̄.}

24: Somebody: When you say ``don't go,'' they don't listen, those guys.

25: CB: How many of them are there, those Thai girls?

26: TY: There's only one, and I've never even gotten to see her kisser! She just
stayed down [in her home] all the time.\footnote{By this slang word I translate Thuyi's jocular coinage phû-mɛ̂ʔ\textit{,} reordering syllables of the ordinary word mɛ̂ʔ-phû}

27: CB: She probably didn't want either one of you.

28: Somebody: They're supposed to be very pretty, the two of them, the prettiest,
they say.

29: Pa-eh: It's those guys Cà-cɔ́, Cà-g̈âʔ, and Cà-qā.  They went to
them [the Thai girls] without being invited.

30: TY: If it wasn't for them, it would have been great for us!

31: CB: Those guys are blushing like girls [ɔ̀-ma-pa qhe ni yò]!

32: Somebody: Even though they [the girls] said ``Go home!'' they [the suitors]
didn't listen.  When they said ``you ought to be ashamed! [this is really shameful]''
they didn't listen, so...

33: Pa-eh: [imitating girl's voice] ``I didn't invite you!  And yet you came.''

34: CB: Oh, here we are, picking tea. and you act this way.  You'll never earn
a living that way!

35: Cà-cɔ́: It's like these guys are just squirting water at each other!

36: CB: Maybe it's because the weather's hot they just keep fooling around, playing
around.

37: Pa-eh: The fact is, young men are happy, and---

38: Somebody: It's actually very good, what the two of them have been saying.
If one of them doesn't get the Thai girl, the other one will, they're saying.

39: TY: If they [the other group of suitors] hadn't come, I bet everything would
have been fine for the two of us.\footnote{qha-dɛ̀ʔ is an adverbial expression, here used as a full verb.}

40: Pa-eh: It would have been fine.

41: Jesus Christ [pòthôo, For crying out loud], when you tend your fields and
pick your tea this way, there's no way you're going to come up in the world.\footnote{Lit., ``it will not fit that you will become like other people.''}
When people teach you, you ignore what they say, for two years now!  You've been
acting this way, both of you...And have you finished hoeing your swiddens [upland
fields]?

42: Pa-eh: I haven't finished, haven't finished---my field either.

43: TY: Well, as for having my fields, as far I've only finished one\textit{tû}\footnote{This word is not well explained in DL. (tê tû mì ɛ̀) From the context here it looks as if one \textit{tû} contains somewhat more than 20 \textit{rai}}
and 20 \textit{rai}.

44: CB: Only one\textit{tû} and 20 \textit{rai}.

45: TY: Yeah.

46: So you got [about] two\textit{tû }from your field!  Hey, Cà-mÁ,\footnote{mÁ `be lucky in hunting'} when
will your father and you do the picking, of your tea?

45: Cà-mÁ: Well, mine, I guess I can't manage to pick yet.  I have to finish
holing my swidden first.

46: CB: After you finish hoeing you'll have at the picking.

47: Cà-mÁ: This field [I mine] I've been thinking I'll probably have to hoe for
two weeks.  Two weeks and five days.

48: CB: Two weeks and five days.

49: Cà-mÁ: Yeah, two weeks and five days.  After I've finished hoeing I'll go
picking, until I've finished the job.\footnote{mâ pə̀ cɛ = pə̀ e ve thâ qha-qà}  I haven't dug [the necessary holes]
for planting my tea-orchard garden yet.\footnote{šu\textit{ }: i.e. the Boss (ša-thê-ló) from whom Huey Tad rents the tea-orchards}

50: Somebody: It'll all be dried up by that time!

51: CB: That's what I thought too!  Mine also I haven't gotten to try picking yet,
that tea.  But still, more that not being able to pick the tea, in my opinion this
is what I think: more than having tea or not having tea, I think not having rice
is a cause for anxiety, so [only] after finishing hoeing the [rice] field do I
intend to pick [the tea].

52: Cà-mÁ: Mine [my rice field] I still have to hoe for a month and a half ``fifteen
days''.  Once I've finished hoeing I also plan to pick.

53: Somebody: By that time yours will be all rough.

54: Somebody else: And tell him that he still hasn't even cleared the land for
his tea garden.

55: CB: Speaking of his tea-garden, it's kept in a densely overgrown state [he
leaves it all overgrown with weeds]

56: Somebody: It's because you're lazy, all you guys.

57: Somebody else: If\textit{he}\footnote{Here the classifier mà, usually referring to objects, is applied to a person, creating a colloquial and somewhat pejorative sense.]} sees it, it will never do, it will never do.

58: CB: Maybe even [mâ šī ɛ̀ ``I don't even know''] when he sees it, he'll
take it back from you.  There's also another guy,18 Cà-qɛ́,\textit{ }like that,
not taking care of his tea-garden!

59: TY [playing role of Cà-qɛ́]: Well, I've had a good talk with the Boss, about
this.

60: CB: What did he say to you then?

61: Cà-qɛ́: What he said was, when you have free time for [this] work, just
carefully slash away the undergrowth\footnote{phɔ̂ `slash' away undergrowth---an important part of the Agricultural Cycle} and then do the picking.  If you don't
do this [slash the undergrowth], the tea won't be any good, he said.

62: CB: Did he tell you that the shoots would get all shriveled up?

63: Cà-qɛ́: Yep, he did.

64: CB: Hmm.

65: TY: When I finish the work...

66: CBL Hey, Cà-lɔ̂, Headman,\footnote{Càbo here, showing this is a play, calls the headman Cà-lɔ̂ (the name of my citibred consultant who accompanied me on my trips to H.T.) instead of his real name Càbí} what about you---haven't you picked your tea
already this year yourself?

67: Headman: Oh, I'm a great tea-picker, you know!\footnote{This sentence is glossed ``Well, I guess I'll really pick tea for a living'' in DL, under ló-lâʔ}

68: CB: As for your picking, you keep saying ``I'll pick, I'll pick!'', but I don't
see that any tea has come [from you]!

69: Headman: I'm not free yet...

70: CB: You're only saying that with your mouth! Whatever your do [it's like that]...Anyhow,
maybe it's because you're lazy, isn't it!

71: Cà-qɛ́: Yes, A while ago for once he did pick.

72: Headman: ...Wherever I have picked it's nice and cleared off!\footnote{yà qhe phə̂ʔ kɨ̀ qôʔ ve \emph{pɔ̂-yì} dàʔ ve ɛ̀ʔ  This word glossed in my fieldnotes with Chinese [Characters], but in DL 861 it's ``a cleared-off area (as from cattle grazing a place clean)''.}

73: Cà-qɛ́: Oh, as for the headman, he hasn't done it [any tea-picking]!  Last
year he once picked with me, and he did pick very well.  [But] this year, one has
to say that he hasn't been so diligent, the headman.

74: Headman: It's just as I said.\footnote{ô qhe qha-šwí yò.}  In the places where I've picked, the bottoms
of the tea-plants are beautiful [là꞊khɨ́ pɨ́].

75: Cà-qɛ́: He does it well!  That guy, if he's free of [other] work and manages
to do it.  But I haven't seen him do it yet, as of now.

76: Somebody: So he's good at it---but he doesn't manage to pick much of it!  If
it's just that he's not good at it...\footnote{i.e., we could do with less expertise and more actual work}

77: Somebody: As for getting it, he's not capable of getting very much, not like
us, when it comes to tea-picking, that guy.\footnote{The translation attempts to capture the multiple syntactic inversions in the Lahu: g̈a tí qo cɨ̂-cɨ̀ mâ g̈a pɨ́ ve, hɛ̀-Á qhe cɛ-cɛ, là phə̂ʔ ve tí qo, yɔ̂.}

78: Somebody: Because he's so good at it!

79: CB: Other people---I've heard that others have gotten five or six kilos, while
he just got a kilo and a half, for heaven's sake!

80: Since he's all duded up, he can't pick tea properly!

81: Somebody else: In order to be able to live on it for a long time you've got
to take care of [one's crop] very well!  Your (pl.) sprouts are starting to bend
and get knocked over\footnote{tà ɔ́n chɛ̂ nɔ́ chɛ̂ ve} when they're only one span high.  This whole year they
won't sprout again!  They'll all dry up!

82: It's just that they're trying to get a lot, by doing it that way.

83: CB: Wow, Yâ-pā-ɛ́\textit{ }breaks off each individual sapling into two
sections and puts them in [the ground].  You're lucky the Boss didn't see you!\footnote{Lit. ``It's because the Boss wasn't looking.''}

84: TY: Well, that's just the way we have to do it!

85: Pā-ɛ́: The reason is, we do it to get a better crop.\footnote{g̈a câ mâ àʔ/tù te ve ``do it to get a lot to eat''}  As for mine,
the tea plants haven't died at all!  The ones that I picked---

86: Somebody: There are lots of people---

87: Pā-ɛ́: We make [break] each sapling into two sections.

88: Somebody: Don't tell me they didn't die!  They did!---

89: TY: Anyway, we really ought to pick it fast!  Otherwise the Boss might take
[the tree-plantations] back some day.

90: CB: Both of you [Pā-ɛ́ \& Thû-yì] are young men, so you should concentrate
your minds28, and if you do the work [necessary] to earn a living, if you can settle
down and do that, you'll get plenty to eat, you two.

91:. TY: Well, you---

91:. Pā-ɛ́: The likes of the two of them are really good at it.

92: CB: They tell me [you've been running around after] Thai girls, eh? ground
squirrels, eh? and striped squirrels, eh? Taking your gun and hiding wherever you
find a shady place---you'll never make a living that way!\footnote{slightly revised from actual sentence in my fieldnotes in DL. p. 475}  When people teach
you, listen to what they say, these people!  [You're] so\{stubborn/thick-skulled\}\footnote{chɔ (m)â mɔ̀ jɔ. A formulaic expression expressing rather jocular disapproval},
[you're] hard to get through to---I've never seen the like.\footnote{nà-qā-pɨ phì ɛ̀ te ve.  Differently glossed in DL 742, 890}

93: TY: Well, what do you teach I listen to, I want to listen to.  Please try teaching
me properly!

94: CB: I just teach the best [``This is all I can teach''] I can, to the people.
And you add the word ``properly!''

95: TY: Other people [i.e.,\textit{we} (Y and Pā-ɛ́)], who \{aren't married/don't
have a wife\}, can't [act] the way you do.

96: CB: Well, how much have you usually gotten each year then, when you've finished
picking?  From your garden, Yâ-pā-ɛ́?

97: Yâ-pā-ɛ́: Well, I---

98: CB: I [qhe] mean, what you pick in the course of one year.

99: Pā-ɛ́: I myself, what I pick in one year---

100: TY: Is it more than a hundred?\footnote{Thû-yì jocularly uses N. Thai \textit{lo}ɛ́ʔ\textit{ pàj} [Si. rɔ́ɔj paj] for `more than 100'. Maybe not jocularly, since the Lahu had to use Thai when bargaining with the Boss and his staff at the tea-factory.}

101: Pā-ɛ́: It wasn't just [``didn't stay at''] ``more than a hundred''!  It
was about 300!\footnote{This number is also given in Lahuized N. Thai \textit{sà} \textit{lo}ɛ́ʔ\textit{ }[Si: sǎam rɔ́ɔj].}

102: CB: Is it 300 then?!  So, it's about 300\footnote{This time Cà-Bo gives the numeral in Lahu: \textit{š}ɛ̂ʔ\textit{ ha}}, you say, in one year, from his.
Well then, A-kí, about how many hundred have you usually gotten in one year,
from yours?

103: A-kí: Well as for me, I'm not able to pick everything [I've got]!  Since
\{I'm all alone/I have to do it all by myself\}, every year my tea goes rough .\footnote{If tea is left too long growing, it becomes hard and unusable}
I don't have a lot of people.  But anyway, it one year I just get about 150 kilo!
As for tea, it's there.  I [just] can't pick it [all].

104: CB: Hmm-

105: A-kí: I have a lot of work, it's not just this [tea-picking] that I have
to do.

106: CB: Yes indeed.  That's the way it goes.  We hillfolk cultivate mountain fields\footnote{swiddens, unirrigated, requiring backbreaking labor to cultivate.  See the Agricultural Cycle in DL \_\_\_\_\_\_.}
for a living, since we don't have [irrigated] paddy-fields or hó-nâ.\footnote{CB uses both the Lahu (ti-mi) and Shan (hó-nâ; cf. Si. râj-naa) for ``paddy-field.''}

107: A-kí: As for that, if we keep fussing with our tea, we won't have enough
to eat!  When we're free [of other work], once in a while we just try to earn some
pocket-money [``seek to use money''].\footnote{I.e., whatever money they earn from tea is just for little luxuries, not vital to survival like the rice crop}

108: CB: Yes. Well then, Headman, didn't you say that you were going to go ask
for labor\footnote{g̈â ca lɔ̀ qɔ̂ ve.  Custom of helping out a fellow villager in times of intense work in the fields, the understanding that this kindness will eventually be reimported when the shoe is on the other foot} to [help you] cultivate?  This year?

109: Headman: Well, I am trying to ask [``go and ask''] for help.  From among you
guys, my relatives---

110: CB: If you are going to ask for help, that's very fine!  No matter how many
people come from each household [respond to your request]---whether it's two people
or three people.

111: TY: Nobody will come!  Don't go asking them, don't go asking them!

112: Headman: Since I won't be giving them any wages, whether I ask for help or
not \{is my business/depends on me\}.  If I ask them and they don't come, other
[šu]\footnote{The referents of the two occurences of šu `remote 3rd person' in the sentence is not clear: lɔ̀ lɛ mâ là qo, šu àʔ tho pî ò ve lɛ, kâ ò qo mâ là ve qo, šu àʔ cɔ̀ ve yò.} people will be told, and when these others [they] hear that they didn't
come, it will be their fault [šu àʔ cɔ̀ ve yò]\footnote{I.e., it will not be to their credit, but it's up to them}

113: CB: Yes.  Do try asking! Even if one person doesn't come, another person will---that's
the way people are.\footnote{Translating the afterthought topic ``chɔ lɛ̀''}  What day are you planning to go ask for help?

114: Headman: I'm planning to go and try asking on Wednesday.\footnote{The headman naturally uses the X'n way of naming days of the week.}

115: CB: On Wednesday.  Please don't do it on Wednesday.  It's [too] hard to carry
the hoes back.\footnote{The idea seems to be that as the work-week goes on, it becomes more and more difficult to drag one's tools to the fields.}  If you do it after the weekend, do it on Monday.  As long as
people have carried the hoes back on Saturday.

116: Headman: Well, then, I'll do it on Tuesday!

117: CB: Mm-hm.  If you do it on Tuesday, that'll be fine.

118: Headman: [giving the reason he wants to do it Tues. instead of Mon.] For one
day I just [have/ want to] do to my [field/ swidden] and pick up [my tools] and
carry them back.  After doing a few licks of work in the swidden too.

119: CB: I've heard that you [guys] now get one and a half baht [per kilo] per
person for picking tea.\footnote{Cà-bo is apparently speaking of how much a Lahu might earn picking tea for the Thai.}  As for me, I would only get 25 satang, [that's what]
they [gave] me.  Even for one kilo.

120: Headman: Wow.

121: TY: It's because you're [i.e., Cà-bo] no good at picking!  Because you're
not like other people.

122: CB: That's not so!  If you look at what I pick, it's the same [as others].

123: Headman: I don't think it's like that [i.e., that they discriminate in pay
according to who the picker is].  That it's ``each person a different kind''.\footnote{tê g̈â tê yān ``one person one kind'' yān\textit{ <} Shan (Su, jàan), i.e., each person differently.}

124:CB: Maybe they\textit{do} discriminate.\footnote{tê g̈â tê yān te ve may be paraphrased as: te ve kÁ qha-šū kàʔ pî ve ɔ̀-phû mâ šū ``Although the work done is the same, the wages given are not the same.''\}}  They may like/favor one person
and hate another one.

125: Pā-é: If you don't agree on the terms carefully [``don't talk with each
other''], that's what happens.

126: Headman: Well, I'm going to discuss this carefully again.

127: CB: You relatives, if you're friendly with him [the Boss]\footnote{Lit ``if you know his words and his breath''}, you just ought
to help me speak.

128: TY: It's probably because you're not acquainted with each other.\footnote{TY uses the Thai expression hû-câʔ-kā (Si. rúucàk kan) `be acquainted with each other', instead of Lahu šī dàʔ ve.  Probably because a Thai person is involved.}

129: CB: No, we're not acquainted [šī dàʔ], me anyway.

130: TY: Well, we all are acquainted with him.

131: CB: Hmm---

