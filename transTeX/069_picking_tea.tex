\setcounter{footnote}{0}


1. T: Well, let's talk a little bit about picking tea. A-kí, have you tried
your hand at picking tea yet this year?

2. A-kí: Well, I've only managed to go picking for one day so far.\footnote{Note the 4-verb concatenation: \textbf{g̈a} 'get to V' + \textbf{ca} 'go and V' + \textbf{phə̂ʔ} 'pick' + \textbf{ni} 'try and V'.} Since I'm
so busy hoeing my rice-field. But my tea \textit{is} doing fine! I just haven't
managed to pick it yet. I'll just finish hoeing my field first.

3. T: So how is it, Thû-yì? Have you gotten any tea-picking done yet?

4. Ty: Oh, I've picked for two days already.

5. T: Is it good and thick?

6. Ty: Well, a little bit of it is doing nicely. Some of it has gotten rough though.

7. T: Oh, then have you already tried carrying it down to where they sell the
tea\footnote{A four-verb concatenation: \textbf{ca} 'go and V' + \textbf{pû} 'carry to V' + \textbf{hɔ̂} 'sell' + \textbf{ni} 'try and V'. The place in question was actually a tea-factory owned by \textbf{šathê-ló}, the "Big Boss".} way over there?

8. Ty: I've already tried going there.

9. T: This year how much do they say you get for one kilo?

10. Ty: They say about a baht and a half.

11. T: A baht and a half. Wow, last year we only got one baht and 20 satang.
\footnote{There are 100 satang in one baht. In the 1960's the exchange rate was 20 baht to the dollar. The actual difference between one baht plus 20 satang vs. one baht plus fifty satang was minuscule.} This year we get a baht and a half!

12. Ty: Those were the leaves, now we're talking about the shoots.\footnote{Ordinary tea-leaves went for 1.2 baht (one baht plus 20 satang) per kilo, while the tender shoots went for 1.5 baht (one baht plus 50 satang) per kilo.}

13. T: This year you're going to pick the shoots too, are you! You know, when
you do something like plucking the shoots the plant nearly dies!

14. Ty: They told us to do that. Pick the shoots this year, then next year you
get to pick the leaves again.\footnote{The same principle as pruning a tree.}

15. T: Next year you pick the leaves again, they say.

16. Ty: Yeah.

17. T: Well, well! That's all right, all right then, in that case! Pā-ɛ́,
\footnote{\textbf{Pā-ɛ́} (short for \textbf{Yâ-pā-ɛ́} ``Sonny, Little Boy'') was my chief consultant during my 1977 fieldtrip.} this year how many kilo did you get when you picked your tea?

18. Pa-eh: Only five kilo.

19. T: Only five kilo. What happened, the tea wasn't growing thick, in your orchard?

20. [Whisper]: Because he's lazy!

21. T: Because he's lazy. Too bad, that's how it goes with lazy people. They
can't earn a proper living.

22. Pa-eh: Because Thûyì keeps telling me to go visit Thai girls---because that's
all \textit{he} does!

23. Ty: We \textit{have} to go visit Thai girls. After you work a little you just
want to see a woman. We don't have wives yet!

24. Somebody: When you say ``don't go,'' they don't listen, those guys.

25. T: How many of them are there, those Thai girls?

26. Ty: One of them there I've never even gotten to see the kisser\footnote{By this slang word I translate Thuyi's jocular coinage \textbf{phû-mɛ̂ʔ}, which is a reordering of the syllables of the ordinary word \textbf{mɛ̂ʔ-phû}.} of! She
just stayed down [in her house] all the time.

27. T: She probably didn't want either one of you.

28. Somebody: They're supposed to be very pretty, the two of them, the prettiest
there are, they say.

29. Pa-eh: It's those guys Cà-cɔ́, Cà-g̈âʔ, and Cà-qā. They went to
them without being invited.

30. Ty: If it wasn't for them, it would have been great for us!

31. T: Those guys are blushing like girls!

32. Somebody: Even though they [the girls] said ``Go home!'' they [the suitors]
didn't listen. When they said ``You ought to be ashamed!'' they didn't listen,
so...

33. Pa-eh: [imitating girl's voice] ``I didn't invite you! And yet you came.''

34. T: Oh, here you are, picking tea to earn money. and you act this way. You'll
never earn a living that way!

35. Cà-cɔ́: It's like these guys are just squirting water at each other!

36. T: Maybe it's because the weather's so hot that they just keep fooling around,
playing around.

37. Pa-eh: The fact is, young men are carefree, and---

38. Somebody: It's actually very good, what the two of them have been saying.
If one of them doesn't get the Thai girl, the other one will, they're saying.

39. Ty: If they [the other group of suitors] hadn't come, I bet everything would
have been fine

for the two of us.

40. Pa-eh: It would have been fine.

41. T: My God, my God, if this is how you tend your fields and pick your tea,
there's no way you're going to come up in the world. When people teach you, you
ignore what they say, for two years now! You've been acting this way, both of
you. And have you finished hoeing your swiddens?

42. Pa-eh: I haven't finished, haven't finished---my field either.

43. Ty: Well, as for hoeing my fields, so far I've only finished one \textit{tû}
and 20 \textit{rai}.\footnote{From the context here it looks as if one \textbf{tû} contains somewhat more than 20 \textit{ray. }See \textbf{tê} \textbf{tû} \textbf{mì} \textbf{ɛ̀} (DL 613).}

44. T: Only one \textit{tû} and 20 \textit{rai}.

45. Ty: Yeah.

46. T: So you got about two \textit{tû }from your field! Hey, Cà-má, Mary's
father,\footnote{This name derives from the verb \textbf{má} `be lucky in hunting'. This man was also called \textbf{Mɛ́lè-pa} ``Mary's father'', by the nomenclatural phenomenon of teknonymy, ubiquitous in Southeast Asia.} when will you be picking your tea?

47. Cà-má: Well, I guess I can't manage to pick mine yet. I have to finish hoeing
my swidden first.

48. T: After you finish hoeing you'll tackle the picking.

49. Cà-má: This field [of mine] I've been thinking I'll probably have to hoe
for two weeks. Two weeks and five days.

50. T: Two weeks and five days.

51. Cà-má: Yeah, two weeks and five days. After I've finished hoeing I'll go
picking, until I've finished the job. I haven't dug [the holes] for planting my
own tea-garden yet.

52. Somebody: It'll all be dried up by that time!

53. T: That's what I thought too! I haven't been able to try picking that tea
of mine yet either. But still, more than not being able to pick the tea, this is
what I think: more than having tea or not having tea, I think not having \textit{rice}
is a cause for anxiety, so only after finishing hoeing the [rice] field do I intend
to pick [the tea].

54. Cà-má: I still have to hoe my field for a month and fifteen days. Once I've
finished hoeing I also plan to do the picking.

55. Somebody: By that time yours will be all rough.

56. Somebody else: And tell him that he still hasn't even cleared the land for
his tea garden!

57. T: Speaking of his tea-garden, he leaves it all overgrown with weeds.

58. Somebody: It's because you're lazy, all you guys.

59. Somebody else: If \textit{he }\footnote{\textbf{šu} : i.e. the Big Boss (\textbf{ša-thê-ló}) from whom the Huey Tat villagers rent the tea-orchards.} catches sight of it, it will never do, it
will never do.

60. T: Maybe when he sees it, he'll even take it back from you. There's also
another guy, Cà-qɛ́,\textit{ }who's\textit{ }like that, not taking care of his
tea-garden!

59. Cà-qɛ́: Well, I've had a good talk with the Boss about this.

60. T: What did he say to you then?

61. Cà-qɛ́: What he said was, when you have free time for this work, just carefully
slash away the undergrowth and then do the picking. If you don't do this, the
tea won't be any good, he said.

62. T: Did he tell you that the shoots would get all shrivelled up?

63. Cà-qɛ́: Yep, he did.

64. T: Hmm.

65. Ty: When I finish the work...

66. T: Hey, Cà-bí, Headman, what about you---haven't you picked your tea already
this year yourself?

67. H: Oh, I'm a great tea-picker, you know!\footnote{In DL:1379, under \textbf{ló-lâʔ}, this sentence is glossed ``Well, I guess I'll really pick tea for a living'' -- an equally possible translation in a different context.}

68. T: You and your picking! You keep saying ``I'll pick, I'll pick!'', but I
don't see that any tea has come [from you]!

69. H: I haven't had time yet.

70. T: You're only saying that with your mouth! Whatever your do [it's like that].
Anyhow, maybe it's just because you're lazy, isn't it!

71. Cà-qɛ́: Yes, a while ago for once he did pick.

72. H: I'm not too lazy to clear my land! Wherever I've picked it's spic-and-span!
\footnote{The basic meaning of \textbf{pɔ̂-yì}, here translated 'spic-and-span' is 'a cleared-off area (as from cattle grazing a place clean)' (DL:861).} It's

the best it can be!

73. Cà-qɛ́: Oh, the headman hasn't done any [tea-picking]! Last year he once
picked with me, and he did pick very well. But this year, one has to say that
he hasn't been so diligent, the headman.

74. H: It's just as I said. In the places where I've picked, the bottoms
of the tea-plants are beautiful.

75. Cà-qɛ́: He does it well, that guy, if he's free of other work and manages
to do it. But he hasn't done it yet, as of now.

76. Somebody: So he's good at it---but he doesn't manage to pick much! If it's
just that he's good at it...\footnote{I.e., we could do with less expertise and more actual work!}

77. Somebody: As for how much he gets, unlike us he's not capable of getting very
much, when it comes to tea-picking, that guy.\footnote{The translation attempts to capture the multiple syntactic inversions in the Lahu: \textbf{g̈a} \textbf{tí} \textbf{qo}, \textbf{cɨ̂-cɨ̀} \textbf{mâ} \textbf{g̈a} \textbf{pɨ́} \textbf{ve}, \textbf{nɛ̀-á} \textbf{qhe} \textbf{cɛ-cɛ}, \textbf{là} \textbf{phə̂ʔ} \textbf{ve} \textbf{tí} \textbf{qo}, \textbf{yɔ̂}. The most straightforward ordering of these 5 syntactic elements would be: \textbf{g̈a} \textbf{tí} \textbf{qo}, \textbf{yɔ̂}, \textbf{nɛ̀-á} \textbf{qhe} \textbf{cɛ-cɛ}, \textbf{là} \textbf{phə̂ʔ} \textbf{ve} \textbf{tí} \textbf{qo}, \textbf{cɨ̂-cɨ̀} \textbf{mâ} \textbf{g̈a} \textbf{pɨ́} \textbf{ve}. [1-5-3-4-2]}

78. Somebody: Because he's so good at it!\footnote{A sarcastic remark.}

79. T: Other people---I've heard that others have gotten five or six kilos, while
he just got a kilo and a half, for heaven's sake!

80. Since he's all duded up, he can't pick tea properly!

81. Somebody else: In order to be able to live on it for a long time you've got
to take care of [your crop] very well! Your shoots are starting to bend and get
knocked over when they're only one span high. This whole year they won't sprout
again! They'll all dry up!

82. It's just that they're trying to get a lot,\footnote{\textbf{g̈a} \textbf{mâ} \textbf{ve} \textbf{àʔ} \textbf{te} \textbf{ve} \textbf{ɛ̀ʔ}: This sentence exemplifies the accusative of purpose (not recognized in GL). Later in the text a similar expression occurs: \textbf{g̈a} \textbf{câ} \textbf{mâ} \textbf{àʔ} \textbf{te} \textbf{ve} 'do it in order to get a lot to eat'. This use of the accusative particle \textbf{àʔ} or \textbf{thàʔ} is identical to that of the purposive particle \textbf{tù}.} by doing it that way.

83. T: Wow, Yâ-pā-ɛ́\textit{ }breaks off each individual sapling into two
sections and puts them in [the ground]. You're lucky the Boss didn't see you!

84. Ty: Well, that's just the way we have to do it!

85. Pā-ɛ́: The reason is, we do it to get a better crop. As for mine, the tea
plants haven't died at all! The ones that I picked---

86. Somebody: There are lots of people---

87. Pā-ɛ́: We break each sapling into two sections.

88. Somebody: Don't tell me they didn't die! They did!

89. Ty: Anyway, we really ought to pick it fast! Otherwise the Boss might take
them [the tea plantations] back some day.

90. T: Both of you [Pā-ɛ́ and Thû-yì] are young men, so you should concentrate
your minds, and if you do the work to earn a living, if you can settle down and
do that, you'll get plenty\footnote{\textbf{šɨ} \textbf{ɛ̀} 'awfully much; plenty', ult. < \textbf{šɨ} 'die'. See DL:1232.} to eat, you two.

91. Ty: Well, you---

91. The likes of the two of them are really good at it.

92. T: They tell me [you've been running around after] Thai girls, eh? ground
squirrels, eh? and striped squirrels, eh?\footnote{\textbf{fâʔ-šwɛ}: 'red-cheeked ground squirrel' \textit{[Dremomys sp.]}; \textbf{fâʔ-gàʔ} 'striped ground squirrel' \textit{[Lariscus sp.]}. For slightly different identifications, see DL 1308, 1306.} Taking your gun and hiding wherever
you find a shady place\footnote{\textbf{mû-cha} \textbf{bà} \textbf{ve} (Nspec + V) 'be shaded from the sun'.}---you'll never make a living that way! When people
teach you, listen to what they say, those people! You're so thick-skulled and
stubborn\footnote{\textbf{nà-qā-pɨ} phì \textbf{ɛ̀} \textbf{te} \textbf{ve} 'be thick-skulled', lit., "be flat in the forehead". A less metaphorical meaning 'have a fat and pudgy face' is recognized in DL 742 and 890.}, hard to get through to---I've never seen the like.\footnote{\textbf{chɔ} (m)â \textbf{mɔ̀} \textbf{jɔ}. A formulaic expression expressing rather jocular disapproval.}

93. Ty: Well, what you teach I listen to, I want to listen to. So please try
teaching me properly!

94. T: I just teach the best I can, to the people. And you add the word ``properly''!
\footnote{Cabo pretends to be a bit miffed at this impertinence.}

95. Ty: Other people who don't have a wife can't act the way you do.

96. T: Well, how much have you usually gotten each year then, when you've finished
picking? From your garden, Yâ-pā-ɛ́?

97. Yâ-pā-ɛ́: Well, I---

98. T: Like what you pick in the course of one year.

99. Pā-ɛ́: I myself, what I pick in one year---

100. Ty: Is it more than a hundred?\footnote{\textbf{Thû-yì} jocularly uses N. Thai \textbf{lwɛ̀} \textbf{paj} [Si. rɔ́ɔj \textbf{paj}] for `more than 100'. On the other hand, perhaps he was not being particularly jocular, since the Lahu had to use Thai when bargaining with the Boss and his staff at the tea-factory.}

101. Pā-ɛ́: It wasn't only ``more than a hundred''! It was about 300!\footnote{This number is also given in Lahuized N. Thai: \textbf{šàn} \textbf{lwɛ̀} [Si: sǎam rɔ́ɔj].}

102. T: Is it 300 then? So he says his [yield] is about 300\footnote{This time \textbf{Cà-bo} gives the numeral in Lahu: \textbf{šɛ̂ʔ} \textbf{ha}.}, in one year.
Well then, A-kí, how many hundred have you usually gotten in one year, from yours?

103. A-kí: Well as for me, I'm not able to pick everything [I've got]! Since
I have to do it all by myself, every year my tea goes rough.\footnote{If tea is left growing for too long, it becomes hard and unusable.} I don't have
a lot of people [to help]. But anyway, in one year I just get about 150 kilo!
As for the tea, it's there. I just can't pick it all.

104. T: Hmm-

105. A-kí: I have a lot of work, it's not just this [tea-picking] that I have
to do.

106. T: Yes indeed. That's the way it goes. We hillfolk cultivate mountain fields
\footnote{Swiddens, unirrigated fields in the hills, require backbreaking labor to cultivate.} for a living, since we don't have [irrigated] paddy-fields or hó-nâ.\footnote{\textbf{Cà-bo} uses both the Lahu (\textbf{ti-mi}) and Shan (\textbf{hó-nâ}; cf. Si. râj-naa) words for ``paddy-field.''}

107. A-kí: As for that, if we keep fussing with our tea, we won't have enough
to eat! When we're free [from other work], once in a while we just try to earn
some pocket-money.\footnote{I.e., whatever money they earn from tea is just for little luxuries, not vital to survival like the rice crop.}

108. T: Right! Well then, Headman, didn't you say that you were going to go ask
for labor\footnote{\textbf{g̈â} \textbf{ca} \textbf{lɔ̀} \textbf{qɔ̂} \textbf{ve}: the custom of helping out a fellow villager in times of intense work in the fields, with the understanding that this kindness will eventually be reciprocated when the shoe is on the other foot} to [help you] cultivate this year?

109. H: Well, I am trying to ask for help. From among you guys, my relatives---

110. T: If you are going to ask for help, that's very fine! No matter how many
people respond from each household---even if it's only two or three people.

111. Ty: Nobody will come! Don't go asking them, don't go asking them!

112. H: Since I won't be giving them any wages, whether I ask for help or
not is my business. If I ask them and they don't come, other people will be told,
and when these others hear that they didn't come, it will be to their discredit.

113. T: Yes. Do try asking! Even if one person doesn't come, another person will---that's
the way people are.\footnote{Translating the afterthought topic \textbf{chɔ} \textbf{lɛ̀} 'as for people'.} What day are you planning to go ask for help?

114. H: I'm planning to go try asking on Wednesday.\footnote{The headman and his interlocutors naturally use the Christian way of naming days of the week.}

115. T: On Wednesday. Please don't do it on Wednesday. It's [too] hard to carry
the hoes back.\footnote{The idea seems to be that as the work week goes on, it becomes more and more difficult to drag one's tools to the fields.} If you do it after the weekend, do it on Monday. As long as
people have carried the hoes back on Saturday.

116. H: Well, then, I'll do it on Tuesday!

117. T: Mm-hm. If you do it on Tuesday, that'll be fine.

118. H: For one day I just want to go to my field and pick up [my tools]
and carry them back. After doing a few licks of work in the field too.

119. T: I've heard that you guys now get one and a half baht [per kilo] per person
for picking tea. Me now, I would only get 25 satang, that's what they gave me.
Even for one kilo.

120. H: Wow.

121. Ty: It's because you're no good at picking! Because you're not like other
people.

122. T: That's not so! If you look at what I pick, it's the same [as others].

123. H: I don't think it's like that--that it's ``each person a different
way.''\footnote{\textbf{tê} \textbf{g̈â} \textbf{tê} \textbf{yān} ``one person one kind'': \textbf{yān} < Shan (Siamese\textit{ jàaŋ}), i.e., treating each person differently.}

124. T: Maybe they \textit{do} discriminate. They may favor one person and hate
another one.

125. Pā-é: If you don't agree on the terms carefully, that's what happens.

126. H: Well, I'm going to discuss this carefully again.

127. T: You relatives, if you're friendly with him [the Boss]\footnote{Lit. ``if you know his words and his breath''.}, you just ought
to help me speak.\footnote{Note the 4-verb concatenation: \textbf{g̈a} 'must' + \textbf{ga} 'help' + \textbf{qôʔ} 'speak' + \textbf{cɔ̂} 'ought to'.}

128. Ty: It's probably because you're not acquainted with each other.\footnote{TY uses the Thai expression \textbf{hû-câʔ-kā} (Si.\textit{ rúucàk kan}) `be acquainted with each other', instead of Lahu \textbf{šī} \textbf{dàʔ} \textbf{ve}, probably because a Thai person is involved.}

129. T: No, we're not acquainted, me anyway.

130. Ty: Well, we all are acquainted with him.

131. T: Hmm---

