
\textbf{\#157 I'm feeling sad}

[Told in Red Lahu in a chanting style; the speaker was more than a bit tipsy, and
from Sentence 5 on was chewing betel.]

1. Today I went to Na-lí a-a-and there were no friends the-e-ere, and I'm so sad
I could die. \footnote{\textbf{šɨ-la-o}: 'so much one could die; to death'. An emphatic sentence-final tag, probably calqued on Thai V + \textbf{cətaaj}.}

2. Si-i-inging up there I we-e-nt, [but now] my heart is sa-a-ad and I'm going
to di-i-ie.

3. I went to Na-lí and came back, but no friends were the-e-re, not a single friend
was the-e-re, and I feel so cra-a-ppy \footnote{\textbf{chɛ̂ bɔ̀}: lit. ``be weary of living'', i.e. 'be totally wretched'.} I could just die.

4. I came ho-o-me and put my stuff in the house, and went to take a dip in the
river.

5. I went into the wa-a-ater, and there was another young guy there-- \footnote{The last clause of this sentence was inaudible because of the speaker's betel-chewing.}

6. It took two or three hours \footnote{\textbf{ko-mon}: the speaker uses the Tai word for 'hour' (cf. Si. \textbf{chûamoŋ}], or simply its first syllable \textbf{ko-} (cf. Si. \textbf{chûa-}), instead of the usual Lahu word \textbf{nālī} ( < Bs < Indic).} to go up there [to Na-lí], and two or three
hours to come ba-a-ack, and I put my stuff in the house---[but] I went up there,
I just went to have a chat. \footnote{At this point the speaker lapses into incoherence.}

