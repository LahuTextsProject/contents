
{\large{}\textbf{\#1 }}{\large{}\textit{\textbf{Remarks on the Lahu}}}

{\large{}\textit{[Cà-l}}{\large{}ɔ}{\large{}\textit{̂]}}

1. I'd like to tell you something now about the Lahu people.

2. In the past the people called Lahu were not folks who lived here in Thailand.

3. They were people who lived in Burma.

4. But if one asks why they came here to Thailand to live, you all know [the answer]
very well.

5. Burma right now is not a country where you can live comfortably.

6. It's not like in the past when everybody could think what they wanted and do

what they wanted! \footnote{When the context is clear, as here, the general 3rd person pronoun \textbf{yɔ̂} can be used in a plural sense, like the more specific \textbf{yɔ̂-hɨ}.}

7. It's because ever since the new policy of Red China has come out nobody can
do what they want to do.

8. So since you can't do as you like, the people are unhappy.

9. So I think that's why you all had to come fleeing here.

10. The thing is, even some of the Lahu living in Burma were not really people

who used to live in Burma from times past.

11. They used to live in China, in a place called Yunnan Province.

12. As for why they came to live in Burma, two or three hundred years ago they
came down from China, fleeing down into Burma.

13. When they came fleeing down into Burma they also really suffered a lot.

14. So then after living a while in Burma, soon the Burmese government, that Burmese
regime, \footnote{The speaker first uses \textbf{acúyàʔ}, the Burmese-derived word for 'government', then the Thai-derived synonym \textbf{lâthâbân}.} came out with those Chinese-type rules, so the Lahu came fleeing down
into Thailand, like the Shan.

15. Lots of Lahu, Lisu, and Chinese have come fleeing down into Thailand.

16. The reason why they've come fleeing down here is the same for everybody.

