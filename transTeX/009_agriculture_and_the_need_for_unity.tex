
{\large{}9 Agriculture and the Need for Unity}

{\large{}[Tape 1,  Side 2 \#4 and Tape II, Side 1 \#1]}

{\large{}1: So, this time we'll have a discussion about how we Lahu grow our crops.
Well, whereabouts will all of you in the village grow your crops this year? }

{\large{}2: Well, I'll be down below. Where I farmed last year. Probably I'll try
Elephant's-Death-Valley.[1]}

{\large{}3: Mm, I'll be over there <gesture>. I'll work around
Thai-lai.[2] I won't come down below. The elephants eat [the crops] up all the
time [down there].}

{\large{}4: Ha, ha! }

{\large{}5: Oh, my brothers! When human beings[3] are few and not numerous,[4]
they should seek their livelihood together. Don't each work in a different place.
If someday one of you should get sick or fall on bad times,[5] there would be nothing
he could do. We mustn't have our heads full of stones, you know.[6]}

{\large{}6: Well, as far as I'm concerned, I'm going to work over there. Down below
I will not go. I'll work all by myself.}

{\large{}7: Ha, ha!}

{\large{}8: Well, in that case you really ought to pay attention to the words of
our forefathers: ``When we cultivate the soil, if we work all together,
even if sickness or bad times should befall, to look after one another, to help
one another, to do for one another is easy!``   }

{\large{}9: Can't you work that way, too? Must everybody work all by himself? }

{\large{}10: Ah, me! Ah, me!}

{\large{}11: It's absolutely right, that saying! We really ought to work together.
}

{\large{}12: When you get sick and weak, you can't go lifting and carrying if you're
all alone.}

{\large{}13: Well, whatever happens, even if you say ``I won't listen!``
to my words of advice as we talk to each other, even if you people each work the
land by yourself, as you have been intending to--wherever you work, at least work
well! Cultivate the soil properly, and try to earn your food and drink, so that
you will not be inferior to others.[8]}

{\large{}14: For my part, I'm going to clear my land.[9] I don't care! Even if
it's all by myself, even if I'm the only one, I'll clear my land over there. I'll
be there clearing it off, from now on.}

{\large{}15: But we can't act this way, brothers! Now, at this moment, you know
we are few and not numerous! We ought to earn our living together, happily and
cheerfully, and filled with love. People like us, we Lahu, you know that there
aren't masses and hordes[13] of us! There are only a few of us! If someday somebody
gets sick, it should not be that no one sees it; if someone is about to die, it
should not be that no one sees it. There aren't too many of us, and that's a fact!
}

{\large{}16: As for me, I'm going to hack out an oldfield.[14]}

{\large{}17: You're hacking out an oldfield, are you?}

{\large{}18: Yes.}

{\large{}19: Well, even from an oldfield one can hack out a living. }

{\large{}20: It's just that it's kind of overgrown with weeds.}

{\large{}21: [Yeah], the weeds are thick [in an oldfield].}

\begin{center}
{\large{}* * *}
\end{center}

\leftskip=0pt
{\large{}22: Well, has everybody in our village finished clearing the trees from
his land this year? }

{\large{}23: Oh no, I haven't finished my clearing yet. I haven't finished, but
there's nothing I can do about it. Some Thais set my land the hell on fire for
me![15] I don't know now if I'll even be able to do my second-burning,[16] either.}

{\large{}24: Yeah, mine hasn't burned over thoroughly, either! I guess I'll still
have to re-burn for two or three days now. }

{\large{}25: Ah, if only you had listened that day to the words your elders spoke,
if only you had been working together in one group, as a unit, they wouldn't have
set fire [to your land]. If you had looked after it, if you had taken care of it.
But today, since you paid no attention to your elders' words, and each acted according
to his own way of thinking, they set fire to you and here you are unable to do
your second-burning. }

{\large{}26: There's nothing to be done. You didn't listen that time, you paid
no attention to what was said, and now this is what happened! All you can do [at
this point] is re-burn as much as possible[18] and [try to] live off that much.
This year your fields aren't burned off, and who knows whether you'll get any rice
to eat or not. }

{\large{}27: Ah, my brothers, all of you! If we are to live in loving harmony--even
if  someone once disregarded advice--we must help one another! We mustn't squabble
with each other like this. We ought to keep help each other in the future. If we
do this we will earn our food and drink, our clothing and garments, and we will
not be inferior to others, so that the Lahu community[20] will prosper.[21] On
the other hand, if we don't act [properly] this way, when one person gets in trouble,
others will suffer for it.[23]}

\begin{center}
{\large{}* * *}
\end{center}

\leftskip=0pt
{\large{}Well, has everybody finished reburning his field by now? If you've finished,
say so now![24]}

{\large{}28: Oh, it's all finished--mine, that is.}

{\large{}29: Well, if you're really finished, it's time now to plant the rice!
Have you all tried planting your rice yet?}

{\large{}30: Well, I've only managed to begin trying to plant a little so far.
This week I've got to begin trying to do more and more.[25]}

{\large{}31: Well then, Jali, how many acres[26] have you finished [planting] already?
}

{\large{}32: Oh, I've still only finished one acre. I've only managed to plant
one acre so far.}

{\large{}33: One acre, eh?}

{\large{}34: What you've been planning -- how many acres have you been thinking
of planting?}

{\large{}35: According to my plans, I've been thinking of planting five acres.}

{\large{}36: Five acres. }

{\large{}37: Mm-hm.}

{\large{}38: Yeah, if you do that you'll certainly earn your food and drink and
won't be less well-off than anyone else. }

{\large{}39: Well, Kheh-ki, how many acres will }{\large{}\emph{you}}{\large{}
plant? }

{\large{}40: Well, I've been thinking of planting five or six acres, too. I'll
simply try to do as much as I can. I'm just beginning to work on it now--and the
time still isn't too late.}

{\large{}41: Right. Go to it! Well, then, Sexton[30], how many acres will you plant?
}

{\large{}42: Oh, I think I'll plant ten acres! }

{\large{}43: Wow!}

{\large{}44: But at this point, all I've gotten done is to make a field-hut.[31]}

{\large{}45: Well, do your best! As our forefathers used to say:}

{\large{}Slow to till and toil--rats!}

{\large{}Slow to seek and drive away--rats!}

{\large{}Slaves and thralls of others!}

{\large{}If you're slow to till and toil:}

{\large{}Rat food, bird food.[32]}

{\large{}Everybody ought to hurry up and reburn his land properly, and cultivate
it. }

{\large{}46: I haven't even finished my reburning yet!}

{\large{}* * *}

{\large{}47: Tomorrow for sure[33], I'm planning to go way off over there, to the
place where they say some people smeared hot peppers on a man's prick long ago,[34]
to scoop out a honeycomb to lick. }

{\large{}48: Oh, I'm going too! As far as honey-scooping goes, I've seen a bees'-hollow
myself. Over there above the river-bed on Porcupine Mountain.[35] Say, let's go
together, shall we? }

{\large{}49: Yeah, we can certainly go together. }

{\large{}50: Early tomorrow morning when it's time to go you just call me, too,
okay? The two of us will go off together at the crack of dawn.[36] I've seen a
bees'-hollow myself.}

{\large{}51: Okay, but don't do to me what somebody else did that time.[37]}

{\large{}52: If I say I'll go, I'll go. Don't talk nonsense to me. }

{\large{}53: Let's go together, happily and with good cheer! }

{\large{}54: Because we two have never yet broken a promise. Whenever anybody discusses
something and decides on it, we just have a go at it.[38] We'll go early tomorrow
morning. }

{\large{}55: Right, we've never broken [a promise] in the past.}

{\large{}56: Say, let's take the dogs along, too! Pastor! Kheh-ki! Jali! We'll
go shoot ourselves some macaques[39] and little gibbons.[40] We'll go hunt all
kinds of porcupines and spend the night out[41]!}

{\large{}57: Well, then, Jali, you can bring your dog along for us!}

{\large{}58: Yeah, sure I can take mine along. But he can't hunt monkeys, you know!}

{\large{}59: I bet he can hunt barking-deer, though. Even if he can't hunt monkeys.
}

{\large{}60: Sure he can hunt, sure he can hunt barking-deer.}

{\large{}61: So, that's how we'll go then!}

{\large{}62: If you just shout and make noise at the hunting-grounds, even if the
dogs don't know how to  give chase, those monkeys will come out [all by themselves].
But as far as barking-deer are concerned, if the dogs don't chase them, they won't
come out. At the hunting-grounds we'll chase them and beat them and get them, we'll
chase them and shoot them and get them! We'll sneak up on them and shoot them.
}

{\large{}63: This is the time when we've already gone and secured places to grow
our crops, and almost managed to finish planting our rice and so let's just keep
on going hunting [for game] to eat, and earn our living in joy and gladness. }

\begin{center}
{\large{}* * *}
\end{center}

\leftskip=0pt
{\large{}64: Well, Sexton, they say some wild boars have eaten your paddy, is that
right? }

{\large{}65: Yes.}

[TAPE II, Side 1]

66: Well, since wild boars have eaten rice from the Sexton's field,  tomorrow we've
got to help hunt them down and beat them dead for him, right? [200]

67: Yeah, you've got to help me hunt them down and beat them to death . The rice
itself is very good [this year in my field] .

68: Otherwise you won't have anything to eat. If \textit{they} eat it all up.

69: Whether we catch them or not, let's go drive them away for him. If we drive
them away, they won't come back for a while, at least. We'll just go and drive
them away for him. Whether or not we catch them we'll go give them a chase. Tomorrow
we'll go.

70: Right!

\begin{center}
* * *
\end{center}

\leftskip=0pt
Well, now the time has come for harvesting our rice!

71: Yep, rice-harvest time has come. Tomorrow I plan to go start trying to reap.
Since the wild pigs are even eating away at it.

72: Kheh-ki, have you managed to try harvesting yet?

73: Yes, I plan to try to harvest. My field has just begun to ripen a little too
now. Tomorrow I'll go reap, I guess.

74: Jali, what about you?

75: Well, mine is being harvested.  Even today.

76: I see. Well, how many paddy-mounds\footnote{\textit{cà-p}ū: 'mound of paddy, left to dry in the fields for a couple of weeks between harvesting and threshing'. \textit{p}ū is the classifier for these paddy-mounds or rice-stoops.}  have you gotten this year--I mean \emph{you}?\footnote{\textit{n}ɔ̀\textit{ à}ʔ 'you (accusative).' I.e., ``it's \textit{you} I'm asking (not Jali).``}

77: Five mounds.

78: Five mounds, eh?

79: Uh-huh.

80: Five bushels probably isn't enough to live on.

81: It doesn't even come to five bushels, unfortunately. In my field this year
the paddy is no good, since they set the damn thing on fire for me, so I'll really
only get \textit{three} bushels, Pastor. I have no idea how I'll ever be able to
feed myself now.

82: Ah, you can't think your way out [of a situation like this] !\footnote{\textit{d}ɔ̂\textit{ mâ t}ɔ̂ʔ: lit., ``think not come out``.}

83: How about you, Jali?

84: Well, I've got ten mounds, ten mounds for me.

85: Hm, you're a little better off, then.\footnote{The verb particle \textit{ š}\emph{ɔ} 'still' conveys the idea that 'you're still in the realm of comfortable living.'}  Shaw-lu  \footnote{This is the proper name of the Kachin (Jingpho) man resident in Huey Tat.}, how many bushels
did you get?

86: At the moment, I've only got five mounds. But I haven't finished my reaping
yet.

87: I see.

88: For my part, I've only got five mounds. Because the wild pigs were eating it,
too. Half of it.

89: Hm... well....

90: And that's why we kept saying before ``Work\textit{ together}, work
\textit{together}! If everybody works all by himself, the rats eat it, the birds
eat it, the wild pigs eat it!`` It's because you don't listen to what people
tell you, you guys!

91: There's an old saying, ``If you would eat, smell.``\footnote{I.e., ``Look before you leap.``}  That
is, if you're going to eat something, try smelling it first. Oh, my children!\footnote{The Pastor here lapses into sermonizing style.}
Cultivating the land is also like that. If you plan to do it, you've got to think
about it first. Furthermore, if you want to \textit{speak}, you have to think about
it first, I tell you. Even if we just intend to speak, we must carefully think
over [what we should say] . When we don't listen to people's advice sometimes,
and disobey, that's the reason why we have to suffer for it this way....

\begin{center}
* * *
\end{center}

\leftskip=0pt
Well, have you tried threshing your paddy already this year?

92: Yeah, I've threshed it out\footnote{It is misleading (and boring) invariably to translate the post-head versatile verb \textit{a-ni} by 'try to Verb.' It sometimes serves merely to 'round out' the verb-phrase  ('have a go at V' ing'), and is often best left untranslated.}  already.

93: How much grain did you get?

94: I got ten baskets\footnote{The classifier {\large{}\textit{pîʔ }}'basketful' is used synonymously with \textit{t}û(\textit{n)} or \textit{p}û .}  per paddy-mound.

95: If you really get ten baskets per mound from your field, that's not bad at
all. Since you've gotten five mounds.

From mine I get only\footnote{The noun-particle ɛ̀ \textsubscript{after a numeral-plus-classifier sometimes simply serves to 'round off' the quantity expression (much like Thai }\textsubscript{\textit{sàk }}\textsubscript{or Japanese }\textsubscript{\textit{gurai),}}\textsubscript{ and need not be translated. Sometimes, however, it retains its original nuance of 'only.'}  \textsubscript{[53] }\textsubscript{\textit{ch}}\textsubscript{ɔ}\textsubscript{\textit{ kh}}\textsubscript{ɔ̂}\textsubscript{\textit{ š}}\textsubscript{\textit{\emph{i}}}\textsubscript{\textit{ ve mâ hê}}\textsubscript{ʔ. lit, `` don't understand human words.``}  \textsubscript{[54] The victim of the incident to be recounted never appears as the overt subject of a sentence. }  \textsubscript{[57] The Lahu carry things on their backs or heads (pû ve). The Thai prefer to carry baskets by attaching them to shoulder-poles (tâʔ ve).}}  six baskets or [sometimes]  as much as seven. Sometimes
I only get two basketsful, or one and a half!

96: Jali, what about yours?

97: Well, mine now, sometimes I get ten basketsful and sometimes fifteen.

98: I see. Well, now  I'll teach you children one point! Let everybody carry back
his grain and store it inside his house! In this country the Thais are very mean,
and they're apt to set it on fire. If the Thais burn it up, we'll have nothing
to eat.

99: No matter how much of it there is, even if I have to spend all day tomorrow
carrying mine, I'll probably get [the job]  finished. I'll carry it home and store
it, all of it. It's a hard thing  to go and do. However, since these Thais are
such irrational people[53] ...

100: Will everybody carry it back to store?

101: They'll carry it back to store [if they know what's good for them] . What
[this fellow] [54]  carried back came to ten basketsful. But all together [his
crop]  had come to fifteen baskets. So that there were still five basketsful [lying
around in the fields] . And since you people don't listen to your elders' advice,
some elephants came and trampled [the rest] !

102: Ah, me! That's why we said on that day long ago: ``When people instruct
you, listen to what they say! Listen to what they say! When you cultivate the land,
work where the others are working,`` we said. But today even the elephants
have again trampled [on somebody's crop] ! That's no way to earn a living! When
people instruct you, you don't heed their words.

\begin{center}
* * *
\end{center}

\leftskip=0pt
103: Has everybody finished carrying his grain back to store in his own hearth
and home?

104: Well, I haven't managed to store mine yet. This weekend I plan to have some
Thais carry the rice [to my home]  [on their shoulders] .[57]

105: Yes, do [have them]  carry it all up and store it away! Immediately.

106: Well, I'm finished already.. Because mine didn't amount to very much. [But]
I've thought it over, and next year, after we've celebrated New Year's, when it's
time to work again, from then on I'll work together [with the others] .

107: Okay.

