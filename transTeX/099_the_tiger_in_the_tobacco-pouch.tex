
99 The Tiger in the Tobacco Pouch

1 Once there was a certain man.

2 He was a tobacco-smoker.

3 So he had all kinds of tobacco.

4 Well, he went looking for [a certain kind of] tobacco at a Shan's.

5 Then the Shan said --- rather, he said to the Shan, ``Mister, don't you have
any šú ɔ̀-ɛ́?''\footnote{The Lahu does not know the Shan word for `cheroot, cigarillo,' so he uses the Lahu periphrasis šú ɔ̀-ɛ́ `little tobacco'. He has some of the wrapped up (with other belongings) in a large package, in case it should prove necessary to show the Shan just what kind he wants, but he has not unwrapped it yet (see below, sentences 7 and 9). [Ironically enough, ɔ̀-ɛ́ may ultimately be a loanword from Shan itself (cf. Shan ʔè `few'), though it is perhaps mangled beyond a Shan's recognition.]}

6 At this the Shan [said], ``What is this `šú ɔ̀-ɛ́'?'', so he repeated [`said']
``šú ɔ̀-ɛ́''.

7 Then the Shan said to him,\footnote{The words Pî-chɔ̂ yɔ̂ àʔ `the Shan (said) to him' are tacked onto the end of the sentence as an afterthought. Such inversions are common in colloquial Lahu.} ``Oh!\footnote{The Shan now thinks he understands, and recoils in horror. The Lahu word šú `tobacco' sounds something like the Shan [shə̌] `tiger'. Perhaps the Shan understands the word ɔ̀-ɛ́ `little thing' (see note 1); this would lend plausibility to his hypothesis that there was a `little tiger' or tiger-cub in the Lahu's package, since a big one would not fit. This is not essential to the story, however, since the sequence of vowels u-ɔ-ɛ in šú ɔ̀-ɛ́, when rapidly pronounced, sounds very much like the schwa-vocalism of the Shan form.} In that case don't up wrap it yet, don't
unwrap it yet!

8 I'll go look for a policeman first!''

9 Then, since the Shan hadn't understood, he unwrapped it and put it down [for
him to see] --- he'd been talking about smoking-tobacco!

