\setcounter{footnote}{0}

[Medic who had received training in Mae Taeng]

1. At this time, we brethren\footnote{\textbf{ɔ̀-ví-ɔ̀-ni}: lit. `elder and younger siblings'; the most common vocative term used when addressing a group of Lahu.} are very happy that a foreigner\footnote{\textbf{Kâlâ-phu}: lit. `white Indian'. Term is used for Euro-Americans, like Thai \textit{fàràŋ.} The reference is to the author. African Americans are called \textbf{Kâlâ-phu=nâʔ}, lit. ``black white Indian''.} and a Lahu
\footnote{I.e., \textbf{Cà-lɔ̂}, my chief consultant on this first fieldtrip (1965-66), a recent immigrant to Thailand from Shan State, then living in Chiang Mai, not in a Lahu village.} have become friends with us, and that we are seeing each other this evening.

2. Although we live in different places\footnote{\textbf{tê} \textbf{g̈â} \textbf{tê} \textbf{kà}: lit. ``one-person-one-place''.}, our being able to meet this evening
is due to the great grace of God.

3. And in this connection, at this time we would like to record\footnote{\textbf{te} \textbf{kə} \textbf{gâ}: \textbf{kə} `insert, put in' is used for `to record'.} a hymn.

4. I'll teach you this hymn that has been composed, just as it is in our prayer-books.\footnote{\textbf{g̈ɨ̀-ša} \textbf{ve} \textbf{tɔ̂-khɔ̂}: lit. ``God's words''}

\textbf{Hymn}

The doors of God \{are still/ stay\} open

for all (us) sinners.

Therefore, O brethren

do not hesitate!\footnote{Lit., ``do not stay for a long time''.}

If time runs out

you won't have time to be happy.

The doors of God are still open.

Come in, come in!

Before Jesus' door closes it's not too late!

If time runs out

you won't have time to be happy.

Speaker: That's how it ends.

