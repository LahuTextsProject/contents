
\textbf{88 Building an Airstrip }

\textbf{Shatodo} IX-1

Dav 18: We are now making a landing place for airplanes in Shatodo village. [Planes
with] how many engines do they say will be able to land?

19: Well, they say that ones with three engines will land. But right now we still
haven't even finished making the airstrip. We haven't finished hauling away the
trees we've cut down already. And the lac stinging us is making it hard for us
now, I tell you!\footnote{The lac palm has a stinging itchy substance on its bark which bothers the Lahu workers when they haul the trees away.}

20: How many days will it take to be finished?

21: Well, it might even be another week before it's finished.

22: The trees haven't been burned off yet.

23: Towards evening yesterday the trees were chopped down. We did set fire to them,
but they haven't finished burning yet. And that's because everybody doesn't work
at the same time.

24: Some days, you see, only twenty-odd people, some days only thirty-odd people
work, so that's why it's not finished.

25: Isn't it done ``one person per household,`` as they say?

26: They say one person per household, because now the time to work -- the time
for planting the paddy --- has arrived, and so the people work in rotation with
each other. Some days we've been split up into groups of twenty-odd people each
to work -- some days there are as many as thirty, other days there are only about
twenty.

27: Ha! There are days when two people come from a household too!

28: That's right, that happens, too.

29: There are also times when one person comes, aren't there?

30: There are also times when one person comes.

31: This business of splitting up the people into two groups means that everybody
doesn't go to work in a single bunch in the same place at the same time. Sometimes
we even go to work split up into \textit{more} than two groups. Everybody has his
own work to do at home besides. So that's how it is building the airfield. It's
because of things like this that we can't get it done very fast.

32: It's been sixteen days now that we've been building this airfield, up to today.
Even so it's not finished yet.

Dav 33: How many liters of rice do you get per day for building the airfield --
for working?

34: Three liters we get. Plus six dried salted fish. Lately we've been getting
four liters. [But] when we first started working in the beginning, no matter how
we worked they only gave us \textit{three} liters of paddy a day and \textit{five}
dried fish.\footnote{For a slightly different account, see sentence XX (170-171 in Jim's original translation).}

35: That's probably not enough to live on, an amount like that.

36: This evening I tried begging them, and I got five liters of rice. I got seven
or eight dried fish, too. If you don't ask for something extra they don't give
it to you.

37: Do they say that when the airfield is built you all will get to ride [in the
planes]?

38: Well, they say that when it's finished, even if we wanted to go visit Chiang
Mai we'd be allowed to ride. They'll let us ride, they say. I don't know anymore
whether this is true or not.

39: Well, when a plane comes we'll just climb in for a ride! Even if they don't
offer to take us!

40: Because we've suffered for it, haven't we?

41: We have suffered for it.

42: Every bone in my body is aching from dragging away the bamboo stumps and the
tree stumps!

43: A lac tree over there really irritated me one time a few days ago. My forehead
was so itchy I thought I would die. But even though I told this to the police,\footnote{The Thai Border Police were supervising the building of the airstrip.}
and they said they would blast it out for us, they didn't hurry about blasting
it away for us. Then today they finally did blast it out. But even so, it stung
me again, would you believe it!

Dav 44: If you chew up some raw sticky rice and smear it on the place where the
lac has irritated you, it'll get better, you know.

45: Say, I didn't know that!

46: By daybreak tomorrow it should be all burnt off. The fire's burning very hard.

47: I doubt if it'll be all burnt off.

48: It will be over. By daybreak. The inside part is all burnt up already.

49: Burnt up already, is it?

50: Well, I still think it probably won't be finished.

51: The other day an airplane came, this airplane came -- and came in the wrong
direction and sank into a mud-hole! We had to lift it out again. A little plane
-- a tiny one, yea big.

52: It's because he'd never come before, wouldn't you say?

53: Because he'd never come before. The others would come down in a certain direction,
but he came down in another direction and sank into the mud-hole.

54: But you people managed to help him get back out again, so it turned out all
right.

55: This evening now I was working. And they still hadn't managed to blast the
trees out yet. So far, we've only bored the holes\footnote{For sticking the dynamite in.}, and the police official says
he'll blast tomorrow morning.

56: Well, but didn't he say that the ones we had bored before he would blast this
evening?

57: No. He's a real bastard.

58: What he said was --

59a: He lies. He makes other people do the work, and he just hangs around doing
nothing!

59b: He says that he won't blast with a big grenade. It's wasteful, he says. When
you just blast with a little tiny one -- he says that these large grenades cost
two thousand baht a piece. That's what he told us.

60: All this about spending this much or that much, what do \textit{we} know about
it!

61: He ought to do the blasting for us.

62: Oh, this policeman is always chewing us out too! He goes around bitching at
us all the time, this policeman.

63: Even if he spends that much it's not his money! It's the government's money!

64: He's a pain in the neck.

65: They gave it to him to blast with. He ought to blast.

66: But that's all he keeps telling us, you see!

\begin{center}
[A pause]
\end{center}

\leftskip=0pt
Dav 67: Is there any more [to say]?

68: Building this airfield we -- it extends into our fields too, so that we even
have to dig up all the rice we already planted. Because if we don't dig it up,
there's nothing to be done.

69: In that case you've just got to go ask for compensation.

70: As for going and asking for compensation, he doesn't want to give it to us,
because it's government land. ``Never mind,`` he says, ``extend
your fields in any direction you please!''\footnote{To make up for the land lost to the airstrip. Lit: ``in whichever direction you cultivate onwards, cultivate onwards!``} And so, since this police official
is the boss, you go ask him for something and he gives you exactly twenty baht.\footnote{Equivalent to US\$1.00 in those days.}

71: You can't win. No matter how little they give us it's better than nothing.\footnote{Lit: ``no matter how much they give us, if we get that much it's good''.}

72: He tells us to make this place for airplanes to land -- but does he say when
one will really land here, I wonder!

73: He says they'll land one the very day it's finished -- that's what he says,
anyway.

74: I wonder what they look like, flying-machines like that! I've never even seen
one! Not in all my born days.\footnote{Lit: ``even when I was just coming to be aware of people's speech''.}

75: Say, friend, have you ever gotten to ride in one?

76: No, I never have! I've been wanting to but I don't have the money. And nobody
has offered to give me a ride! Someday I \textit{will} try asking somebody for
a ride, though.

77: I doubt if anybody would \textit{offer} to give you a ride. ``I don't
have any money,`` tell them! ``Could I please have a ride? I'd
like to see what it's like to ride for a while,`` you've got to say to
them.

78: I'd like to just grab onto a wing and try flying around. Even the shortest
side of it is six cords long.\footnote{A `cord' or `fathom' (Lahu \textit{lò}) is the distance between the fingertips of the outstretched arms.} I've never seen anything like it in all my born
days.

79: The rear end of it was sticking upwards, see? He was going to drop some supplies
and when he had the head of it pointing up -- I mean, when he had the head pointing
downwards and the tail sticking right up he let the stuff come falling down!

<Titters>

Paul 80: Didn't it fall into the River Kok?\footnote{The NÁam Kòk, a river south of the village, arises in Shan state and empties into the Mekong, nearly forming the Thai-Burmese border at certain points.}

81: Yesterday they were dropping things again and they almost hit people! <laughing>
If it hits you, you die.

82: Well then, when it's coming you should go run off and stay under a tree. Otherwise
you might get hit.

Dav 83: When the three-headed airplane lands, you ought to touch it carefully,
stroke it carefully, and look it over carefully, too. While it's coming down from
above, its top, its bottom, its surface you should all carefully --

84: Hey, you ought to look over its body carefully, too.

87: It's beautiful to look at.

88: It's all sparkling white, like when you've just brushed your teeth!

89: That's what it's like, what we call an ``airplane''.

90: Well, when that big airplane comes down to a stop, I want to look at it, give
it a good once-over. It's very beautiful.

Paul 91: This airplane, who do they say made it? Do they say that Lahu made it?

<laughter>

92: They say that white men made it.

Paul 93: Someday in the future the Lahu will be able to make them too, these airplanes.

Dav 94: I bet they will! That is, some day. If we build them.

95: Come on, the Lahu can't do it. The Lahu have no education.

96: Well, you're wrong! Two of our people are already away studying now.

Paul 97: You just go cut some bamboo tubes and put in some gasoline and set it
alight and there you are! It doesn't matter if it goes or not.

98: If it's Jalaw,\footnote{This is Paul's original Lahu name (Cà-lɔ̂ in formal transcription).} he probably \textit{could} do it.

99: Sure it could be done. But all we know is its \textit{shape}. It wouldn't be
able to fly around in the air.

100: Well, a \textit{lorry }now, even a Lahu could make one, I bet.

101: It goes gliding along the ground, a lorry.

102: We'd scoop out a nice hole in a log, eh?

<laughs>

103: But everybody has his own lorry, you know! The lorry that your father and
mother gave you. The one you go walking around on every day!\footnote{He is referring to one's legs.}

Dav 104: That's a good one!

105: That's a good one.

106: It can't run out of gas. And it can't break down either, right?

\begin{center}
[Pause]
\end{center}

\leftskip=0pt
107: I guess we're all finished now.

\begin{center}
***
\end{center}

\leftskip=0pt
108: Hurry up and do it!

110: Granddad, you're an old man, you shouldn't get up to work so early every morning!
You're liable to catch a fever.

Old man 111: I won't catch anything. Since I'm a guy who's used to staying in the
water all the time.

112: This old fellow now, when he really gets the urge they say he even goes off
before daybreak, off to the fields.

Old man 113: I do it because I'm afraid I won't get anything to eat otherwise.

Dav 114: I wouldn't be surprised if he played the young lover too!

Old man 115: Not me, not me!

116: You do. We know all about you.

117: He's trying to handle two things at once. The airstrip and his fields.

One of the young guys 118: I tell you, in the mornings I can't move a muscle.\footnote{Lit: ``can't shake it''.}
In the afternoons I work on the airstrip, chop out knots of roots, chop up logs
that have been chopped down. In the mornings I can't move a muscle. Well, this
old guy is probably playing the lover. Just like the Pastor said, eh?

118: I bet he does do it a little bit.

119: In the morning, when it's just getting light enough for him to see --

120: I wouldn't be surprised if he was playing around a little.

<laughter>

121: He's a very hard worker.

122: When it's just getting light enough to see a little bit, he's gone, this old
guy.

123: Say, granddad, aren't you tired from working on the airstrip on top of hoeing
your fields?

Old man 124: I'm not tired at all. I'm not sick, you know.

125: Goddam, even young guys like us are so tired from working on the airstrip
that we can't move a muscle in the mornings. By the time we can get up the sun
is already way up there!

Old man 126: As soon as the cock crows I can't sleep, old man that I am. I'm worried
that I won't have anything to eat.

Young man 127: Young man that I am, I don't know anything about all that. I don't
know about \textit{not} getting anything to eat. All I know is that when I'm sound
asleep it's great to be sound asleep. I don't understand anything else.\footnote{This young man is egging the older man on to sermonize him.}

Old man 128: If you stay asleep you'll get no rice to eat! We've got to hoe our
own mountain-fields on top of doing the government's work. Early in the morning
at the crack of dawn we've got to till our fields -- in the afternoons we've got
to do the government's work.

129: These people of ours are pretty stubborn. They [the Border Police] said to
us ``If you're going to work, everybody work at the same time!''---but
while some of us are working, others are just looking on idly. So they told us,
``When you work, \textit{everybody} should work at once! And when you rest,
let everybody rest together! Don't just stand around and gawk while somebody else
is working!`` The fact is, they said, ``if the time to plant your
paddy comes before you've finished this work [the airstrip], you won't be able
to plant at all, so hurry up with the work!`` But, even so, they're very
lazy.

130: This weekend let's work together all at once, every soul in the village, and
get it done! If we do finish it fast --

131: If we don't finish the work, it will be everybody's fault.

132: If we do finish it fast, we'll manage to get our fields hoed in time, see?

133: Yeah, if we don't finish we won't manage to hoe our fields.

134: That's what we said to that crowd. We told them to hurry up and work together,
all at the same time, but they don't listen. Anyway, on any given day only some
of them are at work. ``Sure we work, sure we work, sure we work!'' they
say, but it's just not true. All they want is to go stay in the forest and spend
the night there.\footnote{The implication is spending it with a girl.} Ah, these Lahu are hard to reason with!

135: Yes sirree!\footnote{This phrase is in the north country dialect (Lâhū Mə́-nə́).} Getting together this weekend and working in a group -- if
we\textit{ don't} finish, if we don't manage to plant the paddy in good time, by
next year the village might not even exist anymore.

136: If there's not enough rice to eat that's what will happen. Each of us will
go his own way to earn a living. If such a time should come.

137: Or else we'd even become thieves! If there's nothing for us to eat.

138: Even if we have nothing to eat I daresay we won't be thieves.

139: Well, people like us two now, no matter how it may go with us, since we're
``their people'',\footnote{I.e. Thai citizens.} since it'd be because we've done as they've ordered us, if
we should have nothing to eat all we'd have to do would be to ask them to feed
us. So you see, we won't have to speculate back and forth or talk back and forth!
We'll just ask then to feed us, if we're going hungry. Because they've made us
work for them -- because we've become ``their people.``

140: While the men are working over there [on the airstrip], let the \textit{women
}go do the weeding, I say. You'd think the women would be working at it every day.
But they haven't even managed to hoe the weeds yet, some of them. And they haven't
gotten to plant the paddy yet either.

Old man 141: Next year we'll starve, I tell you [if it's up to those women].

<Laughter>

142: These women \textit{do} work at times -- some of them anyway. Some of them
are lazy as hell though. If their husbands aren't working right along with them,
they won't work either.

143: They're good-for-nothing then.\footnote{Lit: ``They don't manage to lick (i.e. eat)''. That is, their activities are ineffectual.} They insist on working together with the
males, those women.

David 144: They can't part from their menfolk!

145: Even the menfolk are tired of their being that way.

147: This year I haven't been able to plant a single seeding of pepper! Or even
sesame either! In the days to come we probably won't even have any clothes to wear
anymore.\footnote{Chili-peppers and sesame were among the Lahu's cash crops. Normally they would sell them to the Thai in exchange for clothes, salt, etc.}

148: I haven't managed to plant any peppers either.

149: It's the same way with me. If I can't even get the weeding done how can I
go planting peppers! I haven't even planted my paddy yet.

150: I only weeded for three days, then I had to go back to digging the aeroplane
field again.\footnote{He uses an old-fashioned Burmese-derived word for `plane'.}

Women 151: I am \textit{too} a hard worker! I've tried to work --

152: Ah, me! Ever since this work came up I've been weary to the bones!\footnote{Lit: ``if we talk about being tired it's the most!``}

153: That's the truth.

154: It came up right at the time we have to cultivate our fields, the time for
us to till the soil and plant the rice -- so it's miserable for us this year.

155: If this job hadn't come up this year, we would've taken out a loan of rice
from the Northern Thai as usual and gotten as much as we wanted.\footnote{Note the contrary-to-fact use of the verb particle \textit{ò}. The Lahu of this village normally get rice on credit from their Thai neighbors until the harvest. But this year, seeing that the Lahu were having planting delays, the Thai were unwilling to extend credit.} If they had
only done this to us when the rice planting was over it would've been fine, see?

156: If the paddy were all planted we wouldn't mind working [on the airstrip] every
single day.

157: Or else if they had even done it next year\footnote{I.e. After this year's harvest but before the next planting.} it would've been fine and dandy.
But instead this year we're going hungry, and if we just work for them over there
now I don't see what use it will be for us next year either. This year we starve,
and next year we'll keep on starving.

158: That's just what I told them [the Border Police], a while ago. ``Right
now we don't have the time yet. Let us do it next month,`` I said, so then
they said to me, ``Next month it can't be done. The rainy season has already set
in. It can't be done.``

159: We've got to obey them. Because the likes of us are a little race of people.
Our work is no match for the government's work.

160: Mm-hm.

161: This very morning a policeman said that any of us who wanted to work on the
airfield on a permanent basis would get three liters of rice a day per man, without
planting his own paddy at all, just working over there all the time. ``Is
there anybody who would want to do that?`` he said. ``No, there
aren't,`` I told him. ``If you only make it that much, it won't
be enough to live on. The reason it wouldn't be enough they say, is that they have
so many children they couldn't live on that much,`` I said to him.

162: One person could live on three liters, all right. But it's just not enough
to feed a wife and children.

163: We have too many children for only three liters!

Women 164: With lots of children three liters is only a mouthful.

165: With my family, it wouldn't even \textit{be} a mouthful! Those three liters
of theirs.

166: As far as I'm concerned even three liters would be enough to eat. Since there
aren't many people [in my house].

167: There are only three of you, so it's enough to eat.

168: God, there are so many people in my family. At a single meal we use up four
liters of rice!

169: When we were just beginning to work on the airstrip we used to get tobacco
to smoke, even. And we would get bananas to munch and candy too! But now, for the
morning meal they don't give us a thing to eat! As for the evening meal, we get
one cigarette and one piece of candy, and that's all. It's not the same now as
it used to be when we were just beginning to work. They don't feed us properly!

170: When we were just beginning to work -- the amount of rice we got when we were
just beginning to work was four liters. But now we don't get that much any more.\footnote{For a somewhat different account, see sentence XX (34 in Jim's original translation).}

171: It wasn't four liters! When we were just beginning to work we'd sometimes
get even five liters, sometimes even as much as six liters, I tell you.

172: Now we only get three liters out of it! Today now, there were eight Thai working
along with us too. These Thai don't work very hard, you know! They're no good at
working! When a Thai is on the job it takes him two days to clear away a termite
hill -- and even then it's not finished.

173: Why, if we Lahu don't help them, those Thai can't even dig out a clump of
bamboo roots. Tomorrow some people are coming from Venkheh,\footnote{Name of a Red Lahu village.} and if more of
\textit{our} people go too, we should get a lot more done.

174: Once these people from Venkheh chopped through a tree that was over there
on the side and made it fall over into the middle [of the area that had been cleared]!
These Venkheh guys.

175: They can't even chop down trees!

176: They can't even chop trees --

177: Because they chop without looking to see the direction! Those guys just don't
figure out the direction right.

178 <Defending them> That's not it. It happened because there
was already another one that somebody had cut down right next to it.\footnote{So there wasn't room to chop the second one in such a way that it would fall outside the cleared area.}

179: So the next day they had to drag it away.

180: You're wrong. If they had only chopped on a different side it wouldn't have
turned out the way it did. But they chopped it on \textit{this} side <gesture>
and it came falling smack into the middle. Or else, for God's sake, if somebody
had climbed up the tree and tied a rope to it, and if lots of people had stood
over there and pulled, it would've been fine.

181: It was the climbing up that was too much for them!

182: Oh, come on, with all those people if one man couldn't make it somebody else
would've. Besides if they'd only asked us to, then \textit{we} would've climbed
up for them.

184: We'll have to be back on the job again tomorrow. We've got to cut away the
green\footnote{\textit{nɔ́ ɛ̀ ve} `green', i.e. not yet dried up, not dead, so hard to deracinate.} root-clumps, don't we? Yeah, that fellow A-meh said, ``Let's
take that tree over there that you've already knocked a hole in, and blast it out
too with a small stick of dynamite.\footnote{\textit{à-mī-šī } (lit: ``fire pellet'') is used to mean `grenade' or `dynamite'.} Because if we take a \textit{big} charge
of dynamite and blast, the hole will be too deep, and if the hole's too deep you'll
have to spend too long scraping earth together to fill it. You still haven't finished
with filling in the holes from the last two, three, or four trees that we blasted!
And if you just keep burning the trees out\footnote{I.e. by setting fire to them rather than by blasting them with dynamite.}, he said, ``It'll waste a lot of
time. If you waste time that's how you won't be finished soon enough,``
he said. ``So, if you explode a small stick of dynamite, so it only blasts
away the surface, then if you just do a little cutting here and there and drag
[the pieces of wood] away, and fill in the earth again, you'll be all done, right?``
he said.

185: That policeman is a hard man to talk to, all right!

186: If we make a little suggestion to him he just curses us out. Today now, he
said that one of us -- you know, that little guy -- wasn't working hard enough,
so he told him, ``When the day comes that the communists get down here you're the
one they'll make fertilizer out of!''

Dav 187: It's because our people won't listen to reason. From one point of view
it was right that they scolded him that way. Since he's a disobedient fellow anyway,
and also doesn't understand their language on top of it, he's hard to deal with.
When we talk to him he doesn't listen. And when \textit{they} talk to him he doesn't
understand. What's tough is that he doesn't do what you tell him.

188: He doesn't listen and he doesn't understand! Right?

189: When they tell him ``Don't pull!`` he pulls. And when we try
to say anything to him he acts as if he was above us. If there's something he doesn't
understand, he ought to listen to us.

