\setcounter{footnote}{0}

1. When we were fleeing here to Thailand from where\footnote{The directional idea `from' is seldom expressed overtly in Lahu. The `place from which' is usually mentioned without any directional particle, followed immediately by the `place to' plus locative particle. See GL 3.89.} we used to live in Burma,
when we were fleeing here along the roads and byways, we lived off jungle-greens\footnote{\textbf{phòʔ-tu}: specifically `pith-root shoots'.}
and by damming streams for fish as we came -- that's how we made it here.

2. And there was a big flock of kids along, too -- er, walking along --.

3. And we had to stay on the road a very long time, too.

4. Getting here\footnote{\textbf{kàʔ} is here being used elliptically for \textbf{chò} \textbf{kàʔ} `here'.}, when it had gotten to be over a month, we had come as far as
Thayakwei,\footnote{Name of a Christian Lahu village in Shan State, which used to comprise 67 houses. See DL: 675.} where we made camp.

5. Looking for banana tree creepers\footnote{Creeper with edible leaves growing on banana plants.} to live off and damming forks in streams,
that's how we got here.

6. We made it to Thailand.

7. [whisper] Think carefully before you speak!\footnote{A hissed reprimand by a friend of the speaker. The beginning of this account is indeed somewhat diffuse.}

8. Fleeing here from Burma we crossed the Nam-Si river\footnote{A river in Shan State. See DL: 737.} over to this side.

9. The time we stayed at Thayakwei was about two months -- we were there for two
months.

10. When we were there --

11. [whisper] Don't laugh!\footnote{The speaker is a giggly young woman.}

12. -- we were all suffering from hunger and cold.

13. We would go and eat the other Lahus' mustard-greens.\footnote{They were reduced to living off the Lahu who were already settled in that part of Thailand.}

14. We'd go and eat their food.

15. We were at Thayakwei for two months.

16. We would chop down sticky rattan\footnote{\textbf{mɛ́-ni-gɔ̂}: lit. ``cat-rattan'', a small sticky kind of rattan only marginally edible at best.} to eat.

18. We'd go eat banana plants.

19. We'd dam for fish to eat.

20. We came here begging for something to eat.\footnote{\textbf{lɔ̀-lɛ̀ʔ-lɔ̀-câ}: lit. ``beg-lick-beg-eat.''}

21. After\footnote{The many occurrences of the word \textbf{qɔ̀ʔ} as main verb or prehead versatile verb in this narrative serve as `narrative-lubricators', indicating that the various events took place in an ordered sequence, and giving the speaker extra instants to think.} staying there for a long time\footnote{\textbf{tâ-vâ}: lit. ``all day long'' (< Shan). Here used imprecisely to mean `for a long time.'} we continued on to Shajieh.\footnote{A Chinese village.}

22. When we arrived at Shajieh we and the Chinese -- er, all the Chinese welcomed
us gladly and cheerfully, and set down cups of tea\footnote{\textbf{g̈ɨ̀-hɔ}: lit. ``hot water''. This is the Lahu word for light-colored Chinese tea. The strong dark-orange variety that the Lahu themselves habitually drink is called \textbf{là-g̈ɨ̀}.} for us to drink.

23. And they fed everybody rice, too.

24. After we ate we took to the road\footnote{\textbf{qɔ̀ʔ} \textbf{phɔ} \textbf{la} \textbf{ve}: lit. ``again came fleeing''.} again, until we got to the village of Pomu\footnote{A village near the town of Farng, about 115 km north of Chiang Mai.}
over there, where we stayed and spent the night, and we Lahu traded all sorts of
things with them for food.

25. We traded away our silver buttons,\footnote{A traditional Black Lahu woman's costume had some 500 silver buttons sewn on it.} and even our bracelets we traded away
to fill our bellies.\footnote{'Fill our bellies' translates \textbf{lɛ̀ʔ}, lit. ``to lick'', more earthy than \textbf{câ} in the sense of `eat.'}

26. We had nothing left.

27. In poverty and distress we kept on begging for our food.

28. Then our brethren who were already living here in Huey Tat\footnote{An advance party of Lahu had already settled in Huey Tat. When they got established they sent for their relatives and friends still in Burma.} went to lead
us back.

29. They came to lead us back.

30. They came to lead us here and they took care of us joyfully and gladly, and
so we arrived back here at Huey Tat.

31. Having to flee here was a very distressing time for us.\footnote{Lit: ``The time that we had to flee here was very troublesome.''}

32. In the midst of our troubles we had been separated from one another.\footnote{Lit: ``one person hadn't even gotten to see the other.''}

33. So inasmuch\footnote{`Inasmuch as' is very similar morphologically to Lahu VP + \textbf{ve} \textbf{ɔ̀-qhɔ} \textbf{lo}, lit. ``inside VP.''} as we have finally, by the grace of God, managed to see each
other again, we now have great and abundant reason to praise God.

34. That's all there is to say.

