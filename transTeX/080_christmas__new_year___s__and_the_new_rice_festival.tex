
80 Christmas, New Year's, and the New Rice Festival

Cà-bo (abbreviated both ``C'' and ``T'' [teacher]; Headman (``H''), Thû-yì,
the Kachin, Khù-Mɔ̂, Yâ-Pā-ɛ́

\. H: Well, this Christian group of ours, we Lahu living in Huey Tat village, will
celebrate Christmas when the year 1965 is almost over and just before 1966 arrives.
All of us, from olden times until today, have observed this custom from generation
to generation.  Therefore on December 24th Christmas will be here for us. \footnote{It is not clear why the date isn't Dec. 25th.  Rather like the Jewish practice} Everybody
should prepare in \{in advance/beforehand\} before December 24th arrives, everything
that we'll be using, like rice and wood.  Unhusked rice, hulled rice---prepare
them both properly, everybody prepare carefully the food for us to eat at Christmas.
When the time comes, we shouldn't do any work.  I'm telling this to everybody.

2 . T: Well, yes, we should prepare and make ready.  You're saying that we'll enter
the Christmas season after dark on the 24th, right?

\. H: Yes.  We'll \{enter/begin\} Christmas \{on the evening of/after dark on\}
the 24th.

\. T: Uh-huh.

\. H: When [Jesus] was born.

\. T: Aren't we indeed commemorating the birth of Jesus!  So in our Lahu language
how should we say it?  ``Christmas'' is an English \footnote{kâlâ-phu-khɔ̂: lit. ``white Indians' language''.} word.

\. H: Well, we could call it ``Welcoming Jesus' Birth around the New Year'' in
\{Lahu/our Lahu language\}.

\. T: Well, if I think about [it in] our Lahu language, I think we ought to call
it ``The Happy Festival.'' \footnote{pwɛ̂-ha-lɛ̀}  Jesus' birth was [a] happy [event], so everybody
please call it ``The Happy Festival.''  So, when shall we celebrate New Year's?

\. H: Celebrating New Year's?

1\. T: Yes.

1\. H: Well, the day that we celebrate New Year's \footnote{H. was unprepared for this change of subject to New Year's, and goes back to}---if we \{celebrate/start
celebrating\} ``the Happy Festival'' on December 24th, when the 25th arrives, we
can really have a good time.  In the daytime.

1\. T: Yes.

1\. H: When [Xmas] is over, how many days should we still take off, in \{your/you
guys'\} opinion?

1\. T: Well, we only celebrate New Year's once, at the beginning of the year.
I think we ought to take off three or four days.

1\. H: Mm-hmm,  So you're saying, for New Year's, for these two years [old and
new], it's only three or four days?

1\. T: Not at all!  The ``Happy Festival''  [alone] takes three days!

16 1/\. H: Oh.

1\. T: So then, when it's time to celebrate New Year's, we should still take off
three or four days.

1\. H: Three days and four days. \footnote{I.e., three days for Xmas and 4 days for New Year's} So, you[`re saying] it comes to seven days.
If it's three days plus four days.

1\. T: If it extends to seven days, we'll have a proper rest, all of us.

2\. H: I'll bet \footnote{``I'll bet'' translates the nf nɛ̀-ɔ́ `suppositional'.} you all want to \{take off work/celebrate\}. \footnote{chɛ̂: `to be in a place, dwell'; in this context it means `to stop work during}  All you kids.

2\. Pà-ɛ́: All of us want to celebrate.

2\. Kachin: I want to celebrate.

2\. H: Oh, they want to celebrate!  Hey Kachin, do you want to celebrate, seven
days for New Year's?

2\. Kachin: I do, I do! \footnote{``Want to celebrate, want to celebrate.''}

2\. H: [\textit{laughs}]

25 1/\. T: Well then, whose pig(s) shall we kill for Christmas?

2\. Pà-ɛ́: Yours!

2\. T: Mine?!  Mine probably couldn't be divided up! \footnote{I.e., it wouldn't be big enough to share among the villagers.}  It's just not very big,
that pig.

2\. Pà-ɛ́: It could, it could! [be divided up]

2\. Somebody: Hey, Khù-mɔ̂, you kill one [too]!  If there's not enough to divide
up!

3\. H: Oh, [mine] is a sow!  Its meat probably doesn't taste good!

3\. T: If we kill Khù-mɔ̂'s too, then once it's New Year's we won't have any
left to kill, in that case.

3\. Kachin: I'll kill one too!

3\. H: Well, Kachin, I guess yours will [make it] be enough!

3\. T: Is yours a male?

3\. Kachin: I've got a male, and I have a female too.

3\. T: Oh.

3\. H: Please kill the bigger one!

3\. T: OK then, Headman, how many kilos \{would you like to get/will you get\}?

3\. H: Oh, I'll take five kilos.  This year, when we're relaxing and having a good
time we'll cook it up [the pork dishes] and set it aside \footnote{Until the guests come, when it will be heated up and served.}, that's how we'll
do it.

4\. Pà-ɛ́: Hey, son-in-law! \footnote{Pā-ɛ́ is role-playing here as if he were the Headman's father-in-law---he} Aren't you getting \footnote{Approximate quantification via reduplication: tê chi kîlô-lô.  Could also} about 10 kilos, I guess?

4\. T: Ten kilos, eh?

41a. Pà-ɛ́: Yeah.

4\. H: What about you?

4\. T: Well, I guess I'd like two kilos [for Xmas] and three kilos [for New Year's].

4\. H: (laughing) In that case it comes to 5 kilos!  If we do the math. \footnote{tà ni a qo ɔ̄: lit. ``if [we] begin to look [at it].''}

4\. T: Well then, Cà-qu-ní \footnote{This is Thû-yì's nickname as transcribed in my 1965 field notes.  It might}, how many kilos are you getting?

4\. Thûyì: I'll take 4 kilos and 5 kilos.

4\. H: (laughing) 4 kilos and 5 kilos.

4\. T: 9 kilos.

4\. H: My goodness, that's plenty then.

5\. T: That's probably not enough.  \{Look for/find\} another animal [=pig], one
more.  Let's still look for one more pig.  So, let's think over whose we should
find.

5\. H: Let's kill Cà-lɔ̂'s \footnote{More play-acting.  Cà-lɔ̂ was my chief consultant on the 1965-66 fieldtrips,}. Cà-lɔ̂, I guess we should kill yours.

5\. T: The big male.

5\. Cà-lɔ̂: I'll kill it.

5\. H: A-ha.

5\. T: Cà-lɔ̂ says he'll kill one of his.  In that case there'll be enough!

5\. H: Since it's very big, eh?

5\. T: That's plenty, plenty then.  Until the end of the New Year's celebrations,
nobody go off anywhere, please!  [If you're] all by yourself that's no fun.  If
a person doesn't take time off to celebrate, that's depressing.  The thing is to
take time off.  Do as we have discussed.

5\. H: Right.

5\. T: We \{can't must not\} do as we have discussed.  \{Make your plans/ think
it over\} carefully right now.  Otherwise, when the time eventually comes around
later, it'll be like they say, ``By the time you realize what's happening you can't
do anything about it.'' \footnote{T. here segues into two proverbs one after the other.  The first one means} Or, as they say, ``By the time a woman feels she's
pregnant she has burdens aplenty!''---when the time comes there'll be nothing you
can do.

6\. H: So everybody should make their plans \{in good/ at the right\} time. \footnote{ɔ̀-yâ thâ lɛ̀}
Every household.  Because we in Huey Tat are so few in number, aren't we!  If
we try counting the houses [in the village], there are maybe 20 of them. \footnote{On my most recent trip to Huey Tat in 20\_\_, the village had grown to some} There
just aren't that many people!  To be precise about the [number of] people, \footnote{Lit. ``If we say carefully the people''}
if we say it, in Northern Thai, there are just ``pɛ̀ʔ sîʔ pÁi''! \footnote{``More than eighty.'' Cf. Siamese pɛ̀ɛt sìp paj} If
we say it in Lahu, there are only hí chi g̈â [80 people], these people of ours.
\{Talk it over, talk it over/ discuss it, discuss it\}, everybody try talking
it over, from now on whatever you do you ought to do it together.  How many deep
each time [for X-mas and New Year's respectively] \footnote{This meaning is conveyed by the reduplication of qhà-nî ni `how many days?'} will everybody take off
work to celebrate \footnote{This is an important issue.  In a subsistence culture based on constant work}, you \{graybeards/ gray heads/ gray-headed elders\} ought
to count heads and think it over, right? [laughs] [Otherwise] one of these days
if somebody wants to take off and somebody else doesn't want to, if one says ``Do
it this way,'' ``Do it that way'' there will be all kinds of \{things to quarrel
about/ \_\_\_ of contention/ controversies\}, won't there!  Everybody think it
over carefully.

6\. Pā-ɛ́: If we just take off \{as long as / the way\} we usually do it should
be enough.

6\. H: Yeah.

6\. T: If we just do things brashly without regard to others' opinions \footnote{pò-mɔ̀ʔ-pò-šê ``exceed that which is appropriate'' \texttt{<} Shan.} we
could become a laughing stock!

6\. H: Right.  Well then, everybody agrees I guess.  Taking time off the way we
usually have done, like our ancestors of old also used to celebrate for generation
after generation.  Is that right?

6\. Somebody: That's right then.

6\. Thû-Yì: There must be at least one person who doesn't agree.

6\. Pā-ɛ́: \{I doubt it/ there probably isn't\}

6\. H: [If you don't agree] speak out!  Otherwise later on [you might say] ``I
wasn't satisfied but I didn't say anything! I'm the kind of person who doesn't
say anything when he doesn't agree.'' Say what's on your mind!  If you don't agree,
please tell us!

6\. Pā-ɛ́: Everybody agrees probably.

7\. H: OK. If [you all] agree that's fine!  By the grace of God, in joy and happiness
we [will] celebrate the new fruits of our labor \footnote{cà-šɨ-ɔ́ câ ve `eat [celebrate] the New Rice festival', lit., ``eating}, and having reached the days
and nights of the New Year, everyone happily will live in peace and harmony; and
we'll just try to keep on doing all kinds of work, won't we!

7\. T: Well then, when shall we celebrate the New Rice Festival this year?  The
time to celebrate it is getting close!  After we've finished discussing the matter
of celebrating New Year's, let's \{try going on to talk/ have a go at talking\}
about the New Rice Festival.  This is the time to do it---otherwise some people
will be around \footnote{In this sentence chɛ̂ has its basic meaning of `be in a place; stay', rather} but others aren't---it'll be impossible to discuss later.

7\. H: Well, we've always celebrated the New Rice Festival in October, every year!

7\. T: October is getting close!

7\. H: If it's not on this coming October the eighth \footnote{The Clf tâʔ, from the verb tâʔ `climb', traditionally referred to days}, we've celebrated it
in the past on the 4th, the New Rice Festival.

7\. T: But then, if we do it on the 4th, how in the world will we be able to carry
the rice home?

7\. H: Hmm---

7\. T: It's not something that can be done in just two or three days.  It's the
rainy season!  It's not the hot season!

7\. H: Well, we've celebrated it on the 4th, in the past!

7\. T: As for when they celebrated it in the past, just because we remember \{that/
a certain\} year, a certain time, it's impossible to celebrate it at that same
time every year.  Some years there's no rain, other years the rain goes on and
on. \footnote{mû-yè mu `the rain is high'}

8\. H: We've also [sometimes] celebrated it on the 14th, the New Rice.

8\. T: In that case let's just do it on the 14th.

8\. H: Mm-hm.  The New Rice---

8\. T: Who---you're going to invite people, right? [i.e., from their villages]

8\. H: Well, I think it would be good to invite our Lahu \{relatives/ brethren\}
who live in all the nearby villages!

8\. T: In that case you guys have to think about food and drink now!

8\. Somebody: Well, as for food, I'll contribute a pig all by myself!

8\. T: All by yourself?

8\.  Somebody: Yes.

8\. Somebody else: In that case, the meat---

9\. T: In that case who all will contribute extra food for second helpings? \footnote{ɔ̀-phâʔ-ɔ̀-lə́: phâ `to exceed'; lə̀ `be left over, extra' (\texttt{<}}
If there's only a small amount it won't be enough.  Since you're inviting other
people.  What do you say?

9\. H: I don't actually have very much pork.  Because mine is a female.  There's
only three or four kilos \footnote{A scrawny sow indeed!}, I'd say.

9\. Pā-ɛ́: If there's not enough, I'll kill \{an elephant/ one of my elephants\}
and contribute it.

9\. Somebody: An elephant?

9\. Pā-ɛ́: Yep.

9\. T: My goodness, people could never finish paying for it, I bet!  An elephant---[laughter]
Is it a male or a female, this elephant of yours?

9\. Pā-ɛ́: It's a male, I think. \footnote{The suppositional use of nɛ̀-ɔ̄ `probability' is humorous here, since}

9\. T: A male.  My goodness, my goodness!  Who all is there who doesn't eat elephant
meat?

9\. Pā-ɛ́: People don't eat it, elephant meat.

9\. H: Oh, it's very tasty.

10\. T: I'd like to try eating it myself, elephant meat!  Since I've never eaten
it.

10\. H: Elephant meat is tough and chewy, so it's delicious!  I've \{gotten/ had
a chance\} to eat it before.

10\. Thû-yì: Let's divide it into shares!  I'll take my whole share [of meat
for the festival] from the elephant meat!

10\. T: Dividing it into shares, I doubt you'd be able to pay for it.

104a. H: How many thousand baht---

105a. T: How many thousand baht for yours---

104b. H:---will you charge for it?

105b. T:---are you intending to contribute it for?

10\. Pā-ɛ́: I'm not changing anything.  I'm contributing it for free.

10\. T: So you'll kill it and contribute it?

10\. Pā-ɛ́: Yeah.

10\. T: That's great then!

11\. Somebody: It sounds like he's saying he'll divide it up into shares.

11\. Somebody else: I hear that Khù-mɔ̂ also has an elephant to kill and contribute,
an elephant.  To eat it at the New Rice Festival.

11\. Khù-mɔ̂: Since I'm doing it for charity, I won't accept any payment.

11\. H: For my part I've got one pig!  A sow, with only about seven or eight kilos
of meat.

11\. T: Well, I'll also contribute three basketfuls [tû] of rice \footnote{tû : \texttt{<} Shan \textit{tun}.  1 tû = 1 pîʔ = 1 pû}.  Look for
other contributors of rice!  Cà-qu-ní (= Thû-yì), how much will you put in?

11\. Thû-yì: For my part I guess I'll put in one or two baskets full [tû].

11\. T: Just one or two basketfuls [tû].  Hey, you Kachin over there!  Jingpho
guy! \footnote{A longtime and well-liked resident of Huey Tat, this Jingpho man evidently}

11\. Kachin: Oh, five or five and a half basketsful [tû].

11\. T: Five and a half basketsful [tû].  Well, that's great.  So Cà-lâ over
there!  How much will you put in?

11\. Cà-lâ: Who, me?  Well, I'll contribute three \textit{tû }and two \textit{tû}!

12\. T: Three \textit{tû }and two \textit{tû}.  Yes, that's plenty, that's plenty.
Well then, when you've invited people and they've come, don't do anything embarrassing,
anything that would bring shame to anybody, OK, all of you?  As for this bunch
of young guys, don't you go courting the girls.  If you want to do some courting,
just do it at home.

12\. Pā-ɛ́: I do want to court the girls.

12\. T: As for courting now, do go and court each other.  But don't do bad things,
any of you. \footnote{This is an intentional pun: mâʔ   dàʔ   ve  ɔ̀-cə̀ /   mâ  dàʔ}

12\. Thû-yì: I probably will!  The others will too, I bet.  So maybe I will too
