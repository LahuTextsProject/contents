\setcounter{footnote}{0}

\textit{Headman Cà-bí}

1. Well, then, let me tell you another story. Listen carefully, everybody.

2. Long, long ago, the story goes, there were some lazy men, three traders.\footnote{\textbf{tâ(n)-kā} \textbf{te} \textbf{ve} `to trade; go around trading.'}

3. They were resting\footnote{Lit., ``sitting.''} in a certain place where several roads met, where three
roads met.

4. The first man\footnote{Lit., ``at first, one man...''}, saying he was going off to trade, had sat down there\footnote{He sat down on his way, and didn't bother to get up again.},
and would say ``Ah, where are you going?'' [to those who passed].

5. One day another lazy man appeared.

6. [In response to the other's usual question] he said, ``Oh, I'm doing
some trading.''\footnote{It is probable that the Headman reversed two sentences and meant to have ``where are you going`` come \textit{after} ``one day another lazy man appeared.''}

7. ``Have a seat here.''

8. After a while a [third] man came along [one of the] roads to the place where
the three roads met.

9. He arrived there, and they all had a conversation.

10. ``What do you do?,'' they said.

11. ``Nothing much.\footnote{Lit: ``I don't do anything.''} I go around trading,'' he said.

12. ``What are you good at?'' they said.

13. All of them were sitting there and talking back and forth.

14. ``Well, I know how to make crossbows,'' he said.

15. ``Well, then, you--what can you do?''

16. ``Oh, I know how to predict the future.\footnote{\textbf{mɔ́-tɔ̂} \textbf{thèʔ} \textbf{ve}. There is apparently a folk-etymology here. \textbf{mɔ́-tɔ̂} is prob. < Thai \textit{mɔ̌ɔ-duu} ``seeing master'', but there are also Lahu roots \textbf{mɔ̂} `be an elder; be venerable' and \textbf{tɔ̂} `words.'}'' These three men
[were talking like that].

17. Thereupon they said, ``What do you have to say, then, if you tell fortunes?
Try predicting the future, maestro!\footnote{The Thai-derived vocative \textbf{mɔ́} `teacher' is used.}'' they said.

18. Well, he spoke, screwing his mouth up from one side to the other\footnote{Fortune-tellers often speak out of the side of their mouths, to appear more impressive.}: ``Hm,
in a short while, at midday, we'll see\footnote{The benefactive verb-particle \textbf{lâ} implies that the hawk's actions are directed at the men: `for their benefit' in the sense that they will be the witnesses of the action.} that a hawk carrying a princess--a
king's daughter--in his beak will fly over the lake here,'' he said, foretelling
the future.

19. ``Well, didn't \textit{you} say you could make crossbows?''

20. ``Yes, I can (make crossbows).'' And he made a crossbow.

21. Well, another of them said, ``And what can \textit{you} do?''

22. ``I know how to swim,'' he said.

23. As they were all talking like this, that which had just been foretold would
come to pass.

24. It would happen according to the prediction.

25. So, the crossbow was all made, and they stayed there together.

26. When midday came, there was the hawk carrying the king's daughter, and he was
already over the lake.\footnote{Lit., ``and he had gotten as far as above the lake.''}

27. So he took careful aim and fired off a shot, and they fell into the lake.

28. When they had fallen into the lake, the man who could swim immediately struck
out swimming, and by his swimming rescued the princess who was out there--the young
girl.

29. When he had swum back with her, the three of them made conflicting claims on
[the girl].

30. They were arguing over who would marry her.

31. Everybody wanted to marry her, [and the fortune-teller said] ``I foretold
it, so I, the maestro, ought to marry her!"

32. ``Well, you couldn't get her just by telling your fortunes--I had to
swim out and get her!"

33. ``If I hadn't shot him down with my crossbow you couldn't have gotten
her by your swimming!"

34. Thus the three of them vied with each other.

35. You people think about it! Which one of them probably should have married her?
Think about it, everybody.

36. Pà-ɛ́: You mean the queen\footnote{Pà-ɛ́ uses the word \textbf{hɔ́-khâ-ma} instead of \textbf{hɔ́-khâ} \textbf{yâ-mî}.}? The one who shot [the arrow] ought to get
her.

37. T: The swimmer!

38. Ty: It's the fortune-teller who should get her! If he hadn't foretold it they'd
never have known.

39. T: Even if he foretold it, if he didn't swim he couldn't get her.

40. H: Well, they took the case to the king and told him about it--and how did it
come out? Who was the one who created [this situation]\footnote{\textbf{phā} may mean either `pass by' or `create' (both < Shan), though neither makes much sense here. The Headman brought this narrative to an abrupt close, but retold the story on another occasion. See Version II.} first? Was it the fortune-teller?


