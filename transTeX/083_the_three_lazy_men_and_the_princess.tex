
\textbf{\#83. The Three Lazy Men and the Princess }

114 Headman

Well, then, let me tell you another story.115 Listen carefully, everybody.

116 Long, long ago, the story goes, there were some lazy men, three traders.\footnote{tâ k\emph{a} te ve 'to tg̈ade, be engaged in commerce.'}117
They were resting\footnote{Lit., ``sitting.``} in a certain place where several roads met, where three roads
met.118 The first man,\footnote{Lit., ``at first, one man...``} saying he was going off to trade, had sat down there,\footnote{He sat down on his way, and didn't bother to get up again.}
and would say ``Ah, where are you going?`` [to those who passed].119
One day [another] lazy man appeared.120 [In response to the other's usual question]
he said, ``Oh, I'm doing some trading.``\footnote{It is probable that the Headman reversed two sentences and meant to have ``where are you going`` come \textit{after} ``one day another lazy man appeared.``}121 ``Have
a seat here.`` 122 After a while a [third] man came along [one of the]
roads to the place where the three roads met.123 He arrived there.124 And they
all had a conversation.

125 ``What do you do?,`` they said.

126 ``Nothing much.\footnote{Lit: ``I don't do anything.``} I go around trading,`` he said.

127 ``\{What can you do (well)? / What are you good at?\}`` they
said.

128 All of them were sitting there and talking back and forth.

129 ``Well, I know how to make crossbows,`` he said.

130 ``Well, then, you--what can you do?``

131 ``Oh, I know how to \{tell fortunes / predict the future\}\footnote{mɔ́-tɔ̂ thèʔ ve. There is apparently a folk-etymology here. mɔ́-tɔ̂ is prob. < Thai mɔɔ-duu, but theg̈e ag̈e Lahu roots mɔ̂ 'be an elder, vevenerableand tɔ̂ 'words.'}``--these
three men [were talking like that].132 ``Well, what can you do?``
133 ``Oh, I can make crossbows.`` 134 Thereupon they said, ``What
do you have to say, then, if you tell fortunes? 135 Try predicting the future,
maestro,\footnote{The Thai vocative mɔ́ 'teacher' is used.}`` they said.

136 Well, he spoke, screwing his mouth up from one side to the other\footnote{Fortune-tellers often speak out of the side of their months, to appear more impressive.}: 137 ``Hm,
in a short while, at midday, we'll see\footnote{The benefactive $P_v$ lâ implies that the hawk's actions are directed at the men: 'for their benefit' in the sense that they will be the witnesses of the action.} that a hawk carrying a princess--a king's
daughter--in his beak will fly over the lake here,`` he said.138 Thus he
foretold.139 ``Well, didn't\textit{you} say you could make crossbows?``
140 ``Yes, I can (make crossbows).`` 141 And he made a crossbow.142
Well, another of them said, ``And what can\textit{you} do?``

143 ``I know how to swim,`` he said.144 As they were all talking
like this, that which had just been foretold \{came to pass / was fulfilled\}.145
It happened according to the prediction.146 So, the crossbow was all made, and
they stayed there together.147 When midday came, there was the hawk carrying the
king's daughter, and he was already over the lake.\footnote{Lit., ``and he had gotten and far as above the lake.``}148 So he took careful aim
and fired off a shot, and they fell into the lake.149 When they had fallen into
the lake, the man who could swim immediately struck out swimming, and by his swimming
rescued the princess who was out there--the young girl.150 When he had swum back
with her, the three of them [began to] \{fight over / lay conflicting claims to\}
the girl.151 They were arguing over who would marry her.152 One of them, now, one
of them wanted to marry her, saying ``I foretold it, so I, the maestro,
ought to marry her!`` 153 ``Well, you couldn't get her just by
telling your fortunes--I had to swim out and get her!`` 154 ``If
I hadn't shot him down with my crossbow you couldn't have gotten her by your swimming!``
155 Thus the three of them vied with each other.

156 You people think about it! Which one of them probably should have married her?
Thank about it, everybody.

157 Pà-ɛ́: You mean the queen\footnote{Pà-ɛ́ uses the word hɔ́-khâ-ma instead of hɔ́-khâ yâ-mî.}?

158 The one who shot [the arrow] ought to get her.

159 The swimmer!

160 T-y: It's the fortune-teller who should get her! If he hadn't foretold it they'd
never have known.

161 Even if he foretold it, if he didn't swim he couldn't get her.

162 H: Well, they took the case to the king and told him about it--and how did
it come out? ``Who was the one who \{passed by / created\}\footnote{ph\emph{a} may mean either 'pass by' or 'create' (both < Shan), though neither makes much sense here. The King is supposed to be talking. The Headman brought this narrative to an abrupt close, but retold the story on another occasion. See below, \# \_\_\_\_\_.} first?
Was it the fortune-teller?

\begin{center}
TO BE CONTINUED
\end{center}

\leftskip=0pt
