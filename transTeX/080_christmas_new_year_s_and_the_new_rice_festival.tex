\setcounter{footnote}{0}

1. H: Well, this Christian group of ours, we Lahu living in Huey Tat village, will
celebrate Christmas when the year 1965 is almost over and just before 1966 arrives.
All of us, from olden times until today, have observed this custom from generation
to generation. Therefore on December 24th Christmas will be here for us.\footnote{This is reminiscent of the Jewish practice of beginning holidays at nightfall of the preceding day.} Everybody
should prepare in advance before December 24th arrives, everything that we'll be
using, like rice and wood. Unhusked rice, hulled rice---prepare them both properly,
everybody prepare carefully the food for us to eat at Christmas. When the time
comes, we shouldn't do any work. I'm telling this to everybody.

2. T: Well, yes, we should prepare and make ready. You're saying that we'll begin
Christmas after dark on the 24th, right?

3. H: Yes. We'll begin Christmas after dark on the 24th.

4. T: Uh-huh.

5. H: When [Jesus] was born.

6. T: Aren't we indeed commemorating the birth of Jesus! So in our Lahu language
how should we say it? ``Christmas'' is an English word.

7. H: Well, we could call it ``Welcoming Jesus' Birth around the New Year'' in
our Lahu language.

8. T: Well, if I think about our Lahu language, I think we ought to call it ``The
Happy Festival.'' Jesus' birth was happy, so everybody please call it ``The Happy
Festival.'' So, when shall we celebrate New Year's?

9. H: Celebrating New Year's?

10. T: Yes.

11. H: Well, the day that we celebrate New Year's\footnote{The headman was unprepared for this change of subject to New Year's, and goes back to talking about Christmas.}---if we start celebrating
``the Happy Festival'' on December 24th, when the 25th arrives, we can really have
a good time. In the daytime.

12. T: Yes.

13. H: When [Christmas] is over, how many days should we still take off, in you
guys' opinion?

14. T: Well, we only celebrate New Year's once, at the beginning of the year.
I think we ought to take off three or four days.

15. H: Mm-hmm, So you're saying, for New Year's, for these two years [old and
new], it's only three or four days?

16. T: Not at all! The ``Happy Festival'' [alone] takes three days!

H: Oh.

17. T: So then, when it's time to celebrate New Year's, we should still take off
three or four days.

18. H: Three days and four days.\footnote{I.e., three days for Christmas and four days for New Year's.} So, you[`re saying] it comes to seven days.
If it's three days plus four days.

19. T: If it extends to seven days, we'll have a proper rest, all of us.

20. H: I'll bet\footnote{``I'll bet'' translates the \textbf{nɛ̀-ɔ̄} (Puf) `suppositional'.} you all want to take off work.\footnote{\textbf{chɛ̂} `be in a place, dwell'. In this context it means `to stop work during a festival'.} All you kids.

21. Pà-ɛ́: All of us want to celebrate.

22. Kachin: I want to celebrate.

23. H: Oh, they want to celebrate! Hey Kachin, do you want to celebrate, seven
days for New Year's?

24. Kachin: I do, I do!

25. H: [\textit{laughs}]

T: Well then, whose pig(s) shall we kill for Christmas?

26. Pà-ɛ́: Yours!

27. T: Mine?! Mine probably couldn't be divided up! It's just not very big, that
pig.

28. Pà-ɛ́: It could, it could!

29. Somebody: Hey, Khù-mɔ̂, you kill one too! If there's not enough to divide
up!

30. H: Oh, [mine] is a sow! Its meat probably doesn't taste good!

31. T: If we kill Khù-mɔ̂'s too, then once it's New Year's we won't have any
left to kill, in that case.

32. Kachin: I'll kill one too!

33. H: Well, Kachin, I guess yours will [make it] be enough!

34. T: Is yours a male?

35. Kachin: I've got a male, and I have a female too.

36. T: Oh.

37. H: Please kill the bigger one!

38. T: OK then, Headman, how many kilos would you like to get?

39. H: Oh, I'll take five kilos. This year, when we're relaxing and having a good
time we'll cook it up [the pork dishes] and set it aside\footnote{Until the guests come, when it will be heated up and served.}, that's how we'll
do it.

40. Pà-ɛ́: Hey, son-in-law!\footnote{\textbf{Pā-ɛ́} is role-playing here as if he were the Headman's father-in-law. He was actually an unmarried youth of about 17 when this was recorded.} Aren't you getting\footnote{\textbf{tê} \textbf{chi} \textbf{kîlô-lô}: This approximate quantification is expressed via reduplication. One could also say \textbf{tê-chi-chi} \textbf{kîlô}, by reduplicating the classifier.} about ten kilos, I
guess?

41. T: Ten kilos, eh?

41a. Pà-ɛ́: Yeah.

42. H: What about you?

43. T: Well, I guess I'd like two kilos [for Xmas] and three kilos [for New Year's].

44. H: (laughing) In that case it comes to five kilos! If we do the math.\footnote{\textbf{tà} \textbf{ni} \textbf{a} \textbf{qo} \textbf{ɔ̄}: lit. ``if [we] begin to look [at it].''}

45. T: Well then, Cà-qu-ní\footnote{This is Thû-yì's nickname, as transcribed in my 1965 field notes. It might mean `hairless one' (cf. \textbf{ni} \textasciitilde{ \textbf{ní}} `red; naked'). But DL 449 has \textbf{Cà-qu-nɛ̂ʔ} as his nickname.}, how many kilos are you getting?

46. Ty: I'll take four kilos and five kilos.

47. H: (laughing) Four kilos and five kilos.

48. T: Nine kilos.

49. H: My goodness, that's plenty then.

50. T: That's probably not enough. Look for another animal, one more. Let's still
look for one more pig. So, let's think over whose we should find.

51. H: Let's kill Cà-lɔ̂'s.\footnote{More play-acting. \textbf{Cà-lɔ̂} was my chief consultant on the 1965-66 fieldtrips, and lived in Chiang Mai, not in Huey Tat, and didn't have a pig to his name!} Cà-lɔ̂, I guess we should kill yours.

52. T: The big male.

53. P: I'll kill it.

54. H: A-ha.

55. T: Cà-lɔ̂ says he'll kill one of his. In that case there'll be enough!

56. H: Since it's very big, eh?

57. T: That's plenty, plenty then. Until the end of the New Year's celebrations,
nobody go off anywhere, please! [If you're] all by yourself that's no fun. If
a person doesn't take time off to celebrate, that's depressing. The thing is to
take time off. Do as we have discussed.

58. H: Right.

59. T: We must do as we have discussed. Make your plans carefully right now.
Otherwise, when the time eventually comes around later, it'll be like they say,
``By the time you realize what's happening you can't do anything about it.''\footnote{\textbf{T}. here segues into two proverbs one after the other. The first one means literally ``When your thoughts catch up you cannot catch the thing'' (When think-catch cannot chase-catch).}
Or, as they say, ``By the time a woman feels she's pregnant she has burdens aplenty!''
When the time comes there'll be nothing you can do.

60. H: So everybody should make their plans in good time. Every household. Because
we in Huey Tat are so few in number, aren't we! If we try counting the houses,
there are maybe twenty of them.\footnote{On a trip to Huey Tat in 2004, the village had grown to some 100 houses with perhaps 400 people.} There just aren't that many people! To be
precise about the number of people, if we say it in Northern Thai, there are just
\textit{p}ɛ̀ʔ\textit{ sî}ʔ\textit{ páj}!\footnote{``More than eighty.'' Cf. Siamese pɛ̀ɛt sìp \textbf{paj}} If we say it in Lahu, there
are only eighty people, these people of ours. Talk it over, talk it over, everybody
try talking it over---from now on whatever you do you ought to do it together.
How many days each time [for Christmas and New Year's respectively]\footnote{This meaning is conveyed by the reduplication of \textbf{qhà-nî} \textbf{ni} `how many days?'} will
everybody take off work to celebrate\footnote{This is an important issue. In a subsistence culture based on constant work in the growing season, villagers are often loath to take time off from vital work in the fields.}, you gray-headed elders ought to count
heads and think it over, right? \direct{laughs} Otherwise one of these
days if somebody wants to take off and somebody else doesn't want to, if one says
``Do it this way'' or ``Do it that way'' there will be all kinds of things to quarrel
about, won't there! Everybody think it over carefully.

61. Pā-ɛ́: If we just take off as long as we usually do it should be enough.

62. H: Yeah.

63. T: If we just do things brashly without regard to others' opinions\footnote{\textbf{pò-mɔ̀ʔ-pò-šê} ``exceed that which is appropriate'' < Shan.} we
could become a laughing stock!

64. H: Right. Well then, everybody agrees I guess. Taking time off the way we've
usually done, like our ancestors of old also used to celebrate for generation after
generation. Is that right?

65. Somebody: That's right then.

66. Ty: There must be at least one person who doesn't agree.

67. Pā-ɛ́: I doubt it.

68. H: [If you don't agree] speak out! Otherwise later on [you might say] ``I
wasn't satisfied but I didn't say anything! I'm the kind of person who doesn't
say anything when he doesn't agree.'' Say what's on your mind! If you don't agree,
please tell us!

69. Pā-ɛ́: Everybody agrees probably.

70. H: OK. If [you all] agree that's fine! By the grace of God, in joy and happiness
we will celebrate the new fruits of our labor\footnote{\textbf{cà-šɨ-ɔ́} \textbf{câ} \textbf{ve} `eat [celebrate] the New Rice festival', lit., ``eating the rice from the new paddy'': a harvest festival carried over from pre-Christian times, celebrated some time in October.}, and having reached the days
and nights of the New Year, everyone will happily live in peace and harmony, and
we'll just try to keep on doing all kinds of work, won't we!

71. T: Well then, when shall we celebrate the New Rice Festival this year? The
time to celebrate it is getting close! After we've finished discussing the matter
of celebrating New Year's, let's try talking about the New Rice Festival. This
is the time to do it---otherwise some people will be around\footnote{In this sentence \textbf{chɛ̂} has its basic meaning of `be in a place; stay', rather than the extended sense `take time off for a holiday'.} but others won't
be---it'll be impossible to discuss later.

72. H: Well, we've always celebrated the New Rice Festival in October, every year!

73. T: October is getting close!

74. H: It's either on the eighth day after the new moon of October\footnote{The classifier \textbf{tâʔ}, from the verb \textbf{tâʔ} `climb', traditionally referred to days of the waxing moon, but when the Christian Lahu reckon dates according to the Western calendar they use the same classifier like our ordinal numerals (\textbf{tê} \textbf{tâʔ} `the first' [``one climbing''], \textbf{nî} \textbf{tâʔ} `the 2nd', etc.), with the counting beginning from the first of the calendar month, not from the new moon.}, or else
we've also done it on the fourth day---celebrating the New Rice Festival.

75. T: But then, if we do it on the fourth, how in the world will we be able to
carry the rice home?

76. H: Hmm---

77. T: It's not something that can be done in just two or three days. It's the
rainy season! It's not the hot season!

78. H: Well, we've celebrated it on the fourth, in the past!

79. T: As for when they celebrated it in the past, just because we remember a certain
year, a certain time, it's impossible to celebrate it at that same time every year.
Some years there's no rain, other years the rain goes on and on.\footnote{\textbf{mû-yè} \textbf{mu}: lit. ``the rain is high''}

80. H: We've also [sometimes] celebrated it on the fourteenth, the New Rice.

81. T: In that case let's just do it on the fourteenth.

82. H: Mm-hm. The New Rice---

83. T: You're going to invite people [from other villages], right?

84. H: Well, I think it would be good to invite our Lahu relatives who live in
all the nearby villages!

85. T: In that case you guys have to think about food and drink now!

86. Somebody: Well, as for food, I'll contribute a pig all by myself!

87. T: All by yourself?

88. Somebody: Yes.

89. Somebody else: In that case, the meat---

90. T: In that case who all will contribute extra food for second helpings?\footnote{\textbf{ɔ̀-phâʔ-ɔ̀-lə́}: \textbf{phâʔ} `to exceed'; \textbf{lə́} `be left over, extra' (< Tai; cf. Si. \textit{ly̌a}).}
If there's only a small amount it won't be enough. Since you're inviting other
people. What do you say?

91. H: I don't actually have very much pork. Because mine is a female. There's
only three or four kilos\footnote{A scrawny sow indeed!}, I'd say.

92. Pā-ɛ́: If there's not enough, I'll kill one of my elephants and contribute
it.

93. Somebody: An elephant?

94. Pā-ɛ́: Yep.

95. T: My goodness, people could never finish paying for it, I bet! An elephant---\direct{laughter}
Is it a male or a female, this elephant of yours?

96. Pā-ɛ́: It's a male, I think.\footnote{The suppositional use of \textbf{nɛ̀-ɔ̄} `probability' is humorous here, since there is absolutely no difficulty in determining whether an elephant is male or not.}

97. T: A male. My goodness, my goodness! Who all is there who doesn't eat elephant
meat?

98. Pā-ɛ́: People don't eat it, elephant meat.

99. H: Oh, it's very tasty.

100. T: I'd like to try eating it myself, elephant meat! Since I've never eaten
it.

101. H: Elephant meat is tough and chewy, so it's delicious! I've had a chance
to eat it before.

102. Ty: Let's divide it into shares! I'll take my whole share [of meat
for the festival] from the elephant meat!

103. T: Dividing it into shares, I doubt you'd be able to pay for it.

104. H: How many thousand baht---

105. T: How many thousand baht are you intending to contribute yours for?

106. Pā-ɛ́: I'm not charging anything. I'm contributing it for free.

107. T: So you'll kill it and contribute it?

108. Pā-ɛ́: Yeah.

109. T: That's great then!

110. Somebody: It sounds like he's saying he'll divide it up into shares.

111. Somebody else: I hear that Khù-mɔ̂ also has an elephant to kill and contribute,
an elephant. To eat at the New Rice Festival.

112. Khù-mɔ̂: Since I'm doing it for charity, I won't accept any payment.

113. H: For my part I've got one pig! A sow, with only about seven or eight kilos
of meat.

114. T: Well, I'll also contribute three baskets full of rice.\footnote{\textbf{tû}: < Shan \textbf{tun}. 1 \textbf{tû} = 1 \textbf{pîʔ} = 1 \textbf{pû}} Let's look
for other contributors of rice! Cà-qu-ní (= Thû-yì), how much will you put
in?

115. Ty: For my part I guess I'll put in one or two baskets.

116. T: Just one or two baskets. Hey, you Kachin over there! Jingpho guy!\footnote{A longtime and well-liked resident of Huey Tat, this Jingpho man evidently made the trip from Burma to Thailand in the early 1950s with the rest of the villagers.}

117. Kachin: Oh, five or five and a half baskets.

118. T: Five and a half baskets. Well, that's great. So Cà-lâ over there!
How much will you put in?

119. Cà-lâ: Who, me? Well, I'll contribute three baskets\textit{ }and two baskets!

120. T: Three baskets\textit{ }and two baskets. Yes, that's plenty, that's plenty.
Well then, when you've invited people and they've come, don't do anything embarrassing,
anything that would bring shame to anybody, OK, all of you? As for this bunch
of young guys, don't you go courting the girls. If you want to do some courting,
just do it at home.

121. Pā-ɛ́: I do want to court the girls. \direct{laughter}

122. T: As for courting now, do go and court each other. But don't do bad things,
any of you.\footnote{This is an intentional \textbf{pun}: \textbf{mâʔ} \textbf{dàʔ} \textbf{ve} \textbf{ɔ̀-cə̀} `the courting thing'/ \textbf{mâ} \textbf{dàʔ} \textbf{ve} \textbf{ɔ̀-cə̀} `bad things' : \textbf{mâʔ} `to court', \textbf{dàʔ} Pv `mutual', \textbf{mâ} `negative', \textbf{dàʔ} Vadj `good'.}

123. Ty: I probably will! The others will too, I bet. So maybe I will too
\direct{laughter}.

124. Pā-ɛ́: So, you guys, it's up to you all to think this over.

125. T: In any case lots of girls will come.

126. H: They'll come, they'll come.

127. Ty: What kind of girls?

128. T: From up there, around Šá-to-dō, Mɛ-mə̂-nwɛ̀ and Vê-khɛ̄ villages
they'll come!

129. H: You should mention Vê-khɛ̄ and Vê-làʔ.

130. T: Yes.

131. Ty: Ah, that's exactly what I've been thinking about.

132. H: (laughing) You'll be very happy, won't you, when the girls come.

133. Somebody: Up and at `em, you guys! [laughter]

134. H: They're all young men, and when the New Rice Festival comes, when Christmas
comes, they're just hoping and waiting for girls. Since they're all young guys.
\footnote{\textbf{chɔ-há-pā} `a young man of marriageable age; a young bachelor'.}

135. T: They also contribute a lot of money, some of them. They have plenty of
money. Well, Cà-lɔ̂, you'll come too, won't you, when the time to celebrate
the New Rice comes.

136. P: Oh, I'll certainly come.

136a. T: Ah.

137. H: Come and enjoy yourselves, the two of you!\footnote{Talking to \textbf{Cà-lɔ̂} (my consultant who lived in Chiang Mai) and myself.} Don't shave off that beard.
We'll offer you pork fat until it's dripping.

138. T: This beard is all bushy, as it'll be a sight to see when it's sticky with
pork fat!

139. JAM: A sight to see, eh?

140. T: Yeah, a great sight to see.

