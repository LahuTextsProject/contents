\setcounter{footnote}{0}

Dramatis Personae:

H = Headman Cà-bí

T = Teacher Cà-bo

P = Paul Cà-lɔ̂

1. H: Well now, if we talk about how the Lahu people set up a new village, it's
like this.

2. If the place where we live is not good for farming, for earning a living from
the land, if the place isn't good that's what we do!

3. Since we don't want to stay there, a new --- looking for a good new place for
growing crops is what we call ``setting up a new village,'' this time in a good
place.

4. T: Well then, haven't you been talking and talking this year about going up
north there to Nâ-qhày?\footnote{This location remains to be identified.}

5. Isn't that right?

6. Isn't it said that there are irrigated fields up there --- that's what they're
saying.

7. H: I've gone there. We've already gone to look and see whether they still have
irrigated fields.\footnote{Note the sequence of Pv's \textbf{á} \textbf{ò} \textbf{šɔ̄}. \textbf{á} (\textasciitilde{} \textbf{tā}) `durative' + \textbf{ò} `change of state' + \textbf{šɔ̄} `still'.}

8. Everything there is just great.

9. The houses there are very nice too.

10. T: If you pay attention to what people say, they're saying that there's not
even any water there!

11. H: For God's sake,\footnote{\textbf{pòthôo}: a Tai-derived interjection (``by the Buddha!''), used frequently also by the Christian Lahu.} as far as water goes, there's a river called Yɛ̂-nâ\footnote{To be identified.}
which comes down from the high mountains with so much water that you can't use
it all.

12. It's very good. [But] down below to the south, there is a foamy place in the
river\footnote{\textbf{g̈ɨ̀-phû-twɛ̂ʔ}: see DL 1145.} where the water doesn't reach, in the hot season.

13. T: They've gone and lied to you, I bet --- and you're also lying!

14. H: I'm not lying. I've gone to see it and come back!

15. Didn't you just see me [coming back]?

16. T: Oh.

17. P: When you guys set up a new village, what sort of things do you have to consider?
[inaudible words]

18. Whether it's a place where you can get water easily, or whether it's very high
up in the mountains, or whether there are lots of trees around?

19. Whether it's an easy place to come and go on the paths?

20. What are all the kinds of things that you consider before setting it up, setting
up a village like this?

21. T: Well, as for that ---

22. P: In other words, whether it's a place where it's pleasant and good to live?

23. H: That's right.

24. When setting up a village, all the good points --- er, of --- you have to consider
the good points before setting up a village, they say.

25. If it's good for cultivation, if water is plentiful, if everything is convenient,
then we'd set up a village.

26. T: Where there's plenty of game and fish.

27. H: It's easy, if it's like that.

28. T: We're looking for someplace where it's easy to dam up a stream, to hunt
for crabs!

29. When we Lahu are looking for a place for a village ---

30. P: Water --- if it's far from a place where you can draw water to drink you
probably can't make a go of it.

31. T: If it's far from where you draw water to drink, you can't.

32. Our Lahu ancestors used to suffer\footnote{\textbf{bɔ̀} \textbf{à}: lit. ``were tired out''.} from that!

33. Since you drink water day after day, if there's no water nearby you can't make
it!

34. P: Like when there aren't very many trees, right?

35. T: Yeah, right.

36. H: When there are lots of trees, it's easy to gather wood to burn, easy to
chop down bamboo, and all that.

37. P: So, looking at that sort of thing, considering all that, you get to set
it up.

38. H: Mm-hm.

39. P: Why is it that when the Lahu set up a village, every year, every time, they
only set it up in the mountains?

40. Why don't they go look in the plains?

41. Oh, the plains are tough, for us Lahu it's tougher.

42. Since even in the old days we never used to get to do it or get ahold of [anything].\footnote{\textbf{g̈a} \textbf{te} \textbf{ve} `get to do'; \textbf{cɔ̀} \textbf{la} \textbf{ve} `come to have'. I.e., we never managed to set up a village in the plains.}

43. To live in the plains is hard!

44. P: Oh, you are talking about malaria.\footnote{\textbf{nà} \textbf{phə̂ʔ} \textbf{ve}: have any serious disease associated with living in the plains, in which one runs a high fever, especially malaria, but also typhoid, cholera, plague, hemorrhagic fever, dengue fever, etc.}

45. T: Yes. When we have chills and fever, there's no money to buy medicine to
take, for us hillfolk.

46. Well then, our real dwelling place is up there in the high reaches\footnote{\textbf{yɛ̂} \textbf{qhɔ}: `in the upper reaches of the hills' (\textbf{yɛ̂} `high cold area, very high land').} where
the mountains are high, yeah, where there are plenty of gibbons, plenty of squirrels\footnote{\textbf{fâʔ-thɔ̂ʔ} `common squirrel' [\textit{Callosciurus sp.}]},
lots of red-cheeked ground squirrels\footnote{\textbf{fâʔ-šwɛ} [\textit{Dremomys rufigensis}].}!

47. We love to live in places like that!

48. That's what we've been looking for now.

49. H: Yes indeed, that's the way it is!

50. But now those of you haven't gone don't know what Nâ-qhày is like.

51. If you haven't been there, you don't know.

52. I've been there already!

53. I've looked it over carefully!

54. P: Well, when you make a village, how do you divide up the land --- I mean
the places where you'll build a house?

55. H, T: Well ---

56. P: Whatever place each person wants to get, he gets. He can just go look at
it and take it\footnote{There is a 4-verb concatenation here: \textbf{g̈a} \textbf{ca} \textbf{ni} \textbf{yù} (\textbf{g̈a} `be able', \textbf{ca} `go and', \textbf{ni} `look at', \textbf{yù} `take').}, is that right?

57. T: Yep. Whatever place each person wants, if the place pleases him he goes
and takes it.

58. P: ``I'll build in this place, I'll build a house\footnote{\textbf{yɛ̀} \textbf{te} \textbf{ve} (OV) `build a house' (``make a house''), \textbf{te} (V) `do; make'.} in this place, in that
place I'll make a garden'' --- is that how it's done?

59. H: Yes, yes, yes.

60. P: When you build a house, how do you do it?

61. Like, do you help each other, one person to another?

62. Or else does each person build his own house by himself?

63. Oh, when we build houses, we certainly help each other! During the year when
we're going to set up [a new village] we can't avoid helping each other.

64. If each person does everything he's supposed to do, if he does what he ought
to do all by himself, then the whole village will pitch in and help him.\footnote{A three-verb concatenation: \textbf{phôʔ} \textbf{ga} \textbf{pî} (\textbf{phôʔ} `do in a group', \textbf{ga} `help', \textbf{pî} `3rd p. beneficiary').}

65. Sometimes we can manage to do it in a single day, building a house.

66. P: You can do it in only one day?

67. T: Yes.

68. P: So then in order to build the houses of a whole village properly, I wonder
how many days, how many months it takes.

69. T: Oh, it doesn't take many months, you know.\footnote{The 2nd person pronoun \textbf{nɔ̀} is added as an afterthought at the end of the sentence, signaling that the speaker is being attentive to the listener's question.}

70. P: In order for everybody to have a proper house to live in!

71. T: If everybody properly puts his effort\footnote{\textbf{ú-khə̀ʔ} \textbf{cɨ́} \textbf{ve}: lit. ``wear a head strap (for carrying)''; used figuratively to mean `apply elbow grease; put one's shoulder to the wheel; put one's nose to the grindstone'.} into building his own house,
it wouldn't even take a month.

72. It could be done in one or two weeks.

73. P: So when the houses are all built, it's like what other people call it, isn't
it?

74. [I mean,] in Shan and Northern Thai they call it ``eating a new village,''
don't they, and they have a big party.

75. T: Well, as for that ---

76. P: Don't they call it ``climbing up into a new house''?

77. T: We ``grace-begging'' Lahu\footnote{\textbf{šú-tɔ̂\textit{(n)-p}ā}: lit. ``those who beg for grace.'' Epithet used by the animist Lahu for the Christian Lahu, here used ironically by Teacher. The word is < Shan \textit{sun-t}ɔŋ < Bs. \textit{hsu' tâuN} (WB \textbf{chu}' tôŋ). See DL: 1193.} don't do it that way.

78. As for climbing up into a new house, before other people can climb into it,
he [the owner] himself has to go climb up and sleep in it first!

79. That's what other people say about us Lahu.

80. P: Don't other people call it \textit{khyn hə́n màj}?\footnote{This is the Shan expression that means ``climbing into a new house,'' equivalent to Siamese khŷn bâan \textbf{màj}. The Siamese cognate to Shan \textbf{hə́n} is \textit{ryan}, which also means `house; household', although \textit{bâan} (which also means `village') is now the usual Siamese word for `house'.}

81. H: Well, that's Shan ---

82. T: In the Shan language they say \textit{khyn hə́n màj}.

83. We Lahu call it \textit{y}ɛ̀\textit{-ší tâ}ʔ\textit{ ve}, but before
anybody else can climb up, he himself [the owner] [60] must climb up first, before
anybody else.

84. P: Right

85. T: Well, I've heard that in your village\footnote{Even though Teacher and Headman really live in the same village (Huey Tat), the Teacher is pretending to live elsewhere.}, when [you] said ``let's go up
there,'' some people --- some wanted to go, but some didn't want to go!

86. I just don't know what we, the two of us\footnote{I.e. the Teacher and the Headman.} can do with people like that
[who don't want to go].

87. H: Everybody will go. They'll go in their own sweet time.

88. T: If somebody takes so much time about it, later on if he doesn't get a good
place he'll scold people!

89. H: Well, as for that, the land is broad!

90. Land, now, the land you [could] have to live on, there's so much of it that
you could never run out of places to live!\footnote{\textbf{mâ} \textbf{chɛ̂} \textbf{pə̀}: lit. ``not finish living'' (\textbf{mâ} `negative', chɛ̂\textit{ `live'}, \textbf{pə̀} `V exhaustively').}\footnote{Unfortunately this has not been true for the past several decades, as overpopulation has led to land disputes in Thailand between the hillfolk and the plains populations.}

91. T: I'm afraid that\footnote{This clause is introduced by \textbf{à-mù} (Conj.) `lest; or else' (unpleasant hypothesis). See DL 100-101.} --- if it's not a good place, a person wouldn't want
to live there.

92. H: If he doesn't want to live there, just let him not go!

93. It's up to him!

94. T: You're being unreasonable\footnote{\textbf{chɔ} \textbf{khɔ̂} \textbf{â} \textbf{hêʔ}, lit. ``it isn't human speech''.} about this!

95. H: I'm right! He took yours, yours.

96. T: Well, the two of us can't do anything about it.

97. H: So somebody who hasn't come yet like that---

98. So if anybody doesn't want to come\footnote{The deictic viewpoint has now changed from going (\textbf{qay}) to the new village to coming to it (\textbf{là}).}, wherever he wants to live in the future
is up to him!

99. T: But if unfortunately somebody should be unhappy, even a person who had gone
up there before other people, he might someday [want to] come back again!

100. It's hard to reason with us Lahu people.

101. H: Oh, if they manage to stay there and not come back, they can make a go
of it up there.

102. All of you guys, if you manage to go, you're called \textit{men}, so you must
come to a decision!

103. You're not children! Don't fool around!\footnote{\textbf{qôʔ} \textbf{gɨ̂}, `say for fun; talk thoughtlessly'.}

104. T: Men, men so-called, flap their mouths a lot\footnote{\textbf{mɔ̀ʔ-qɔ} \textbf{vây} \textbf{ve}: lit. ``be fast in the mouth''.}, you know.

105. H: They won't go back. So we've got to talk things through to a conclusion.

106. Whatever we do, let's go and do it, okay?

107. But for a while at least, stop flapping your mouths and quarreling!

108. There's no point in arguing.

109. I told you that if you wouldn't go before, don't go at all, but you wouldn't
listen.\footnote{\textbf{tâ} \textbf{qay} \textbf{qôʔ} \textit{\textbf{pî}} \textbf{ve} \textbf{nɔ̀} \textbf{â} \textbf{na} \textbf{ɛ̀ʔ}: This sentence has a somewhat anomalous \textbf{pî} instead of \textbf{lâ} for 3->2 benefaction.}

110. And now you say you won't go.

111. Aren't you a man then?

112. T: What I said was, if I would go, every last person should go.

113. H: They \textit{will} go, every last one of them.

114. Even if they wouldn't go [before], now they'll go.

115. T: I tell you they won't go, that bunch of them.

116. H: If they don't go, afterwards [those who have gone] will say ``Come!'',
that's what they'll say.

117. Yes, once you've gone and looked over the good land for cultivation\footnote{\textbf{g̈a} \textbf{câ} \textbf{kɨ̀} (Ndeclaus): ``place to get to eat'', i.e., `place to earn a livelihood (by agriculture)'.},
the really fine land, you won't want to leave.

118. I myself, when I went there to look it over a while ago, I said to them [those
in Nâ-qhày], ``If they [from Huey Tat] don't go, don't worry about it.''\footnote{\textbf{mâ} \textbf{qay} \textbf{qo} \textbf{tâ} \textbf{ni} \textbf{qôʔ} \textbf{pî} \textbf{ve}: here \textbf{ni} `look at' means `pay attention to; take into consideration'.}

119. What will those who \textit{have} seen it do?\footnote{\textbf{ni} \textbf{ò} \textbf{ve} \textbf{ɔ̄} \textbf{qhà-qhe} \textbf{te} \textbf{è}: i.e., it's more important to consider what those people who have seen it will do after they move there.}

120. I myself have never been there, so I'll listen to what you say, and go.

121. Then if the time comes when there's nothing to eat, you'll know about it.

122. H: If you don't do it you'll starve anyway.

Somebody: 123. They've gone already, they've gone already!

124Y. ou deep thinkers\footnote{\textbf{dɔ̂꞊ná-lɨ̂ʔ} \textbf{ve}: \textbf{dɔ̂} `think', lɨ̂ʔ (Vadj) `deep' < Tai; \textbf{ná} (Vadj) is the native Lahu word for `deep'.} have gone already!

125. T: All of them are just following their hearts!\footnote{\textbf{ni-ma-qa-pɨ} \textbf{tí} \textbf{tɛ̂ʔ} \textbf{ve}: ``only measure their hearts''.}

Somebody: 126. Can you guys make a living there?

127. H: You mean growing crops?

128. The new rice, the new crops are just coming out now.

129. If you want to have an increase in rice production.

130. T: That's fine then, if that's the way it is.

131. What kind of fields\footnote{\textbf{à-thòʔ-ma} \textbf{ca} \textbf{te} \textbf{câ}: short for \textbf{à-thòʔ-ma} \textbf{ca} \textbf{te} \textbf{câ} \textbf{kɨ̀} ``what kind of place to go and earn a living''.} do you say they have up there?

132. Do you say swiddens or paddy-fields?

133. They plan to cultivate paddy fields. The first year I went there they were
still living off swiddens in the high mountains.

134. While they were still cultivating swiddens they also worked paddy-fields.

135. P: What kinds of crops will you plant to eat?

136. H: After we prepare the paddy fields we'll plant rice; we'll cultivate swiddens
in the hills, and make gardens to grow several kinds of things, and make smelly
bean\footnote{\textbf{nɔ̂ʔ-qhɛ̂}: lit. ``shit-beans'' (also called \textbf{nɔ̂ʔ-kɨ̂} ``rotten beans''): a kind of soybean fermented to make odoriferous condiments.} fields.

137. We'll grow maize, bananas, and sugar-cane!

138. All kinds of pumpkins.

139. We'll earn plenty of money up there, I tell you.

141. The plains up there in the Nâ-qhày valley are all excellent, they say.

142. Other people haven't worked them, so I'll just do it myself.

143. P: Are you saying you claimed\footnote{\textbf{vɛ} (V) `lay claim to something, usually by leaving a visible symbol of ownership'.} [that land] first?

144. T: I have claimed it already.\footnote{The Teacher is playing different roles in this conversation. Previously he was reluctant to move up there, but now he's saying that he's been planning to go all along.}

145. H: But then you won't be able to cultivate all of it by yourself.

146. You should just cultivate enough for one person to manage easily.

147. T: I want to earn money by selling [stuff] to others.\footnote{\textbf{hɔ̂} \textbf{lɛ̀ʔ} \textbf{ve}: `sell in order to eat', i.e., ``sell to earn (money) to live on''. Here \textbf{lɛ̀ʔ} `lick' is used as the informal equivalent of \textbf{câ} `eat'.}

148. H: So you'll sell for a living.

149. But you don't have the government's permission, I'll bet.

150. T: Down there on the high ground above the river there won't be anybody who
could climb up onto the bank on the rocky side to grab it for himself.

151. So I've claimed it, down there.

152. H: There's plenty [of land] around the rocky cliff, so you go ahead and claim
some!

153. Wherever you want to claim, claim it!

154. There's lots of it [land]. We can't use it all up.

155. The Akha guy said, just you look at it, look at it --- it's good.

156. Plenty of game, plenty of monkeys, even if you want to eat fish, for God's
sake!

157. When you go there, even though it's called the mountains, the land doesn't
go up and down.

158. It's flat and smooth --- wherever you go it's very easy to earn a living!
\footnote{Note the correlative use of \textbf{qhɔ̀}: \textbf{qhɔ̀} + V1+ \textbf{qhɔ̀} + V2 ``wherever V1, there V2''.}

Somebody: 159. When I went there recently I was awfully\footnote{\textbf{šɨ-e-la-yò} `so much one could die'; verb intensifier, undoubtedly calqued on Thai \textit{cətaaj}.} thirsty!

160. T: Listen, listen, listen to him!

161. H: You're going to be really hot down there, in the plains, in the hot season.

162. T: It's not flat land. How are we ever going to make a go of cultivating paddy-fields?

163. H: Up there, over there --- in the riverbed they've channeled [the water]
as it falls from the high mountains so that when it reaches down below there's
so much water that you can't use it all.

164. The Nâ-qhày River is no small thing.\footnote{\textbf{mâ} \textbf{ɨ̄} \textbf{ve} \textbf{mâ} \textbf{hêʔ}: ``is not the case that it is not big.''}

165. It has two valleys, Mɛ-thà-lây and Kɛ̀-pa-tâwʔ.\footnote{< Thai kɛ̀ɛ\textit{ŋ-pan-táw} (kɛ̀ɛŋ `cataract').} ,\footnote{Note the unusual use of the pluralizer \textbf{hɨ} after an inanimate noun.}

166. As for the paddy-fields, you can't even cross them when it rains.

167. T: Not long ago a bunch of us went there. We had been sent to look it over,
and Nâ-qhày was just floating knee-deep in sand!

168. H: That's because the water had come flowing down, and that's a good fertilizer
if you're working paddy-fields.\footnote{Paddy-fields, being irrigated, require a handy water-source.}

170. In the hot season it's all sand, there's no water down there.

171. T: With sand whatever you plant is no good.

172. H: It's fine! That's just where to plant crops.

173. Other people are earning a living [there]. You just go and check it out.

174. P: Isn't it far from a road for cars, your place?

175. H: Cars can get all the way to it up there.

176. T: So cars can get all the way up there, you see!

177. P: Then that's great for you!

178. H: That senior headman\footnote{kānān (< Thai \textit{kamnan}): a senior headman whose jurisdiction is a \textit{tambol}, or subdistrict, rather than a single village. Translated by the Lahu as \textbf{qhâʔ-šɛ-lón} `great headman'.} up in Nâ-qhày, what was his name now? I've forgotten
it!

179. Yeah, there are a whole group of senior headmen there.

180. They have cars, big trucks\footnote{\textbf{lɔ̄lī} (< Brit. Eng. lorry): `motor vehicle (of any kind)'. A large vehicle or truck may be specified by adding the augmentative morpheme -\textbf{ló}.}, my God, lots of them!

181. T: Their cars, they have two of them\footnote{\textbf{nî} \textbf{khɛ}: the animal classifier \textbf{khɛ} is used to count motor vehicles, since they move and make noise.}, I tell you!

182. P: So we Lahu will probably get to ride in them, I bet.

183. H: We'll get to ride, we'll get to ride!

184. P: Whereabouts would you go to market, if you wanted to go to a [Thai] market?
To Chiang Dao?\footnote{A town about 60 km. north of Chiang Mai, on the road to Farng.}

185. H: Up there, way up there, you come to Farng.\footnote{A large town about 120 km. north of Chiang Mai.}

186. T: Oh, you'll get to come around to the area around Farng!

187. P: Oh, the Farng area, is it?

188. H: Up there in the city there's a road which goes through to a place where
there's a big rice-mill, right up there!

189. As for a market, there's a big entrance to it in Farng.

190. T: Oh, I see.

191. P: So you come to the Farng market and go right in, is that right?

192. To buy things.

193. H: Yeah, we go buy things, whatever, all kinds of stuff to go buy.

194. You can even buy an ox to raise.

195. It's very easy to get an oxcart to pull things for a living, up there!

196. P: Wow, that's probably all right then!

197. H: Mm-hm, you won't have to carry anything at all on your back to earn a living,
on the plains.

198. Staying here I don't see how we can survive!

199. If we can just manage to settle up in that area.

200. T: As soon as one thing has been done and come to pass, and has almost succeeded,
another thing comes up to be done.

201. When one thing has come to pass and almost succeeded, another thing comes
up to be done.\footnote{The Teacher is repeating himself here.}

202. I'll follow your advice.\footnote{\textbf{nɔ̀} \textbf{qhâ}, lit. ``your way''.} I'll really just keep following your advice!

203. Just this once, I'll be patient and try to follow your advice.

204. But if that place also turns out to be a dud, I'll never take your advice
again!

205. H: We'll never leave again [if we move up there].

206. If we do manage to get up there, we'll stay there forever afterwards and die
there of old age!

207. For our children's and grandchildren's generations.

208. We have planned to make it succeed!

209. We're not just toying with the idea!\footnote{\textbf{dɔ̂} \textbf{gɨ̂}: ``think for fun''.}

