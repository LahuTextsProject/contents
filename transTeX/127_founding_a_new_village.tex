
[badly recorded, especially the beginning]

Dramatis Personae:

H = Headman Cà-bí

T = Teacher Cà-bo

P = Paul Cà-lɔ̂

1. H: Well now, if we talk about how the Lahu people set up a new village, it's
like this.

2. If the place where we live is not a good place for farming, for earning a living
from the land, if the place isn't good that's what we do!

3. Since we don't want to stay there, a new --- looking for a good new place for
\{growing crops/farming\} is what we call ``setting up a new village,'' this time
in a good place.

T: 4. Well then, haven't you been talking and talking this year about going up
north there to Nâ-qhày?\footnote{This location remains to be identified.}

5. Isn't that right?

6. Isn't it said that there are irrigated fields up there --- that's what they're
saying.

H: 7. I've gone there. We've already gone to look and see whether they have irrigated
fields.\footnote{Note the sequence of Pv's Á ò šɔ̄. Á (\textasciitilde{} tā) `durative'}

8. Everything there is \{just great/ very pleasant

9. The houses there are very nice too.

T: 10. If you \{pay attention/listen\} to what people say, they're saying that
there's not even any water there!

H: 11. For God's sake,\footnote{pòthôo: a Tai-derived interjection (``by the Buddha!''), used frequently} as far as water goes, there's a river called Yɛ̂-nâ\footnote{To be identified.}
which comes down from the high mountains [with so much water that you can't use
it all]. \texttt{<}Inaudible --- my guess as to what the missing words are ---
check MM's transcription\texttt{>}

12. It's very good. [But] down below to the south, there is a foamy place in the
river\footnote{g̈ɨ̀-phû-twɛ̂ʔ: see DL 1145.} where the water doesn't reach, in the hot season.

T: 13. They've gone and lied to you, I bet --- and you're also lying! [inaudible
words]

H: 14. I'm not lying. I've gone to see it and come back!

15. Didn't you \{just/recently\} see me [coming back]?

T: 16. Oh.

P: 17. When you guys set up a new village, what sort of things do you \{consider/
look at\}? [inaudible words]

18. Whether it's a place where you can get water \{easily/ properly\}, or whether
it's very high up in the mountains, or whether there are lots of trees around?

19. Whether it's an easy place to \{come and go on the paths/walk back and forth\}?

20. What are all the kinds of things that you consider before setting it up, setting
up a village like this?

T: 21. Well, as for that ---

P: 22. In other words, whether it's a place where it's pleasant and good to live?

H: 23. That's right.

24. [When] setting up a village, all the good points --- er, of --- you have to
consider the good points before setting up a village, they say.

25. If it's good for \{agriculture/cultivation/farming\}, if water is plentiful,
if everything is \{easy/convenient\}, then we'd set up a village.

T: 26. [Where] there's plenty of game and fish.

H: 27. It's easy, if it's like that.

T: 28. We're looking for someplace where it's easy to dam up a stream, to hunt
for crabs!

29. When we Lahu are looking for a place for a village ---

P: 30. Water --- if it's far from a place where you can draw water to drink you
probably can't \{do/ make a go of\} it.

T: 31. If it's far from where you draw water to drink, you can't.

32. Our Lahu ancestors used to suffer\footnote{bɔ̀ à: lit. ``were tried out''.} from that!

33. Since you drink water day after day, if there's no water nearby you can't \{make/
do\} it!

P: 34. Like when there aren't very many trees, right?

T: 35. Yeah, right.

H: 36. When there are lots of trees, it's easy to gather wood to burn, easy to
chop down bamboo, and \{everything/ all that\}.

P: 37. So, looking at that sort of thing, considering all that, you get to set
it \texttt{<}a new village\texttt{>} up.

H: 38. Mm-hm.

P: 39. Why is it that when the Lahu set up a village, every year, every time, they
only set up in the mountains?

40. Why don't they go look [for something] in the plains?

41. Oh, the plains are tough, for us Lahu it's tougher.

42. Since even in \{older times/ the old days\} we never used to get to do it or
get ahold of [anything].\footnote{g̈a te ve `get to do'; cɔ̀ la ve `come to have'. I.e., we never managed}

43. [inaudible, something to do with diseases in the plains] ... it's hard!

P: 44. Oh, you are talking about malaria.\footnote{nà phə̂ʔ ve: have any serious disease associated with living in the plains,}

T: 45. Yes. When we have chills and fever, there's no money to buy medicine to
take, for us hillfolk.

46. Well then, our real dwelling place is up there in the high reaches\footnote{yɛ̂ qhɔ: `in the upper reaches of the hills' (yɛ̂ `high cold area, very} where
the mountains are high, yeah, where there are plenty of gibbons, plenty of squirrels\footnote{fâʔ-thɔ̂ʔ `common squirrel' [Callosciurus sp.]},
lots of red-cheeked ground squirrels\footnote{fâʔ-šwɛ [Dremomys rufigiensis].}!

47. We love to live in places like that!

48. That's what we've been looking for now.

H: 49. Yes indeed, that's the way it is!

50. [But] now [inaudible] ... you guys haven't gotten up there.\footnote{I.e., up to Nâ-qhày [sentence \#4].}

51. If you haven't been there, you don't know.

52. I've been there already!

53. I've looked it over \{carefully/ properly\}!

P: 54. Well, when you make a village, how do you divide up the land --- I mean
places where you'll build a house?

H, T\} 55. Well ---

P: 56. Whatever place each person wants to get, he gets; he can just go look at
it and take it\footnote{There is a 4-verb concatenation here: g̈a ca ni yù (g̈a `be able', ca `go}, is that right?

T: 57. Yep. Whatever place each person wants, if the place pleases him he goes
and takes it.

P: 58. ``I'll build in this place, I'll build a house\footnote{yɛ̀ te ve (OV) `build a house' (``make a house''), te (V) `do; make'.} in this place, in that
place I'll make a garden'' --- is that how it's done?

H: 59. Yes, yes, yes.

P: 60. When you build a house, how do you do it?

61. Like, do you help each other, one person to another?

62. Or else does each person build his own house by himself?

63. Oh when we build houses, we certainly help each other!

63a. During the year when we're going to set up [a new village] we can't [not]
help each other.

64. If each person does everything he's supposed to do, if [he does] what he ought
to do all by himself, then the whole village will pitch in and help him.\footnote{A three-verb concatenation: phôʔ ga pî (phôʔ `do in a group', ga `help',}

65. Sometimes we can manage to do it in a single day, building a house.

P: 66. You can do it in only one day?

T: 67. Yes.

P: 68. So then in order to build the bases of a whole village properly, I wonder
how many days, how many months it takes.

T: 69. Oh, it doesn't take many months, you know.\footnote{The 2nd person pronoun nɔ̀ is added as an afterthought at the end of the}

P: 70. In order for everybody to have a proper house to live in!

T: 71. If everybody properly \{puts his effort into building/ concentrates on\}
his own house, it wouldn't even take a month.

72. It could \{take/be done in\} between one and two weeks.

P: 73. So when the houses are all built, it's like what other people call it, isn't
it?

74. [I mean,] in Shan and Northern Thai they call it ``eating a new village,''
don't they, and they have a \{festival/big party\}.

T: 75. Well, as for that ---

P: 76. Don't they call it ''climbing up into a new house''?

T: 77. We ``grace-begging'' Lahu \footnote{šú-tɔ̂(n)-pā: lit. ``those who \{beg/pray\} for grace.'' Epithet used} don't do it that way.

78. As for climbing up into a new house, before other people can climb into it,
he [the owner] himself has to go climb up and sleep in it first!

79. That's what other people say about us Lahu.

P: 80. Don't other people call it khyn hə́n mài?\footnote{This is the Shan expression that means ``climbing into a new house,'' equivalent}

H: 81. Well, that's Shan ---

T: 82. In the Shan language they say khyn hə́n mài.

83. We Lahu call it yɛ̀-ší tâʔ ve,\footnote{This is a literal equivalent of the Tai expression.} but before anybody else can climb
up, he himself [the owner] must climb up first, before \{anybody else/other people\}.

P: 84. Right

T: 85. Well, I've heard that your village,\footnote{Even though teacher and Headman really live in the same village (Huey Tat),} when [you] said ``let's go up there,''\footnote{To check out nâ-qhày. See sentence \#4.}
some people --- some wanted to go, but some didn't want to go!

86. I just don't know what we, the two of us, can do with people like that [who
don't want to go].

H: 87. Everybody will go.\footnote{qay tù ve dê yò: dê is here a rare non-reduplicated occurrence of dê-dê} They'll go \{slowly/in their own sweet time\}.

T: 88. If \{they take so much time about it/``it's slowly-slowly''\} \{in the future/later
on\} if somebody doesn't get a good place they'll \{scold/curse\} people!

H: 89. Well, as for that, the land is broad!

90. Land, now, the land you [could] have to live on, there's so much of it that
you could never run out of places to live!\footnote{mâ chɛ̂ pə̀: ``not finish living''.}\footnote{Unfortunately this has not been true for the past several decades, as overpopulation}

T: 91. I'm afraid that \footnote{à-mù (conj.) `lest; or else' (unpleasant hypothesis). See DL 100-101.} a --- if it's not a good place, \{he/a person\} wouldn't
want to live there.

H: 92. If he doesn't want to live there, \{let him just/just let him\} not go!

93. It's up to him!

T: 94. You're being unreasonable \footnote{chɔ khô â hêʔ, lit. ``it isn't human speech''.} about this!

H: 95. I'm right! [INAUDIBLE: ``he took yours, yours'']

T: 96. Well, the two of us can't do anything about it.

H: 97. So somebody who hasn't come yet---

98. So if anybody doesn't want to come,\footnote{The deictic viewpoint has now changed from going to the new village (qay)} wherever he wants to live in the future
is up to him!

T: 99. But if unfortunately [somebody] should be unhappy, even somebody who had
gone up there before other people, he might someday [want to] come back again!

100. It's tough for us Lahu people.

H: 101. Oh, if they manage to stay there and not come back, they can make a go
of it up there.

102 All of you guys, if you manage to go, you're called men, so you must come to
a decision!

103. You're not \{children/kids\}! Don't fool around!\footnote{qôʔ gɨ̂, `say for fun; talk thoughtlessly'.}

T: 104. Men, men so-called, flap their mouths a lot,\footnote{mɔ̀ʔ-qɔ vây ve: lit. ``fast in the mouth''.} you know.

H: 105. They won't go back. So we've got to talk things through to a conclusion.

106. Whatever we do, let's go and do it, okay?

107. \{But/otherwise\} don't keep talking thoughtlessly and arguing with each other!

108. There's no \{point/advantage\} in arguing.

109. I told you that if you wouldn't go before, don't go at all, but you wouldn't
listen.\footnote{CHECK. This sentence has an anomalous pî instead of lâ for 3-\texttt{>}2}

110. And now you saw you won't go.

111. Aren't you a man then?

T: 112. What I said was, if I \{would/were to\} go, every last person should go.

H: 113. They will go, every last one of them.

114. Even if they wouldn't go before, now they'll go.

T: 115. I tell you they won't go, that bunch of them.

H: 116. If they don't go, afterwards [those who have gone] will say ``Come!'',
that's what they'll say.

117. Yes, once you've gone and looked over the good land for cultivation,\footnote{g̈a câ kɨ̀ (Ndeclaus): ``place to get to eat'', i.e., `place to earn a} the
really fine [land], you won't want to leave [qay].

118. I myself, when I went there to look it over a while ago, I said to them [those
in Nâ-qhà], ``If they [from Huey Tat] don't come [qay], don't worry about it.''\footnote{mâ qay qo tâ ni qôʔ pî ve: here ni `look at' seems to mean `pay attention}

119. What will those who have seen it do?\footnote{ni ò ve ɔ̄ qhà-qhe te è: i.e., it's more important to consider what those}

120. \{As for me, I have/I myself have\} never been there, so I'll listen to what
you say and go.

121. Then if the time comes when there's nothing to eat, you'll know about it.

H: 122. If you don't do it you'll starve anyway.

Somebody else: 123. They've gone already, they've gone already! You deep thinkers\footnote{dɔ̂꞊nÁ-lɨ̂ʔ ve: lɨ̂ʔ (Vadj) `deep' \texttt{<} Tai; nÁ (Vadj) is}
have gone already!

T: 124. All of them are just following their hearts!\footnote{ni-ma-qa-pɨ tí tɛ̂ʔ ve: ``only measure their hearts''.}

[125. yɔ̂-hɨ a-cí cɔ̀ lɛ --- à-mù a-cí ... gà la ve mâ hêʔ] UNINTELLIGIBLE

Somebody else: 126. ... Can you guys \{live/make a living\} there?

H: 127. You mean growing crops?

128. The new rice, the \{first fruits/new crops\} are just coming out now.

129. If you want to have an increase in rice production.\footnote{nɔ̀-hɨ \{ɔ̀ʔ=ɔ̄\} mâ la gâ ve qo ɔ̄.}

T: 130. That's fine then, if that's the way it is.

131. What kind of fields\footnote{à-thòʔ-ma ca te câ: short for \textasciitilde{} ca te câ kɨ̀ ``what} do you say they have up there?

132. Do you say swiddens or paddy-fields?

133. They intend to \{do/cultivate\} paddy fields. The first year I went there
they were still living \{from/off\} swiddens in the high mountains.

134. While they were (still) cultivating swiddens they [also] worked paddy-fields.

P: 135. What kinds [of crops] will you plant to eat?

H: 136. After we prepare the paddy fields we'll plant rice; we'll cultivate swiddens
in the hills, and make \{gardens/plantations\} to grow several kinds of things,
and make shit-bean\footnote{nɔ̂ʔ-qhɛ̂ (also nɔ̂ʔ-kɨ̂ ``rotten beans''): kind of soybean fermented} fields.

137. We'll \{plant/grow\} \{maize/corn\}, bananas, and sugar-cane!

138. All kinds of pumpkins.

139. We'll earn plenty of money up there, I tell you.

T: 140. [INAUDIBLE] ... qhe te qay tù ve qôʔ-qo-pɔ̀ʔ.

141. The plains up there in the Nâ-qhày valley are excellent, they say.

142. Other people haven't worked then, so I'll just do it myself.

P: 143. Are you saying you claimed\footnote{vɛ (V) `lay claim to something, usually by leaving a visible symbol of ownership'.} [that land] first?

T: 144. I have claimed it already.\footnote{hɔ̂ lɛ̀ʔ ve: `sell in order to eat', i.e., ``sell to earn (money) to}

H: 145. But then you won't be able to cultivate all of it by yourself.

146. You should just cultivate enough for one person to manage \{easily/carefully\}.

T: 147. I want to earn money by selling\footnote{The Teacher is playing different roles in this conversation---before he was} [stuff] to others.

H: 148. So you'll sell for a living.

149. But you don't have the government's permission, I'll bet.

T: 150. Down there on the high ground above the river there won't be anybody who
could climb up onto the bank on the rocky side to grab it for himself.

151. So I've claimed it, down there.

H: 152. There's plenty \texttt{<}of land\texttt{>} around the rocky cliff, so you
go ahead and claim some!

153. Wherever you want to claim, claim it!

154. There's lots of it [land]. We can't use it all up. [INAUDIBLE WORDS]

155. The Akha\footnote{tɔ́-kɔ in notebook, but the correct form is tɔ̀-kɔ.} guy said, just you look at it, look at it ... [words missing]
it's good.

156. Plenty of game, plenty of monkeys, even if you want to eat fish, for God's
sake!\footnote{pòthôo: `by the Buddha'. The X'n Lahu have adopted this Tai expletive.}

157. When you go there, even though it's \{in/called\} the mountains [the terrain]:
it doesn't go up and down.

158. It's flat and smooth --- wherever you go it's very easy, to earn a living!\footnote{ca câ ve: lit. ``seek to eat, look for a living''.}[44a]

Somebody: 159. When I went there recently I was awful\footnote{šɨ-e-lɛ-yò `so much one could die'; verb intensifier, probably calqued} thirsty!

T: 160. Listen, listen, listen to him!

H: 161. You're going to be really hot down there, in the plains, in the hot season.
[sentence invented by Jim]

T: 162. ... It's not flat land. How are we ever going to \{make a go of/earn our
living from\} cultivating paddy-fields?

H: 163. Up there, over there ... in the riverbed they've channeled [the water]
as it falls from the high mountains so that when it reaches down below there's
so much water that you can't see it all.

164. The Nâ-qhày River is no small thing.\footnote{mâ ɨ̄ ve mâ hêʔ: ``is not the case that it is not big.''}

165. It has two valleys, Mɛ-thà-lây and Kɛ̀-pa-tâwʔ.\footnote{\texttt{<} Thai kɛɛ̀ŋ-pan-tÁw (kɛɛ̀ŋ `cataract').}[47a]

[166. Paddy-fields ... [partly inaudible] you can't even cross them when it rains.]

T: 167. Not long ago a bunch --- a bunch of us went there, we had been sent to
look it over, and Nâ-qhày was just floating knee-deep in sand!

H: 168. That's because the water has come flowing down, and it's [made the land]
nice and fertile.

169. So you can have paddy-fields.\footnote{Paddy-fields, being irrigated, require a handy water-source.}

170. In the hot season it's all sand, there's no water down there.

T: 171. With sand \{you can't plant anything/whatever you plant is no good\}.

H: 172. It's fine! That's just where to plant crops.

173. Other people are earning a living [there]. You just go and check it out.

P: 174. Isn't it far from a road for cars, your place?

H: 175. Cars can get all the way to it up there.

T: 176. So cars can get all the way up there, you say!

P: 177. Then that's great for you!

H: 178. That senior headman\footnote{kānān (\texttt{<} Thai kamnan): a senior headman whose jurisdiction is} up in Nâ-qhày, what was his name now? I've forgotten
it!

[Something inaudible said by somebody else]

H: 179. Yeah, there are a whole group of senior headmen there.

180. They have cars, big trucks,\footnote{lɔ̄lī (\texttt{<} Brit. Eng. lorry): `motor vehicle (of any kind)'. A} my God, lots of them!

T: 181. Their cars, they have two of them,\footnote{nî khɛ: the animal classifier khɛ is used to count motor vehicles.} I tell you!

P: 182. So we Lahu will probably get to ride in them, I bet.

H: 183. We'll get to ride, we'll get to ride! Up there [inaudible].

P: 184. Whereabouts would you go to market, if you wanted to go to a [Thai] marked?
To Chiangdao?\footnote{A town about 60 km. north of Chiang Mai, on the road to Farng.}

H: 185. \{Up there/To the north), way up there, you come to Farng.\footnote{A large town about 120 km. north of Chiang Mai.}

\{Simultaneous: T: 186. Oh, you'll get to come around to the area aroun Farng!

\{Simultaneous: P: 187. Oh, the Farng area, is it?

H: 188. Up there in the city there's a road which goes through to a place where
there's a big rice-mill, right up there!

189. As for a market, there's a big entrance to it in Farng.

T: 190. Oh, I see.

P: 191. So you come to the Farng market and go right in, is that right?

192. To buy things.

H: 193. Yeah, we go buy things, whatever, all kinds of stuff to go buy.

194. You can even buy \{an ox/a cow\} to raise.

195. It's very easy to get an oxcart to pull things for a living, up there!

P: 196. Wow, that's probably all right then!

H: 197. Mm-hm, you won't have to carry anything at all on your back to earn a living,
on the plains.\footnote{I.e., you can transport things by oxcart.}

198. Staying here I don't see how we can survive!

199. If we can just manage to settle up in that area.

T: 200. As soon as one thing has been done and come to pass, and has almost succeeded,
another thing comes up to keep on doing.

201. When one thing has come to pass and almost succeeded, another thing comes
up to keep on doing.\footnote{T. is repeating himself here.}

202. I'll follow your advice,\footnote{nɔ̀ qhâ, lit. ``your way''.} I'll really just keep following your advice!

203. Just this once [more] I'll be patient and try to follow your advice.

204. But if that place also turns out \{to be a dud/not to work\}, I'll never take
your advice again!

H: 205. We'll never leave again [if we move up there].

206. If we do manage to get up there, we'll certainly\footnote{pə̂-lɛ (written pə̂-ê in notebook). Glossed `even so, anyway, at least'} stay there forever and
die there of old age!

207. For our children's and grandchildren's generations.

208. We have planned to make it succeed!

209. We're \{not just toying\footnote{dɔ̂ gɨ̂: ``think for fun''.} with the idea!/really serious about it!\}

