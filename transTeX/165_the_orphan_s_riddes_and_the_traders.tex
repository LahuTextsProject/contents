\setcounter{footnote}{0}

or

Parlaying a Chick to an Elephant-Goad

Note: The Orphan hero of this story, as in others in this collection [give Refs]
is a sort of Trickster figure [give Refs]

(Recorded in Pashú Village)\footnote{Lahu village in the Chiengdao \textit{nikhom}, above \textbf{Hwè-tàʔ} (dissolved \textbf{ca}. 1969). See DL:803, GL p. xliv.}

1. \direct{clears throat}

Our great teacher\footnote{\textbf{šālā-ló}: polite term of reference for the author.} has come to visit our Pashu village, and he says he'd like
to hear a traditional Lahu story.\footnote{\textbf{à-pòn} qhāy \textbf{khɔ̂}: lit. ``ancestors' story-telling language''.}

2. So I'll just say one or two words.

***

3. Once upon a time there was an orphan child called Cà-šɨ́-thî.\footnote{\textbf{Cà-šɨ́-thî}: name given to a boy born at the dawn of the new moon (\textbf{šɨ́} `new', \textbf{thî} `dawn').}

4. Since he was very poor he had nothing to eat from day to day.

5. So when he was off scooping up frogs, he put dried things into starlings' nests.

6. Then he had a riddling contest\footnote{\textbf{tɔ̂} \textbf{fá-e} \textbf{dàʔ} \textbf{ve}: lit. ``use hidden words with each other.'' The meaning of this expression is somewhat different from that of English \textit{riddle}. A better translation might be \textit{deceptive question}.} with a great Chinese trader.\footnote{As will be mentioned explicitly later, the impoverished orphan had to live by his wits. The orphan and the trader apparently hit it off, since they continued on their travels together.}

7. ``What's that inside [the nest] over there?'', the orphan asked him.

8. The Chinese trader said, ``What should there be? There are only bird's eggs!,''
he said.

9. ``Wrong!'' he [the orphan] said. ``Inside there is a dried-up frog.''

10. So why don't you just climb up and get it, I'll watch you,'' he [the orphan]
said.

11. When he [the trader] went climbing up to fetch it and took a look at it, [the
orphan said] ``Do you see the dried-up frog?''\footnote{Evidently the trader did see it.}

12. So they kept going on and on, and when they reached a certain place over there
\footnote{\textbf{ô} \textbf{ɔ̄} or \textbf{ô} \textbf{kàʔ} `over there' is used throughout this story as a vague locative like `in a certain place'.}, the orphan looked around and saw another starling's nest.

13. ``That starling's nest over there, what's inside it?'' speaking in riddles
with him.

14. ``There are starling's eggs in it over there,'' he said.

15. ``Wrong!'' he said. ``In it over there, in that bird's nest over there, there's
a dried-up eel,'' he said.

16. But the Chinese trader didn't want to believe what this orphan, Cà-šɨ́-thî,
was saying.

17. Since he didn't believe it, Cà-šɨ́-thî climbed up to look and found a
dried-up eel.

18. So in this way the Chinese trader riddled with him, but nowhere\footnote{\textbf{qhà-nî} \textbf{kà} \textbf{kàʔ}: lit. ``even how many places.''} could he
beat Cà-šɨ́-thî.

19. So they kept going on and on, and when they got to a certain place, he [the
orphan] said, ``Here, inside that tree-hollow, what is there?''

20. ``Inside that tree-hollow there are bird's eggs,'' said the Chinese trader.

21. Cà-šɨ́-thî said, ``You're wrong! Inside that tree-hollow there is a super-dried-up
\footnote{\textbf{ɔ̀-gwê} \textbf{šwîʔ}: \textbf{ɔ̀-gwê} (\textasciitilde{} \textbf{ɔ̀-kwi}) `something dried out; \textbf{šwîʔ} (Nlim) `the utmost; highest degree'. GL 3.632, 6.114; DL: 1249-50.} pa-dâ-qō\footnote{Snakehead murrel: a kind of edible black fish about two inches long [\textit{Channa striatus}], called paa-kâa in N. Thai. DL:802.} fish,'' he said.

22. But the Chinese trader didn't believe there was a super-dried-up pa-dâ-qō
there.

23. ``Well then, if you don't believe it, why don't you just climb up and have
a look,'' he said.

24. ``Okay, okay, in that case I'll go try to climb up and get it,''\footnote{A four-verb concatenation: \textbf{ca} (vV) `go and', \textbf{tâʔ} (V\textsubscript{h}) `climb', \textbf{yù} (Vv) `V and take', \textbf{ni} (Vv) `try'.} he said.

25. When he went climbing up to try and get it, [the orphan said], ``Is there a
super-dried-up pa-dâ-qō up there?''

26. In this way, since Cà-šɨ́-thî was an orphan, he had to make a living by
his riddles.

27. So they kept going on and on, and in a certain place over there, [the orphan]
said, ``Well, what is there on top of that tree?''

28. When [the trader] looked over there he said, ``I don't see anything at all!''

29. ``You're wrong!'' he said. ``On top of that tree there's a little baby dove.''

30. So Cà-šɨ́-thî scrambled\footnote{\textbf{gâʔ} (lit. ``scratch'') is used as a Vv connoting lively action, here translated by `scrambled'.} up to look.

31. Since the Chinese trader didn't believe him, when he climbed up to look, he
got a little baby dove.

32. When he got the baby dove, the Chinese trader said, ``Oh, sell it to me! This
baby dove of yours.''

33. ``I can't do it,'' he said. ``Not me. It's mine to take care of.''

34. Then he said, ``Oh, please sell it, this thing of yours! So how much would
it cost?''

35. So the orphan said to the Chinese trader, ``How much will you give me? How much
will you give? How much would you be willing to pay?\footnote{\textbf{ni-qhâ} \textbf{šɨ} \textbf{ve}: (lit. ``heart-path is stable/ dead'') `be satisfied, be willing'.}, Cà-šɨ́-thî said.

36. ``Well, then, for this baby dove I'll give you five rupees\footnote{\textbf{ŋâ} \textbf{thɛ̀ʔ}: a \textbf{thɛ̀ʔ} is a Burmese rupee or \textit{kyat}.},'' he said.

37. So the Chinese trader gave him five rupees.

38. He [the orphan] went back, and when he got home he bought a chicken\footnote{Actually a baby chick, as we find out later in the sentence.}, and
he went off with that chicken he bought, and he released the baby

chick over there in a place where somebody was feeding a pig.

39. ``Don't do that! My pig will trample it to death!'' he [the pig's owner] said.

40. But he didn't listen. He set it loose.

41. So the pig chomped it to death over there.

42. When the pig bit it to death, he [the orphan] sobbed and sobbed, ``Give me
a pig, a piglet!'' he kept screaming.

43. He blubbered and blubbered so much that he [the pig's owner] couldn't stand
to hear it, so he gave him a piglet.

44. Then he went to let it [the piglet] loose where somebody's horse was being
fed.

45. ``Let mine eat something too!''

46. ``Don't do that! Otherwise my horse will kick it to death,'' he said.

47. But he didn't listen. He went and let it loose in there.

48. So the horse chomped it to death.

49. When the horse had chomped it to death, again he sobbed violently, there next
to where the horse was standing.

50. So that guy couldn't stand to hear it, and gave him a little foal.

51. When he [the horse's owner] had given him [the orphan] the foal, he went to
release it in a place over there where somebody's elephant was being fed.

52. Unfortunately the elephant chomped it to death.

53. So he sobbed and sobbed, and it was unbearable, so he was given a goad for
prodding elephants.

54. Then [the elephant owner] said, ``This goad of mine, if you poke a dead person
with it he'll come back to life,'' so he went over there to where there was a dead
person, and having been told that if he prodded him he'd come back to life, (the
orphan) poked him with all his might, but he didn't come back to life.

55. So then they took him and threw him out [from where the funeral was].

