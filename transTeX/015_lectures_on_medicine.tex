\setcounter{footnote}{0}


1. Well, these are the things the great medical doctor\footnote{A government doctor. The speaker received several weeks of paramedical training at a government center in Mae Taeng, about 40km south of Huey Tat. Every weekend he would return to the village and give a talk in church on Sunday morning, telling about what he had learned during the week.} [was saying].

2. And what the great doctor tells us we have to learn every day, down there.

3. Well, what the great doctor said was, that when we Lahu are sick and ailing
\footnote{\textbf{nà-ve-gɔ̀-ve} (Elabv) `be sick and wretched'.} we only come [for treatment] when it's too late.

4. And the government is very concerned, he says, whenever such a person is lost.

5. So, brothers and sisters, he told me to let you hear and let you know [what
you should do].

6. Well then, no matter what ails a person, whether he has a headache, or no matter
how he may be feeling unwell or sick\footnote{\textbf{chɛ̂-hā-cɔ̀-hā} (Elabv) `be unwell and having a hard time'.}, he ought to get to [the doctor's] quickly
and in good time.

7. Come in good time, he says!

8. Down there the government\footnote{The speaker uses both the Thai- and Burmese-derived words for `government' together, in a sort of compound.} has set up a hospital.

9. So now nobody has to pay any money [for medical treatment].\footnote{This seems like an overly optimistic description of the real situation in the late 1960's.}

10. Don't keep on suffering all by yourselves, he says.

11. Brothers, if you arrive down there after suffering and suffering for a long
time, when you're practically dead, even if they try to restore your strength you
can only survive for a little while, he says.

12. You have no strength.

13. You're just too weak.

14. When that time comes, even if he [tries to] give you back your strength, your
strength is no longer enough.

15. So when somebody is getting to feel slightly unwell, whoever it is who's sick,
a group [of us] should talk it over a little, and we should all cheerfully take
him and put him into [the hospital] in good time--even if he can't walk.\footnote{It would actually have been quite an undertaking to carry a sick person down the mountain path to the highway, a trip that took a healthy person three hours walking under his own steam.}

16. Also, when it happens that a woman is to give birth to a child, you know when
the time for the birth has arrived, when the month has arrived.

17. If you [women] all stay up here in the mountains [for your deliveries], if
you can manage to give birth all by yourselves, you do.

18. But if you can't give birth you just die, he says.

19. He says they have seen several cases like that.

20. For this reason, from this year on, when the time comes for a woman to give
birth, if you watch her and realize that she has a hard time giving birth, let
her come down while there is still time to the hospital in the valley, he says.

21. Then when she gives birth, when she has finished giving birth -- er -- he says
you're not supposed to sleep with the mother.\footnote{The speaker's hesitation betrays his slight embarrassment at having to refer to sexual matters.}

22. For four days you shouldn't put the baby with the mother.

23. He says that they observe this rule.

24. For four days it shouldn't stay with the mother.

25. That way the mother regains her full strength and energy.\footnote{\textbf{ɔ̀-g̈â-ɔ̀-šā} (Elabn): lit. ``strength and flesh''.}

26. She doesn't lose her strength and energy.

27. Then she will always be full of strength and vigor, and her physical condition
\footnote{\textbf{ɔ̀-šɨ̄-ɔ̀-šā} (Elabn) : lit. ``blood and flesh''.} and strength will not deteriorate.

28. That way her strength remains great, he says.

29. When she arrives down there and has had her baby, they take care of the child
separately\footnote{\textbf{yâ-qhâ-yâ}: ``the child all by itself''. For a description of the N1 + \textbf{qhâ} + N1 construction, see GL:79-80.} at the same time [she is regaining her strength].

30. For four days they feed it cow's milk.

31. They take care of it.

32. So this woman, the mother, once she has already borne the child, then she can
sleep in the same room

with it.

33. After they've been in the same room -- er -- when all four days have passed,
if she says she wants to see her baby, she may have it put down with her.

34. But, it's fine to have them take care of it for you\footnote{The change of person from 3rd to 2nd is signaled by the verb-particle \textbf{lâ}, which cannot be used for 3rd to 3rd person benefaction. See GL:324-330.} right up until the
time you're led back home.

35. Seeing that you don't have to spend a red cent [for it], he says you might
as well\footnote{V + \textbf{phɔ̂} \textbf{dàʔ} \textbf{ve}: lit., ``the V side is good,'' i.e., `the alternative of V'ing is better.' This is similar to the Japanese construction V hoo \textbf{ga} ii.} let them take care of it.

36. But, about the matter of being allowed to go home, down there they have this
rule for everybody who enters [the hospital], the doctor says.

37. They don't let you be sent home for six or seven days..

38. That's a rule that they follow, he says.

39. After six or seven days you can be sent back.

40. So, everybody, there is no need to suffer!

41. Instead of continuing to die for nothing in the same old way,\footnote{\textbf{chi} \textbf{qhe} \textbf{chi} \textbf{qhe}: lit. ``like this like this'', i.e. `in the same old way'.} come [to
the hospital] early and in good time, and be cured, he says.

42. Down there the government has arranged it for you.

43. The other day, from over there on the mountain they call Doi Suthep\footnote{Doi Suthep (Thai Dɔɔj Sùʔthêep) A wooded mountain near Chiang Mai with a famous Buddhist temple on its summit. There is a Hmong village (Hwè-tìʔ) on a nearby slope.}, there
was a Hmong woman I think he said it was\footnote{Khè-mèoʔ \textbf{qôʔ} \textbf{nā} \textbf{â} \textbf{šī} \textbf{ò}: lit. ``whether he said it was a Hmong I don't know now."} who came to give birth to a child,
because there on the mountain she couldn't give birth.\footnote{Lit: ``a Hmong [woman] came and as for her coming to give birth to a child, [it was because] over there on the mountain she couldn't give birth".}

44. She had been in labor\footnote{Lit: ``her giving birth had reached four days.''} for four days, he said.

45. She was in labor, [but] the child was blocked.\footnote{\textbf{yâ} \textbf{khá} \textbf{ve}: lit. ``the child is blocked''. This expression is used to refer to most mishaps of pregnancy, including miscarriage.}

46. But she didn't come [to the hospital] quickly.

47. Since four days had passed the mother was already very weak, he said.\footnote{The quotative final unrestricted particle \textbf{cê} must now be translated `he said', since the speaker is recounting a story that took place in the past. Earlier in the text, when the speaker was conveying the doctor's prescriptive remarks that were meant to have present relevance, the tag `he says' was preferable.}

48. When she arrived down there she could no longer speak.

49. She had very little strength left.

50. Then they tried giving her stimulants\footnote{\textbf{ɔ̀-g̈â} \textbf{nâʔ-chî}: lit. ``strength-medicines.''} to restore her strength , but although
they did, it was of no use.\footnote{V + \textbf{dê} \textbf{yò}: `it is futile to V.'}

51. She had very little strength, it was almost all gone, so when she arrived down
there there was nothing to be done but to take out the child [by Caesarean section].

52. The mother had very little strength left.

53. She had very little strength left.

54. Well, when they took out the child, the mother died on the spot, he said.

55. Because it was too late.

56. Don't let this happen to you,\footnote{Lit. ``may you not do like this''.} he says!

57. ``Let all your brothers hear this!'' he told me.

58. This great doctor is an adviser to an important committee\footnote{Presumably the `committee' had jurisdiction `over' Chiengmai.} in a government
bureau [with jurisdiction] over Chiang Mai.

59. So that doctor down there says that everybody should come down to be treated.

