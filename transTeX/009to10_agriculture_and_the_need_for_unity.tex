
1: So, this time we'll have a discussion about how we Lahu \{grow our crops / practice
agriculture / farm our land\}. Well, whereabouts will all of you in the village
grow your crops this year?

2: Well, I['ll be] down below. Where I farmed last year. Probably I'll try elephant's-Death-Valley.[1]

3: Mm, \textit{I['ll} be] over \textit{there} [gesture]. I'll work around Thai-lai.[2]
I won't stay--er, I won't come down below. The elephants eat [the crops] up all
the time [down there].

4. Ha, ha!

5: Oh, my brothers! When human beings[3] are few and not numerous,[4] [they should]
seek their livelihood together. Don't each work in a different place. If someday
[one of you] should get sick or fall on bad times,[5] [there would be] nothing
he could [do]. We mustn't have our heads full of stones, you know.[6]

6: Well, as far as I'm concerned, I'm going to \{farm / work\} over \textit{there}.
Down below I will not go. I'll work all by myself.

7: Ha, ha!

8: Well, \{if that's the way you feel (or it is) / in that case\}, you really ought
to pay attention to the words of our forefathers: \texttt{"}When we cultivate the
soil, if we work all together,[7] even if sickness or bad times should befall,
to look after one another, to help one another, to do for one another is easy!\texttt{"}


9: Can't \textit{you} work that way, too? Must everybody work all by himself?

10: Ah, me! Ah, me!

11: It's absolutely right, that saying! We ought to work together.

12: When you get sick and weak, you can't go lifting and carrying if you're all
alone.

13: Well, whatever happens, even if you say \texttt{"}I won't listen!\texttt{"}
to my words of advice as we talk to each other, even if you people each work the
land by yourself, as you have been intending to--wherever you work, [at least]
work well! Cultivate the soil properly, and try to gain your food and drink, so
that you will not be inferior to others.[8]

14: For my part, I['m going to] clear my land.[9] Even if it's all by myself, even
if I'm the only one, I'll clear my land \textit{over there}. I'll be [there] clearing
it off, from now on.

15: But we can't act this way, brothers! Now, at this moment, [you know] we are
few and not numerous[10]! We ought to earn our living together, happily and cheerfully,[11]
and filled with love.[12] People like us, we Lahu, you know that there aren't masses
and hordes[13] of us! There are only a few of us! If someday somebody gets sick,
[it shall not be that] no one sees it; if someone is about to die, it should not
be that no one sees it. There aren't too many of us, and that's a fact!

16: As for me, I'm going to hack out an oldfield.[14]

17: You're hacking out an oldfield, are you?

18: Yes.

19: Well, even from an oldfield one can hack out a living.

20: It's just that it's kind of overgrown with weeds.

21: [Yeah], the weeds \textit{are} thick [in an oldfield].

\begin{center}
* * *
\end{center}

\leftskip=0pt
22: Well, this year--has everybody in our village finished clearing the trees from
his land this year?

23: Oh no, I haven't finished [my] clearing yet. I haven't finished, but there's
nothing I can do about it. Some Thais set my land the hill on fire for me![15]
I don't know now if I'll be able to do my second-burning,[16] either.

24: Yeah, mine hasn't burned over thoroughly, either! I guess I'll still have to
re-burn for two or three days now.

25: Ah, if only you had listened that day to the words your elders spoke, if only
you had been working together today in one group, as a unit, they wouldn't have
set fire [to your land]. If you had looked after it, if you had taken care of it....[17]
But today, since you paid no attention to your elders' words, and each acted according
to his own way of thinking, they set fire to you and here you are unable to do
your second-burning.

26: There's nothing to be done. You didn't listen that time, you paid no attention
to what was said, and now this is what happened! All you an do [at this point]
is re-burn as much as possible[18] and [try to] \{cultivate that much / live off
that much\}. This year your fields aren't burned off, and who knows whether you'll
get any rice to eat or not.

27: Ah, my brothers, all of you! When [we live] in loving harmony--even if  someone
once disregarded advice--we must help one another! We mustn't squabble with each
other like this. By helping each other in the proper way--er, by doing this--we
will earn our food and drink and will not be inferior to others. And if we can
[also] clothe and garb[19] ourselves like the others, the Lahu community[20] will
prosper.[21] On the other hand,[22] if we don't act [properly] this way, when one
person gets in trouble, others will suffer for it.[23]

\begin{center}
* * *
\end{center}

\leftskip=0pt
Well, the fields--er, has everybody finished reburning his field by now? If you've
finished, say so now![24]

28: Oh, it's all finished--mine, that is.

29: Well, if you're really finished, it's time now to plant the rice! Have you
all tried planting [your] rice yet?

30: Well, I've only managed to begin trying to plant a little so far. This week
I'll get to begin trying to do more and more.[25]

31: Well then, Jali, how many acres[26] have you finished [planting] already?

32: Oh, I've still only finished one acre. I've only managed to plant one acre
so far....[27]

33: One acre, eh.

34: \{Your plans now-- / What you've been planning--\} how many acres have you
been thinking of planting?

35: According to my plans, I've been thinking of planting five acres.[28]

36: Five acres.

37: Hm-hm.

38: Yeah, if you do that you'll certainly earn your food and drink and won't be
\{inferior to others / less well-off than anyone else\}.

39: Well, Kheh-ki, how many acres will \emph{you} plant?

40: Well, I've been thinking of planting five or six acres, too. I'll simply try
to do as much as I can. I'm just beginning to work on it now--and the time still
isn't too late....[29]

41: Right. Go to it! Well, then, Sexton[30], how many acres will \emph{you} plant?

42: Oh, I think I'll plant \textit{ten} acres!

43: Wow!

44: But at this point, all I've gotten done is to make a field-hut.[31]

45: Well, do your best! As our forefathers used to say:

Slow to till and toil--rats!

Slow to drive away--rats!

Slaves and thralls of others!

If you're slow to till and toil:

Rat food, bird food.[32]

Everybody ought to hurry up and reburn his land properly, and cultivate it.

46: I haven't even finished my reburning yet!

47: Tomorrow--for sure[33], I'm planning to go way off over there, to the place
where they say some people smeared hot peppers on a [man's] prick long ago,[34]
to scoop out a honeycomb to lick.

48: Oh, I'm going to! As far as honey-scooping goes, I've seen a [bees']-hollow
\{too / myself\}. Over there above the river-bed on Porcupine Mountain.[35] Say,
let's go together, shall we?

49: Yeah, we can certainly go together.

50: Early tomorrow morning when it's time to go you just call me, too, okay? The
two of us will go off together at the crack of dawn.[36] I've seen a [bees']-hollow
myself.

51: Okay, [but] don't do to me what somebody else did that time.[37]

52: If I [say I'll] go, I'll go. Don't talk nonsense to me.

53: Let's go together, happily and with good cheers!

54: Because we two have never yet broken a promise. Whenever anybody discusses
something and decides on it, we just have a go at it.[38] We'll go early tomorrow
morning.

55: Right, we've never broken [a promise] in the past.

56: Say, let's take the dogs along, too! Pastor! Kheh-ki! Jali! We'll go shoot
ourselves some macaques[39] and little gibbons.[40] We'll go bag ourselves all
kinds of porcupines[41]!

57: Well, then, Jali, you can bring your dog along!

58: Yeah, sure I can take mine along. [But] he can't hunt monkeys, you know!

59: I bet he can hunt barking-deer, though. Even if he can't hunt monkeys.

60: Sure he can hunt, sure he can hunt barking-deer.

61: So, that's how we'll go then!

62: Well, as far as deer are concerned--er, I mean, if you just shout and make
noise at the hunting-grounds, even if the dogs \{don't know how to / can't\} give
chase, those monkeys will come out [all by themselves]. But as far as barking deer
are concerned, if the dogs don't chase them, they won't come out. At the hunting-grounds
we'll chase them and beat them and get them, we'll chase them and shoot them and
get them! We'll sneak up on them and shoot them.

63: This is the time when we've already gone and secured places to grow our crops,
and almost managed to finish planting our rice and so let's just keep on going
hunting [for game] to eat, and earn our living in joy and gladness.

\begin{center}
* * *
\end{center}

\leftskip=0pt
64: Well, Sexton, they say some wild boars have eaten your paddy, is that right?

65: Yes.

[TAPE II, Side 1]

66: Well, since wild boars have eaten rice from the Sexton's field,  tomorrow we've
got to help hunt them down and beat them [dead] for him, right?

67: Yeah, you've got to help me hunt them down and beat them [dead]. The rice itself
is very good [this year in my field].

68: Otherwise you won't have anything to eat. If \textit{they} eat it all up.

69: Whether we catch them or not, let's go drive them away for him. If we drive
them away, they won't come back for a while, at least. We'll just go and drive
them away for him. Whether or not we catch them we'll go give them a chase. Tomorrow
we'll go....

70: Right....

\begin{center}
* * *
\end{center}

\leftskip=0pt
Well, now the time has come for harvesting our rice!

71: Yep, rice-harvest time has come. Tomorrow I plan to go start trying to reap.
Since the wild pigs are eating away at it.

72: Kheh-ki, have you managed to try harvesting yet?

73: Yes, I plan to try to harvest. My field has just begun to ripen a little, too,
now. Tomorrow I'll go reap,[42] I guess.

74: Jali, what about you?

75: Well, mine is being harvested.[42] Even today.

76: I see. Well, how many bushels[43] have you gotten this year--I mean \emph{you}?[44]

77: Five bushels.

78: Five bushels, eh.

79: Uh-huh.

80: Five bushels probably isn't enough to live on.

81: It doesn't even come to five bushels, unfortunately.[45] In my field this year
the paddy is no good, since they set the damn thing on fire for me, so I'll really
only get \textit{three} bushels, Pastor. I have no idea how I'll ever be able to
feed myself now.

82: Ah, you can't think your way out [of a situation like this]![46]

83: How about you, Jali?

84: Well, I've got ten bushels, ten bushels for me.

85: Hm, you're a little better off, then.[47] Shaw-lu [The Kalin], how many bushels
did you get?

86: At the moment, I've only got five bushels. But I haven't finished my reaping
yet.

87: I see.

88: For my part, I've only got five bushels. Because the wild pigs were eating
it, too. Half of it.

89: Hm... well....

90: And that's why we kept saying before \texttt{"}Work\textit{ together}, work
\textit{together}! If everybody works all by himself, the rats eat it, the birds
eat it, the wild pigs eat it!\texttt{"} It's because you don't listen to what people
tell you, you guys!

91: There's an old saying, \texttt{"}If you would eat, smell.\texttt{"}[48] That
is, if you're going to eat something, try smelling it first. Oh, my children![49]
Cultivating the land is also like that. If you plan to do it, you've got to think
about it [first]. Furthermore, if you want to \textit{speak}, you have to think
about it first, I tell you. Even if we just intend to speak, we must carefully
think over [what we want to say]. When we don't listen to people's advice sometimes,
and disobey, that's the reason why we have to suffer for it this way....

\begin{center}
* * *
\end{center}

\leftskip=0pt
Well, have you tried threshing your paddy already this year?

92: Yeah, I've threshed it out[50] already.

93: How much grain did you get?

94: I got ten baskets[51] per bushel [of paddy].

95: If you really get ten baskets per bushel from your field, that's not bad at
all. Since you've gotten five bushels. From mine I only[52] get six baskets or
[sometimes] as much as seven. Sometimes I only get two basketsful, or one and a
half!

96: Jali, what about yours?

97: Well, mine now, sometimes I get ten basketsful and sometimes fifteen.

98: I see. Well, [now] I'll teach you children one \{point / thing\}! Let everybody
carry back his grain and \{put / store\} it inside his house! In this country the
Thais are very mean, and they're apt to set it on fire. If the Thais burn it up,
we'll have nothing to eat.

99: No matter how much of it there is, even if I have to spend all day tomorrow
carrying mine, I'll probably get [the job] finished. I'll carry it home and store
it, all of it. It's [a] hard [thing] to go and do. However, since these Thais are
such irrational people\.[53]...

100: Will everybody carry it back to store?

101: They'll carry it back to store [if they know what's good for them]. What [this
fellow][54] carried back came to ten basketsful. But all together [his crop] had
come to fifteen baskets. So that there were still five basketsful [lying around
in the fields]. And since you people don't listen to your elders' advice, some
elephants came and trampled [the rest]!

102: Ah, me! That's why we said on that day long ago: \texttt{"}When people instruct
you, listen to what they say! Listen to what they say! When you cultivate the land,
work where the others are working,\texttt{"}[55] we said. But today even the elephants
have again trampled [on somebody's crop]! That's no way to earn a living! When
people instruct you, you don't heed their words.

\begin{center}
* * *
\end{center}

\leftskip=0pt
103: Has everybody finished carrying his grain back to store in his own hearth
and home[56]?

104: Well, I haven't managed to store mine yet. This weekend I plan to have some
Thais carry the rice [to my home] [on their shoulders].[57]

105: Yes, do [have them] carry it all up and store it away! Immediately.

106: Well, I'm finished already.. Because mine didn't amount to very much. [But]
I've thought it over, and next year, after we've celebrated New Year's, when it's
time to work again, from then on I'll work together [with the others].

107: Okay.

