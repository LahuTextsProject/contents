\setcounter{footnote}{0}

1. So, this time we'll have a discussion about how we Lahu practice agriculture.
Well, whereabouts will all of you in the village grow your crops this year?

2. Well, I'll be down below. Where I farmed last year. Probably I'll try Elephant's-Death-Valley.\footnote{\textbf{Hwei-cà-tái} <No. Thai; lit. ``valley-elephants-die''.}

3. Mm, I'll be over \textit{there} [gesture] . I'll work around Thai-lai.\footnote{A rocky area near the village, also called by the Thai \textit{phənǎŋ-dɛɛŋ} `Red Wall.' \textbf{ô} \textbf{ɔ} `over there' (i.e., `off to the side'), as distinct from \textbf{mô} \textbf{ɔ} `down below.'}
I won't stay--er, I won't come down below. The elephants eat [the crops] up all
the time down there.

4. Ha, ha!

5. Oh, my brothers! When human beings\footnote{'Human beings' rather than `people' is intended to convey the sense of \textbf{chɔ} \textbf{qôʔ} \textbf{ve} \textbf{lɛ̀} ``what we call people'' (cf. Japanese \textit{ningen to yuu no wa}).} are few and not numerous,\footnote{\textbf{mâ-mâ-mâ-g̈ɨ̀} (Elab-noun) lit. ``not many not numerous.''} they should
seek their livelihood together. Don't each work in a different place. If someday
one of you should get sick or fall on bad times,\footnote{\textbf{nà-la-gɔ̀-la} (Elab-verb) ``become sick become wretched.''} there would be nothing he
could do . We mustn't have our heads full of stones, you know.\footnote{Proverb: \textbf{ó-qo} \textbf{há-pɨ} \textbf{kə} \textbf{a} \textbf{ve} \textbf{mâ} \textbf{hêʔ} ``One shouldn't act stupidly.'' The `you know' renders the non-final unrestricted particle \textbf{lɛ} `since; because.'}

6. Well, as far as I'm concerned, I'm going to work over \textit{there}. Down below
I will not go. I'll work all by myself.

7. Ha, ha!

8. Well, in that case you really ought to pay attention to the words of our forefathers:
''When we cultivate the soil, if we work all together,\footnote{\textbf{tê} \textbf{pɔ̂ʔ} \textbf{tí} `all together': lit. ``at the same time.''} even if sickness
or bad times should befall, to look after one another, to help one another, to
do for one another is easy!"

9. Can't \textit{you} work that way, too? Must everybody work all by himself?

10. Ah, me! Ah, me!

11. It's absolutely right, that saying! We ought to work together.

12. When you get sick and weak, you can't go lifting and carrying if you're all
alone.

13. Well, whatever happens, even if you say ``I won't listen!''
to my words of advice as we talk to each other, even if you people each work the
land by yourself, as you have been intending to--wherever you work, at least work
well! Cultivate the soil properly, and try to gain your food and drink, so that
you will not be inferior to others.\footnote{\textbf{šu} \textbf{lɔ} \textbf{šu} \textbf{tù} \textbf{te} \textbf{ve}: ``do so as to be in the same situation as others.'' This phrase is often used to express the ambition of the Lahu as a people to become as well off as other groups. Another phrase for the same concept is \textbf{ɨ̄-la-mu-la} \textbf{tù} \textbf{te} \textbf{ve}, ``to do so as to become big and high''--i.e., to make progress in the world's eyes.}

14. For my part, I'm going to clear my land.\footnote{\textbf{hɛ} \textbf{thu} \textbf{ve} `to clear away the trees from a new field-site': one of the first steps in the Lahu agricultural cycle.} Even if it's all by myself, even
if I'm the only one, I'll clear my land \textit{over there}. I'll be [there] clearing
it off, from now on.

15. But we can't act this way, brothers! As we have said, you know we are few and
not numerous\footnote{See fn. 6. The `you know' is intended to connivery the sense of the the final unrestricted particle \textbf{ve-ɔ}.} ! We ought to earn our living together, happily and cheerfully,\footnote{\textbf{ha-lɛ̀-ha-qa} (Elab-adv) `in joy and gladness'.}
and filled with love.\footnote{\textbf{hàʔ-hàʔ-pɛ̂n-pɛ̂n} `lovingly': a reduplication of the verb \textbf{hàʔ-pɛ̂n}.} People like us, we Lahu, you know that there aren't
masses and hordes\footnote{\textbf{ɔ̀-mo} \textbf{ɔ̀-cú} (Elab-noun) `a large group'.} of us! There are only a few of us! If someday somebody gets
sick, it should not be that no one sees it; if someone is about to die, it should
not be that no one sees it. There aren't too many of us, and that's a fact!

16. As for me, I'm going to hack out an oldfield.\footnote{\textbf{hɛ-g̈ɔ̂} \textbf{phɔ̂} \textbf{câ} \textbf{ve}: \textbf{hɛ-g̈ɔ̂} `field-bones' - a field that has already yielded a crop the year before. \textbf{phɔ̂} `to hack clear the undergrowth with a long sword.' The heavier work, \textbf{hɛ} \textbf{thu} \textbf{ve} (fn.9), is of course not necessary in the case of an `oldfield.'}

17. You're hacking out an oldfield, are you?

18. Yes.

19. Well, even from an oldfield one can hack out a living.

20. It's just that it's kind of overgrown with weeds.

21. Yeah, the weeds \textit{are} thick [in an oldfield] .

\begin{center}
* * *
\end{center}

22. Well, this year--has everybody in our village finished clearing the trees from
his land this year?

23. Oh no, I haven't finished clearing yet. I haven't finished, but there's nothing
I can do about it. Some Thais set my land the hell on fire for me!\footnote{As a malicious act or through carelessness. This greatly complicated the timing of the various slash-and-burn operations.} I don't
know now if I'll even be able to do my second-burning,\footnote{\textbf{hɛ} \textbf{ji} \textbf{ve}. The reburying of the field to consume the material left over from the first burning (\textbf{hɛ} \textbf{tú} \textbf{ve}).} either.

24. Yeah, mine hasn't burned over thoroughly, either! I guess I'll still have to
re-burn for two or three days now.

25. Ah, if only you had listened that day to the words your elders spoke, if only
you had been working together today in one group, as a unit, they wouldn't have
set fire [to your land] . If you had looked after it, if you had taken care of
it...\footnote{The second element \textbf{lō} in the compounds \textbf{ni-lō} and \textbf{hàʔ-šá-lō} are apparently derived from Burmese \textbf{lou} `be necessary; need sthg'; see to the needs of'.} But today, since you paid no attention to your elders' words, and each
acted according to his own way of thinking, they set fire to you and here you are
unable to do your second-burning.

26. There's nothing to be done. You didn't listen that time, you paid no attention
to what was said, and now this is what happened! All you an do [at this point]
is re-burn as much as possible\footnote{\textbf{qhà-qhe} \textbf{cɛ} \textbf{g̈a} \textbf{chi} \textbf{qhe} \textbf{cɛ} V: `to V as much as possible.'} and try to live off that much. This year your
fields aren't properly burned off, and who knows whether you'll get any rice to
eat or not.

27. Ah, my brothers, all of you! When we live in loving harmony--even if someone
once disregarded advice--we must help one another! We mustn't squabble with each
other like this. By helping each other in the proper way--er, by doing this--we
will earn our food and drink and will not be inferior to others. And if we can
also clothe and garb\footnote{\textbf{g̈a-və̀ʔ-g̈a-dɛ} (Elab-verb) `have clothes to wear'.} ourselves like the others, the Lahu community\footnote{\textbf{Lâhū} \textbf{tê} \textbf{yɛ̀} ``the Lahu house.''} will
prosper.\footnote{\textbf{pɔ-ša} \textbf{la} `come to be well-off (lit. ``become easy-born'').'} Otherwise\footnote{\textbf{mâ} \textbf{hêʔ} \textbf{qo} `otherwise' (lit. ``if it is not so'').} if we don't act [properly] this way, when one person
gets in trouble, others will suffer for it.\footnote{Lit. `when one person is in trouble, it is the same as another person being in trouble.'} Well, has everybody finished reburning
his field by now? If you've finished, say so now!\footnote{Lit. `if [you've] finished, say ``finished''.'}

28. Oh, it's all finished--mine, that is.

29. Well, if you're really finished, it's time now to plant the rice! Have you
all tried planting your rice yet?

30. Well, I've only managed to begin trying to plant a little so far. This week
I'll get to begin trying to do more and more.\footnote{Two splendid multiversatile concatenations in successive sentences! \textbf{g̈a} \textbf{tàn} \textbf{ti} \textbf{a-ni} `manage to begin trying to plant'; \textbf{g̈a} \textbf{tàn} \textbf{te} \textbf{qay} \textbf{a-ni} `get to begin trying to do more and more.'}

31. Well then, Jali, how many acres\footnote{\textbf{tû} `the area one can sow with the seeds from one basket.'} have you finished [planting] already?

32. Oh, I've still only finished one acre. I've only managed to plant one acre
so far...\footnote{The suspension marks `...' are meant to translate the suspensive non-final unrestricted particle \textbf{lɛ}.}

33. One acre, eh?

34. Your plans now--how many acres have you been thinking of planting?

35. According to my plans, I've been thinking of planting five acres.\footnote{Lahu quotative constructions are often both introduced and ended by the same citation marker: cf. \textbf{yɔ̂} \textbf{qôʔ} \textbf{ve} ``X'' \textbf{chi} \textbf{qhe} \textbf{qôʔ} \textbf{ve} \textbf{yò} Lit., ``what he said was X, he said.'' A less literal and more literate translation of this sentence would be simply ``what I've been planning is to plant five acres.''}

36. Five acres.

37. Hm-hm.

38. Yeah, if you do that you'll certainly earn your food and drink and won't be
less well-off than anyone else.

39. Well, Kheh-ki, how many acres will \textit{you} plant?

40. Well, I've been thinking of planting five or six acres, too. I'll simply try
to do as much as I can. I'm just beginning to work on it now--and the time still
isn't too late...\footnote{Here the suspension marks translate the causal non-final unrestricted particle \textbf{lɛ}, homophonous with the ordinary suspensive particle. See note 27.}

41. Right. Go to it! Well, then, Sexton\footnote{\textbf{Pù-cɔ̂} < Shan ``church-person.'' This person had the job of looking after the physical appearance of the hut used as a church by the village.} , how many acres will \textit{you}
plant?

42. Oh, I think I'll plant \textit{ten} acres!

43. Wow!

44. But at this point, all I've gotten done is to make a field-hut.\footnote{A hut erected in the fields. During periods of intense work in the fields, the cultivator cooks, eats, and sleeps in his field-hut for days on end without returning to the village.}

45. Well, do your best! As our forefathers used to say:

Slow to till and toil--rats!

Slow to drive away--rats!

Slaves, slaves and thralls of others!

If you're slow to till and toil

Then it's rat food, bird food.\footnote{A good example of an old poetic form: five lines, the first four with 6 syllables and the last with 4. A prose paraphrase: ``If you're slow in tilling your land, the rodents (rats, porcupines, squirrels, etc.) will ruin it for you; if you're slow about driving the animals off your land, the rodents will ruin you. You will become the slaves of others. If you're slow in your tilling, your land (and you) will become food for the rats and the birds.''}

Everybody ought to hurry up and reburn his land properly, and cultivate it.

46. I haven't even finished my reburning yet!

47. Tomorrow--for sure\footnote{The `for sure' renders \textbf{qo-qɔ̀ʔ}, more emphatic than plain \textbf{qo}.} , I'm planning to go way off over there, to the place
where they say some people smeared hot peppers on a guy's prick long ago,\footnote{The incident in question involved a group of mischievous boys who played this trick on a man as he lay sleeping in the shade.}
so I can scoop out a honeycomb to eat.

48. Oh, I'm going too! As far as honey-scooping goes, I've seen a [bees'] -hollow
too. Over there above the river-bed on Porcupine Mountain.\footnote{\textbf{fâʔ-pɛ́} is the `brush-tailed porcupine,' (\textit{Atherurus macrourus}) a large species that lives among the rocks.} Say, let's go together,
shall we?

49. Yeah, we can certainly go together.

50. Early tomorrow morning when it's time to go you just call me, too, okay? The
two of us will go off together at the crack of dawn.\footnote{\textbf{tê} \textbf{nàʔ} \textbf{šá} \textbf{ɛ̀} \textbf{te} \textbf{lɛ}: \textbf{tê} \textbf{nàʔ} `early morning' + \textbf{šá} \textbf{ɛ̀} `intensitive.'} I've seen a [bees'] -hollow
myself.

51. Okay, but don't do to me what somebody else did that time.\footnote{That is, on a previous occasion somebody had made a similar arrangement with him, but ended up by not keeping the appointment.}

52. If I [say I'll] go, I'll go. Don't talk nonsense to me.

53. Let's go together, happily and with good cheer!

54. Because we two have never yet broken a promise. Whenever anybody discusses
something and decides on it, we just have a go at it.\footnote{\textbf{qay} \textbf{ni} \textbf{ve} `have a try at going; go and see what happens.' Lit: ``To the extent that anybody discusses and decides, we just go and see.''} We'll go early tomorrow
morning.

55. Right, we've never broken [a promise] in the past.

56. Say, let's take the dogs along, too! Pastor! Kheh-ki! Jali! We'll go shoot
ourselves some macaques\footnote{\textbf{mɔ̀ʔ-ní-qwē} `Rhesus macaque' (\textit{Macaca rhesus rhesus}).} and little gibbons.\footnote{\textbf{mɔ̀ʔ-nâʔ} `white-headed gibbon' (\textit{Hylobates lar}).} We'll go hunt ourselves all
kinds of porcupines and spend the night out\footnote{\textbf{fâʔ-pu-fâʔ-pɛ́} (Elab-noun) `all kinds of porcupines' \textbf{fâʔ-pu} `Asiatic porcupine (\textit{Hystrix brachyura}).' For \textbf{fâʔ-pɛ́}, see fn. 35.} !

57. Well, then, Jali, you can probably bring your dog along!

58. Yeah, sure I can take mine along. [But] he can't hunt monkeys, you know!

59. I bet he can hunt barking-deer, though. Even if he can't hunt monkeys.

60. Sure he can hunt, sure he can hunt barking-deer.

61. So, that's how we'll go then!

62. If you just shout and make noise at the hunting-grounds, even if the dogs don't
know how to give chase, those monkeys will come out [all by themselves] . But as
far as barking deer are concerned, if the dogs don't chase them, they won't come
out. We'll beat the bushes and drive them, we'll chase them and shoot them and
get them! We'll sneak up on them and shoot them!

63. This is the time when we've already gone and secured places to grow our crops,
and almost managed to finish planting our rice and so let's just keep on going
hunting [for game] to eat, and earn our living in joy and gladness.

\begin{center}
* * *
\end{center}

64. Well, Sexton, they say some wild boars have eaten your paddy, is that right?

65. Yes.

66. Well, since wild boars have eaten rice from the Sexton's field, tomorrow we've
got to help hunt them down and beat them [dead] for him, right?

67. Sexton: Yeah, you've got to help me hunt them down and beat them dead.\footnote{This sentence contains an impressive six-verb sequence: \textbf{g̈a} (vV) `must' + \textbf{ga} (vV) `help' + \textbf{g̈àʔ} (V\textsubscript{h}) `hunt down' + \textbf{dɔ̂ʔ} (V\textsubscript{h}) `kill' + \textbf{pə} (Vv) `set into vigorous motion' + \textbf{pî} (Vv) `give/benefactive'.}
The rice itself is very good [this year in my field] .

68. Otherwise you won't have anything to eat. If \textit{they} eat it all up.

69. Whether we catch them or not, let's go drive them away for him. If we drive
them away, they won't come back for a long time, at least. We'll just go and drive
them away for him. Whether or not we catch them we'll go give them a chase. Tomorrow
we'll go...

70. Right!

\begin{center}
* * *
\end{center}

Well, now the time has come for harvesting our rice!

71. Yep, rice-harvest time has come. Tomorrow I plan to go start trying to reap.
Since the wild pigs are eating away at it.

72. Kheh-ki, have you managed to try harvesting yet?

73. Yes, I plan to try to harvest. My field has just begun to ripen a little, too,
now. Tomorrow I'll go reap,\footnote{\textbf{g̈ə̀ʔ} \textbf{qay} `go and reap'; but in sentence \#75 \textbf{g̈ə̀ʔ} \textbf{qay} means `is being reaped, is reaping, continues to (be) reap(ed).' The post-head versatile verb \textbf{qay} means either `go and V\textsubscript{h}' or `continue to V\textsubscript{h}.'} I guess.

74. Jali, what about you?

75. Well, mine is being harvested. Even today.

76. I see. Well, how many bushels\footnote{\textbf{pū} (Cl\textsubscript{f}) `mound of paddy; rice-stoop', here translated `bushel'.} have you gotten this year?

77. Five bushels.

78. Five bushels, eh.

79. Uh-huh.

80. Five bushels probably isn't enough to live on.

81. It doesn't even come to five bushels, unfortunately.\footnote{\textbf{še} (Pv): this particle here expresses regret at the verbal event.} In my field this
year the paddy is no good, since they set the damn thing on fire for me, so I'll
really only get \textit{three} bushels, Pastor. I have no idea how I'll ever be
able to feed myself now.

82. Ah, you can't think your way out [of a situation like this] !\footnote{\textbf{dɔ̂} \textbf{mâ} \textbf{tɔ̂ʔ}: lit., ``think not come out''}

83. How about you, Jali?

84. Well, I've got ten bushels, ten bushels for me.

85. Hm, you're a little better off, then.\footnote{The Pv \textbf{šɔ} `still' conveys the nuance `you're still in the realm of comfortable living.'} Shaw-lu [the Kachin] , how many
bushels did you get?

86. At the moment, I've only got five bushels. But I haven't finished my reaping
yet.

87. I see.

88. For my part, I've only got five bushels. Because the wild pigs were eating
it, too. Half of it.

89. Hm... well....

90. And that's why we kept saying before ``Work\textit{ together}, work
\textit{together}! If everybody works all by himself, the rats eat it, the birds
eat it, the wild pigs eat it!'' It's because you don't listen to what people
tell you, you guys!

91. There's an old saying, ``If you would eat, smell.''\footnote{I.e., ``Look before you leap.''} That
is, if you're going to eat something, try smelling it first. Oh, my children!\footnote{The Pastor lapses into sermon-style at this point.}
Cultivating the land is also like that. If you plan to do it, you've got to think
about it first. Furthermore, if you want to \textit{speak}, you have to think about
it first, I tell you. Even if we just intend to speak, we must carefully think
over [what we want to say] . When we don't listen to people's advice sometimes,
and disobey, that's the reason why we have to suffer for it this way...

\begin{center}
* * *
\end{center}

Well, have you tried threshing your paddy already this year?

92. Yeah, I've threshed it out\footnote{It is misleading (and boring) invariably to translate the V + (\textbf{a-)ni} by `try to V\textsubscript{h}.' It sometimes serves merely to `round out' the verb ('have a go at Vh'ing'), and is best left untranslated.} already.

93. How much grain did you get?

94. I got ten baskets\footnote{\textbf{pîʔ} (Cl\textsubscript{f}) `basketful; container holding 4 imperial gallons'} per bushel [of paddy] .

95. If you really get ten baskets per bushel from your field, that's not bad at
all. Since you've gotten five bushels. From mine I only\footnote{The P\textsubscript{n } \textbf{ɛ̀}, used with Num+Clf, sometimes simply serves to `round off' the quantity expression (much like Thai \textbf{sák} or Japanese \textit{gurai}) and need not be translated. Sometimes however it retains its original nuance of `only.'} get six baskets or
sometimes as much as seven. Sometimes I only get two basketsful, or one and a half!

96. Jali, what about yours?

97. Well, mine now, sometimes I get ten basketsful and sometimes fifteen.

98. I see. Well, now I'll teach you guys one thing! Let everybody carry back his
grain and store it inside his house! In this country the Thais are very mean, and
they're apt to set it on fire. If the Thais burn it up, we'll have nothing to eat.

99. No matter how much of it there is, even if I have to spend all day tomorrow
carrying mine, I'll probably get [the job] finished. I'll get it all carried home
and stored. It's hard to go and do. However, since these Thais are such irrational
people\footnote{\textbf{chɔ} \textbf{khɔ̂} \textbf{šī} \textbf{ve} \textbf{mâ} \textbf{hêʔ}. lit., ``they don't understand human words.''} ....

100. Will everybody carry it back to store?

101. They'll carry it back to store [if they know what's good for them] . What
[this fellow]\footnote{The victim of the incident to be recounted never appears as the overt subject of a sentence.} carried back came to ten basketsful. But all together [his
crop] had come to fifteen baskets. So that there were still five basketsful [lying
around in the fields] . And since you people don't listen to your elders' advice,
some elephants came and trampled [the rest] ! You won't have enough to eat!

102. Ah, me! That's why we said on that day long ago: ``When people instruct
you, listen to what they say! Listen to what they say! When you cultivate the land,
work where the others are working,''\footnote{Lit., ``wherever the others work, work there (''here'') [too] .''} we said. But today even the elephants
have again trampled [on somebody's crop] ! That's no way to earn a living! When
people instruct you, you don't heed their words.

\begin{center}
* * *
\end{center}

103. Has everybody finished carrying his grain back to store in his own hearth
and home\footnote{\textbf{á-qhɔ-yɛ̀-qhɔ} (Elab-noun) \textbf{á-qhɔ-yɛ̀-qhɔ} `hearth and home'.} ?

104. Well, I haven't managed to store mine yet. This weekend I plan to have some
Thais carry the rice [to my home] [on their shoulders] .\footnote{The Lahu carry things on their backs or heads (\textbf{pû} \textbf{ve}). The Thai prefer to carry baskets by attaching them to shoulder-poles (\textbf{tâʔ} \textbf{ve}).}

105. Yes, do [have them] carry it all up and store it away! Immediately.

106. Well, I'm finished already, because mine didn't amount to very much. [But]
I've thought it over, and next year, after we've celebrated New Year's, when it's
time to work again, from then on I'll work together [with the others] .

107. Okay.

