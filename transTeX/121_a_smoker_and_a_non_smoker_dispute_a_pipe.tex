\setcounter{footnote}{0}

<See previous text for a longer version>

1. Well, I'll just tell a story about a smoker and a non-smoker in the old days,
okay?

2. To all of you.

3. Once upon a time a person who was not a smoker saw somebody else's beautiful
pipe and coveted it

very much, wanting badly to get it for himself.

4. This smoker was very happy with his pipe.

5. Well, one day, since the non-smoker wanted to get that [so badly], he stole
it and made off with it.

6. Since he stole it, these two guys then went to court.

7. These two people, the smoker and the non-smoker, went at each other in a great
court case.

8. When they arrived at the official place\footnote{\textbf{ɨ̄} \textbf{kɨ̀} \textbf{ɔ̄}: lit. ``at the big place''.} for their trial, they had to litigate
\footnote{\textbf{tá} \textbf{pî} \textbf{ve}: The post-head versatile verb \textbf{pî} here indicates that an action by a third person affects other 3rd persons.} for a very long time,

those guys.

9. It could not be done for them quickly.

10. When they got there the smoker had been very angry, since he treasured his
possession so much.

11. The non-smoker wanted it badly, since the pipe was so beautiful, so he stole
it.

12. So then having discussed it vigorously, and argued and argued with each other,
when they got to the

great courtroom, the officials\footnote{khú \textbf{tê} \textbf{phā} : khú `chief; lord; bigshot' < Tai (cf. Si \textit{khun}).} thought it over and instructed them thus: ``If
you are the smoker,

crumple up\footnote{\textbf{chɨ̂ʔ}: `crumple up; wad into a ball'. DL:557.} the [right amount of] tobacco,'' they said.

13. So they made the pipe-thief crumple tobacco, and also made the smoker crumple
tobacco, both of them.

14. As for the non-smoker, the tobacco that he took and stuffed into the pipe was
way too much!

15. It just wouldn't fit into the pipe when he tried to pack it in.\footnote{\textbf{dôʔ} \textbf{mâ} \textbf{dɔ} : lit. ``pack not fit.'' \textbf{dɔ} is a \textit{resultative complement} specifying the successful or non- successful completion of the act of packing in.}

16. As for the smoker, he only stuffed a small amount into the pipe, so it fit
properly.

17. So that non-smoker who desired somebody else's possession and stole it did
not win.

18. He committed a grievous offense\footnote{\textbf{phîʔ} `commit an offense' < Tai (cf. Si \textit{phìt})}, not even being a smoker himself, yet still
desiring

another's possession and stealing it.

19. When he was forced to stuff in tobacco, it was way too much, while the smoker
put in exactly

enough\footnote{\textbf{gà-e}: this word consists of the post-head versatile verb \textbf{gà} `V to a conclusion' plus the verb-particle \textbf{e} `transitive motion', fused into a single syllable [\textbf{gày]}. See DL: 400-1.}, so the smoker was able to get his pipe back.

