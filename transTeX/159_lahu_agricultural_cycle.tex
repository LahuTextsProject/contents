
{1 Paul: Now what I'd like to know is, how do we Lahu people cultivate our
swiddens, from the day that we start to cultivate until we harvest, how does it
go?}

{2 Tell me one stage at a time, okay? That's what I would like to know.}

{3 In the beginning you look for a place to clear a swidden, and when you
find such a place, how do you clear it of trees, how many days do you have to spend
clearing it, that sort of thing.}

{4 Cà-bo: The way we Lahu cultivate our swiddens for a living is like this.}

{5 The first thing is we must clear away the undergrowth from the swidden.}

{6 After we finish clearing away the undergrowth, we clear away the big
trees.}

{7 As far as the time it takes to clear away the big trees goes, we have
to do it for about two months.}

{8 And after we've cleared away the big trees, we do the first burning of
the swidden.}

{9 Once this first burning is over, we do a second burning.}

{10 As for the time it takes to do the second burning, it doesn't take very
long.}

{11 It only takes one or two weeks.}

{12 So as for the amount of time it takes to go on to plant the fields,
it takes about a month and a half for us to manage to plant the rice.}

{13 After the planting is done, we have to do the weeding.}

{14 And now for the amount of time it takes for weeding, it's this way.}

{15 Some years when we cultivate the fields the weeds are very high, so
we can't do the weeding right away.}

{16 Sometimes it takes about a month to do the weeding.}

{17 Sometimes it takes about two months to do the weeding.}

{18 If they're very overgrown we must keep weeding right up until harvest
time.}

{19 But the time it takes to harvest isn't very long.}

{20 For those who have large swiddens to cultivate it takes about one month;
for those who don't have a large swidden to cultivate, it just takes two or three
weeks to do the harvesting, to be able to harvest.}

{21 Paul: Well then, what you were saying about clearing away the undergrowth
just now, how do you actually do it?}

{22 And what you called doing a second burning, how do you do that?}

{23 I still don't understand. This clearing away undergrowth and doing a
second burning. How do you do that burning, and how do you clear away the undergrowth?}

{24 Cà-bo: So what we call clearing away the undergrowth is like this.}

{25 We must clear away the undergrowth in the hot season.}

{26 When we get rid of these overgrown weeds, the weeds decrease greatly
--- when they become fewer the swidden is not overgrown --- so we Lahu say.}

{27 For this reason we have to clear the undergrowth as soon as possible.}

{28 And doing the second burning of the field is like this.}

{29 After we've done the first burning, we gather and pile up together all
the stuff which didn't get burned and set fire to it --- this is called doing the
second burning.}

{30 Paul: So what you said was that clearing away the undergrowth still
didn't clear away the big trees, right?}

{31 Before you've cleared away the big trees, you finish pulling out and
getting rid of the weeds little by little, that's what you do.}

{32 Clearing away the undergrowth goes that way, you said, right?}

{33 Cà-bo: Yep. We can say that's the way it goes.}

{34 You can't just clear away the big trees first.}

{35 You have to clear away and kill all the weeds under the trees in the
field, all of them.}

{36 And it's not just a question of clearing away and killing those weeds.}

{37 It becomes fertilizer --- as it rots it becomes fertilizer and it's
very good for the rice plants.}

{38 Paul: So then when you harvest the rice, how do you harvest it, do you
help each other out?}

{39 Or else does everybody do the harvesting by himself?}

{40 Cà-bo: When we Lahu harvest the rice, it's not exactly like the way
the plains folk help each other.}

{41 Sometimes a person is not able to harvest his own rice since he's unwell
or miserably poor, and when we realize this the whole village goes and manages
to help him out.}

{42 Unless a person is sick and ailing, everybody has his own swidden to
cultivate; and sometimes the fields are not even in one area, so there wouldn't
be time to go off to this person's place and that person's place [to help them
out].}

{43 It's just in times of real need that we go and help them out.}

{44 Paul: Now I'd just like to know about harvesting the rice.}

{45 Harvesting the rice, how does the rice have to get to be so you know
when the time has come to harvest it?}

{46 How does the rice have to look so that you know that it's ripe?}

{47 How many days does it take to do the harvesting, would you say?}

{48 When harvesting is done, how do you stack up [the rice]?}

{49 Tell me how you go about threshing it.}

{50 Cà-bo: Harvesting the rice is like this.}

{51 When we look it over this is what we have to know.}

{52 After the rice has sprouted when we see that its grains have hardened,
we know it's time.}

{53 When the grains have hardened and become yellow we know it's time to
do the harvesting, so we must do the harvesting.}

{54 After the harvesting, we must stack up the rice.}

{55 This is why stacking up the rice is important:}

{56 If we don't stack up the rice plants, the rice is no good. Besides,
it can't be threshed loose [so that the grains fall off the stalks easily].}

{57 Paul: When you say ``can't be threshed loose'' what do you mean?}

{58 Cà-bo: Not come loose --- the meaning of ``not come loose'' is that
when you thresh it the grains don't fall completely off.}

{59 Paul: Don't come falling off.}

{60 Cà-bo: Yes, some of it is still hanging on. That's what we mean.}

{61 Paul: How many days do you have to leave it stacked up before you do
the threshing?}

{62 Cà-bo: Properly speaking our custom is that you can do the threshing
after it's been stacked up two or three nights.}

{63 But if you don't have a lot of time and you have a great deal stacked
up, some years it has to be stacked up for a whole month.}

{64 Some years it happens that you have to leave it for two or three weeks.}

{65 When you've left it like this after you've finished stacking it up,
even if it rains, even if it's very hot, there's nothing to worry about. It doesn't
get wet.}

{66 Paul: So then when you've threshed the rice, when it's all threshed,
do you carry it to store at home? How do you carry it to store?}

{67 Cà-bo: Threshing the rice goes this way.}

{68 If the swidden is far away you do it right there in the field.}

{69 You make a house in the field, you make a huge storage basket, and you
do the threshing right there in the field.}

{70 But if the field isn't far away, if it's right outside the village,
you can bring it right back home directly.}

{71 Paul: So then after the rice grains have been carried and put away it's
all over, right?}

{72 So how do you do this? When the threshing is done these --- er, what
do you call it? --- the stalks[1], how do you call them in Lahu?}

{73 How do you deal with them?}

{74 Cà-bo: When the threshing is done we Lahu don't prepare [the stalks]
[for other uses] the way the people in the towns do.}

{75 We just take all those rice stalks that have been threshed away and
stack them up into piles.}

{76 And if we have to recultivate an old field we carry all those rice stalks
and scatter them all over the field, and set fire to them, and when we plant rice
the next time it will be good for the crop.}

{77 Paul: Well, so then I have come to understand a lot about how we Lahu
clear our swiddens for cultivation, how we cultivate the rice in the field, haven't
I?}

{78 So then, at the same time you are cultivating the rice, what other things
are you also growing besides?}

{79 Planting all kinds of things, planting cornfields, whatever fields you
plant, things like that.}

{80 Cà-bo: It's not that we only cultivate [rice in] our swiddens.}

{81 We Lahu at the same time we are cultivating [rice in] our swiddens we
also plant all kinds of legumes, various kinds of bananas, kinds of taro, all kinds
of things which are good to eat, all kinds of edible things.}

{82 Besides, we are also planting cornfields so we have to cultivate the
cornfields in a separate plot of land.}

{83 If we didn't have cornfields, since we Lahu raise pigs to eat, if we
didn't have anything to feed the pigs with, we'd have no way of earning money.}

{84 Paul: Well, that's great then.}

{85 Now the thing to worry about more is that the Thai authorities don't
want to let you clear swiddens, this is the most worrisome thing, isn't it?}

{86 If the Lahu aren't allowed to clear swiddens, they wouldn't have anything
to eat, would they?}

{87 Even if you don't have any money, it doesn't matter.}

{88 If you just have rice you can live, wouldn't you say?}

{89 This matter of not being able to clear [new] swiddens and having to
rely on old ones by putting fertilizer on them, this is very hard to do.}

{90 So then going to find a new place, as you said already, every single
place has an owner, but even though it's said that they have owners, you never
see anybody actually cultivating [those places].}

{91 We even see places abandoned for ten years, or even twenty years.}

{92 So then I don't know what can be done in this situation, right?}

{93 If you all got together, some people who knew their language, if you
could just go down there to the government office, to the officials, and have a
discussion about this, it would probably be good.}

{94 Cà-bo: Yes. It's not just that we have to be familiar with other people's
languages and customs or that we all have to be able to speak their languages.}

{95 But we've never succeeded in getting what we've asked for, what we had
planned to get by asking for it, as best we could.}

{96 We won't know whether we succeed or fail until we try.}

{97 And as for the authorities, what they tell us is, ``Go find a place
in the plains!''.}

{98 But we can't find such a place, no matter how we look.}

{99 Because of this the government is not taking pity on us and is not helping
us out, so that for us Lahu who can neither clear swiddens nor have paddy fields,
even though it is already difficult now, in the future it will get even more and
more difficult.}

{100 There is probably no way that we can come up in the world.}

{101 Paul: They tell you to plant coffee, fruit orchards, all kinds of things,
so couldn't you earn a living that way?}

{102 How could you do that, things like that?}

{103 Cà-bo: It's this way. If we Lahu succeeded in doing what they tell
us, we'd be able to earn a living.}

{104 But even if we wanted to plant a fruit orchard, we can't buy the fruit
trees.}

{105 We don't have the money to buy them. We don't have the money to do
it.}

{106 It takes a very long time. It can't be done quickly.}

{107 For this reason, while we still haven't gotten anything [from the government],
getting food to eat will be a very difficult thing for us Lahu.}

{108 The second thing is the matter of buying the fruit trees. If we don't
have the money we can't buy them.}

{109 Since you just have to spend money for everything, being able to earn
a living is very hard for us Lahu.}

