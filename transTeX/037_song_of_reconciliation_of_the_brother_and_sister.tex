
37 Song of Reconciliation of the Brother and Sister

1. \{Long, long ago/ once upon a time\} in Lahu country, in the land of Meh \footnote{mû-mɛ̂-mì-mɛ̂: mû-mì `country'. This is the traditional former abode of the Lahu, in Yunnan, called mɛ̂ probably because the Chinese Lahu were said to use animals to plow their swiddens (mɛ̂ `to plow'); in Thailand mountain fields (swiddens) are always cultivated by hand, unlike lowland wet-rice fields, where oxen and buffalo are used.},
there lived an older brother \footnote{ɔ̀-u-phâ: an archaic kin term with a variety of related meanings: (1) older brother of a girl; (2) a girl's mother's brother; (3) wife's brother; (4) man's brother in law.  The basic meaning is `guardian of a young girl'.} and a younger sister \footnote{ɔ̀-nù-ma: Correlative of ɔ̀-u-phâ. Archaic kin term meaning (1) sibling's daughter; niece; (2) female word of an elder brother or maternal uncle.}, who used to divide up
portions of game to eat.

2. I'd just like to tell a little story about how they didn't get along with each
other.

3. One day \footnote{a-lɔ́ tê ni: lit. `on the first day'.} the younger sister caught a great sambhur deer and brought it \footnote{khɨ́-yɨ̄: \textit{Cervus unicolor.}},
which she properly shared out with her brother.

4. Afterwards, the elder brother caught a porcupine \footnote{fâʔ-pu: Asiatic porcupine [\textit{Hystrix brachyura}]} and brought it.

5. But then he didn't share it with his sister.

6. So the sister was very \{hurt/resentful/offended\}, and she went off downstream.

7. The brother went climbing off upstream.

8. One day they met each other again, and were so happy that they sang this little
song:

9. <sister> Only long ago could we live together

Is that not so, oh, my brother!

10. <brother> It is so, oh, my sister!

11. <narrator> Indeed so! They hadn't seen each other for ages!
