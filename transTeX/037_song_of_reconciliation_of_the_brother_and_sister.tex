\setcounter{footnote}{0}

1. Once upon a time, in Lahu country, in the land of Meh\footnote{\textbf{mû-mɛ̂-mì-mɛ̂}: an elaborate expression built on \textbf{mû-mì} `country'. This is the traditional former abode of the Lahu, in Yunnan. It is claimed that this land was called \textbf{mɛ̂} because the Chinese Lahu used animals to plow their swiddens (\textbf{mɛ̂} `to plow'). In Thailand mountain fields (swiddens) are always cultivated by hand, unlike lowland wet-rice fields, where oxen and buffalo are used.}, there lived an older
brother\footnote{\textbf{ɔ̀-u-phâ}: an archaic kin term with a variety of related meanings: (1) older brother of a girl; (2) a girl's mother's brother; (3) wife's brother; (4) man's brother in law. The basic meaning is `guardian of a young girl'.} and a younger sister\footnote{\textbf{ɔ̀-nù-ma}: Correlative of \textbf{ɔ̀-u-phâ}. Archaic kin term meaning (1) sibling's daughter/niece; (2) female word of an elder brother or maternal uncle.}, who used to divide up portions of game to
eat.

2. I'd just like to tell a little story about how they didn't get along with each
other.

3. First the younger sister caught a great sambhur deer\footnote{\textbf{khɨ́-yɨ̄}: \textit{Cervus unicolor.}}, which she brought
[home] and properly shared out with her brother.

4. Afterwards, the elder brother caught a porcupine.\footnote{\textbf{fâʔ-pu}: Asiatic porcupine [\textit{Hystrix brachyura}].}

5. But then he didn't share it with his sister.

6. So the sister was very offended, and she went off downstream.

7. The brother went off climbing upstream.

8. One day they met each other again, and were so happy that they sang this little
song:

9. Sister: Only long ago could we live together

Is that not so, oh, my brother!

10. Brother: It is so, oh, my sister!

11. Narrator: Indeed so! They hadn't seen each other for ages!\footnote{Lit., ``only long ago had they seen each other.''}

