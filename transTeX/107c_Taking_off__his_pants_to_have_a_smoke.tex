\setcounter{footnote}{0}

1. Once there were two men.\footnote{A Black and a Yellow Lahu.}

2. The two of them were going together along a road.

3. [The Yellow Lahu] said to the Black Lahu, "When we get up there on the
mountain, let's put in

some tobacco and smoke for a while."\footnote{Lit., ``let's put in and roll (wad up; crumple) tobacco for a while.''}

4. So then the Black Lahu said, "Well we're up here now, so I'll just take
off my pants, and..."

5. ``What the hell are you doing?'' he asked.

6. "When we were talking just now, didn't you say you were going to tie
us up?" he said.\footnote{A rather complicated set of puns is involved here. In Black Lahu the synonym compound \textbf{phɛ-chɨ̀ʔ} means 'tie up' and \textbf{hā} \textbf{phɛ̂} means 'take off one's pants'. In Yellow Lahu, on the other hand, \textbf{chɨ} means 'to smoke' (YL \textbf{a-šú} \textbf{chɨ} \textbf{che} is equivalent to BL \textbf{šú} \textbf{dɔ̀} \textbf{ve}). The BL is pretending to misunderstand what the YL said, by interpreting YL \textbf{chɨ} as BL \textbf{chɨ̀ʔ} 'tie up', and purposely conflating BL \textbf{phɛ} 'tie up' and \textbf{phɛ̂} 'take off'. He thus removes his pants so that the YL can use them as a rope to tie him up.}

\textit{Note: An over-elaborate explanation of this joke was offered in JAM 1969,
pp. 202. The current explanation (arrived at by consultation with my YL consultant
Aaron Maung Maung Tun, June 2017) is much better.}

