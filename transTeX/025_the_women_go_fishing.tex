\setcounter{footnote}{0}

<All female speakers. Recorded at Huey Tat, March 9, 1965>

1. A: Which section [of the stream] shall we dam up?

2. B: We'll just go dam\footnote{\textbf{tɔ̄} \textbf{câ}: lit. ``dam-eat.'' As a post-head versatile verb \textbf{câ} `eat' means `to V for a living'.} it at Mɛ-thà-lây\footnote{\textbf{Mɛ-thà-lây} is a small Thai village on a stream flowing at the foot of a mountain near Huey Tat.}!

3. C: I'll go too!

4. D: [I'll] go, [I'll] go.

5. A: Carry along a mattock\footnote{\textbf{thɔ̄-qwɛ̀ʔ} (also \textbf{kɔ̂-ŋwɛ̂ʔ}): a digging tool resembling a trowel, with a curved handle. See plate \#18 in DL.}! And take along a hoe too!

6. B: Of course we'll carry them.

7. A: And carry a plastic sheet too!

8. B: We've got to carry [one]. If we didn't, since the water's especially high,
we probably wouldn't be able to dam it.

9. C: Well then, one way to do it would probably be to dam one or two sections
of this fork in the stream.\footnote{\textbf{lɔ-qa}: a fork in a stream such that the two branches rejoin later.}

10. A: That's what we'll have to do. The two of you, mother and child\footnote{One of the women (``B'') has brought her small child along.}, go dam
it properly\footnote{\textbf{qha-dɛ̀ʔ} (Adv.): A culturally important adverb, `properly; right; thoroughly; the way one should'.} upstream.\footnote{The fuller form of this word is \textbf{g̈ɨ̀-ú} (lit. ``water-head'').} I'll do it downstream.

11. B: I'll do it right, You wait for me the way you should.

12. C: They're probably over on that side, the fish.

13. D: Wherever you're planning to stay [and wait], that's just where you have
to dam it.

14. B: Why don't you dam it on that side. Dam it on that side! That's where they
probably are, the fish.

15. A: I'll dig some earth and pile it up for you. You two, mother and child,
go pick up stones to stack.

16. C: All right, I'll carry them and plaster\footnote{The idea is to plaster mud onto the piles of stones to impede the flow of water.} them over with all my might.
\footnote{\textbf{g̈â-thèʔ} : another culturally important adverb, very much like Japanese \textit{isshōkenmei} `energetically; diligently; with sincere effort'.}

17. A: Just do it right! If you don't, how will we get anything to eat?

18. C: By God, we've just got to do the damming with our foreheads dripping with
sweat\footnote{\textbf{kɨ̄-ni} \textbf{nā} \textbf{pâʔ-šôʔ} \textbf{te} \textbf{ve}: lit. `do so that one's forehead drips with sweat.'}, even while we're carrying our babies and trying to keep

them amused.\footnote{\textbf{yâ-pû-yâ-qa} : `have kids on one's hands to keep entertained' (``child-carry-child-play'').}

19. B: I've got to do it even when my baby is waiting to eat!

20. D: Yes, yes, my old man\footnote{\textbf{lɔ̂-pū}: < Chinese (Mand. \textit{lǎo-fù}) ``old guy.'' Familiar reference term used by a woman when speaking of or to her husband: `my old man'.} is also waiting to eat at home.

21. A: You, mother and child, dig up mud properly upstream and plaster it on.\footnote{The interjection \textbf{pōthôo} \textasciitilde{} \textbf{thôo} (ult. < Tai `by the Buddha') is freely used by these Christian Lahu. To avoid tedious repetition I translate it in several different ways, e.g.``my God'';``my, my;'' ``wow'', etc.}
I'll go and dam downstream.

22. C: Whew, I'm giving it all I've got. Since I'm so hungry!

23. A: Wow, [13a] the fork has been dammed just right! The water is nicely blocked
off!

24. Wowie, tons\footnote{\textbf{a-cí-cí} \textbf{mâ} \textbf{hêʔ}: lit. ``not just a little bit'', i.e., `lots and lots'.} of fish are coming down!

25. Wowie zowie,\footnote{This extreme interjection is meant to convey the exaggerated intonation of \textbf{pōthôo-ō-o}.} they're coming right into the fish basket!\footnote{\textbf{thɛ-qō} \textasciitilde{} \textbf{phɛ-qō}: huge basket, usually used for storing paddy; \textbf{ŋâ꞊thɛ-qō} `basket for catching fish' (here \textbf{ŋâ} is interpreted as the first member of a compound). The sentence could also be interpreted with \textbf{ŋâ} `fish' as the topic (``the fish are coming right into the basket'').}

26. It's full to the brim already!

27. We can't even budge it!

28. Act fast! Hurry and pick them up and come down here!

29. My, my, my, they've climbed up here too!

30. This is just what we've been looking for today.

31. I'm so hungry, so let's do our best!

32. A: My goodness, there are so many big hu-qɔ̀ʔ\footnote{A kind of small, edible, \textbf{hard-to-catch} fish, whose bite can inflict pain.} fish coming down---in
droves!

33. Wow, how are the two of you doing?\footnote{Lit., ``How are the two of you doing to eat?'' i.e., what are you doing that will bring us food?}

34. Hurry up and come here now---hurry up and come!

35. B: I don't think they [the fish] are going down[stream]---I think they're probably
going up.

36. A: No, no! Hurry and come down---just come down!

37. B: From what I see it looks as if they're climbing up!

38. A: No! They are \textit{not} climbing up! Hurry and come down, the two of you,
mother and child!

39. B: They came up, they came up, they came up! My God, to-da-a-y\footnote{\textbf{yàʔ-hi} \textbf{lɛ̀} : the topic particle \textbf{lɛ̀} is pronounced with very exaggerated intonation as \textbf{lɛ̀-ɛ̄-ɛ́-ɛ}, a phenomenon especially characteristic of women's speech.} we got
so many you could die\footnote{\textbf{g̈a} \textbf{šɨ} \textbf{e} (\textbf{la}) \textbf{yò}: lit. ``must die''; an intensive expression, similar to Thai \textit{cə-taaj}. Cf. Eng. locutions like ``the foie gras was to die for.''}!

40. C: Do it fast, hurry and pick them up and come here!

41. D: When we get home we'll eat up a storm\footnote{\textbf{vǎ(n)} is a verb (< Shan \textbf{vǎŋ} `strike, beat') usable as a substitute for other verbs indicating violent, sudden, or extreme actions.}!

42. C: It's going to get dark\footnote{\textbf{mû} \textbf{phə̂ʔ} \textbf{ve} `get dark'; lit. ``the sky is revealed''; i.e.the blackness of space and the stars, which are hidden during the day, are ``revealed'' at night.} soon.

43. D: Even if it's dark we'll just light our way home with pine-torches.

44. A: Just by damming this one fork we probably got five or six kilos of fish!

45. D: We sure did! Our old men will be happy---I bet they'll clap their hands
for joy tonight.

46. B: Yeah, my old man will be so-o-o happy too---wow, I bet he'll be pacing back
and forth giggling\footnote{\textbf{hɔ-hí} \textbf{hɔ-hí} \textbf{te} \textbf{ve} (AE onomat + V) `giggle' [inadvertently left out of DL, where also the variant \textbf{hí-hí} \textbf{te} \textbf{ve} appears].} all over the house! \direct{laughter}

47. D: \direct{pretending to be home} Eat your fill, old man! I've come
back! Eat all you want! Here, eat up!

48. B: Wow, even though there were so many fish today, the mother and child didn't
listed to what they were told..

49. Since they didn't listen, we only got five or six kilos today.

50. Oh boy, this big soft-shelled turtle\footnote{\textbf{pa-fá} ( < Tai). This turtle is classified as a fish (Si. \textbf{plaa}) in Lao and Northern Tai, so that nominal vegetarians may eat it in good conscience.} is for my old man!

51. D: Hey, old man, hurry and cook these up!

52. Pick the big one for our kid and cook it properly for him!.

53. Cook the little one for the two of us to eat, mixed with banana-creeper\footnote{\textbf{á-pɔ̂-g̈wɛ̂}: A creeper with edible leaves that grows on banana trees (\textbf{á} \textbf{pɔ̂} `banana').}
and mimosa-shoots.\footnote{\textbf{jàʔ-cā-ɛ̀}: edible shoots of a small thorny bush, perhaps a kind of acacia or mimosa [\textit{Dichrostachys cinera}]}

54. \direct{taking an opposite stance} Hey, old man---hey, old man!
Feed this little tiny one to the kid, and let's scarf down this big one ourselves!

55. B: Be careful to remove the inedible parts\footnote{This idea is conveyed by the verb \textbf{ši} `take out inedible parts'} when you feed it to the kid,
old man.

56. A: [This] crab bit me---ouch, that hurts!

57. B: Grandma, that fart really stinks! That fart stinks! Where is it coming from?

58. X: Right here, right here, right here!

59. B: Where did she go?

60. X: Grandma! Right here!

61. B: Wow! Grandma let a bombshell!

