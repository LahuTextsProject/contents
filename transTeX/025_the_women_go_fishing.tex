
25 The Women Go Fishing

<All female speakers. Recorded at Huey Tat, March 9, 1965>

1. A: Which section [of the stream] shall we dam up?

2. B: We'll go dam \footnote{tɔ̄ câ: lit. ``dam-eat.'' As a post-head versatile verb, câ `eat' means `to verb-head for a living'.} it at Mɛ-thà-lây \footnote{Mɛ-thà-lây is a small Thai village on a stream flowing at the foot of a mountain near Huey Tat.}!

3. C: I'll go too!

4. D: [I'll] go, [I'll] go.

5. A: Carry along a mattock \footnote{thɔ̄-qwɛ̀ʔ (syn. kɔ̂-ŋwɛ̂ʔ): a digging tool resembling a trowel, with a curved handle. See plate \#18 in DL.}! And take along a hoe too!

6. B: Of course we'll carry them.

7. A: And carry a \{rubber/ plastic\} shut too!

8. B: We must carry [one]. If we don't, since the water is especially high, we
probably won't \{succeed in damming/ be able to dam\} [the stream].

9. C: Well then, one path [to take is that] we'll probably have to dam one or two
sections of this fork in the stream .\footnote{lɔ-qa: a fork in a stream/river such that the two branches rejoin later.}

10. A: That's what we'll have to do. Both of you, mother and \{child \footnote{One of the women (``B'') has brought her small \{son/ child\} along}/son\},
go dam it properly upstream.\footnote{The fuller form of this word is \textbf{g̈ɨ̀-ú} (lit. ``water-head``).} I'll do it downstream.

11. B: I'll do it right, You wait for me the way you should .\footnote{qha-dɛ̀ʔ (Adv.): A culturally important adverb, `properly; right; thoroughly; the way one should'.}

12. C: I suppose I should stay \{over there/ on that side\}.

13. D: Wherever you \{plan to stay/ think of staying\}, that's just where you have
to dam it.

14. B: Why don't you dam it on that side. Dam it on that side! That's where they
probably are, the fish.

15. A: I'll dig some earth and/ to block it up for you.  You two, mother and child,
(go) pick up stones to \{pile/stack\} up.

16. C: All right, I'll carry them and plaster \footnote{The idea is to plaster mud onto the pile of stones to impede the flow of water.} them up with all my might. \footnote{g̈â-thèʔ (Adv.): another culturally important adverb, very much like Japanese \textit{isshōkenmei} `energetically, diligently, with sincere effort'.}

17. A: Just do it properly! If you don't, how will we get anything to eat?

18. C: (We've) just got to work our butts off \footnote{kɨ̄-ni  nā  pâʔ- šôʔ  te  ve: lit. ``do so that one's forehead drips with sweat.''  sweat forehead copiously flow do NOM} damming it! Even while \{we're/
I'm\} carrying \{our babies/ my baby\} and playing with \{them/ him\} \footnote{yâ- pû- yâ- qa (elab. adv.): have kids on one's hands to keep entertained.   child carry child play}

19. B: I've got to do it even when my baby is waiting to eat!

20. D: Yes, yes, my old man \footnote{lɔ̂- pū: < Chinese (Mand. lǎo-fù) ``old guy.'' Familiar ref. term used by a woman when speaking of her husband to others: `my old man'.} is also waiting to eat at home.

21. A: You, mother and child, dig up mud properly upstream and plaster it on. \footnote{The speaker adds ``you two mother and child'' as an afterthought after saying she'll go dam downstream.}
I'll go and dam downstream.

22. C: Whew, I'm giving it all I've got.  And I'm hungry besides!

23. A: Wow, the fork has been dammed just right! The water is nicely blocked off!

24. Wowie, tons \footnote{a-cí-cí mâ hêʔ: lit. `not just a little bit', i.e., `lots and lots'.} of fish are coming down!

25. Wowie zowie, \footnote{This extreme interjection is meant to convey the exaggerated intonation of pōthôo-ō-o.} they're coming right into the fish basket! \footnote{thɛ-qō \textasciitilde{} phɛ-qō: huge basket, usually used for storing paddy. ŋâ꞊thɛ-qō `basket for catching fish' (here ŋâ is interpreted as the first member of a compound). The sentence could also be interpreted with ŋâ `fish' as the topic (``the fish are coming right into the basket'').}

26. It's full to the brim already!

27. We can't even \{budge it/pick it up\}!

28. Act fast! Hurry up and pick them up and come down here!

29. My, my, my, they've climbed up here too!

30. This is just what we've been looking for today.

31. I'm so hungry, so let's do our best!

32. A: My goodness, there are so many big hu-qɔ̀ʔ \footnote{A kind of small, edible, hard-to-catch fish, whose bite can inflict pain.} fish coming down---in
droves!

33. Wow, how are the two of you doing? \footnote{Lit., ``How are the two of you doing to eat?'' i.e., what are you doing that will bring us food?}

34. Hurry up and come here now---hurry up and come!

35. B: I don't think they [the fish] are going down[stream]---I think they're probably
going up.

36. A: No, no! Hurry and come down---just come down!

37. B: From what I see it looks like they're climbing up!

38. A: No! They are \textit{not} climbing up! Hurry and come down, the two of you,
mother and child!

39. B: They came up, they came up, they came up! My God, to-da-a-y \footnote{yàʔ-hi lɛ̀ : the topic particle lɛ̀ is pronounced with very exaggerated intonation as lɛ̀-ɛ̄-ɛ́-ɛ, a phenomenon especially characteristic of women's speech.  today  TOPIC} we got
so many you could die \footnote{An intensivizing expression g̈a  šɨ  e  yò, similar to Thai Vɨ \_ taaj. Cf. Eng. locutions like ``the foie gras was to die for.''  get to die away DECL}!

40. C: Do it fast, hurry and pick them up and come here!

41. D: When we get home we'll eat up a storm \footnote{vâ is a verb (< Shan vǎŋ `strike, beat') usable as a substitute for other verbs indicating violent, sudden, or extreme actions.}!

42. C: It's going to get dark soon.

43. D: Even if it's dark we'll just light our way home with pine-torches.

44. A: Just by damming this one form we probably got five or six kilos of fish!

45. D: We sure did! Our old man will be happy---I bet they'll clap their hands
for joy tonight.

46. B: Yeah, my old man will be so-o-o happy too---wow, I bet he'll be pacing back
and forth giggling \footnote{hɔ-hí hɔ-hí te ve (AE onomat + V) `giggle' [inadvertently left out of DL, where also the variant hí-hí te ve appears].} all over the house! [laughter]

47. D: [pretending to be home] Eat your fill, old man! I've come back! Eat all
you want! Here, eat up!

48. B: Wow, today the fish didn't listen when the mother and child spoke.

49: [Since they didn't listen], we got five or six kilos of them, just today.

50: Oh boy, this big soft-shelled turtle \footnote{pa-tÁ ( < Tai). This turtle is classified as a fish (Si. plaa-)} is for my old man!

51. D: Hey, old man, hurry and cook these up!

52. Pick the big one for our kid and cook it properly for \{him/her\}.

53. Cook the little one for the two of us to eat, mixed with Á-pɔ̂-g̈wɛ̂
