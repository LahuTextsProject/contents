
\textbf{\#86 C. The Smoker and the Non-smoker}

83 H: Well, once upon a time there were two men.

84 One of them was a tobacco-smoker, he used to smoke all the time.

84 He was a great smoker.

85 The other one didn't smoke.

86 His pipe--the smoker's pipe was a very fine one.

87 The inside\footnote{ɔ̀-kɛ 1. 'the innermost, hardest part of a log; the pith' 2. 'the inside of a tube'} of it was red and smooth, and it was streamlined\footnote{phi ɛ̀ʔ 'flat.' Here probably refers to its elegant shininess. CHECK.} and shiny,
so that the non-smoker wanted to have it, and stole it away from him.

88 When he had gone and stolen it, the two of them then [a few words are inaudible]
and they make a big count-case\footnote{à-mù-ló phɛ̀ʔ dàʔ ve "to have a great case happen mutually"} out of it.

89 They made a big case of it, they were fighting with each other, and (the time
came that) they reached the great official--the lord--the king\footnote{The headman decides to make his story impressive by making the judge a king.}--and this king
was a very shrewd man.\footnote{Lit: "as for this king, his cleverness was very great."}

90 He made the non-smoker \{make a / wad of\} tobacco, to put in the pipe.

91 And he had the smoker crumple some tobacco too.

92 "Both of you crumple it and put it down here," he said.n93
He didn't have them wad it up one tobacco--er, I mean one\textit{man} after the
other\footnote{The headman's slip of the tongue occurred at the beginning of the sentence, but the English word-order requires that it be translated late in the sentence.} [while they were there] together.

94 [Rather] he summoned one man at a time and had him \{crumple it / wad it up\}--the
tobacco.

95 Well, then, after the tobacco had been wadded and placed [before him], he summoned
the smoker.

96 "Come here, you!" he said to the owner of the pipe.

96 He came.

97 Then he summoned the one who had stolen the pipe too.

98 "Come here!" he said.

99 "If you smoke a pipe--er, you, if the pipe is really yours, try putting
the tobacco you wadded into it," he said.

100 When he said, "Try putting it in," they went at it with a
will\footnote{g̈ɔ̀ kə a lɛ: g̈ɔ̀ is a lively $_{v}$V.}--and the one the smoker had made\footnote{"the smoker's one" (šú dɔ̀ p\emph{a} tê mà).} filled the bowl of the pipe perfectly.

101 The non-smoker's one was a big lump like this . . . \footnote{The inaudible part of the sentence must mean something like "and wouldn't go into the bowl, \textit{since he didn't know} its exact size."} . . . because he didn't
know....

102 He was jealous of someone else's pipe, and stole it, and even made a court
case out of it and went before the king, but he couldn't win against the pipe-smoker!
103 THat fellow went and robbed someone else's property, went and stole someone
else's possession--104 Then, after he had coveted it and stolen it, when it came
to putting it to proper use,\footnote{qɔ̀ʔ te ve qɔ̀ʔ: the first qɔ̀ʔ implies the action takes place \textit{later}. te is a pro-verb for 'wad up'--i.e., 'do an action.' The second qɔ̀ʔ is similar to the Pvnf kàʔ 'even.'} he had no idea of how to do it.

105 When someone forced him to make a wad for it, he crumpled up enough tobacco
for \{anyone / a person\} to put into\textit{two} pipes--he didn't know [\{how
to do it / any better\}].

106 He wanted to get a hold of his pipe so he brought the fellow\footnote{qɔ̀ʔ te ve qɔ̀ʔ: the first qɔ̀ʔ implies the action takes place \textit{later}. te is a pro-verb for 'wad up'--i.e., 'do an action.' The second qɔ̀ʔ is similar to the Pvnf kàʔ 'even.'} to court,
thinking all the time\footnote{"all the time" translates the reduplicated yɔ̂ g̈â yɔ̂ g̈â 'he'd win, he'd win [he thought].'} that if a serious issue were made of it\footnote{ɔ̀-tɛ̀ ɔ̀-na te ve.} he would
win.

107 [But] it was in vain.

108 He went and stole another's property, and behaved like a miserable wretch.\footnote{chɔ lù p\emph{a} 'a ruined person, a person who has gone bad.'}

109 But the real pipe-smoker, in joy and gladness, got [back] his possession.\footnote{In case one desires to impress the moral more strongly, one may add 109a: hɔ́ khi chi chɔ-lù-p\emph{a} àʔ šu mɔ̂ qhɔ̂ ve thàʔ patɔ th\emph{ɔ} qhɔ mɔ-chwɛ̂ yù kə š\emph{e} ve gò cê. "And the king threw the wretch into jail for a long time, since he had stolen what belonged to another."}

110 That's right, my boys! "Don't go sitting on someone else's stool,"
as the saying goes. If it's not your own stool--

111 T-y: (Teasing) Did you say "don't go shitting" or "don't
go sitting"?\footnote{The original says "did you say tâ mi (꞊ don't catch) or tâ mɨ (don't sit)." The humorous impertinence is, if anything, enhanced by the translation.}

112 Don't \textit{sit}, I said! That man who stole the pipe couldn't best the other
when it came right down to it, in the end.\footnote{Lit: "when the doing-thus time came, when the being-the end/last time came."  \textbf{\#86b D. The Three Lazy Men and the Princess }  114 Headman  Well, then, let me tell you another story.115 Listen carefully, everybody.  116 Long, long ago, the story goes, there were some lazy men, three traders.[53]117 They were resting[54] in a certain place where several roads met, where three roads met.118 The first man,[55] saying he was going off to trade, had sat down there,[56] and would say "Ah, where are you going?" [to those who passed].119 One day [another] lazy man appeared.120 [In response to the other's usual question] he said, "Oh, I'm doing some trading."[57]121 "Have a seat here." 122 After a while a [third] man came along [one of the] roads to the place where the three roads met.123 He arrived there.124 And they all had a conversation.  125 "What do you do?," they said.  126 "Nothing much.[58] I go around trading," he said.  127 "\{What can you do (well)? / What are you good at?\}" they said.  128 All of them were sitting there and talking back and forth.  129 "Well, I know how to make crossbows," he said.  130 "Well, then, you--what can you do?"  131 "Oh, I know how to \{tell fortunes / predict the future\}[59]"--these three men [were talking like that].132 "Well, what can you do?" 133 "Oh, I can make crossbows." 134 Thereupon they said, "What do you have to say, then, if you tell fortunes? 135 Try predicting the future, maestro,[60]" they said.  136 Well, he spoke, screwing his mouth up from one side to the other[61]: 137 "Hm, in a short while, at midday, we'll see[62] that a hawk carrying a princess--a king's daughter--in his beak will fly over the lake here," he said.138 Thus he foretold.139 "Well, didn't\textit{you} say you could make crossbows?" 140 "Yes, I can (make crossbows)." 141 And he made a crossbow.142 Well, another of them said, "And what can\textit{you} do?"  143 "I know how to swim," he said.144 As they were all talking like this, that which had just been foretold \{came to pass / was fulfilled\}.145 It happened according to the prediction.146 So, the crossbow was all made, and they stayed there together.147 When midday came, there was the hawk carrying the king's daughter, and he was already over the lake.[63]148 So he took careful aim and fired off a shot, and they fell into the lake.149 When they had fallen into the lake, the man who could swim immediately struck out swimming, and by his swimming rescued the princess who was out there--the young girl.150 When he had swum back with her, the three of them [began to] \{fight over / lay conflicting claims to\} the girl.151 They were arguing over who would marry her.152 One of them, now, one of them wanted to marry her, saying "I foretold it, so I, the maestro, ought to marry her!" 153 "Well, you couldn't get her just by telling your fortunes--I had to swim out and get her!" 154 "If I hadn't shot him down with my crossbow you couldn't have gotten her by your swimming!" 155 Thus the three of them vied with each other.  156 You people think about it! Which one of them probably should have married her? Thank about it, everybody.  157 Pà-ɛ́: You mean the queen[157]?  158 The one who shot [the arrow] ought to get her.  159 The swimmer!  160 T-y: It's the fortune-teller who should get her! If he hadn't foretold it they'd never have known.  161 Even if he foretold it, if he didn't swim he couldn't get her.  162 H: Well, they took the case to the king and told him about it--and how did it come out? "Who was the one who \{passed by / created\}[158] first? Was it the fortune-teller?  \begin{center} TO BE CONTINUED \end{center}  \leftskip=0pt \textbf{Footnotes}}

113 H: Mm-hm.

