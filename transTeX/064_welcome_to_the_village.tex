
64 Welcome to the Village (with a hymn)

[Medic who had received training in Mae Taeng]

1. At this time, we brethren \footnote{ɔ̀-ví-ɔ̀-ni: lit. `elder and younger siblings'; the most common vocative term used when addressing a group of Lahu.} are very happy that a foreigner\footnote{Kâlâ-phu: lit. `white Indian'. Term is used for Euro-Americans, like Thai \textit{fàràŋ.} The reference is to the author. Afro-Americans are called Kâlâ-phu=nâʔ, lit. `black white Indian'.} and a Lahu\footnote{I.e., Cà-lɔ̂, my chief consultant on this first fieldtrip (1965-66), a recent immigrant to Thailand from Shan State, then living in Chiang Mai, not in a Lahu village.}
have become friends with us, and that we are seeing each other this evening.

2. However, amidst our great happiness---er---although we live in different places\footnote{tê  g̈â  tê    kà: lit. ``one-person-one-place''  one CLF  one CLF},
being able to meet this evening is due to the great grace of God.

3. In addition, at this time we would like to record\footnote{te kə gâ: kə `insert, put in' is used for `to record'} a hymn.

4. I'll teach you this hymn that has been composed, just as it is in our prayer-books.\footnote{g̈ɨ̀-ša ve tɔ̂-khɔ̂: lit. ``God's words''}

The doors of God \{are still/ stay\} open

for all (us) sinners.

Therefore, O brethren

do not hesitate!

If time runs out

you won't have time to be happy.

The doors of God are still open.

Come in, come in!

Before Jesus' door closes.

If time runs out

you won't have time to be happy.

Speaker: That's how it ends.

