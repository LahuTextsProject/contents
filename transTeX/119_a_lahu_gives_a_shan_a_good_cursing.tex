\setcounter{footnote}{0}

1. Well then, once there was a certain person.

2. As he was on his way to work\footnote{Lit: ``within (the time) he was going off to earn his livelihood.''} he met a Shan [woman].

3. When he met the Shan,\footnote{Lahu narratives (indeed Tibeto-Burman stories in general) often use the device of linking one sentence to another by echoing the final clause of the preceding sentence in the first clause of the next sentence.} he had many people with him.\footnote{Lit: ``his people were numerous.''}

4. Well, he wanted to say to the Shan ``I'm very hungry, very hungry!''

5. [But] what he said was, ``Madam, my stomach is very full!''\footnote{He wanted to say something like \textbf{yâ} khâo \textbf{nā}, but said instead \textbf{pú} \textbf{kāɨ} \textbf{nā} (appropriate when one politely refuses the offer of more food). The similarity between these expressions is situational rather than phonological. The Lahu man knew a couple of Shan set expressions related to eating, but mixed them up.}

6. So then the Shan [lady] said, "Since your stomach is full, why don't
you7 lie down here for a while?"\footnote{The durative nuance conveyed by English 'for a while' is provided by the verb-particle \textbf{tā}, which indicates that the action of the verb will be prolonged or quasi-permanent.}, and she got a pillow and put it
down for him.

7. Well, he kept thinking that he would get something to eat, so he went on waiting.\footnote{I.e., he thought that the pillow was intended to make him more comfortable while he waited for his meal.}

8. He waited and waited for a long time, [but] he still wasn't getting anything
to eat.

9. When [he saw] he was being treated this way, he got angry.

10. When he got angry, he decided he would curse out the Shan [woman], and said
"Hey, boys, boys\footnote{His companions on the way to the field (See Sentence 3).}! You hustle on ahead.\footnote{I.e., ``keep hurrying along''.}

11. I'm just going to tell this Shan off\footnote{\textbf{dê-lō}: \textbf{dê} 'scold', \textbf{lō} 'attack physically or verbally'.} first!"

12. Then, after the boys had left, he said: "You damned Lahu! I eat your
shit, sir!"\footnote{The Lahu had once probably heard the expression ``damned Lahu'' from the lips of an angry Shan, so this is what comes out instead of his intended ``damned Shan!'' Finally he works himself up to his crowning insult. He would like to assert that the Shan lady is so far inferior to him that his very excrement is good enough for her to eat. Unfortunately. however, he gets his Shan pronouns transposed, so that the direction of coprophagy is reversed, and he ends up volunteering to perform the act for her. Further adding to the humor is his use of the Shan polite final particle \textbf{lɔ̀ʔ} (cf. Thai \textit{khráp}), here translated ``sir.'' One can imagine the poor kind woman's amazement as she tries to understand his angry babble.}

\direct{laughter}

13. Then he went off.

14. When he reached his destination,\footnote{Lit: 'when he got there.'} when he saw his people, he said, "I
really told that Shan a thing or two!"\footnote{He uses the classifier for inanimate objects (\textbf{mà}) instead of the one for people (\textbf{g̈â}). This is sometimes (as here) slightly insulting, though other speakers occasionally use \textbf{mà} in this sense with no such intention.}

