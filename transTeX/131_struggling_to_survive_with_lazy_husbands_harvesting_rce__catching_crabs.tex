
131 Struggling to Survive with Lazy Husbands

1. \{My dear/daughter-in-law\}\footnote{nò à (voc) : nò (\texttt{<} PLB *nam ⪤ ʔnam) was originally a kinship}, when you went to spend the night in the fields
this week, were you able to harvest a lot of \{rice/paddy\}?

2. There was nothing \{[for me] to/I could\} get! Since I was carrying my whining
baby around!\footnote{yâ-pû-yâ-ŋɛ̀}

3. Grandma \footnote{mɛ-thāw: respectful term of address to an elderly woman ( \texttt{<} Shan)}, how's it going with you? Didn't you get anything?

4a. I couldn't get anything.

4b. \{Since/Seeing as how\} I'm an old lady I can't move very fast.

4c. My knees hurt.

5. Younger sister, what about you?

6a. As for me, I did get something.

6b. But I still don't have enough to \{live on/eat\}

6c. I don't \{have/own\} any \{trees/plants\} [that produce things to eat].

6d. They're not bearing fruits.

6c. The rats and birds have eaten them all up.

7. So when will it [the rice harvesting] likely be finished?

8a. It's probably not finished yet, unfortunately!

8b. I guess it'll still be about a week before it can all be harvested.

8c. The starlings will eat the paddy too!

9. If [they/we] keep reaping for a week, will it be finished?

10a. It should probably be finished.

10b. But even when we've finished, there'll still be nothing to \{eat/live on\}.

10c. After it's [the reaping] done I guess we'll have to \{thrash the paddy/do
the threshing\}.

11. We'll do the threshing.

12. When the threshing's done where will you [pl.] build your granary?

13a. I'll probably just carry it [the rice] home to store [``put in''].

13b. A little bit each day.

14. Younger sister, have you \{made/built\} [a granary] in the fields?

15a. I guess I'll have to thresh and store [the paddy] in the fields.

15b. \{Seeing that/Since\} I only have my \{son/child\} [to help].

16. Don't you have anybody to \{carry it/do the carrying\}?

17a. I've got nobody to do the carrying.

17b. There are only the two of us.

18a. Well in that case I can't imagine \footnote{dɔ̂ â tɔ̂ʔ: lit. ``think-not-emerge''} how to go and carry it \{to store/for
storage\}\.\footnote{ca  pû  kə: a three-verb concatenation}

18b. \{I'm so tired/It's so tiring\}.

19a. \{For my part/As for me\}, I can manage to \{carry it/do the carrying\}.

19b. Even though I only have two people.

19c. The reason why we can carry it is because we didn't get very much.

20. How many hundred basketsful \footnote{tû \emph{(}Clf) \texttt{<} Shan tuŋ. See DL: 613.} are you likely to get?

21a. I supposed about one hundred basketsful.

21b. I don't see how it'll be large enough for us to live on.

21c. I don't know what went wrong this year \.\footnote{Lit., ``I don't know how it went [this year] like this.''}

21d. When we pile up the paddy we can't even make [proper] rice-\{heaps/stoops/cocks\}
out of it.

21e. They just like sitting dogs \.\footnote{I.e., the paddy just makes miserable little heaps the size of sitting dogs}

22. How many heaps \{will/did\} you get?

23a. Probably only three or four.

23b. Little teeny ones.

24. In that case it should be enough to live on.

25a. I doubt it would be enough.

25b. We'll probably have to go buy [some] in order to eat.

26. So then next year what will you probably do?

27a. Next year \{I'll/we'll\} just have to borrow and borrow \.\footnote{ca-chî-ca-ŋā (ElabV): when used alone, \textbf{chî} means `borrow something}

27b. From our friends and neighbors.

28. Grandma, where would you go to buy [rice] to live on?

29. Dear me, I'd probably just go to buy it from the Northern Thai \.\footnote{Kɔ́lɔ́ gɛ ɔ̄ šɨ: lit. ``where the N. Thai are; in the N. Thai area'';}

30a. Instead of [going] to the Northern Thai, one probably couldn't go to share
with Lahu friends.

30b. \{Since/seeing that\} nobody at all got [enough] rice.

31a. Oh, next year, we'll probably starve to death, eh?

31b. We don't have any money \footnote{The speaker uses the Thai-derived word \textbf{šatâ(n)} for `money' rather} either.

32. If we could only raise pigs and chickens and sell them [to the Thai] to buy
food. \footnote{vɨ̀ lɛ̀ʔ ve: `buy [things] to eat'}

33. I'm just so damn tired of living this way! \footnote{This sentence ends with a string of four exclamatory Pnf's (yâ o nè lê)}

34. There's nothing we can do about it!

35. Because [we're] not very good at \{agriculture/working in the fields/farming\},
since we're all so old.

\texttt{<}Pause\texttt{>}

36. Cà-hɛ \footnote{A man sitting nearby prompts them.}: Na-thî, say that the work is so hard [in the rice swiddens]
that you're hungry for food to go with the rice \footnote{ɔ̄-chî, lit. ``rice lifter'': food that is eaten with one's rice (cf. Thai}, tell your husband to go hunting
for animals to eat.

37. What kind of food have you gotten to eat with your rice this week, my dear?

38a. We didn't get to eat anything at all, because we've been working so hard \.\footnote{I.e., since the work in the rice fields is so demanding, there's no time for}

38b. If I say to my \{old man/husband\}\footnote{\textbf{lɔ̂-pū}, ult. \texttt{<} Chinese 老夫 (Mand. lăofū)} ``Go hunting for something to eat,''
he doesn't go.

39. Didn't you even get to eat monkey-meat?

40. I tell you we didn't get to eat monkey-meat either!

41. And what kind of food have you been eating?

42a. Today I hiked \footnote{khə̀ `march, hike' (prob. Yunnanese Chinese) is brisker in meaning than} down there to the stream where they trade chickens so I
could scoop up crabs!

42b. I got a turtle to eat.

\texttt{<}squeal of laughter\texttt{>}

43. (Cà-hɛ): Tell \{Cà-mɔ̀ʔ's/Jamaw's\} father \footnote{I.e., the husband of the women he's talking to, who is Jamaw's mother. Referring} to go over to Hwēi-tû
next week to hunt wild boar and spend the night there!

44a. I don't know---whether he'd go or not.

44b. The father of my Jamaw is a real \{lazybones/sluggard\}.

45. Just tell him ``If you don't go we won't have anything to eat''---just tell
him ``Hurry up and go, we're starving! We can't hold out anymore!''

46a. He just won't go [do it] for you.

46b. He's like the my own children \footnote{I.e., my own husband.}

46c. Lazy.

46d. He just wants to stay at home, that guy.

47a. You ought to beat him!

47b. That's the way to treat old farts.

48. Tell him the kids are hungry for meat.

49a. You tell me to hit him, but he's stronger than me \footnote{Lit., ``I can't win [against] him.''}!

49b. And I can't go and divorce him either.

50a. An old geezer like that you've got to \{chase/drive\} [into action] by hitting
him!

50b. The type who doesn't take care of somebody like me.

51a. I don't know what to do.

51b. I'm so \{depressed/miserable\}\.\footnote{phɛ̀ʔ bɔ̀ jâ: lit. ``very tired of living''}

52. Since you married him, you're stuck with him \footnote{Lit. ``You can't separate anymore.'' Cf. tà (V) `want to separate (as an}!

53a. One problem is, he doesn't want to separate.

53b. The other problem is, I can't divorce him.

54a. Getting divorced is not according to God's will.

54b. Since you've already married him.

54c. Ever since you were very \{young/small\}.

55. \{That's just it!/You've put your finger on the problem!\}

56a. We've just got to accept it.

56b. There's nothing to be done!

56c. Even if we don't have anything to eat, we'll just starve together.

57. This week I don't know what we'll have to eat.

58. \texttt{<}Jokingly\texttt{>} You'll just have to kill a chicken in the morning
