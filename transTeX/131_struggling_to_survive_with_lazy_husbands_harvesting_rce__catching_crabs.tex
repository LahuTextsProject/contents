
131 Struggling to Survive with Lazy Husbands

-or-

Harvesting Rice, Catching Crabs, and Lazy Husbands

1. My dear \footnote{nò (< PLB \textbf{*nam }⪤\textbf{ *ʔnam} was originally a kinship term for close female relatives, but is now  usable as a friendly term of address to intimates of either sex. See DL:779.}, when you went to spend the night in the fields this week, were
you able to harvest a lot of rice?

2. There was nothing I could get! Since I was carrying my whining baby around! \footnote{\textbf{yâ-pû-yâ-ŋɛ̀ʔ:} 'carry around a whining child; be saddled with the care of many children'. A similar expression is \textbf{yâ-cɨ́-yâ-ŋɛ̀ʔ.} See DL:428, 1255, 1257.}

3. Grandma \footnote{\textbf{mɛ-thāw}: respectful term of address to an elderly woman (< Shan).}, how's it going with you? Didn't you get anything?

4. I couldn't get anything. Seeing as how I'm an old lady, this morning I couldn't
move very fast.

My knees hurt.

5. Younger sister, what about you?

6. As for me, I did get something. But I still don't have enough to live on.  I
don't have any trees

[that have produced things to eat]. They're not bearing fruit. The rats and birds
have eaten them

all  up.

7. So when will it [the rice harvesting] likely be finished?

8. Well, it's probably not finished yet, unfortunately! I guess it'll still be
about a week before it can

all be harvested. The starlings might well eat the paddy [in the meantime] too!
As for [us] people,

if old folks try to go and do it, the harvesting will never get done.

9. If we try to keep reaping for a week, will it be finished?

10. It should probably be finished. [But even] when we've finished, there'll still
be nothing to live on.

After it [the reaping] is over I guess we'll have to do the threshing.

11. We'll do the threshing.

12. When the threshing's done where will you all build your granary?

13. I'll probably just carry it [the rice] home to store. A little bit each day.

14. Younger sister, have you built [a granary] in the fields?

15. I guess I'll have to thresh and store [the rice] in the fields. Since there's
[only] my son [to help].

16. There's nobody to do the carrying then?

17. I've got nobody to do the carrying. There are only the two of us.

18. Well in that case I can't imagine \footnote{\textbf{dɔ̂ â tɔ̂ʔ}: lit. ``think-not-emerge''.} how to go and carry it to store. \footnote{\textbf{ca pû kə}: a three-verb concatenation comprising \textbf{ca} 'go and', \textbf{pû} 'carry', \textbf{kə} 'put in to store'.}
It's so tiring.

19. For my part, I can manage to do the carrying. Even though I only have two people.
The reason

why we can carry it is because we didn't get very much.

20. About how many hundred basketsful \footnote{\textbf{tû} (Clf) < Shan tuŋ. See DL:613.} are you likely to get?

21. I suppose about one hundred basketsful. I don't see how it'll be large enough
for us to live on.

I don't know what went wrong this year .\footnote{Lit. ``I don't know how it went [this year] like this.''} When we pile up the paddy we can't
even

make [proper] rice-stoops \footnote{\textbf{cà-pū}: a pile of harvested paddy. See DL 445.} out of it. They're just like sitting dogs. \footnote{I.e., the paddy just makes miserable little heaps the size of sitting dogs.}

22. How many heaps will you get?

23. Probably only three or four. Little teeny ones.

24. In that case it should be enough to live on.

25. I doubt it would be enough. We'll probably have to go buy [some] in order to
eat.

26. So then next year what will you probably do?

27. Next year I'll just have to borrow and borrow .\footnote{ca-chî-ca-ŋā (ElabV): when used alone, chî means `borrow something that is to be repaid by  an equivalent' (like money or food), whereas ŋā means `borrow something that is to be returned  as such' (like a tool); but in this elaborate expession the two verbs (here translated as `borrow  and borrow') merely reinforce each other.} From our friends and neighbors.

28. Grandma, where would you go to buy [rice] to live on?

29. Dear me, I suppose I'd just go buy it from the Northern Thai .\footnote{\textbf{Kɔ́lɔ́ gɛ ɔ̄-šɨ}: lit. ``where the Northern Thai are; in the N. Thai area''; Cf. \textbf{šɨ} `region; area'  (DL: 1228). It  now appears that \textbf{ɔ̄-šɨ} is best considered to be a unitary locative noun-particle  meaning 'at a place (where)'. This was not recognized in either GL or DL.}

30. Instead of [going] to the Northern Thai, one probably couldn't go to share
with Lahu friends. Since nobody at all got [enough] rice.

31. Well, next year we'll probably starve to death, eh? We don't have any money
