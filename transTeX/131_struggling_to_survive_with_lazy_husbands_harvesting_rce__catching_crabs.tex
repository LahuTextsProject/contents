\setcounter{footnote}{0}

1. My dear/daughter-in-law\footnote{\textbf{nò à} (voc) : \textbf{nò} (< PLB \textbf{*nam}  \~{} \textbf{ *ʔnam}) was originally a kinship term for close female relatives, but is now usable as a friendly term of address to intimates of either sex. See DL:779.}, when you went to spend the night in the fields this week, were
you able to harvest a lot of rice?

2. There was nothing I could get! Since I was carrying my whining baby around!\footnote{\textbf{yâ-pû-yâ-ŋɛ̀ʔ}: `carry around a whining child; be saddled with the care of many children'. A similar expression is \textbf{yâ-cɨ́-yâ-ŋɛ̀ʔ}. See DL:428, 1255, 1257.}

3. Grandma\footnote{\textbf{mɛ-thāw}: respectful term of address to an elderly woman (< Shan).}, how's it going with you? Didn't you get anything?

4. I couldn't get anything. Seeing as how I'm an old lady, this morning I couldn't
move very fast. My knees hurt.

5. Younger sister, what about you?

6. As for me, I did get something. But I still don't have enough to live on. I
don't have any trees

[that have produced things to eat]. They're not bearing fruit. The rats and birds
have eaten them

all up.

7. So when will it [the rice harvesting] likely be finished?

8. Well, it's probably not finished yet, unfortunately! I guess it'll still be
about a week before it can

all be harvested. The starlings might well eat the paddy [in the meantime] too!
As for [us] people,

if old folks try to go and do it, the harvesting will never get done.

9. If we try to keep reaping for a week, will it be finished?

10. It should probably be finished. [But even] when we've finished, there'll still
be nothing to live on.

After it [the reaping] is over I guess we'll have to do the threshing.

11. We'll do the threshing.

12. When the threshing's done where will you all build your granary?

13. I'll probably just carry it [the rice] home to store. A little bit each day.

14. Younger sister, have you built [a granary] in the fields?

15. I guess I'll have to thresh and store [the rice] in the fields. Since there's
[only] my son [to help].

16. There's nobody to do the carrying then?

17. I've got nobody to do the carrying. There are only the two of us.

18. Well in that case I can't imagine\footnote{\textbf{dɔ̂} \textbf{â} \textbf{tɔ̂ʔ}: lit. ``think-not-emerge''.} how to go and carry it to store.\footnote{\textbf{ca} \textbf{pû} \textbf{kə}: a three-verb concatenation comprising \textbf{ca} `go and', \textbf{pû} `carry', \textbf{kə} `put in to store'.}
It's so tiring.

19. For my part, I can manage to do the carrying. Even though I only have two people.
The reason

why we can carry it is because we didn't get very much.

20. About how many hundred basketsful\footnote{\textbf{tû} (Cl\textsubscript{f}) < Shan\textit{ tuŋ}. See DL:613.} are you likely to get?

21. I suppose about one hundred basketsful. I don't see how it'll be large enough
for us to live on.

I don't know what went wrong this year.\footnote{Lit. ``I don't know how it went [this year] like this.''} When we pile up the paddy we can't
even

make [proper] rice-stoops\footnote{\textbf{cà-pū}: a pile of harvested paddy. See DL 445.} out of it. They're just like sitting dogs.\footnote{I.e., the paddy just makes miserable little heaps the size of sitting dogs.}

22. How many heaps will you get?

23. Probably only three or four. Little teeny ones.

24. In that case it should be enough to live on.

25. I doubt it would be enough. We'll probably have to go buy [some] in order to
eat.

26. So then next year what will you probably do?

27. Next year I'll just have to borrow and borrow.\footnote{\textbf{ca-chî-ca-ŋā} (ElabV): when used alone, \textbf{chî} means `borrow something that is to be repaid by an equivalent' (like money or food), whereas \textbf{ŋā} means `borrow something that is to be returned as such' (like a tool); but in this elaborate expession the two verbs (here translated as `borrow and borrow') merely reinforce each other.} From our friends and neighbors.

28. Grandma, where would you go to buy [rice] to live on?

29. Dear me, I suppose I'd just go buy it from the Northern Thai.\footnote{\textbf{Kɔ́lɔ́} \textbf{gɛ} \textbf{ɔ̄-šɨ}: lit. ``where the Northern Thai are; in the N. Thai area''; Cf. \textbf{šɨ} `region; area' (DL: 1228). It now appears that \textbf{ɔ̄-šɨ} is best considered to be a unitary locative noun-particle meaning `at a place (where)'. This was not recognized in either GL or DL.}

30. Instead of [going] to the Northern Thai, one probably couldn't go to share
with Lahu friends. Since nobody at all got [enough] rice.

31. Well, next year we'll probably starve to death, eh? We don't have any money
\footnote{The speaker uses the Thai-derived word \textit{šatâ(n)} for `money' rather than the native Lahu \textbf{phu}, evidently because the villagers' contact with a money-based economy was mostly restricted to their dealings with the Northern Thai.} either.

32. If we could only raise pigs and chickens and sell them [to the Thai] to buy
food.

33. I'm just so damn tired of living this way!\footnote{This sentence ends with a string of four exclamatory Puf's (\textbf{yâ} \textbf{o} \textbf{nē} \textbf{lê}) which follow a concatenation of three verbs (\textbf{phɛ̀ʔ} `exist', \textbf{bɔ̀} `be tired of', \textbf{càʔ} `very')}

34. There's nothing we can do about it!

35. Because [we're] not very good at working in the fields, since we're all so
old.

\direct{Pause}

36. Cà-hɛ\footnote{A man sitting nearby prompts them.}: Na-thî, why don't you say\footnote{'Why don't you' translates the interrogative vocative noun-particle \textbf{à}.} that the work is so hard [in
the fields] that you're hungry for food to go with the rice\footnote{\textbf{ɔ̄-chî}, lit. ``rice lifter'': food that is eaten with one's rice (cf. Thai \textit{kàp-khâaw}, lit. ``with rice, rice go-with.'' Sometimes translated as `curry', but here more generally as `food'.}, and tell your
husband to go hunting for animals to eat.

37. What kind of food have you gotten to eat with your rice this week, my dear?

38. We didn't get to eat anything at all, because we've been working so hard.\footnote{I.e., since the work in the rice fields is so demanding, there's no time for any other kind of female subsistence activities like cultivating vegetable gardens.}
Even if I say to my old man\footnote{\textbf{lɔ̂-pū}, ult. < Chinese \TC{老夫} (Mand. \textit{lăofū}). See DL:1400.} ``Go hunt for something to eat,'' he doesn't go.

39. Don't you even get to eat monkey-meat or anything?\footnote{\textbf{mɔ̀ʔ-mû-mɔ̀ʔ-šā}: `monkey meat or something'. \textbf{mû} is a bound morpheme that occurs in elaborate expressions, sometimes conveying an indefinite meaning like English `or anything'. See DL: 1003-4 where other nuances of meaning are listed.}

40. I tell you we don't even get to eat monkey-meat either!

41. And what kind of food have you been eating?

42. Today I hiked\footnote{\textbf{khə̀} `march; hike' (probably from Yunnanese Chinese) is brisker in meaning than the usual word \textbf{qay} `go'.} down there to the stream called Jungle Chickens' Shelter
\footnote{\textbf{g̈âʔ-pa} \textbf{tê} \textbf{lɔ̀}: apparently this stream was near a place where jungle-chickens took shelter.} so I could scoop up

some crabs! I got a turtle to eat.

\direct{squeal of laughter}

43. (Cà-hɛ): Tell Cà-mɔ̀ʔ's father\footnote{I.e., the husband of the woman he's talking to, who is the mother of \textbf{Cà-mɔ̀ʔ}. Referring to people as the parent of their child (a practice anthropologists call teknonymy) is widespread in SE Asia.} to go over to Hwè-tû next week
to hunt wild boar and spend the night there!

44. I don't know whether he'd go or not. The father of my Cà-mɔ̀ʔ is a real
lazybones.

45. Just tell him ``If you don't go we won't have anything to eat!'' Just tell
him ``Hurry up and go, we're starving! We can't hold out anymore!''

46. He just won't go [do it] for you. He's lazy like the father of my own children.
\footnote{I.e., `my own husband'.} He just wants to stay at home, that guy.

47. You ought to beat him up! That's the way to treat old farts.

48. Tell him the kids are hungry for meat.

49. You tell me to hit him, but he's stronger than me\footnote{Lit., ``I can't win [against] him.''}! And I can't go and
divorce him either.

50. An old geezer like that you've got to drive [into action] by beating him up!
The type who doesn't take care of somebody like you.

51. I don't know what to do. I'm so miserable!.\footnote{\textbf{phɛ̀ʔ} \textbf{bɔ̀} \textbf{jâ}: lit. ``very tired of living''}

52. Since you married him, you're stuck with him\footnote{Lit. ``You can't separate anymore.'' Cf. \textbf{tà} `want to separate (as an unhappy spouse)', DL:594-5.}!

53. One problem is, he doesn't want to separate. The other problem is, I can't
divorce him.

54. Getting divorced is not according to God's will. Since you've already gone
and married him. Ever since you were very young.

55. You've put your finger on the problem!

56. We've just got to accept it. There's nothing to be done! Even if we don't have
anything to eat, we'll just starve together.

57. This week I don't know what we'll have to eat.

58. \direct{Jokingly} You'll just have to kill a chicken in the morning
.\footnote{This is a wry bit of humor. If she had a chicken there would be no problem.}

59. Kill one tonight while you're at it! To eat tomorrow morning.

60. As for me, I don't have anything to eat.\footnote{The speaker pronounces the verb \textbf{cɔ̀}, `have; be there' with exaggerated drawling intonation.} I don't even have a chicken.
When I go to dam [a stream] for fish, I don't catch anything to eat. Since I have
so many children. if I only catch one [fish]

it likely wouldn't be enough to eat!

61. I have a big male pig. I'll slaughter it to feed you. Next week.

62. Even if you kill it to feed us we don't have anything to buy it with! We don't
have any money.

63. I'll give it to you on credit!

64. I don't see any way to get [money].\footnote{To repay the owner of the pig.} If I should get ahold of a baht, I
have to buy food for my kids to eat, clothes for them to wear---

65. [girl] Enough, enough already! \direct{laughs}

