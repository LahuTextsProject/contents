
125 Church Service

Pastor Cà-bo:

\. Hymn Number 218, 218: ``Trusting in the Lord I shall work.''\footnote{This hymn was published by Isaac Watts in 1707 (refrain by Ralph E. Hudson,}

Verse I:

Trusting in the Lord I shall work

I shall teach His words

Indeed I am not ashamed

To tell about His cross

Chorus:

When I approach the Cross I surely see the light.

My worries have completely disappeared.

Since I have faith, eternal life is mine

Henceforth shall I live in joy and gladness.

Verse II:

I rely upon the name of the Lord

And staunchly do I believe in Him

Surely He will not allow

Shame or distress to come to me!

Verse III:

Exceeding strong His promise šÁ-câʔ and His covenant

His words incontrovertible

All that I entrust to Him

He shall nurture faithfully.

Verse IV:

Before the throne of the Father

He will guide me

He will \{allow/suffer/cause\} me to live

In the New Jerusalem.

Number 218---when we have finished looking at it, let us rise...

1\. Brethren one and all, if it were not for the grace of God, we could not exist
like this, in the world, could we!

1\. Because God has redeemed us, we can all now exist in the world.

1\. We cannot see into people's hearts, can we! But He knows, as it is said.

1\. What---er, why is this? [anticipating sentence (18)]

1\. Since you all \{\{are able to/have gotten to\} celebrate/ have celebrated\}
the New Rice Festival, remembering God, and because you have invited them [others]\footnote{I.e., guests from other villages as well as foreigners.}
to come celebrate the new rice, He is extremely pleased.

1\. Why is that?

1\. Because you have not forgotten the grace of God, you are doing it this way,
He says.

2\. He says---that one---When Jesus lived on the earth, He cured the bodies of
ten people with leprosy, didn't he!

2\. Among these ten people---that one---After they had been cured, nine of them
went home immediately, it is said.

2\. Returning home happily to their wives and children, [saying/they said] ``Now
all my disease of leprosy has disappeared!''

22a. As for the other person, although his body was cured, he did not forget Jesus,
so he went to Jesus and sang his praises, it is said.

2\. ``Because you have cured me now, since my body has become well again, I shall
not forget your \{grace/favor\}.''

2\. \{I shall/let me\} praise your grace!''

2\. [Only] after he had praised Him did he go home to his wife and children, and
go on living with them in joy and gladness.

2\. All people like that are very good, it is said.

2\. Whosoever it may be---right?---when others help him, if he does not \{appreciate/remember\}
it, if he can't acknowledge it, that is not good, is it!

2\. That is not a good person.

2\. As for those people who \{are mindful of /acknowledge\} others, they are good.

3\. There were two people among them. [Edit to: ô-thâ chɔ nî g̈â cɔ̀ ve
cê, once there were two people.]

3\. One was a \{person of God/g̈ɨ̀-ša꞊yâ Christian [quote ref. in DL]\},
and the other was not, it is said.

3\. When it was time to eat a meal, the Christian said a prayer.

3\. He praised God.

3\. As for the other person, since he was \{non-religious/a heathen/gentile/note
on \textbf{lɔ̂kī}, DL 1401\}, he did not say a prayer.

3\. So he said to the religious person, ``Why do you pray when you eat?''

3\. He---as for him, he didn't say a prayer.

3\. All by himself he cultivated a paddy field, cultivated a swidden...

3\. \{All by itself/Automatically\} there was food to eat...

3\. Therefore he ate all by himself.

4\. He was not grateful for the grace of God.

4\. ``Why \{are you grateful to/do you remember\} God?'' he said.

4\. He answered, ``Look! Since I am a Christian, when I eat a meal I remember God,''
he said.

4\. As for the other person...he took all the leftover food home, and he brought
all the piglets into his \{home/house\} and gave them [to his family] to eat, all
the piglets he went and brought them to eat.

4\. He said, ``What you give to God is very little.

4\. ``What God gives back is very great,'' he says.

4\. He has given us eyes, ears, vital forces [strength of breath], hands and feet...

4\. And the sun---He has bestowed upon us all the rays of the sun.

4\. All the mountains and the valleys\footnote{qhɔ-qhôʔ-lɔ̀-qhôʔ} [we] happily get to \{behold/look at\}.

4\. For these reasons, we...the grace of God has been exceedingly great.

5\. The thing is, what we give to God is only a little.

5\. What God bestows on us is very great.

5\. And after we die, we there---in Heaven we will live again.

5\. \{Consider/look at\} this, I'm saying...

5\. Now the fact that we are Christians, this is the same as [the story of] the
banana.

5\. One day a certain mother and father brought a cluster of bananas and gave them
to their tiny little daughter.

5\. Then [they] picked a banana and gave it to [their] daughter.

5\. After this they put the clusters back.

5\. Then he [the father] said, ``Daughter, please feed [one to] your daddy.''

5\. She [the daughter]---she didn't give [anything] to him. She didn't want to
give anything to her father.

6\. He asked her a second time, and she just plucked a tiny bit (from the banana-cluster)
for him, her father.

6\. \{Therefore/so\} we Christians are also just like this, are we not? [cê]

6\. That which we give to God is not much.

6\. What the Father gives to us is very much indeed, I am saying [qôʔ ve]

***

6\. Their...what we offer to God now is just a tiny bit. [cê]

6\. Whether it is money or whether it is food, what we offer to God is only a little.

6\. It's not just a question of food or money.

6\. Let us also offer up our hearts completely to God. [qôʔ ve]

6\. After we have offered [ourselves] to God, God will help us, will he not!

6\. God will redeem us.

7\. \{After all/Finally\} God---just as Jesus suffered and died on the cross...

7\. Because he has redeemed us...

7\. We are sinners. [cê]

7\. So, \{just as/given that\} we sin one against the other, if we do not forgive
each other's sins, even if we have been quarreling and bickering, if we do not
forgive one another, it is unpleasant when one person living on this earth sees
another's face.

7\. Therefore, all of us trusting in God, praying to God, whether we have committed
offenses against each other, or quarreled with each other, just as God has forgiven
us our sins, you also [must] forgive each other, your brother. [cê]

Quotatives \textbf{cê}, \textbf{qôʔ ve }used by the pastor to show he's quoting
scripture/the word of God.

***

7\. In the course of this occasion when we have met together, we will now soon
\{break up/separate\}.

7\. Therefore within this [remaining] time, amid everything [that has happened]
to us all\footnote{ɔ̀-ví-ɔ̀-ni tê phā: ``all older and younger siblings''.} in the past days and nights, because of all the things we folk\footnote{chɔ-yâ}
did that we should not have, we have \{separated into factions/become estranged
from one another\}.\footnote{phâʔ-dàʔ-gɨ̂-dàʔ}

7\. There are many cases of people \{accusing/blaming\} each other \{of/for\} offenses,
my brethren...

7\. [But] for a person to forgive another person's offense [is according to] God's
[will]...and is a very good thing.

7\. Since this is the work we Christians have to do, during the time I myself have
spent on this...

8\. There are so many cases of totally trivial matters [becoming] huge causes for
disputes.

8\. But I have also observed many things about my brethren.

8\. However, I have forgiven all these so-called \{offenses/sins\}.

8\. I am not laying blame on any of you brethren for anything.

8\. And if in the future\footnote{cɔ̀ tâ ve ɔ̀-g̈û-šɨ̄: the use of the perfective Pv tā is curious} it should be the will of God that we cannot agree on
a leader,\footnote{A new headman for the village. This was the casus belli. [Story of headman} even if we can't do it, we must go on living together in amity and
act together according to the will of God,\footnote{g̈a te cɔ̀: ``must cause to be \{there/according\}.'' g̈ɨ̀-ša ve alō} and \texttt{<}loud sound of chairs
scraping\texttt{>}

8\. [hastily] And how while we are all gathered together, after we have finished
singing this hymn, we will break up.

