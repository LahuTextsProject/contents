\setcounter{footnote}{0}

Pastor Cà-bo:

1. Hymn Number 218, 218: ``Trusting in the Lord I shall work.''\footnote{This hymn was published by Isaac Watts in 1707 (refrain by Ralph E. Hudson, 1885), under the title ``At the Cross.'' The style is poetic and strikes one now as pleasantly archaic. The Lahu words have nothing to do with the English original, which may be accessed at: http://library.timelesstruths.org/music/At\_the\_Cross\_Hudson/}

Verse I:

Trusting in the Lord I shall work

I shall teach His words

Indeed I am not ashamed

To tell about His cross

Chorus:

When I approach the Cross I surely see the light.

My worries have completely disappeared.

Since I have faith, eternal life is mine

Henceforth shall I live in joy and gladness.

Verse II:

I rely upon the name of the Lord

And staunchly do I believe and trust in Him

Surely He will not allow

Shame or distress to come to me!

Verse III:

Exceeding strong His promise and His covenant

His words incontrovertible

All that I entrust to Him

He shall nurture faithfully.

Verse IV:

Before the throne of the Father

He will guide me

He will allow me to live

In the New Jerusalem.

Number 218---when we have found it, let us rise.

<The following text is a paragraph-by-paragraph translation by a bilingual
Karen woman pastor (Hsəya-ma Hla Yin) of a Karen preacher's sermon at this church
service in Huey Tat village.>

1. Brethren one and all, if it were not for the grace of God, we could not exist
like this, in the world, could we!

2. Because God has redeemed us, we can all now exist in the world.

3. We cannot see into people's hearts, can we! But He knows, as it is said.

4. Why is this?

5. Since you all have gotten to celebrate the New Rice Festival\footnote{\textbf{cà-šɨ́-ɔ̄} \textbf{câ} \textbf{ve}: `celebrating the new rice' (''eating the cooked rice from the new paddy''), a joyous harvest feast observed in both animist and Christian villages in October. See DL 446. ``Them'' refers to guests from other villages as well as foreigners.}, remembering
God, and because you have invited them to come celebrate the new rice, He is
extremely pleased.

6. Why is that?

7. Because you have not forgotten the grace of God, you are doing it this way,
He says.

8. He says that when Jesus lived on the earth, He cured the bodies of ten people
with leprosy, didn't he!

9. Among these ten people, nine of them went home immediately after they had been
cured, it is said.

10. Returning home happily to their wives and children, they said, ``Now my disease
of leprosy has completely disappeared!''

11. As for the other person, although his body was cured, he did not forget Jesus,
so he went to Jesus and sang his praises, it is said.

12. ``Because you have cured me now, since my body has become well again, I shall
not forget your grace.''

13. I shall praise your grace!''

14. [Only] after he had praised Him did he go home to his wife and children, and
go on living with them in joy and gladness.

15. All people like that are very good, it is said.

16. Whosoever it may be, if he does not remember when others have helped him, if
he can't acknowledge it, that is not good, is it!

17. That is not a good person.

18. As for those people who are mindful towards others, they are good...

19. Once there were two people.

20. One was a person of God, and the other was not, it is said.

21. When it was time to eat a meal, the Christian said a prayer.

22. He praised God.

23. As for the other person, since he was a heathen\footnote{\textbf{lɔ̂kī-yâ}: `member of a religious outgroup; worldly person; gentile; heathen'. Ult. < Sanskrit loka `world'. See DL:1401. This term is opposed to \textbf{g̈ɨ̀-ša-yâ} `Christian', lit. ``godly person''.}, he did not say a prayer.

24. So he said to the religious person, ``Why do you pray when you eat?''

25. As for him, he didn't say a prayer.

26. All by himself he cultivated a paddy field, cultivated a swidden.

27. As if automatically there was food to eat.

28. Therefore he ate all by himself.

29. He was not grateful for the grace of God.

30. ``Why do you remember God?'' he said.

31. He answered, ``Look! Since I am a Christian, when I eat a meal I remember God,''
he said.

32. As for the other person, he had a lot of food left over. So he took it all
home, and poured it out for his piglets to eat around his house, and all the piglets
came to eat.

***

33. He says, ``What you give to God is very little.

34. What God gives back is very great,'' he says.

35. He has given us eyes, ears, vital force, hands and feet.

36. And He has bestowed upon us all the rays of the sun.

37. All the mountains and the valleys\footnote{\textbf{qhɔ-qhôʔ-lɔ̀-qhôʔ}} we happily get to behold.

38. For these reasons the grace of God has been exceedingly great.

39. That which we give to God is only a little.

40. What God bestows on us is very great.

41. And after we die, over there, up there in Heaven we will live again.

42. `` Consider this!'' he says.

43. The fact that we are now Christians is the same as the story of the banana.

44. One day a certain mother and father brought a cluster of bananas to give to
their tiny little daughter.

45. So they picked a banana and gave it to their daughter.

46. After this they put the cluster back down.

47. Then he [the father] said, ``Daughter, please feed [one to] your daddy.''

48. She didn't give [anything] to him. She didn't want to give anything to her
father.

49. He asked her a second time, and she just plucked off a tiny bit for him, her
father.

50. So we Christians are also just like this, are we not?\footnote{The pastor repeatedly uses the quotative morphemes \textbf{cê} and \textbf{qôʔ-ve} in the following sentences to indicate that he is quoting scripture. However, many of the occurrences of \textbf{qôʔ-ve} `it is said; he says' in this text are there mainly because the Karen woman is translating what someone else is saying.}

51. That which we give to God is not much.

52. What the Father gives to us is very much indeed, he says.

53. What we offer to God now is just a tiny bit.

54. Whether it is money or whether it is food, what we offer to God is only a little.

55. It's not just a question of food or money.

56. Let us also offer up our hearts completely to God.

57. After we have offered [ourselves] to God, God will help us, will he not!

58. God will redeem us.

59. Jesus Christ sacrificed himself on the cross in order to save us.

60. We are sinners.

61. So, given that we sin one against the other, if we do not forgive each other's
sins, even if we have been quarreling and bickering, if we do not forgive one another,
it is unpleasant when one person living on this earth sees another's face.

62. Therefore, all of us trusting in God, praying to God, whether we have committed
offenses against each other, or quarreled and disputed with each other, just as
God has forgiven us our sins, you also [must] forgive each other, your brethren.

***

63. In the course of this occasion when we have met together, we will now soon
break up.

64. Therefore within this [remaining] time, amid everything [that has happened]
to us all\footnote{\textbf{ɔ̀-ví-ɔ̀-ni} \textbf{tê} \textbf{phā}: ``all older and younger siblings''.} in the past days and nights, because of all the things we folk\footnote{\textbf{chɔ-yâ}: `human beings'.}
did that we should not have done, we have become estranged from one another.\footnote{\textbf{phâʔ-dàʔ-gɨ̂-dàʔ}: (Elab\textsubscript{v}) `break-up a relationship, split up (as a married couple or a village)'.}

65. There are many cases of people blaming each other for offenses, my brethren.

66. [But] for a person to forgive another person's offense [is according to] God's
[will, and is a very good thing.

67. Since this is the work we Christians have to do, during the time I myself have
spent on this...

68. There are so many cases of totally trivial matters [becoming] huge causes for
disputes.

69. But I have also observed many things about my brethren.

70. However, I have forgiven all these so-called offenses.

71. I am not laying blame on any of you brethren for anything.

72. And if in the future\footnote{\textbf{cɔ̀} \textbf{tâ} \textbf{ve} \textbf{ɔ̀-g̈û-šɨ̄}: the use of the perfective P\textsubscript{v} \textbf{tā} is curious here, giving a meaning like ``the fore-ordained future''.} it should be the will of God that we cannot agree on
a leader\footnote{A new headman for the village. This was the casus belli, since the former headman had fallen into disrepute. I was told that his punishment had been to be tied down in the bright sunlight and covered with honey, so that he was exposed to the stinging of voracious red ants.}, even if we can't do it, we must go on living together in amity and
act together according to the will of God, and \direct{loud sound of chairs scraping}

73. [hastily] And now while we are all gathered together, after we have finished
singing this hymn, we will break up the meeting.

74. Number 326. Number 326.

