
Headman Cà-bí (H) and Cà-g̈âʔ (C-)

[Tape VI, Side 1]

1 H: Well, what shall we do with ourselves today, Cà-g̈âʔ ?

2 Cà-g̈âʔ : Why, we can do whatever we please! \footnote{Lit: ``Whatever we do, we do it!``}

3 H: Well then, let's go hunting, shall we?

4 Cà-g̈âʔ: We'll go hunting!

5 H: Over there, above Kɛ̀-pa-tâwʔ \footnote{Kɛ̀-pa-tâwʔ is a Thai village below Huey Tat, in the valley of the same name.} valley, we'll climb up and hunt above
the Pu-pɨ̂ River, okay?

6 Cà-g̈âʔ : I'd like to go, if that's where [we're going]. The squirrels--there
are plenty of squirrels \footnote{fâʔ is really the generic term for all rodents: squirrels, porcupines, badgers, rats, etc. English 'rodent' has quite a different stylistic value, and is avoided in the translation.} there. Lots of barking-deer too.

7 H: If you catch a barking-deer, do you plan to sell the meat?

8 Cà-g̈âʔ : Well, sure, of course I'll sell it.

9 H: How much per kilo would you sell it for?

10 Cà-g̈âʔ : Oh, I probably wouldn't sell it for less than twelve baht a kilo.

11 H: Hm--if you did get a barking-deer, you'd get back your investment in buying
the gun, wouldn't you?

12 Cà-g̈âʔ: Sure, I certainly would get it back.

13 H: Let's hunt then, let's hunt! Why don't you go around [tracking ]\footnote{cɔ e ve lit., 'go around in circles,' 'wind one's way.'} on that
steep slope over there and force them down into that grove of mɛ̀-šâ bamboo.
I'll go and wait down there.

14  Cà-g̈âʔ: Okay, fine. I'll drive them down for you, all right! Hey, Blackie\footnote{His hunting-dog.}!
Hey! Hey!

15 H: I haven't gotten there yet. Don't do it yet! Let me make my way down first.

16  Cà-g̈âʔ: Be careful now. Wow! There sure are lots of tracks [up here]!

17 H: Okay, wind your way downward--down now.

18  Cà-g̈âʔ: Keep right on their trail down there!

19 H: Drive 'em [down], drive 'em [down]!

20  Cà-g̈âʔ: Hey, hey! Hey, Blackie, Blackie! Hey, something's moving down
there! Be careful now!

21 H: Be quiet, be quiet, a barking-deer's coming down now.

22  Cà-g̈âʔ: It's gone, it's gone.

<Noise made by the deer, then the sound of the gun>

23 H: Hey, now it's gone into the river down there--into the river!

24  Cà-g̈âʔ: What is it?

25 H: It's a big buck, I tell you, a buck.

26  Cà-g̈âʔ: Gee, it'll taste great today--when we boil it up and eat it!

27 H: Oho, its hindquarters have gotten stuck right into the river-bed! Come on
down, come on down! Come help me carry my gun. I can't lift it all by myself. \footnote{Because he'll need both hands to carry the barking-deer.}

28  Cà-g̈âʔ: Aw, you carry it by yourself! I can't do it now. I'm too tired.

29 H: Just come on down first, down here.

30 Cà-g̈âʔ : To where?

31 H: It's below where you are now, below the clump of mɛ̀-šâ bamboo, in the
ravine.

32 Cà-g̈âʔ : Below here, huh?

33 H: Yeah, down there, down there.

34  Cà-g̈âʔ: The gulch is deep, you know! I can't make it.

35 H: Come pick your way down the hill over on that side there.

36  Cà-g̈âʔ: By way of that hill, you say?

37 H: Yeah, by that peak--by that dark-colored one.

38  Cà-g̈âʔ: I can't get down that ridge over there! The ground's too steep.

39 H: [Use] the footholds, the footholds \footnote{ɔ̀-qhɛ̀ʔ 'layer; tier; notch'.} -- there are elephant footprints like
steps [going down the slope].

40  Cà-g̈âʔ: If I make a false step, it's right down into the river!

41 H: So look where you're going! You've got eyes.

42  Cà-g̈âʔ: Even with eyes I'm sure I'll fall! The ground's awfully steep.

43 H: You won't fall, you won't fall. Just get to that clump of suckers over there,
nice and slow--

44  Cà-g̈âʔ: You're sure \footnote{š\emph{i} 'know' + ā 'perfective verb-particle, i.e. 'have something known; be sure of something.'} then, are you? That I won't fall.

45 H: Sure I'm sure!

46  Cà-g̈âʔ: You can't be sure! You'll cause the death of a fellow-man!

47 H: You won't die, you jerk! \footnote{The epithet is conveyed by the exaggerated intonation on šɨ 'die'.}

48  Cà-g̈âʔ: Otherwise even if you do get the animal, I won't live to eat it!

49 H: You just stay up there on the mountaintop and wait. I'll carry it up for
you, if that's the way you

want  it.

50  Cà-g̈âʔ: Well now, in that case I'll just sit myself down nicely up there
for you. \footnote{The use of the benefactive verb-particle lâ is ironic here--``I'll do you a favor and sit down!``} Do be careful carrying it up here now, that thing down there! Shall
I climb up there?

51 H: Sit \footnote{mɨ 'sit' is also used in the sense of 'rest.'} and wait, sit and wait! I'll carry it for you, I'll carry it up
for you.

<The deer is brought up the mountain>

52  Cà-g̈âʔ: Wow, this barking  deer weighs as much as 1 khɔ́ \footnote{khɔ́: a measure of weight equal to two tâʔ. See DL:380, 600.} plus 5
baskets of paddy! It's heavy to carry. Isn't that right?

53 H: Yes. It's one khɔ́ plus five basketful.  There's no doubt about it. \footnote{Lit: ``There's no place for saying``, i.e., there's no gainsaying.}

54  Cà-g̈âʔ: How many animals have you ever gotten--ones like this?

55 H: Why, I've been shooting them down like this and eating them [for so long
that] my teeth have turned yellow already!

56  Cà-g̈âʔ: They're yellow because you smoke tobacco, more likely.

57 H: I have tried smoking those great big Gold Flake \footnote{kâ-thɔ̂ (also pronounced klɛ-thɔ̂ʔ) < Thai klɛ̀t thɔɔŋ 'Gold Flake'; a brand of Thai cigarettes.} cigars, you know.

58  Cà-g̈âʔ: Your teeth are yellow because you smoke, I'd say!

59 H: From eating barking-deer meat my teeth are all blocked up--there's white
crud \footnote{qhɛ̂ 'excrement; any refuse, excretion, or dross.'} stuck in the tips of my teeth.

60  Cà-g̈âʔ: That can't be true! I'm awfully hungry today! Where shall we go
to eat? Let's get with it and cut up the meat.

61 H: I wrapped up [something to eat] and carried it along. A big packet of sticky
rice. \footnote{cà-nɔ̂-\emph{ɔ} 'glutinous rice.' Used as provisions on outings away from home, because of its handy compactness.}

62  Cà-g̈âʔ: Give it here! Let's mix it up [with the meat] and eat it together.

63 H: Mix it up to eat, mix it up to eat! Like what's between the legs of Na-šɨ́'s
father! .\footnote{There is a joke here that the Lahu refused to explain to me because of its raunchy nature, and/or because it involved a particular person in the village. Na-šɨ́ is a girl's name. Lahu adults are often referred to by the names of their eldest child, a widespread practice in Southeast Asia known to anthropologists as \textit{teknonymy}.}

<laughter>

64 Cà-g̈âʔ: It probably doesn't taste too good then. If you call your [food]
by that name.

65 H: It tastes good, I tell you! ``Between the legs of Na-šɨ́'s father``
just means a big fucker \footnote{khɔ-nú pàʔ-cɨ́: khɔ-nú is < Thai khənǒm 'cake'. pàʔ-cɨ́ is probably from pàʔ 'copulate' and cɨ́ 'stick together.' Cf. the Mexican slang word chingada, meaning 'a big lump of something' < chingar 'copulate'.} of a cake, that's all.

66  Cà-g̈âʔ: In a few minutes we'll get this meat cut up and we'll cook the
headman's portion .\footnote{ɔ̀-šɛ 'headman's portion of killed game'. Since one of the hunters is the headman himself, they have the right to cook up that tender morsel immediately. See DL:207 and Plate 56.} We'll sink our teeth \footnote{chèʔ câ 'bite and eat.' The hunters' delight at this prospect is understandable in view of the fact that the ordinary Lahu in this village only gets to eat meat a few times a year.} into it!

67 H; Light a fire. Light a fire then.

<A woman who was listening to the taping breaks in>: The headman's
portion belongs to the elders!

68 Cà-g̈âʔ: Shall I make a fire?

69 H: Mm-hm.

70  Cà-g̈âʔ: Gee, I don't have any pine-splinters. \footnote{a-kɨ́: shards of pitch-pine, used for tinder.} It's hard as hell to
get it lit, isn't it?

71 H: Scratch us together some dry bamboo into a pile. Lay a big fire \footnote{bɨ̀-phu 'large fire; bonfire'. à-mī bɨ-phu is how Paul Lewis translates 'fiery furnace' in his translation of the New Testament (1962).}, strike
a match, and set it ablaze.

72  Cà-g̈âʔ: If it's really [done with] dry bamboo it won't taste good at all,
I tell you.

73 H: Don't let it get scorched in the fire. We'll get diarrhea!

\begin{center}
<laughter>
\end{center}

\leftskip=0pt
74  Cà-g̈âʔ: I'll push it in and 'tube-roast' \footnote{càʔ-lÁ vâ ve: 'cook by inserting the food into a bamboo-tube and holding it over a flame'--``tube-roast.`` Opposed to pì ve 'roast by trying to a stick and holding over a flame'--``spit-roast.`` The lively pro-verb vâ, here meaning 'eat,' like the usual word câ 'eat' itself, occurs as the second member of compounds relating to the preparation of food, but is best omitted in translation.} it then. I'll cut off a
mɛ̀-šâ bamboo tube over there--

75 H: Better tie it to a stick and roast it, 'spit-roast' it! On a great big roasting-stick,\footnote{ɔ̀-pì: 'a spit-roast'--the meat together with the stick to which it is attached.}
as long as your arm.

76  Cà-g̈âʔ: When you spit-roast it, the wood makes it stink of soot! If you
\emph{tube}-roast it, then it's fine!

77 H: Spit-roast it! Scrape (lit. 'beat') those embers over there [together into
one place] and spit-roast it.

78  Cà-g̈âʔ: In that case put some wood in, too. Right over there, then.

79 H: Break off some dry mɛ̀-pɔ̀ bamboo \footnote{mɛ̀-pɔ̀: 'huge species of bamboo, edible when young'. See DL:1018.} and put it in. Pick up some dry
mɛ̀-pɔ̀, some dry bamboo and put it in.

<Sound of match being struck>

80a  Cà-g̈âʔ: Well, now, in a little while, it'll catch nicely and we'll have
it cooked up, right?

H: Mm-hm.  Cà-g̈âʔ: When we're through eating in a few minutes, how shall we
go back [to

the  village]? Why don't we do some more hunting on our way back?

81 H: Let's go back hunting over there outside of Pa-ho village \footnote{Pa-hôʔ: a Thai village near Huey Tat. See DL:803.}, in that narrow
valley--over in those rattan thickets there.

82  Cà-g̈âʔ: But if we're carrying the meat like this, we won't be able to
go through any rattan thickets!

83 H: [If you don't want to hunt anymore on the way back] you go on up there to
the [easy] road along the mountain, while \textit{I} do some sneaky shooting \footnote{jâʔ bɔ̂ʔ ve 'shoot stealthily', i.e. to hunt on noiseless feet.}
down below--then I'll pick up [what I've shot] and carry it up to you!

84 Cà-g̈âʔ : Ha! The way you do it all you can catch is birds! \footnote{Lit: ``if it isn't birds, you don't catch anything to eat.`` A good translation might be 'your shooting is for the birds!'}

85 H: I'll catch him, all right. [There's an animal / He's] sleeping down there
at the river. Big as life, with a hard-on \footnote{nī-qhɛ̀ʔ pé ɛ̀ te ve 'to have an erection'. Used also to mean 'obvious, very much in evidence, as hard to ignore as an erected penis.'}!

<laughter>

86  Cà-g̈âʔ: You've got a foul mouth! \footnote{Lit: ``It's too much, your [speech].``}

87 H: Don't I though!

88  Cà-g̈âʔ: Animals have sharp ears. Things like barking-deer [run away at
the first sound]--because they don't have gall-bladders.\footnote{According to traditional Lahu belief, the gall-bladder is the seat of courage in men and animals.}

89 H: It's still way over there, the mountain after next. It's still far. He probably
can't hear us.

90 Cà-g̈âʔ : The mountain after next in that direction, eh?  H: Mm-hm. Cà-g̈âʔ:
Nope. I bet you won't

be able to go quietly enough.

91 H: I'll flit along softly on my little tippy-toes \footnote{Lit: ``sneaking on tip-toes I'll fly through the air.`` mû-phe 'the region above'}!

\begin{center}
<laughter>
\end{center}

\leftskip=0pt
92  Cà-g̈âʔ: Then your legs will get numb on you.

93 H: I'll twist my steps as I walk on tiptoes. So that I practically turn my [calf-]muscles
inside out.

94  Cà-g̈âʔ: You can't almost turn your muscles inside out that way!

95 H: Sure, I can actually \footnote{The meaning 'actually' is conveyed by the verb-particle šɔ̄ 'still' i.e., ``no matter what you say.``} get my calf turned around to the front [of my leg]!
You just watch as I go twisting along!

\begin{center}
<laughter>
\end{center}

\leftskip=0pt
96  Cà-g̈âʔ: What you say can't be true.\footnote{Lit. ``It is not to such an extent, your [words].``}

97 H: I'll go carefully this time, I tell you.

98  Cà-g̈âʔ: Well then, I'll try going [my way too], right? So I'll climb up
there to the mountain slope, up along the road, and I'll wait for you up there
then, okay?

99 H: Watch me carefully. I'll twist along \footnote{šɨ́ phɛ̂ ``twist forth`` is slang for 'go.'} till I get under that banyan down
there.

100 Cà-g̈âʔ : Right. So I'll sit and wait for you up on that upper ridge then.

101 H: A while back I-kha's father went down there, and a barking-deer jumped out
and ran away!

102  Cà-g̈âʔ: Down there, eh?

103 H: Yeah, over there--Hey, didn't you hear a shot \footnote{vâ is here used as a pro-verb for bɔ̂ʔ 'shoot,' as is clear from qhɛ, the classifier\textbf{ }for gunshots. The particle qha-pâʔ is used especially when auditory perception is at issue. See DL:272.} just now down at the
foot of the mountain?

104  Cà-g̈âʔ: Well, since I'm high up here I can't hear too well, you know.

105 H: Look over there below the mountain with the grove of \textit{mɛ̄-pɔ̂
}-- I mean \textit{mɛ̀-cû-šɛ́} trees! Shots are being fired on that barren
mountain there.

106  Cà-g̈âʔ: What are they shooting at then?

107 H: They're hunting squirrels! A little bunch of miserable \footnote{chɔ lù-kɨ̂ 'wretches'; lit. ``ruined and rotten people``.} young squirts
