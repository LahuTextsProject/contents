
Legitimate Headman Cà-bí \& Cà- âʔ[Tape VI, Side 1]

1 H: Well, what shall we do with ourselves today, Jaga?

2 C- : Why, we can do whatever we please!\footnote{Lit: ``Whatever we do, we do it!``}

3 H: Well then, let's go hunting, shall we?

4 C- : We'll go hunting!

5 H: Over there, above Keh-pa-tau\footnote{Kehpatauʔ is a Thai village below Huey Tad, in the valley of the same name.} valley, we'll climb up and hunt above the
Pu-pui River, okay?

6 C- : I'd like to go, if that's where [we're going]. The squirrels--there are
plenty of squirrels\footnote{fâʔ is really the generic term for all rodents: squirrels, porcupines, badgers, rats, etc. English 'rodent' has quite a different stylistic value, and is avoided in the translation.} [there]. Lots of barking-deer [too].

7 H: If you catch a barking-deer, do you plan to sell the meat?

8 C- : Well, sure, of course I'll sell it.

9 H: How much per kilo would you sell it for?

10 C- : Oh, I probably wouldn't sell it for less than twelve baht a kilo.

11 H: Hm--if you did get a barking-deer, you'd get back your investment in buying
the gun, wouldn't you?

12 C- : Sure, I certainly would get it back.

13 H: Let's hunt then, let's hunt! Why don't you go around [tracking]\footnote{cɔ e ve lit., 'go around in circles,' 'wind one's way.'} on that
steep slope over there\footnote{cô ɔ̀-nu \emph{ɔ} ``over there at that other place.``} and force them [the animals] down into that grove of
meh-sha bamboo. I'll go and wait down there.

14  : Okay, fine. I'll drive them down for you, all right! Hey, Blackie\footnote{His hunting-dog.}! Hey!
Hey!

15 H: I haven't gotten there yet. Don't do it yet! Let me make my way down first.

16  : Be careful now. Wow! There sure are lots of (animal) tracks [up here]!

17 H: Okay, wind your way down--down now.

18  : Keep right on their trail [when they get down] there!

19 H: Drive 'em [down], drive 'em [down]!

20  : Hi, hi! Hi, Blackie, Blackie! Hey, something's moving\footnote{lɔ̀-qÁ-q\emph{o} ``river-hollow.`` Glossed in my notes as 'river-bank,' but I doubt this. CHECK.}! ...\footnote{lɔ̀-qhâ-q\emph{o} ``river-\{path / line\}-hollow.``}

21 H: \{Take it easy, take it easy / Be quiet, be quiet\}, a barking-deer's coming
down now.

22  : It's gone, it's gone.

[Noise made by the deer: Then the sound of the gun: ]

23 H: Hey, now it's gone into the river down there--into the river!

24  : What is it?

25 H: It's a big buck, I tell you, a buck.

26  : Gee, it'll taste great today--when we boil it up and eat it!

27 H: Oho, its hindquarters have sunk right into the river-bed\footnote{lɔ̀-qÁ-q\emph{o} ``river-hollow.`` Glossed in my notes as 'river-bank,' but I doubt this. CHECK.}! Come on down,
come on down! Come help me carry my gun. I can't lift him all by myself.

28  : Aw, you carry it [the gun] by yourself! I can't do it now. I'm too tired.

29 H: Just come on down first, down here.

30  : To where?

31 H: It's below where you are now, below the clump of mi-sa bamboo, in the \{ravine
/ river-bed / gulch\}.\footnote{lɔ̀-qhâ-q\emph{o} ``river-\{path / line\}-hollow.``}

32  : Below here, huh?

33 H: Yeah, down here, down here.

34  : The gulch is deep, you know! I can't make it.

35 H: Come pick your way down the red \{outcropping / peak\}\footnote{qhɔ-cu 'mountain-rounded part'} over on that side
there.

36  : By way of the red peak, you say?

37 H: Yeah, by that red peak--by that dark-colored one.

38  : I can't get down that precipice\footnote{qhɔ qɔ̂-lɔ̂ 'a steep mountain.'} over there! The ground's too steep.\footnote{mì h\emph{a} lit., ``ground is hard.``}

39 H: [Use] the footholds, the footholds\footnote{ɔ̀-qhɛ̀ʔ 'layer; tier; notch'}--there're elephant footprints going
around\footnote{tɔ̀-tɔ̀-nɔ̂ʔ 'all around.' CHECK. The meaning 'a step' (ɔ̀-tɔ̀-nɔ̂ʔ) is also possible: ``There are steps--elephant footprints.``} [the slope].

40  : If I make a false step, it's right down into the river!

41 H: So look where you're going! You've got eyes.

42  : Even with eyes I'm sure I'll fall! The ground's awfully steep.

43 H: You won't fall, you won't fall. Just get to that clump of trees over there,
nice and slow--

44  : You're sure\footnote{š\emph{i} \emph{a} 'know' + perfective particle. ``To have something known`` = be sure} then, are you? That I won't fall.

45 H: Sure I'm sure!

46  : You can't be sure! You'll cause the death of a fellow-man!

47 H: You won't die, you jerk.\footnote{The epithet is conveyed by the intonation on š\emph{i}.}

48  : Otherwise even if you get the animal, I won't live to eat it!

49 H: You just stay up there on the mountaintop and wait. I'll carry it up for
you, if that's the way you want it.\footnote{qhe qo lɛ̀ ``in that case.``}

50  : Well now, in that case I'll just sit myself down nicely up there for you.\footnote{The use of the benefactive Pv lâ is ironic here--``I'll do you a favor and sit down!``}...
Do be careful carrying it up here now--you down there! Shall I climb up there?

51 H: Sit\footnote{mɨ 'sit' is also used in the sense of 'rest.'} and wait, sit and wait! I'll carry it for you, I'll bring it up for
you.

[The deer is brought up the mountain]

52  : Wow, this barking  deer weighs as much as 1taʔ\{plus / and\} 5 baskets\footnote{One tâʔ is about 250 lbs. There are \_\_\_ baskets in a tâʔ. cf. TABLE OF WEIGHTS AND MEASURES, p. \_\_\_.}
of paddy! It's heavy to carry. Isn't that right?

53 H: Yes. It's 1 khɔ́\footnote{Cà- âʔ possibly misspoke himself and meant to say khɔ́. See speech 53.} and 5 qhɔ̀ʔ-lɔ́(or ɔ̀) There's no doubt about
it.\footnote{Lit: ``There's no place for saying`` = there's no gainsaying.}

54  : How many animals have you ever gotten--ones like this?

55 H: Why, I've been shooting them down like this and eating them [for so long
that] my teeth have turned yellow already!

56  : They're yellow because you smoke tobacco, more likely.

57 H: I have tried smoking those great big gold city\footnote{kâ-thɔ̂ < Thai kept thɔɔŋ CHECK THIS + ALSO THE GENERAL MEANING OF 57.} cigars, you know.

58  : Your teeth are yellow because you smoke, I'd say!

59 H: From eating barking-deer meat my teeth are all blocked up--there's white
crud\footnote{qhɛ̂ 'excrement; any refuse, excretion, or dross.'} [stuck] in the tips of my teeth.

60  : \{That can't be true / You're full of it\}.... I'm awfully hungry today!
Where shall we go and eat? Let's get with it and cut up the \{meat / carcass\}.

61 H: I wrapped up [something to eat] and carried it along. A big package of sticky
rice.\footnote{cà-nɔ̂-\emph{ɔ} 'glutinous rice.' Used as provisions on outings away from home, because of its handy compactness.}

62  : Give it here! Let's mix it up [with the meat] and eat it together.

63 H: Mix it up to eat, mix it up to eat--it's a phÁ-k\emph{a} like Na-suh's father
made.\footnote{There is a joke here that the Lahu refused to explain to me because it refers to a particular incident that happened in the village. The meaning of phÁ-k\emph{a} is unknown to me. Na-suh is a girl's name. Lahu adults are often referred to by the names of their oldest(?) child [teknonymy].}

[laughter]

64  : It probably doesn't taste too good then. If you call your [food] by that
name.

65 H: It tastes good, I tell you! ``Na-suh's father's phÁ-k\emph{a}``
means a big fucker\footnote{khɔ-nú pàʔ-cɨ́: khɔ-nú is < Thai khənǒm 'cake.' pàʔ-cɨ́ is from pàʔ 'copulate' and cɨ́ 'stick together.' cf. the Mexican slang word chingada, meaning ``a big lump of something.``} of a cake, that's all.

66  : In a few minutes we'll get this meat cut up and we'll cook the headman's
portion.\footnote{ɔ̀-š\emph{ɛ} \textasciitilde{} ɔ̀-šɛ See \_\_\_\_\_. Since one of the hunters is the headman himself, they have the right to cook up that tender morsel immediately.} We'll sink\footnote{chèʔ câ 'bite and eat.' The hunters' delight and this product is understandable in view of the fact that the ordinary Lahu in this village only gets to eat meat a few times a year.} our teeth into it!

67 H; Light a fire. Light a fire then.

[67a A woman who was listening to the recording breaks in: The headman's portion
belongs to the elders!]

68  : Shall I make a fire?

69 H: Mm-hm.

70  : Gee, I don't have any pine-splinters.\footnote{a-kɨ́: shards of pitch-pine, used for tinder.}... It's hard as hell to get it
lit, isn't it?

71 H: Scratch us together some dry bamboo into a pile. Say a big fire\footnote{bɨ̀-phu: à-m\emph{i} bɨ-phu is how Lewis translates 'furnace' in the Bible.}, strike
a match, and set it ablaze.

72  : If it's really [done with] dry bamboo it won't taste good at all, I tell
you.

73 H: Don't let soot collect on it. We'll get diarrhea!

[laughter]

74  : I'll push it in and 'tube-roast'\footnote{\{lÁ / càʔ-lÁ\} ve: to cook by inserting into a bamboo-tube and holding over a flame--``tube-roast.`` Opposed to pì ve 'to roast by trying to a stick and holding over a flame'--``spit-roast.`` See \#75.  31b. The lively pro-verb vâ, here meaning 'eat,' like the usual word câ 'eat' itself, occurs as the second member of \textbf{opds} relating to the preparation of food, but is best omitted in translation.} it then. I'll cut off a misa-bamboo
tube over there--

75 H: (Better) tie it to a stick and roast it, 'spit-roast' it! On a great big
roasting-stick,\footnote{ɔ̀-pì: 'a spit-roast'--the meat together with the stick to which it is attached.} as long as your arm.

76  : When you spit-roast it, the wood makes it stink of soot\footnote{šɨ̂ʔ-mu ``wood-hair?`` = soot from burning wood. CHECK.}! If you \emph{tube}-roast
it, then it's fine!

77 H: Spit-roast it! Scrape (lit. 'beat') those embers over there [together into
one place] and spit-roast it.

78  : In that case split some wood in, too. Right over there, then.

79 H: Break off some day meh-paw wood\footnote{mɛ̀-pɔ̂: kind of a perennial tree with jute-like bark and leaves like the papaya. Rope is made from the bark. (H.Y.)} and put it in. Pick up some dry meh-paw,
some dry bamboo and put it in.

[Sound of match being struck]

80a  : Well, now, in a little while, it'll catch nicely and we'll have it cooked
up\footnote{The completed action Pv ò is preserved by the English 'have it V'ed.'}, right?

81b H: Mm-hm.

81c  : When we're through eating in a few minutes, \{how / which way\} shall we
go back [to the village]? \{Why don't we / lit. ``won't we``\}
do some more hunting on our way back?

81 H: Let's go back hunting over there outside of Pa-ho village, in that narrow
valley--over in those rattan thickets there.

82  : (But) if we're carrying the meat like this, we won't be able to go through
any rattan thickets!

83 H: [If you don't want to hunt anymore on the way back] you go on up there to
the [easy] road along the mountain, while \textit{I} do some sneaky shooting\footnote{jâʔ bɔ̂ʔ ve 'to shoot stealthily' = to hunt on noiseless feet.}
down below--then I'll pick up [what I've shot] and carry it up to you!

84  : Ha! The way you do it all you can catch is birds!\footnote{Lit: ``if it isn't birds, you don't catch anything to eat.`` A good translation might be 'your shooting is for the birds!'}

85 H: I'll catch him, all right. [There's an animal / He's] sleeping down there
at the river. Big as life, with a hard-on\footnote{n\emph{i}-qhɛ̀ʔ pé ɛ̀ te ve 'to have an erection.' Used also to mean 'obvious, very much in evidence, as hard to ignore as an erected penis.'}!

[laughter]

86  : You've got a foul mouth!\footnote{Lit: ``It's too much, your [speech].``}

87 H: Don't I though!\footnote{This could also be taken to mean ``What I say is true!``}

88  : Animals have sharp ears.\footnote{$V_h$ + pɨ́ often means 'to be good at V'ing.'} Things like barking-deer [run away at the first
sound]--because they don't have gall-bladders.\footnote{According to traditional Lahu belief, the gall-bladder is the seat of courage in men and animals.}

89 H: It's still way over there, the mountain after next. It's still far. He probably
can't hear us.

90a  : The mountain after next in that direction, eh?

90b H: Mm-hm.

90c  : Nope. I bet you won't be able to go \{quiet / stealthily\} enough.

91 H: I'll flit along softly on my little tippy-toes\footnote{Lit: ``sneaking on tip-toes I'll fly through the air.`` mû-phe 'the region above'}!

[laughter]

92  : Then your shins will get numb on you.

93 H: I'll twist my steps as I walk on tiptoes. So that I practically turn my [calf-]
muscles inside out.

94  : You can't almost turn your muscles inside out that way!

95 H: Sure, I can actually\footnote{The meaning 'actually' iš conveyed by the Pv šɔ 'still'--i.e., ``no matter what you say.``} get my calf turned around to the front [of my leg]!
You just watch as I go twisting along!

[laughter]

96  : What you say can't be true.\footnote{Lit. ``It is not to such an extent, your [works].``}

97 H: I'll go carefully this time\footnote{'this time' translates the afterthought chi ɔ̀ʔ.}, I tell you.

98  : Well then, I'll try going [my way too], right? So I'll climb up there to
the mountain slope, up along the road, and I'll wait for you up there then, okay?

99 H: Watch me carefully. I'll \{roll / twist\} along\footnote{šɨ́ phɛ̂ ``twist forth`` is slang for 'go.'} till I get under that
banyan down there.

100  : Right. So I'll sit and wait for you\footnote{mɨ a lâ 'sit and wait for you': mɨ 'sit,' a = ta 'action extended in time,' lâ 1 --> 2 benefactive Pv.} up on that high\footnote{ɔ̀-ma qhɔ qɔ̂-lɔ̂ 'the upper mountain-slope.'} slope then.

101 H: A while back I-kha's father went down there, and a barking-deer jumped out
and ran away!

102  : Down there, eh?

103 H: Yeah, over there--Hey, didn't you hear a shot\footnote{vâ is here used as a pro-verb for bɔ̂ʔ 'shoot,' as is clear from qhɛ, the \textbf{cef }for gunshots. The particle qha-pâʔ is used especially when auditory perception is at issue. See DL:272.} just now down at the spur
of the mountain\footnote{qhɔ-mɨ̂ 'mountain-tip, mountain-end.' This word apparently means 'summit' sometimes. CHECK.}?

104  : Well, since I'm high up here I can't hear too well, you know.

105 H: I'm sure there's shooting down there, coming from that clump of meh-paw\footnote{See note \#34.}
or meh-ju-sheh trees on the barren mountain.\footnote{It's not clear whether H. meant there were both kinds of trees there, or whether mɛ̀-pɔ̂ was a slip of the tongue which he corrected to mɛ̀-cu-šɛ́. This sentence is extremely difficult and presents several unsolved problems. CHECK.}

106  : What are they shooting at then?

107 H: They're hunting squirrels! A little bunch of miserable young squirts\footnote{chɔ lù-kɨ̂ 'ruined and \textbf{rotten} people = wretches.  yâ-qɛ̀ʔ-ku: an insulting term for an adolescent.}
down there.

108  : Who are they then, the ones down there? How many of them are there?\footnote{ɔ̀-mɛ qhà-nî  â le: lit. ``how many names are there?``}

109 H: It's Ja-ha and his crowd and that oaf Thu-yi and his crowd.\footnote{The pluralizer hɨ is usable after proper names. cf. Jse. Tanaka-san-tati 'Tanaka and his group/friends.' qa is an uncomplementary prefix to names. Thû-yì is actually one of the} They're
mumbling away about something down there.[57]

110  : What \{would you say / do they say\} they're up to, those guys?\footnote{te câ ``do to eat``--``to be successful in hunting.`` do for an envisaged advantage--i.e., 'be up to.'}  What
are they looking for--

111 H: The barking-deer jumped in right among them, and they didn't even see it!
It ran away down [the slope]!

112  : They won't get anything to eat.\footnote{The standard Lahu phrase for ``to be unsuccessful in hunting.``}

113 H: They probably won't now. That bunch of kids are awfully lazy.

114  : How many of them \{would you say / do they say\}\footnote{This may be a peculiar use of the quotative pvf cê--it seems to refer to an anticipated 'quotation' to be supplied by the second person. Or maybe it's a reminder that Cà- âʔ has already asked this question, ``How many have come, I asked?!`` Either of these senses may be the one intended in \#110.} have come? Are there
any game-drivers among those who've come down there?

115 H: I saw some drivers just now driving [the game] down from an old field up
over there.

116  : Up over there, huh?

117 H: Mm-hm

118  : Who are they, the game-drivers?

119 H: \{They say / It looks like\} the ones down there who are driving the game
are north-country A-leh\footnote{mə́-nə́ A-lɛ́: mə́-nə́ < Thai myaŋ nyÁ 'north-country.' 'A-leh' is the designation of a Lahu sub-tribe--exact scope unknown.} folk. The A-leh are supposed to have black tips to
their turds, you know.

120  : Come on! You're really too much,\footnote{Lit: ``It is already exaggerated, then.``} you know....

[They fall into a story-telling mood]

Once upon a time there were some people--er--a father-in-law and a son-in-law.
This man got himself a son-in-law, see. He was very happy [about it]. Well, so
he cuts off this great big hunk of meat and boils it--[but] in this meat there
was a little bit of gristle, see?

121 H: Uh-huh.

122  : Well, here they serve it up see, after the wedding, they put it down and
here they are eating and this son-in-law he's sitting right next to his father-in-law,
see. So he takes and bites off this great big piece, the biggest piece of all--and
it happens to be the gristle!

123 H: The gristle, huh?

124  : Right. He wants\footnote{Lit: ``when he \{would (in volitional sense) / tries to\} throw it away again, he can't throw it away.``} to \{spit it out / throw it away\}, but he can't. He's
ashamed before his father-in-law. So he takes and bites into it with all his might,
a little too violently, and the piece of meat rips off and his elbow pokes his
father-in-law in the forehead, right between the eyes, and the father-in-law keels
over! Nobody could do a thing.

125 H: Gee, when you tell this [story] it reminds\footnote{Lit: ``when (you) say this, I had (sthg) once a long time ago.``} me of something that happened
to me once a long time ago.

126  : What was it?\footnote{Lit: ``how was it?``}

127 H: Once when I went over to Meu-heh and Meu-khaw, they boiled up some chicken
to feed me and they put in about three fingers' worth of wild ginger. When I took
and chewed it, I passed the hell out\footnote{chɔ-khɔ̂ mâ nɔ̂: lit. ``not be aware of human speech``. Means either 'be unconscious', or 'be unreasonable'.}!

[laughter]

128  : Then there's a story\footnote{'story' is conveyed by the quotative Pvf cê.} about another person who passed out. He was riding
along on a bicycle. He was headed for a government office building\footnote{šÁ-nÁ-yɛ̀: šÁnÁ < Bse. '\textbf{gout }office,' yɛ́ < Lh. 'house.'} someplace.\footnote{ô \emph{ɔ }qhe ``like over there.``}
Well, on the way to the government office, there were paddy-field dikes,\footnote{ti-mi ŋ-tɛ̂: raised earth-boundaries around an irrigated paddy-field, serving to keep the water from running out.  72. yàʔ-qɔ na pÁ ``the upper part of the road``--i.e., the part the man on the bike had already \textbf{taused}--the part behind him.} see.
\{It was a road that other people didn't go on much / There was nobody else on
the road / It was an unusual route\}. Well, there was this buffalo-herder up behind[72]
him on the road. From where he was behind him, he calls out to this old fellow
[on the bicycle]: ``Where are you going?`` he says. Now [the guy]
happens to be right near the water [in a paddy field], see. So, when he turns around
to look up the road, the bicycle goes right down into the water. [The buffalo-herder]
had just said to him, ``Where are you going?,`` right? So he answers,
``To the off--``\footnote{The original is in Thai: ``paj am--.`` He doesn't have time to finish his phrase ``paj amphəə`` ꞊ ``(I'm) going to the government office.``}

129 H: ``To the off--?``

130  : Yeah, he didn't have time to say 'office,' so he could only say 'off--'
before he sank into the water!

131 H: I see.

132  : Well, the bicycle was swallowed up down there. He couldn't get it back again.
All he could get out was his hat, so he went back home and changed his \{shirt
/ clothes\}. Then he went off again.

133 H: I had a time like that once too. [I was walking] someplace when I slipped
and fell head over heels down off the edge of the path\footnote{This was a serious matter, since he could have fallen several hundred feet to his death.}!

134  : (laughing) Wow, that was too much, something like that!

135 H: When I looked around afterwards there I was smack under a clump of jujubes\footnote{The jujube-branches broke his fall.}!
To think I'd landed down there!

136  : That was really too much!

137 H: I was worried that somebody might see me screwing my way back up\footnote{<qɔ̀ʔ pàʔ qɔ̀ʔ la>: thiš pàʔ is probably to be identified with the verb 'to copulate'--it is here being used as an obscene pro-verb in an <elab>.} [to
the road], so I climbed up in a hurry, hauled myself up and set myself [on the
road] again, then I headed the hell back for home and reached my house. That really
happened to me.

138  : Nothing like that's ever happened to me yet.

139 H; Huh?

140  : I said, that has never happened to me yet.

141 H: It really happened. I skidded and fell downhill, by God. The safety-fence\footnote{kɨ̀-tà: tà ꞊ 'stick,' kɨ̀ ≠ lac-wood? Sticks planted by the side of steep mountain paths. CHECK}
went smash too, I tell you. Against my shins as I fell down through it!

142  : You must've been \{happy / carefree\} in those days!

143 H: Very happy.

144  : Well, how did \textit{you}\footnote{The $P_n$ mɛ is a story topicalizer.} court the girls in those days, then?

145 H: I'd buy a pack of cigarettes someplace, and I'd buy some perfumed powder,\footnote{phû-də hɔ́: phû-də < Bs < Eng 'powder.' hɔ́ < Shan 'sweet-smelling'}
and I'd buy a box of matches, then I'd go scratching my matches down in the middle
of the village, making big red flames. Then the girls would come and say ``Give
me a smoke,\footnote{tɔ ve: The causative of dɔ̀ ve 'to drink, to smoke.'} give me a smoke!``

146  : Then how would you give them a smoke?

147 H: I'd say to them ``Come on over there, let's go over there.``
When we got there I'd give each one of them a cigarette. Then I'd say, ``I've
still got lots more of these, if you'd like to smoke again. Let's go over there....``

148  : Aha. Well, then there's [a story about] another person. In this story it
seems there was a certain woman. She got herself married, and her man was as happy
as could be. (But) when the people were fed [at the wedding feast], he didn't eat,
see. When night fell he said to his wife, ``Give me \{something to eat
/ my rice\}.`` (But) there was no \{food / rice\}. Well, this fellow all
of a sudden took up a--what do you call it--er, an earthen (rice-) pot, see, and
stuck his hand in, and fished around in it for rice with his hand--with his fist.
And it got stuck in there! Fishing around for rice it got stuck, and no matter
what he did he couldn't get it out again. The fellow had a bald head, and the moonlight
outside was shining, and the moonbeams struck his head, you see. As it was hitting
his head, as the light was shining on it, [his wife]\footnote{This story is poorly told. The only way any sense can be made out of it is to assume that 'his wife' is the subject of this clause.} took [the pot] that was
stuck [on his hand] and bashed him on the head with it. She thought it [= his head]
was a rock, you see. And boy, were they a couple that was sorry afterwards!\footnote{bo te ve means 'to repent' among its other shades of meaning.}--Let's
stop here, okay?\footnote{Cà- âʔ is tired of talking into the tape-recorder.} It's enough now, isn't it?

149 H: I don't think it's enough yet. I'm sure there's a lot more [we could say].

150  : I bet it's enough, as much as we've done.

151 H: Hurry and keep going!

152  : My throat is killing me. I want to have a cigarette now. It's enough, okay?

153 H: Do you know what is meant by ``Red Lahu`` and ``Lahu
meu-neu``?\footnote{``If you know when we say 'Red Lahu' and 'Lahu men-ren,' how do we mean?``}

154  : How do you mean?

155 H: When we say 'Red Lahu' and 'Lahu meu-neu.'

156  : Isn't it because the Red Lahu have red stripes on the skirts and jackets
that they wear? So they're called 'Red Lahu.'

157 H: Well, what about 'meu-neu,' then? 'Meu-neu.'

158  : This calling them 'meu-neu' is supposed to be because they live up north-that's
why they're called 'meu-neu' (=North-Country).

159 H: I see. Then what about the ones called ``A-leh``?

160  : [They're called] A-leh because they're supposed to be more ca-\textit{leh}-ver
than others!\footnote{The Lahu original has the pun lɛ̂ʔ 'clever' (< Shan) / a-lɛ́.}

161 H: Oh.

162  : So other people call them ``A-leh.`` Because whatever they
do they're ca-\textit{leh}-verer at it than anybody else.

163 H: They're supposed to be cleverer?

164  : Yes.

165 H: So that's why people give them that name, according to you!\footnote{This sentence and the next are far from clear. The implication seems to be that the A-leh are a sub-tribe of Black Lahus.} Their shit\footnote{qhɛ̂-mɨ̂: lit., 'turd-tips'}
is black,\footnote{nâʔ is used for both brown and black, indeed for any dark color.} too. Now I've also heard it said that the Red Lahu have red shit.
I wonder if it \textit{is }red, like they say!

166  : I daresay yours is black, too, your shit.

167  H: I've never actually picked it out to look at it--my shit-tips....

[laughter]

168  : In that case, do take a good look at it!

169 H: Well, I suppose it is black, at that. It's probably all quite black.

170  : Mine is certainly black--Come on, now, that's enough already!

171 H: What--do you want to have a smoke?

172  : Yeah, that's enough already!

173 H: I've already smoked a big cigar. Aren't you smoking?

(A voice) 174 There aren't even any cigarettes.

175  : Enough now, I tell you!

176 H: You've had enough?

177  : It's awfully long already.

178 H: Oh, if you've had enough that's all right.

