\setcounter{footnote}{0}


1. H: Well, what shall we do with ourselves today, Cà-g̈âʔ ?

2. Cà-g̈âʔ : Why, we can do whatever we please!\footnote{Lit: ``Whatever we do, we do it!''}

3. H: Well then, let's go hunting, shall we?

4. Cà-g̈âʔ: We'll go hunting!

5. H: Over there, above Kɛ̀-pa-tâwʔ\footnote{\textbf{Kɛ̀-pa-tâwʔ} is a Thai village below Huey Tat, in the valley of the same name.} valley, we'll climb up and hunt above
the Pu-pɨ̂ River, okay?

6. Cà-g̈âʔ : I'd like to go, if that's where [we're going]. The squirrels--there
are plenty of squirrels\footnote{\textbf{fâʔ} is really the generic term for all rodents: squirrels, porcupines, badgers, rats, etc. English `rodent' has quite a different stylistic value, and is avoided in the translation.} there. Lots of barking-deer too.

7. H: If you catch a barking-deer, do you plan to sell the meat?

8. Cà-g̈âʔ : Well, sure, of course I'll sell it.

9. H: How much per kilo would you sell it for?

10. Cà-g̈âʔ : Oh, I probably wouldn't sell it for less than twelve baht a kilo.

11. H: Hm--if you did get a barking-deer, you'd get back your investment in buying
the gun, wouldn't you?

12. Cà-g̈âʔ: Sure, I certainly would get it back.

13. H: Let's hunt then, let's hunt! Why don't you go around [tracking]\footnote{\textbf{cɔ} \textbf{e} \textbf{ve} lit., `go around in circles,' `wind one's way.'} on that
steep slope over there and force them down into that grove of mɛ̀-šâ bamboo
.\footnote{\textbf{mɛ̀-šâ} bamboo: a kind of large bamboo that indicates good rice and chili land. See DL:1018.} I'll go and wait down there.

14. Cà-g̈âʔ: Okay, fine. I'll drive them down for you, all right! Hey, Blackie\footnote{His hunting-dog.}!
Hey! Hey!

15. H: I haven't gotten there yet. Don't do it yet! Let me make my way down first.

16. Cà-g̈âʔ: Be careful now. Wow! There sure are lots of tracks [up here]!

17. H: Okay, wind your way downward--down now.

18. Cà-g̈âʔ: Keep right on their trail down there!

19. H: Drive `em [down], drive `em [down]!

20. Cà-g̈âʔ: Hey, hey! Hey, Blackie, Blackie! Hey, something's moving down
there! Be careful now!

21. H: Be quiet, be quiet, a barking-deer's coming down now.

22. Cà-g̈âʔ: It's gone, it's gone.

\direct{Noise made by the deer, then the sound of the gun}

23. H: Hey, now it's gone into the river down there--into the river!

24. Cà-g̈âʔ: What is it?

25. H: It's a big buck, I tell you, a buck.

26. Cà-g̈âʔ: Gee, it'll taste great today--when we boil it up and eat it!

27. H: Oho, its hindquarters have gotten stuck right into the river-bed! Come on
down, come on down! Come help me carry my gun. I can't lift it all by myself.\footnote{Because he'll need both hands to carry the barking-deer.}

28. Cà-g̈âʔ: Aw, you carry it by yourself! I can't do it now. I'm too tired.

29. H: Just come on down first, down here.

30. Cà-g̈âʔ : To where?

31. H: It's below where you are now, below the clump of mɛ̀-šâ bamboo, in the
ravine.

32. Cà-g̈âʔ : Below here, huh?

33. H: Yeah, down there, down there.

34. Cà-g̈âʔ: The gulch is deep, you know! I can't make it.

35. H: Come pick your way down the hill over on that side there.

36. Cà-g̈âʔ: By way of that hill, you say?

37. H: Yeah, by that peak--by that dark-colored one.

38. Cà-g̈âʔ: I can't get down that ridge over there! The ground's too steep.

39. H: [Use] the footholds, the footholds\footnote{\textbf{ɔ̀-qhɛ̀ʔ} `layer; tier; notch'.} -- there are elephant footprints like
steps [going down the slope].

40. Cà-g̈âʔ: If I make a false step, it's right down into the river!

41. H: So look where you're going! You've got eyes.

42. Cà-g̈âʔ: Even with eyes I'm sure I'll fall! The ground's awfully steep.

43. H: You won't fall, you won't fall. Just get to that clump of suckers over there,
nice and slow--

44. Cà-g̈âʔ: You're sure\footnote{\textbf{ši} `know' + \textbf{ā} `perfective verb-particle, i.e. `have something known; be sure of something.'} then, are you? That I won't fall.

45. H: Sure I'm sure!

46. Cà-g̈âʔ: You can't be sure! You'll cause the death of a fellow-man!

47. H: You won't die, you jerk!\footnote{The epithet is conveyed by the exaggerated intonation on \textbf{šɨ} `die'.}

48. Cà-g̈âʔ: Otherwise even if you do get the animal, I won't live to eat it!

49. H: You just stay up there on the mountaintop and wait. I'll carry it up for
you, if that's the way you

want it.

50. Cà-g̈âʔ: Well now, in that case I'll just sit myself down nicely up there
for you.\footnote{The use of the benefactive verb-particle \textbf{lâ} is ironic here--"I'll do you a favor and sit down!"} Do be careful carrying it up here now, that thing down there! Shall
I climb up there?

51. H: Sit\footnote{\textbf{mɨ} `sit' is also used in the sense of `rest.'} and wait, sit and wait! I'll carry it for you, I'll carry it up
for you.

\direct{The deer is brought up the mountain}

52. Cà-g̈âʔ: Wow, this barking deer weighs as much as 1 khɔ́\footnote{\textbf{khɔ́}: a measure of weight equal to two \textbf{tâʔ}. See DL:380, 600.} plus 5
baskets of paddy! It's heavy to carry. Isn't that right?

53. H: Yes. It's one khɔ́ plus five basketful. There's no doubt about it.\footnote{Lit: ``There's no place for saying'', i.e., there's no gainsaying.}

54. Cà-g̈âʔ: How many animals have you ever gotten--ones like this?

55. H: Why, I've been shooting them down like this and eating them [for so long
that] my teeth have turned yellow already!

56. Cà-g̈âʔ: They're yellow because you smoke tobacco, more likely.

57. H: I have tried smoking those great big Gold Flake\footnote{\textbf{kâ-thɔ̂} (also pronounced \textbf{klɛ-thɔ̂ʔ}) < Thai\textit{ klɛ̀t thɔɔŋ} `Gold Flake'; a brand of Thai cigarettes.} cigars, you know.

58. Cà-g̈âʔ: Your teeth are yellow because you smoke, I'd say!

59. H: From eating barking-deer meat my teeth are all blocked up--there's white
crud\footnote{\textbf{qhɛ̂} `excrement; any refuse, excretion, or dross.'} stuck in the tips of my teeth.

60. Cà-g̈âʔ: That can't be true! I'm awfully hungry today! Where shall we go
to eat? Let's get with it and cut up the meat.

61. H: I wrapped up [something to eat] and carried it along. A big packet of sticky
rice.\footnote{\textbf{cà-nɔ̂-ɔ} `glutinous rice.' Used as provisions on outings away from home, because of its handy compactness.}

62. Cà-g̈âʔ: Give it here! Let's mix it up [with the meat] and eat it together.

63. H: Mix it up to eat, mix it up to eat! Like what's between the legs of Na-šɨ́'s
father!.\footnote{There is a joke here that the Lahu refused to explain to me because of its raunchy nature, and/or because it involved a particular person in the village. \textbf{Na-šɨ́} is a girl's name. Lahu adults are often referred to by the names of their eldest child, a widespread practice in Southeast Asia known to anthropologists as \textit{teknonymy}.}

\direct{laughter}

64. Cà-g̈âʔ: It probably doesn't taste too good then. If you call your [food]
by that name.

65. H: It tastes good, I tell you! ``Between the legs of Na-šɨ́'s father''
just means a big fucker\footnote{\textbf{khɔ-nú} \textbf{pàʔ-cɨ́}: \textbf{khɔ-nú} is < Thai khənǒm `cake'. \textbf{pàʔ-cɨ́} is probably from \textbf{pàʔ} `copulate' and \textbf{cɨ́} `stick together.' Cf. the Mexican slang word chingada, meaning `a big lump of something' < chingar `copulate'.} of a cake, that's all.

66. Cà-g̈âʔ: In a few minutes we'll get this meat cut up and we'll cook the
headman's portion.\footnote{\textbf{ɔ̀-šɛ} `headman's portion of killed game'. Since one of the hunters is the headman himself, they have the right to cook up that tender morsel immediately. See DL:207 and Plate 56.} We'll sink our teeth\footnote{\textbf{chèʔ} \textbf{câ} `bite and eat.' The hunters' delight at this prospect is understandable in view of the fact that the ordinary Lahu in this village only gets to eat meat a few times a year.} into it!

67. H; Light a fire. Light a fire then.

\direct{A woman who was listening to the taping breaks in}: The headman's
portion belongs to the elders!

68. Cà-g̈âʔ: Shall I make a fire?

69. H: Mm-hm.

70. Cà-g̈âʔ: Gee, I don't have any pine-splinters.\footnote{\textbf{a-kɨ́}: shards of pitch-pine, used for tinder.} It's hard as hell to
get it lit, isn't it?

71. H: Scratch us together some dry bamboo into a pile. Lay a big fire\footnote{\textbf{bɨ̀-phu} `large fire; bonfire'. \textbf{à-mī} \textbf{bɨ-phu} is how Paul Lewis translates `fiery furnace' in his translation of the New Testament (1962).}, strike
a match, and set it ablaze.

72. Cà-g̈âʔ: If it's really [done with] dry bamboo it won't taste good at all,
I tell you.

73. H: Don't let it get scorched in the fire. We'll get diarrhea!

\begin{center}
\direct{laughter}
\end{center}

74. Cà-g̈âʔ: I'll push it in and `tube-roast'\footnote{\textbf{càʔ-lá} \textbf{vâ} \textbf{ve}: `cook by inserting the food into a bamboo-tube and holding it over a flame'--``tube-roast.'' Opposed to \textbf{pì} \textbf{ve} `roast by trying to a stick and holding over a flame'--``spit-roast.'' The lively pro-verb \textbf{vâ}, here meaning `eat,' like the usual word \textbf{câ} `eat' itself, occurs as the second member of compounds relating to the preparation of food, but is best omitted in translation.} it then. I'll cut off a
mɛ̀-šâ bamboo tube over there--

75. H: Better tie it to a stick and roast it, `spit-roast' it! On a great big roasting-stick,\footnote{\textbf{ɔ̀-pì}: `a spit-roast'--the meat together with the stick to which it is attached.}
as long as your arm.

76. Cà-g̈âʔ: When you spit-roast it, the wood makes it stink of soot! If you
\emph{tube}-roast it, then it's fine!

77. H: Spit-roast it! Scrape (lit. `beat') those embers over there [together into
one place] and spit-roast it.

78. Cà-g̈âʔ: In that case put some wood in, too. Right over there, then.

79. H: Break off some dry mɛ̀-pɔ̀ bamboo\footnote{\textbf{mɛ̀-pɔ̀}: `huge species of bamboo, edible when young'. See DL:1018.} and put it in. Pick up some dry
mɛ̀-pɔ̀, some dry bamboo and put it in.

\direct{Sound of match being struck}

80a. Cà-g̈âʔ: Well, now, in a little while, it'll catch nicely and we'll have
it cooked up, right?

H: Mm-hm. Cà-g̈âʔ: When we're through eating in a few minutes, how shall we
go back [to

the village]? Why don't we do some more hunting on our way back?

81. H: Let's go back hunting over there outside of Pa-ho village\footnote{\textbf{Pa-hôʔ}: a Thai village near Huey Tat. See DL:803.}, in that narrow
valley--over in those rattan thickets there.

82. Cà-g̈âʔ: But if we're carrying the meat like this, we won't be able to
go through any rattan thickets!

83. H: [If you don't want to hunt anymore on the way back] you go on up there to
the [easy] road along the mountain, while \textit{I} do some sneaky shooting\footnote{\textbf{jâʔ} \textbf{bɔ̂ʔ} \textbf{ve} `shoot stealthily', i.e. to hunt on noiseless feet.}
down below--then I'll pick up [what I've shot] and carry it up to you!

84. Cà-g̈âʔ : Ha! The way you do it all you can catch is birds!\footnote{Lit: ``if it isn't birds, you don't catch anything to eat.'' A good translation might be `your shooting is for the birds!'}

85. H: I'll catch him, all right. [There's an animal / He's] sleeping down there
at the river. Big as life, with a hard-on\footnote{\textbf{nī-qhɛ̀ʔ} \textbf{pé} \textbf{ɛ̀} \textbf{te} \textbf{ve} `to have an erection'. Used also to mean `obvious, very much in evidence, as hard to ignore as an erected penis.'}!

\direct{laughter}

86. Cà-g̈âʔ: You've got a foul mouth!\footnote{Lit: ``It's too much, your [speech].''}

87. H: Don't I though!

88. Cà-g̈âʔ: Animals have sharp ears. Things like barking-deer [run away at
the first sound]--because they don't have gall-bladders.\footnote{According to traditional Lahu belief, the gall-bladder is the seat of courage in men and animals.}

89. H: It's still way over there, the mountain after next. It's still far. He probably
can't hear us.

90. Cà-g̈âʔ : The mountain after next in that direction, eh? H: Mm-hm. Cà-g̈âʔ:
Nope. I bet you won't

be able to go quietly enough.

91. H: I'll flit along softly on my little tippy-toes\footnote{Lit: ``sneaking on tip-toes I'll fly through the air.'' \textbf{mû-phe} `the region above'}!

\begin{center}
\direct{laughter}
\end{center}

92. Cà-g̈âʔ: Then your legs will get numb on you.

93. H: I'll twist my steps as I walk on tiptoes. So that I practically turn my [calf-]muscles
inside out.

94. Cà-g̈âʔ: You can't almost turn your muscles inside out that way!

95. H: Sure, I can actually\footnote{The meaning `actually' is conveyed by the verb-particle \textbf{šɔ̄} `still' i.e., ``no matter what you say.''} get my calf turned around to the front [of my leg]!
You just watch as I go twisting along!

\begin{center}
\direct{laughter}
\end{center}

96. Cà-g̈âʔ: What you say can't be true.\footnote{Lit. ``It is not to such an extent, your [words].''}

97. H: I'll go carefully this time, I tell you.

98. Cà-g̈âʔ: Well then, I'll try going [my way too], right? So I'll climb up
there to the mountain slope, up along the road, and I'll wait for you up there
then, okay?

99. H: Watch me carefully. I'll twist along\footnote{\textbf{šɨ́} \textbf{phɛ̂} ``twist forth'' is slang for `go.'} till I get under that banyan down
there.

100. Cà-g̈âʔ : Right. So I'll sit and wait for you up on that upper ridge then.

101. H: A while back I-kha's father went down there, and a barking-deer jumped out
and ran away!

102. Cà-g̈âʔ: Down there, eh?

103. H: Yeah, over there--Hey, didn't you hear a shot\footnote{\textbf{vâ} is here used as a pro-verb for \textbf{bɔ̂ʔ} `shoot,' as is clear from \textbf{qhɛ}, the classifier for gunshots. The particle \textbf{qha-pâʔ} is used especially when auditory perception is at issue. See DL:272.} just now down at the
foot of the mountain?

104. Cà-g̈âʔ: Well, since I'm high up here I can't hear too well, you know.

105. H: Look over there below the mountain with the grove of \textit{mɛ̄-pɔ̂
}-- I mean \textit{mɛ̀-cû-šɛ́} trees! Shots are being fired on that barren
mountain there.

106. Cà-g̈âʔ: What are they shooting at then?

107. H: They're hunting squirrels! A little bunch of miserable\footnote{\textbf{chɔ} \textbf{lù-kɨ̂} `wretches'; lit. ``ruined and rotten people''.} young squirts
\footnote{\textbf{yâ-qɛ̀ʔ-ku}: an insulting term for an adolescent.} down there.

108. Cà-g̈âʔ: Who are they then, the ones down there? How many of them are
there?\footnote{\textbf{ɔ̀-mɛ} \textbf{qhà-nî} \textbf{g̈â} \textbf{le}: lit. ``how many names are there?''}

109. H: It's Cà-há and his crowd and that oaf Thû-yì and his bunch.\footnote{The pluralizer \textbf{hɨ} is usable after proper names, like Japanese -tachi. \textbf{qā}- is an uncomplimentary prefix to names. \textbf{Thû-yì} was actually one of the brightest young men in the village, and a good friend of the Headman! See DL:681.} They're
mumbling away about something down there.

110. Cà-g̈âʔ: What would you say they're cooking up, those guys?\footnote{\textbf{te} \textbf{câ} (vV + V\textsubscript{h}), lit.,``do in order to eat'', i.e. do for an envisaged advantage. \textbf{te-câ} can also be construed as a compound meaning `to cook', so that the translation ``What are they cooking up?'' works well.} What
are they looking for?

111. H: The barking-deer jumped right towards them, but they didn't even see it!
It ran away down [the slope]!

112. Cà-g̈âʔ: They won't get anything to eat.\footnote{The standard Lahu phrase for ``to be unsuccessful in hunting.''}

113. H: They probably won't now. That bunch of kids are awfully lazy.

114. Cà-g̈âʔ: How many of them would you say have come? Are there any game-drivers
\footnote{\textbf{šā-g̈àʔ-pā}: members of the hunting party, armed with knives (and sometimes guns) and accompanied by dogs, who drive the game to where the ``shooters'' (\textbf{šā-bɔ̂ʔ-pā}) are lying in wait.} among those who've come down there?

115. H: I saw some drivers just now driving [the game] down from an old field up
over there.

116. Cà-g̈âʔ : Up over there, huh?

117. H: Mm-hm.

118. Cà-g̈âʔ: Who are they, the game-drivers?

119. H: It looks like the ones down there driving the game are north-country A-leh
\footnote{\textbf{mə́-nə́} \textbf{A-lɛ́}: \textbf{mə́-nə́} `north-country' < Thai myaŋ nyá. `A-leh' is the designation of a minor subgroup of Lahu in Yunnan, considered to be a kind of Lahu Shehleh, rather closely related to Black Lahu. See DL:81.} folk. The A-leh are supposed to have black tips to their turds\footnote{\textbf{qhɛ̂-mə̂} can mean either `turd', or more specifically, `the tip of a turd'. \textbf{nâʔ} means `black', but can also be used for any dark color, e.g. dark brown. When the meaning `black' is insisted upon the intensified form \textbf{nâʔ-tɔ́} is used. See DL:752.}, you know.

120. Cà-g̈âʔ: Come on! You're really too much, you know....

\begin{center}
\direct{They fall into a story-telling mood}
\end{center}

Once upon a time there were some people--er--a father-in-law and a son-in-law.
This man got himself a son-in-law, see. He was very happy about it. Well, so he
cuts off this great big hunk of meat and boils it--but in this meat there was a
little bit of gristle, see?

121. H: Uh-huh.

122. Cà-g̈âʔ : Well, here they serve it up, see, after the wedding, they put
it down and here they are eating and this son-in-law he's sitting right next to
his father-in-law, see. So he takes and bites off this great big piece, the biggest
piece of all--and it happens to have the gristle!

123. H: The gristle, huh?

124. Cà-g̈âʔ : Right. He wants to throw it away, but he can't. He's ashamed
before his father-in-law. So he takes and bites into it with all his might, a little
too violently, and he bites so hard that his elbow pokes his father-in-law in the
forehead, right between the eyes, and the father-in-law keels over! Nobody could
do a thing.

125. H: Gee, when you tell this it reminds\footnote{Lit: ``when (you) say like this, I had (something) once a long time ago."} me of something that happened to
me once a long time ago.

126. Cà-g̈âʔ: What was it?

127. H: Once when I went over to Meu-heh and Meu-khaw\footnote{\textbf{Mə̂-khɔ̂} is a Northern Thai and Karen village a day's walk from Huey Tat. \textbf{Mə̂-hɛ́} is a Northern Thai village near \textbf{Mə̂-khɔ̂}. See DL:1047.}, they boiled up some
chicken to feed me and they put in about three fingers' worth of wild ginger. When
I took and chewed it, I passed the hell out\footnote{\textbf{chɔ-khɔ̂} \textbf{mâ} \textbf{nɔ̂}: lit. ``not be aware of human speech''. According to context, this can mean either `be unconscious', or `be unreasonable'.}!

\begin{center}
\direct{laughter}
\end{center}

128. Cà-g̈âʔ: Then there's a story about another person who passed out. He was
riding along on a bicycle. He was headed for a government office building\footnote{\textbf{šá-ná-yɛ̀}: \textbf{šáná} < Bse. `government office'; \textbf{yɛ̀} `house.'}
someplace.\footnote{\textbf{ô} \textbf{ɔ̄-qhe} ``like over there.''} Well, on the way to the government office, there were paddy-field
dikes\footnote{\textbf{ti-mi} \textbf{ɔ̀-tɛ̂}: raised earth-boundaries around an irrigated paddy-field, serving to keep the water from running out.}, see. It was an unusual way to go. Well, there was this buffalo-herder
up behind\footnote{\textbf{yàʔ-qɔ} \textbf{na} \textbf{pá} ``the upper part of the road''--i.e., the part the man on the bike had already traversed, i.e. the part behind him. Contra DL:1268, where the phrase is glossed `the part of the road remaining to be traversed'.} him on the road. From where he was behind him, he calls out to this
old fellow [on the bicycle]: ``Where are you going?'' he says.
Now [the guy] happened to be right near the water [in a paddy field], see. So,
when he turns around to look back up the road, the bicycle goes right down into
the water. [The buffalo-herder] had just said to him, ``Where are you going?,''
right? So he answered, ``To the off--''\footnote{The original is in Thai: ``\textbf{paj} am--.'' He doesn't have time to finish his phrase \textbf{paj} amphəə ``(I'm) going to the government office.''}

129. H: ``To the off--?''

130. Cà-g̈âʔ: Yeah, he didn't have time to say `office,' so he could only say
'off--' before he sank into the water!

131. H: I see.

132. Cà-g̈âʔ: Well, the bicycle was swallowed up down there. He couldn't get
it back again. All he could get out was his hat, so he went back home and changed
his clothes. Then he went off again.

133. H: I had a time like that once too. [I was walking] someplace when I slipped
and fell head over heels onto a ridge off the edge of the path\footnote{This was a serious matter, since he could have fallen several hundred feet to his death.}!

134. Cà-g̈âʔ: (laughing) Wow, that was too much, something like that!

135. H: When I looked around afterwards there I was smack under a clump of jujubes
\footnote{The jujube-branches broke his fall. The onomatopoetic adverb phə̂n \textbf{kàʔ} can also be used for gusts of wind.}! To think I'd landed down there!

136. Cà-g̈âʔ: That was really too much!

137. H: I was worried that somebody might see me screwing my way back up\footnote{\textbf{qɔ̀ʔ} \textbf{pàʔ} \textbf{qɔ̀ʔ} \textbf{la}: this morpheme \textbf{pàʔ} is probably to be identified with the verb `to copulate'--it is here being used as an obscene pro-verb in an elaborate expression.} [to
the road], so I climbed up in a hurry, hauled myself up and set myself [on the
road] again, then I headed the hell back for home and reached my house. That really
happened to me.

138. Cà-g̈âʔ: Nothing like that's ever happened to me yet.

139. H: Huh?

140. Cà-g̈âʔ: I said, that has never happened to me yet.

141. H: It really happened. I skidded and fell downhill, by God. The safety-fence
\footnote{\textbf{kə̀-tà}: (\textbf{tà} `stick'); sticks planted by the side of steep mountain paths. See DL:357.} went smash too, I tell you. Against my shins as I crashed down through it!

142. Cà-g̈âʔ: You must've been carefree in those days!

143. H: Very carefree.

144. Cà-g̈âʔ: Well, how did \textit{you }\footnote{The noun-particle \textbf{mɛ} is a strong (often contrastive) topicalizer.} court the girls in those days,
then?

145. H: I'd buy them\footnote{There is an anomalous use of \textbf{lâ} here instead of \textbf{pî} for 3rd person benefaction, undoubtedly to make it more vivid, as if he's interacting directly with the girls.} a pack of cigarettes someplace, and I'd buy some perfumed
powder\footnote{\textbf{phû-də} \textbf{hɔ́}: \textbf{phû-də} < Bse. < Eng `powder.' \textbf{hɔ́} < Shan `sweet-smelling'.}, and I'd buy a box of matches, then I'd go scratching my matches down
in the middle of the village, making big red flames. Then the girls would come
and say ``Give me a smoke, give me a smoke\footnote{\textbf{tɔ} \textbf{ve}: The causative of \textbf{dɔ̀} \textbf{ve} `to drink, to smoke.'}!''

146. Cà-g̈âʔ: Then how would you give them a smoke?

147. H: I'd say to them ``Come on over there, let's go over there.''
When we got there I'd give each one of them a cigarette. Then I'd say, ``I've
still got lots more of these, if you'd like to smoke again. Let's go over there..."

148. Cà-g̈âʔ: Aha. Well, then there's [a story about] another person. In this
story it seems there was a certain woman. She got herself married\footnote{\textbf{làʔ-qɔ̄} \textbf{tôʔ} \textbf{ve}: `get married', lit. ``lay arms atop (each other)''.} and her
man was as happy as could be. (But) when the people were fed [at the wedding feast],
he didn't eat, see. When night fell he said to his wife, ``Give me something
to eat." (But) there was no food left. Well, this fellow all of a sudden
took up a--what do you call it--er, an earthen (rice-)pot, see, and stuck his hand
in, and fished around in it for rice with his hand--with his fist. And it got stuck
in there! Fishing around for rice it got stuck, and no matter what he did he couldn't
get it out again. The fellow had a bald head, and the moonlight outside was shining,
and the moonbeams struck his head, you see. As it was hitting his head, as the
light was shining on it, [his wife] took [the pot] that was stuck [on his hand]
and bashed him on the head with it. She thought it was a rock, you see. And boy,
were they a couple that were sorry afterwards!\footnote{\textbf{bo} \textbf{te} \textbf{ve} means `to repent' among its other shades of meaning. See DL:942.}--Let's stop here, okay?\footnote{\textbf{Cà-g̈âʔ} is tired of talking into the tape-recorder.}
It's enough for now, isn't it?

149. H: I don't think it's enough yet. I'm sure there's plenty more [we could say].
\footnote{Note the allofamy between \textbf{pɛ̂} `be enough' and \textbf{pɛ} `be plenty'.}

150. Cà-g̈âʔ: I bet it's enough, as much as we've done.

151. H: Hurry and keep going!

152. Cà-g̈âʔ: My throat is killing me. I want to have a cigarette now. It's
enough, okay?

153. H: Do you know what is meant by ``Red Lahu'' and ``Lahu
Mə̂-nə́"?

154. Cà-g̈âʔ: How do you mean?

155. H: When we say `Red Lahu' and `Lahu Mə̂-nə́.'

156. Cà-g̈âʔ: Isn't it because the Red Lahu have red stripes on the skirts
and jackets that they wear? So they're called `Red Lahu.'

157. H: Well, what about `Mə̂-nə́,' then? `Mə̂-nə́.'

158. Cà-g̈âʔ: This calling them `Mə̂-nə́' is supposed to be because they
live up north-that's why they're called `Mə̂-nə́'.

159. H: I see. Then what about the ones called ``A-leh''?

160. Cà-g̈âʔ: [They're called] A-leh because they're supposed to be more ca-\textit{leh}-ver
than others!\footnote{``Ca-leh-ver'' is an attempt to convey the \textbf{pun} in the Lahu original between \textbf{A-lɛ́} and the adjective \textbf{lɛ̂ʔ} `clever'.}

161. H: Oh.

162. Cà-g̈âʔ: So other people call them ``A-leh.'' Because
whatever they do they're ca-\textit{leh}-verer at it than anybody else.

163. H: They're supposed to be cleverer?

164. Cà-g̈âʔ: Yes.

165. H: So that's why people give them that name, according to you! Their shit is
black, too. Now I've also heard it said that the Red Lahu have red shit. I wonder
if it \textit{is }red, like they say!

166. Cà-g̈âʔ: I daresay yours is black, too, your shit.

167. H: I've never actually picked it out of my butt to look at it!\footnote{Note the 4-verb concatenation: \textbf{qɔ̀ʔ} `V back' + \textbf{qɛ́} `pick out of' + \textbf{ni} `V and see' + \textbf{mɔ̀} `witness the action'.}

\begin{center}
\direct{laughter}
\end{center}

168. Cà-g̈âʔ: In that case, do take a good look at it!

169. H: Well, I suppose it is black, at that. It's probably all quite black.

170. Cà-g̈âʔ: Mine is certainly black--Come on, now, that's enough already!

171. H: What--do you want to have a smoke?

172. Cà-g̈âʔ : Yeah, that's enough already!

173. H: I've already smoked a big cigar. Aren't you smoking?

(A voice) 174 There aren't even any cigars.

175. Cà-g̈âʔ: Enough now, I tell you!

176. H: You've had enough?

177. Cà-g̈âʔ: It's awfully long already.

178. H: Oh, if you've had enough that's all right.

