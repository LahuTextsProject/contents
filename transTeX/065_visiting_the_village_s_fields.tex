\setcounter{footnote}{0}

1. H: \direct{clears throat} This morning we're going to visit
the fields, right? Are the two of you\footnote{\textbf{Cà-lɔ̂} and the author.} going along? Cà-lɔ̂?

2. P: We'll go!

3. H: OK. Well, you're going to take pictures of all the things connected with
our cultivating our swiddens, aren't you?

4. C: Yes, fine!

5. H: Let's go then, right now. You have enough time to go, right?

6. C: We'll go! Say, on the road you're taking are there many Northern Thai\footnote{The Lahu make a clear distinction between \textbf{Kɔ́lɔ́} `Northern Thai' and \textbf{Thây} `Central Thai; Standard Thai; Siamese'.}
villages?

7. H: There are lots of them. As for Northern Thai, there's no shortage\footnote{\textbf{pə̀-lə̂ʔ} (N) `plenty; abundance' < N. Thai.} of
them! There are one or two villages.

8. And a road for motor-cars reaches almost up to our fields over there.

9. Yeah, this village right here\footnote{\textbf{kàʔ} \textbf{kàʔ} is colloquial for \textbf{chò} \textbf{kàʔ} `here'.} is Northern Thai.

10. All of these many trees by the roadside here belong to the ``Big Boss''.\footnote{\textbf{šathê-ló} (< Tai): a wealthy Thai who owned the tea plantations near Huey Tat, and sometimes hired Lahu to help pick the crop. \textit{See Text \#69 ``Picking Tea''.}}

11. And there ahead of us there's another Northern Thai village.

12. All these tea orchards are also Northern Thai.

13. Not Central Thai folks.

14. Yeah, and we have to draw water.\footnote{\textbf{í-kâʔ} \textbf{qho} \textbf{ve} (OV): a very laborious process, filling up gourds or bamboos from a water source and hauling them in tumpline baskets up the mountainside to the village or the fields.} From right here. To drink in the fields.

15. C: Isn't there any water to drink up there?

16. H: No, there isn't. We scoop up all the water here that we use everyplace in
the fields...

17. Hey, how about rolling ourselves a cigar, OK? One for each of us.

18. C: Yeah, that'd be great.

19. H: Oh, when you make a cigar to smoke in a banana leaf it's very harsh.

20. Here, take it, when we make them in corn-husks, they smell great.

21. Here, you smoke one too!

22. Just grab those matches over there!

23. This cigar is hard to light\footnote{\textbf{tú} \textbf{mâ} \textbf{tòʔ} pɨ́:\textit{\textbf{ }}\textbf{tòʔ} `burn'/tú `kindle' are a simplex/causative verb pair, with the former often serving as the \textit{resultative complement }of the latter. For the concept of resultative complement, see Y.R. Chao (1968) \textit{A Grammar of Spoken Chinese }U.C.Press (Berkeley and Los Angeles), pp. 443-446; and GL 4.314.}! It's a little too wet.

24. C: Whose tobacco is it?

25. H: It's the Kachin's\footnote{A Jingpho man was a long-term resident of Huey Tat, having migrated with them from Burma. He was very well liked by the Lahu villagers.} tobacco! Yep, it must not have been well dried, this
tobacco. You can't light it.

26. Hey, these plantations here are where we Lahu pick tea to earn money.\footnote{\textbf{phə̂ʔ} \textbf{câ} `pick to eat'. \textbf{câ} `eat' here functions as a versatile verb meaning `V for a living'. For foods that are consumed for pleasure (e.g., honey, sugar-cane) rather than for mere sustenance, the verb used is often \textbf{lɛ̀ʔ} `lick', rather than \textbf{câ} `eat'.}

27. What's hanging high up on that tree over there is a honeycomb.

28. Have you ever tasted the juice\footnote{\textbf{ɔ̀-g̈ɨ̀} `liquid', here used instead of \textbf{pɛ̂-g̈ɨ̀} `honey' (``bee-juice'')}?

29. C: We have tasted the bee-juice before.

30. H: Did it taste very good?

31. C: Delicious!

32. H: There are twelve nests hanging there. Now, this year there aren't that many
of them hanging.

33. The path to where we draw water to drink is up a very steep slope.

34. When we carry the water we must go along bending our heads [forward to keep
our balance].

35. This road here is the one that we usually walk on.

36. So it's not far now until we get to the fields.

37. When we reach that saddleback\footnote{\textbf{qā-lèʔ}: `a relatively flat or low part of a mountain; saddleback; high plateau'.} up there in the mountains, we'll be able
to see over to where the fields are.

38. Look over there! Those people tilling the fields, they're our Lahu folks.\footnote{\textbf{Lâhū-yâ}: lit. ``Lahu sons; Lahu children.''}

39. \direct{shouting to the workers} Till the fields well, till `em
well now!

40. We'll take a picture of your backs!

41. Field-worker: What are you looking for?

42. We're just visiting. Well, you're tilling the field. Don't exert yourselves
so much!

43. If you don't have anything to eat and drink you'll have no strength.

44. Really tasty food---pork, chicken.\footnote{The headman is gently teasing them. Meat was something eaten only on special occasions, certainly not to be found in the packets of food taken to the fields every day, and not usually eaten in a field-hut. See DL:87.}

45. Prepare the food properly and \textit{then} work.

46. Then you'll have the strength.

47. If you don't have food\footnote{\textbf{ɔ̄-chî}: lit. ``rice-lifter'' (\textbf{chî} `lift up'), i.e. that which is eaten with rice (cf. Thai \textit{kàpkhâaw}), conventionally translatable as `curry', or simply `food'.} to eat you'll have no strength when you're working
in the fields.

48. If you should hook [your tool] onto a tree stump you could topple over backwards!

49. So cook delicious curries to eat [before working].\footnote{The sense of `before' is conveyed by the verb-particle \textbf{šē} `inchoative'}

50. There's an awful lot of hard work to do around here.

51. If you guys drop dead, [your hard work] can't follow you [to the grave]. So
take it easy!

...

52. H: Whew, the sun sure is hot!

53. And the dust stinks so bad you could die.\footnote{\textbf{šɨ} \textbf{e} \textbf{la} \textbf{yò}: cf. the Thai intensifier\textit{\textbf{ }}\textit{cə taaj.}}

54. Today our whole bodies will get pitch-black with dirt.\footnote{This sentence contains two words meaning `all,' each with a slightly different sense: \textbf{qha-pə̀-è} `all [of a thing]; the whole thing' and \textbf{dê-dê} `all (members of an aggregate)'.}

55. Wow, the sun is so hot!

56. Hey, guys, why don't we just go hunting for something to eat!

57. Say, would the two of you like to go?

58. P: We'd love to!

59. H: Well, let's go see to the guns first. If there's one gun per person\footnote{\textbf{tê} \textbf{g̈â} \textbf{tê} \textbf{qhɔ̂}: \textbf{g̈â} `classifier for people', \textbf{qhɔ̂} `classifier for elongated objects, including guns'.}
we can go.

60. C: Hurry and look for them. Two of them.

61. H: There's a guy over there. I don't know whether or not he's going along.

62. C: If it's him, I hear he's never shot a gun in his life.\footnote{Lit. ``I hear he's never shot---as for the thing called a gun.''}

63. H: Well, let's go see then. Cà-qā, please go and ask your father. Your father's
probably not going [hunting with us].

64. Cà-qā: Oh, he actually left yesterday. He's lying in wait up there in the
mountains all by himself, up on a rocky slope!

65. H: I guess he's gone then also.

66. So, let's take some pictures now.

67. Let's take a picture of all of us shooting guns, carrying guns on our shoulders,
aiming our guns at the treetops.

68. P: What kind of guns are they?

69. H: They're flintlocks!\footnote{\textbf{màʔ-tɛ̂ʔ-nâʔ}: a muzzle-loading gun whose fuse must be lit manually, by hammer or match} They're big muskets.\footnote{\textbf{lɔ̂-kà-tù}: more or less synonymous with \textbf{màʔ-tɛ̂ʔ-nâʔ}.}

70. Somebody: That'll be OK, I guess.

71. H: They're very fine. They're Pie-talian!\footnote{\textbf{pítalêiʔ}: evidently a garbled version of `Italian'. Perhaps a conflation of Thai \textit{pyyn} `gun' and ʔ\textit{iitaalii} `Italian'.} They're called Pie-talian.

72. P: What does ``Pie-talian'' mean?

73. H: ``Pie-talian'' means šán-phɔ̀ʔ\footnote{I have been unable to trace this word, which might well be a jocular invention of the headman.}---it's šán-phɔ̀ʔ!

74. P: Well, what does šán-phɔ̀ʔ mean then?

75. H: It means Pie-talian, I told you. šán-phɔ̀ʔ.

76. Cà-g̈âʔ: Where do they say that the Pie-talian guns came from?

77. H: A long time ago they came from a white man's country.

78. Cà-g̈âʔ: Which white man's country?

79. H: A white man's country down there in the west, in the southern part.

80. Cà-g̈âʔ: A place where black people live?

81. H: White people, a white person's country. White people.

82. Cà-g̈âʔ: I bet they're not white, they're probably black.

83. H:The black people's country is still to the south.

\direct{changing the subject}

84. Well, let's take some pictures, one person in each picture, carrying guns on
our shoulders, on our backs\footnote{\textbf{tâʔ} `carry on shoulder'; \textbf{pû} `carry on the back'.}---all of us.

85. Oh boy, today the sun is so hot! We went hunting---we're exhausted.

86. Woman: You men go hunting, we women catch frogs.

87. H: I didn't get to shoot a single animal. Nobody drove any animals towards
me.

88. Down below there you should have followed the animals' trail properly. There
were plenty\footnote{\textbf{cí-cí} \textbf{mâ} \textbf{hêʔ}: a pleonastic expression, lit. ``not just a little bit,'' i.e., `plenty of'.} of barking-deer tracks.

89. You guys drove the animals \textit{away}.\footnote{I.e., instead of towards me.} They went down to the foothills
over there.

90. If you had only shouted and driven them over there towards me on the high ground
above the river, I would have figured out how to catch a barking-deer!

91. The barking-deer are still over there. Under the rocks.

92. Woman: Your wife and children are waiting for something to eat at home.

93. H: Oh, those damned barking-deer! They don't show themselves where people are.

94. Since you guys drove [the animals] that way for me, today we just went wandering
around wasting time.

95. Very soon now we'll go hunting together properly in the jungle.

96. Woman: That's the way it is! We're not going to get anything to eat.\footnote{From the hunting of you men.} If
we [women] go catch some frogs we'll have something to eat.

97. H: Well, there's an old proverb [about this]. That's not the way it should
be. ``Think not that ye shall always fail, nor suppose that ye shall always succeed.''
Don't think you won't ever get anything to eat, whether you get anything or not
[on any specific occasion].

98. Never get weary of making forays into the jungle. You should think ``I'll catch
something to eat,'' and go look for it!

---------------------------------------------------------------------------------------------------------------

