
65 Visiting the Village's Fields

1. Headman: <clears throat> ``This morning we're going to visit
the fields, right? Are the two of you \footnote{Cà-lɔ̂ and the author} going along? Cà-lɔ̂?

2. Cà-lɔ̂: We'll go!

3. H: OK. Well, you're going to take pictures of all the things connected with
our cultivating our \{swiddens/ mountain fields\}, aren't you?

4. C: Yes, fine!

5. H: Let's go then, right now. You have \{the/ enough\} time to go, right?

6. C: Go! Say, on the road you're \{going on/ taking\} are there many Northern
Thai \footnote{The Lahu make a clear distinction between \textbf{Kɔ́lɔ́} `Northern Thai' and \textbf{Thây} `Central Thai; Standard Thai; Siamese'.} villages?

7. H: There are lots of them. As for Northern Thai, \{there are plenty/ there's
no shortage\} \footnote{\textbf{pə̀-lə̂ʔ} (N) `plenty; abundance' < N. Thai} of them! There are one or two villages.

8. And a road for motor-cars \{reaches/ goes\} almost up to our fields over there.

9. Yeah, this village right here \footnote{\textbf{ kàʔ kàʔ} is colloquial for\textbf{ chò kàʔ} `here'  Nsd Pn} is Northern Thai.

10. All of these many trees by the roadside here belong to the ``Big Boss'' .\footnote{\textbf{šathê-}(< Bs.) \textbf{ló} ( < Tai): a wealthy Thai who owned the plantations near Huey Tat, and sometimes hired Lahu to help pick the crop. See Text \#69 ``Picking Tea''.}

11. And there ahead of us there's another Northern Thai village.

12. All these tea \{orchards/ plantations\} are also Northern Thai.

13. Not Central Thai folks...

14. Yeah, and we have to draw water .\footnote{\textbf{ í-kâʔ qho ve} (OV): a very laborious process, filling up gourds or bamboos from a water source and schlepping them in tumpline baskets up the mountainside to the village or the fields.} Right here. To drink in the fields

15. C: Isn't there any water up there?

16. H: No, there isn't. We scoop up all the water here that we use everyplace in
the fields...

17. Hey, how about rolling ourselves a \{cigar/ smoke\}, OK? One for each of us.

18. C: Yeah, that'd be great.

19. H: Oh, when you make a cigar to smoke \{out of/ in\} a banana leaf it's very
harsh.

20. Here, when we make them in corn-husks, they smell great.

21. Here, you smoke one too!

22. Just grab those matches over there!

23. This cigar is hard to light \footnote{\textbf{tú mâ tòʔ  pɨ́}:  kindle not burn be able  \textbf{tòʔ} `burn'/\textbf{tú} `kindle' are a simplex/causative verb pair, with the farmer often serving as the \textit{resultative complement }of the latter. For the concept of resultative complement, see Y.R. Chao \_\_\_\_ and GL \_\_\_\_.}! It's a little too wet.

24. C: Whose tobacco is it?

25. H: It's the \{Kachin's/ Jingpho's\} \footnote{A Jingpho man was a long-term resident of Huey Tat (prob. came with them < Burmese), and was very well liked by the Lahu villagers.} tobacco! Yep, it must not have been
well dried, this tobacco. You can't light it...

26. Hey, these tea gardens here we Lahu pick to earn money...\footnote{\textbf{phə̂ʔ câ}:  pick    eat  \textbf{câ }`eat' here functions as a versatile verb meaning `V for a living'. The verb used for foods that are consumed for pleasure (e.g., honey, sugar-cane) rather than for mere sustenance, is often \textbf{lɛ̂ʔ }`lick', rather than \textbf{câ }`eat'.}

27. Over there what's hanging high up on a tree is a honeycomb.

28. Have you ever \{tasted/ licked\} the juice \footnote{\textbf{ɔ̀-g̈ɨ̀} `liquid', here used instead of \textbf{pɛ̂-g̈ɨ̀ }`honey' (``bee-juice'')}?

29. C: We have tasted \{the juice/ honey\} before.

30. H: Did it taste very good?

31. C: Delicious!

32. H: There are twelve nests hanging there. Now, this year there \{aren't that
many/ are rather few\} of them hanging.

33. The \{path/ road\} to where we \{scoop up/ draw\} water to drink is up a very
steep slope.

34.When we carry the water we must go home bending our heads [forward to keep our
balance].

35. This road here is the road that we usually \{walk on/ take\}.

36. So it's not far now until we \{get to/ reach\} the fields.

37. When we get/ reach up there onto the saddleback \footnote{\textbf{qā-lèʔ}: `a relatively flat or low part of a mountain; saddleback; high plateau'.} in the mountains, we'll
be able to see over there where the fields are.

38. Look over there! Those people tilling the fields, they're our Lahu folks .\footnote{\textbf{Lâhū-yâ:} lit. ``Lahu sons; Lahu children.'' cf. Heb. B'nei Yisrael.}

39. <shouting to the workers> Till the fields well, till `em
well now!

40. We'll take a picture of your backs!

41. Field-worker: What are you looking for?

42. We're just visiting. Well, you're tilling the field. Don't exert yourselves
so much!

43. If you don't have anything to eat and drink you'll have no strength.

44. Really tasty food---pork, chicken .\footnote{H. is gently teasing them. Meat was something eaten only on special occasions, certainly not in the packets of food taken to the fields every day, or usually eaten in a field-hut. [Explain \textbf{Á-po-qo} DL 87.]}

45. Prepare the food properly and \textit{then} work.

46. You'll have the strength.

47. If you don't have \{food/ curries\} \footnote{\textbf{ɔ̄-chî}: lit. ``rice-lifter'' (\textbf{chî }`lift up'), that which is eaten with rice (cf. Thai \textbf{kàpkhâaw}), conventionally translatable as `curry', or simply `food'.} to eat you'll have no strength---when
you're working in the fields.

48. \{If/ When\} you should hook [your tool] onto a tree stump you could topple
over backwards!

49. Cook delicious curries to eat [before working]. \footnote{`Before working' conveyed by the Pv \textbf{šē }`inchoative'}

50. There's an awful lot of hard work [to do] around here.

51. If you guys drop dead, [your hard work] can't follow you [to the grave]. Take
it easy!

...

52. H: Whew, the sun sure is hot!

53. And the dust stinks so bad you could die. \footnote{\textbf{šɨ e la yò}: cf. the Thai intensifier \textbf{cə taaj}.}

54. Today our whole bodies will all get pitch-black with dirt. \footnote{This sentence contains two words meaning `all,' each with a slightly different sense. \textbf{qha-pə̀-è }`all [of a thing]; the whole thing' and \textbf{dê-dê }`all (members of an aggregate)'.}

55. Wow, the sun is so hot!

56. Hey, guys, why don't we just go hunting for something to eat!

57. Say, would the two of you like to go?

58. Cà-lɔ̂: We'd love to!

59. H: Well, let's go see to the guns first. If there's one gun per person \footnote{\textbf{tê g̈â tê qhɔ̂}:\textbf{ g̈â} `Clf for people', \textbf{qhɔ̂} `Clf for elongated objects, incl. guns'; \textbf{tê} `one'.}
we can go.

60. C: Hurry and look for them. Two of them.

61. H: There's a guy over there. I don't know whether or not he's going along.

62. C: If it's him, I hear he's never shot a gun in his life. \footnote{Lit. ``I hear he's never shot---as for the thing called a gun.''}

63. H: Well, let's go see then. Cà-qā, please go and ask your father. Your father's
probably not going [hunting with us].

64. Cà-qā: Oh, he actually left yesterday. All by himself he's lying in wait
up there in the mountains, up on a rocky slope!

65. H: I guess he's gone then, \{he also/ him too\}.

66. So, let's take some pictures now.

67. Let's take a picture of all of us shooting guns, carrying guns on our shoulders,
aiming our guns \{at treetops/ high at the trees\}.

68. Cà-lɔ̂: What kinds of guns are they?

69. H: They're flintlocks! \footnote{\textbf{\emph{màʔ-tɛ̂ʔ}}\textbf{-nâʔ}: a muzzle-loading gun whose fuse must be lit manually, by hammer or match  < Shan    gun} They're big muskets. \footnote{\textbf{lɔ̂-kà-tù}: more or less synonymous with \textbf{màʔ-tɛ̂ʔ-nâʔ}.}

70. Somebody: that'll be OK, I guess.

71. H: They're very fine. They're Pie-talian! \footnote{\textbf{pítalêiʔ}: evidently a garbled version of `Italian'-cf. Thai \textbf{ʔiitaalii}. Perhaps a conflation of Thai \textbf{pyyn} 'gun' and \textbf{ʔiitaalii} 'Italian'.} They're called Pie-talian.

72. Cà-lɔ̂: What does ``Pie-talian'' mean?

73. H: ``Pie-talian'' means šÁn-phɔ̀ʔ \footnote{I have been unable to trace this word, which might well be a jocular invention of the headman.}---it's šÁn-phɔ̀ʔ!

74. Cà-lɔ̂: Well, what does šÁn-phɔ̀ʔ mean then?

75. H: It means Pie-talian, I told you. šÁn-phɔ̀ʔ.

76. Cà-g̈âʔ: Where do they say that the Pietalian guns came from?

77. H: A long time ago they came from a white man's country.

78. Cà-g̈âʔ: Which white man's country?

79. H: A white man's country down there in the west, in the southern part.

80. Cà-g̈âʔ: A place where black people live?

81. H: White people, a white person's country. White people.

82. Cà-g̈âʔ: I bet they're not white, they're probably black.

83. The black people's country is still to the south. <changing the subject>

84. Well, let's take some pictures, one person in each picture, carrying guns on
our shoulders, on our backs \footnote{\textbf{tâʔ}: `carry on shoulder'; \textbf{pû }`carry on the back'.}---all of us.

85. Oh boy, today the sun is so hot! We went hunting---we're exhausted.

86. Woman: You men go hunting, we women catch frogs.

87. I didn't get to shoot a single animal. Nobody drove any animals towards me.

88. Down below me where you were you should have followed the animals' trail properly.
There were plenty \footnote{\textbf{cí-cí mâ hêʔ}: a pleonastic expression, lit. ``not just a little bit,'' i.e., `plenty of'.} of barking-deer tracks.

89. You guys drove the animals \textit{away}. \footnote{I.e., instead of towards me.} They went down from the mountain
peak over there.

90. If you had only shouted and driven them over there towards me on the high ground
above the river, I would have caught a barking-deer!

<unintelligible passage>

91. The barking-deer are still over there. Under the rocks.

92. Woman: Your wife and children are waiting for something to eat at home.

93. H: Oh, those damn barking-deer! They don't show themselves where people are.

94. Since you guys drove [the game] that way for me, today we just went wandering
around wasting time.

95. Very soon now we'll go hunting together properly in the jungle.

96. Woman: That's the way it is! We're not going to get anything to eat. \footnote{From the hunting of you men.} If
we [women] go catch some frogs we'll have something to eat.

97. H: Well, there's an old proverb [about this]. You're wrong. ``Think not that
ye shall always fail, nor suppose that ye shall always succeed.'' Don't think you
won't ever get anything to eat, whether you get anything or not [on any specific
occasion].

98. Never get weary of \{going/ making forays\} into the jungle---think ``I'll
catch something to eat,'' and go look for it!

---------------------------------------------------------------------------------------------------------------

