\setcounter{footnote}{0}

1. Ty: Well, Khumaw, where would you like to go anyway?

2. K-m: I've come to visit.

3. Ty: Aren't you planning to go visit over \textit{there} now?\footnote{I.e., `wouldn't you rather go find some girls?'}

4. K-m: No, I came to visit \textit{here}.

5. Ty: Pa-eh, what about you?

6. H: He's probably going to go around visiting on a day like today, Sunday. There's
no sense in just hanging around.\footnote{\textbf{gɨ̂} \textbf{chɛ̂} `be doing for pleasure, be fooling around' vs. \textbf{ca} \textbf{gɨ̂} \textbf{tô} `go around for pleasure, go on a round of visits.'} You shouldn't\footnote{Clause + \textbf{(m)â hêʔ o} `one shouldn't Clause'; lit. ``it is not (a case of) clause.''} just lie around blinking
your eyes sleepily! You're just wasting time!

7. K-m: We'll go court the girls over there.

8. H: So you're going courting, huh?

9. K-m: Don't you guys want to go?

10. Ty: Where to?

11. Over there--you know.

12. T: Over there, at Me-jwe, like you said!

13. H: Up there around Ven-pa-may, up there under a big banyan tree there's a little
place where they do it! Don't you know the way [to get there]?

14. Ty: I'm definitely going to go!

15. Pà-ɛ́: We have three or four people [who want to go]!

16. H: Get together and go, get together and go, you young fellows. When I was your
age\footnote{Lit. ``when I was like you.''} I used to go, all right!

17. Kachin: You should please cut me in on the deal!\footnote{Lit: ``Please share together with me.'' The Kachin (Jingpho) was an older man, something of a clown. He was still single, since Lahu girls hesitated to marry members of different ethnic groups.}

\direct{Laughter. Everybody talks at once.}

18. H: How many are there, did you say? How many girls?\footnote{Less colloquially: \textbf{yâ-mî-há} \textbf{qhà-nî} \textbf{g̈â} \textbf{cɔ̀} \textbf{le} \textbf{qôʔ} \textbf{ve}.}

19. Pà-ɛ́: There are four or five of them.

20. H: Four or five?

21. Pà-ɛ́: Yes.

22. Ty: How many good-looking ones are there?

23. Pà-ɛ́: They're all good-looking.

24. Ty: There's one who isn't so bad as the others anyway!\footnote{\textbf{khɛ̂} \textbf{ve} < Shan \textit{khɛn} `superior': `be the best of a poor lot; be comparatively good.'}

25. Pà-ɛ́: There is one---she's a little better than the others.

26. Ty: That's the one! The one for me!

27. : No, no, no! ... She's mine!

28. : Like hell she's yours\footnote{\textbf{thôo} \textbf{khɔ̂} + X: ``what nonsense to say X!'' \textbf{thôo} ꞊ \textbf{pòthôo} (Interj.), \textbf{khɔ̂} `words.'}!

29. Pà-ɛ́: She's mine. I'll give you guys the second-rate\footnote{\textbf{ɔ̀-hɔ́} `that which is below; something inferior.'} ones.

30. Ty: Well, in that case I'm not going---I won't go.

31. H: I'll divide them up for you, I'll divide them up for you. As for Numero Uno
\footnote{The headman uses Thai numbers for the rankings, for comic effect. This is approximated by the Spanish translations.}, she'll be for this gentleman. Numero Dos is for that guy, and this guy gets
Numero Tres.

32. Ty: Hey, that's no good, Numero Uno is mine!

33. H: Oh, you can't act that way!

34. T: Good grief, you have no sense of shame\footnote{Lit. ``you don't even know the way of shame.''}!

35. Pà-ɛ́: Numero Uno is mine!

36. Ty: Well, don't fight about it like that!

37. Pà-ɛ́: The one who is Numero Uno is mine, I tell you! She's not yours.

38. Ty: You guys are friends, and there you are fighting! You're sure not acting
like friends.

39. Pà-ɛ́: We're laying conflicting claims to a girl, see?\footnote{\textbf{vɛ} \textbf{dàʔ} \textbf{ve} (V+Pv): `to vie; dispute a claim with one another.'}

40. T: He says they're each claiming the same girl.

41. Kachin: Well, if you people don't want her, why don't you parcel her out to
me!

42. Ty: She's tiny, why don't you tell him! That ``Numero Uno'' of his.

43. Pà-ɛ́: Yes, Numero Uno certainly\textit{ is} mine.

44. Ty: Let's take her---if only we succeed in courting her.

45. H: Well, so this Numero Uno is a tiny one, is she!

46. Pà-ɛ́: You guys are impossible, just impossible! We can't go on arguing this
way.

47. T: Are you going to have a fight?

48. Pà-ɛ́: We'll have a fight.

49. T: Who was the first one to see that girl?

50. Pà-ɛ́: I saw her first.

51. Ty: After he saw her he said he'd give her to me if I wanted her! Right there
on the road [was where he said that]!

52. T: Then whoever saw her first, that fellow gets the good-looking one. As
for the inferior ones, the ugly ones, we'll give them all to the Little Red Savage
\footnote{\textbf{Cà-qu-ní} ``The Red Naked One.'' This is Thû-yì's nickname.} here!

\direct{Laughter}

53. Pà-ɛ́: The Kachin said he'd take one like that.

54. Ty: Well, whoever wants that kind can have them.\footnote{``If it's that kind, whoever wants to get them, gets them.''} Pà-ɛ́, let's you and
me go visiting someplace else! It's not as if there were no other fish in the sea
\footnote{Lit: ``because it is not the case that there are [girls] only in this one place here.''}, you know. Otherwise\footnote{I.e., if there were no girls anywhere else.}, no matter how nicely we discussed it, we'd have
to come to blows, friends that we all are.

55. Pà-ɛ́: There are some over around Pá-lón and Pá-màʔ.

56. H: There are millions of them, I bet, when it comes to that.

57. Pà-ɛ́: Plenty of `em, plenty of `em.

58. H: Oh, so you're very well informed, are you?\footnote{Lit. ``Do you know very well?''}

59. Pà-ɛ́: I know whereof I speak!\footnote{Lit. ``I really know!''}

60. Ty: I'll bet they're probably not very good-hearted.

61. Kachin: I know a thing about it too.

62. Pà-ɛ́: Since there are pretty ones \textit{and} ugly ones---

63. T: Well, if you observe how a female\footnote{\textbf{ɔ̀-sɛ̄-ma}, lit. `a female body.'} does her work, and her behavior
\footnote{\textbf{tâ-hêʔ-tâ-càʔ} `behavior; what one has accomplished in life' < Shan.}, then you'll know!---whether she has a \textit{kood} \footnote{Càbo says \textbf{tàʔ} instead of \textbf{dàʔ} `good', imitating the Kachin's tendency to devoice Lahu stops.} character or not.

64. H: Look carefully [for a wife], my boys! It's not as if [you were buying] a
basket or something, you know. You don't get one today and throw it away tomorrow!
\footnote{\textbf{hə̂ʔ} means both `get' and `marry.' \textbf{bà} means both `throw away' and `divorce.'}

65. Pà-ɛ́: They \textit{are} like baskets!

66. H: According to the wise old saying, [a wife] is the ``flower of the hearth
and home.''\footnote{\textbf{yɛ̀-vêʔ-qa-vêʔ}: \textbf{vêʔ} `flower'; yɛ\textit{̀...qa} `elaborate couplet for `house'.}

67. T: If you want to split up from somebody like that, it's your responsibility!

68. H: You're not buying a naked rabid dog, so if you don't choose carefully, you're
liable to get tired of being married to her. So watch your step!

69. T: If you want to break up after you're married just go ahead and marry
someone like that.

70. Pà-ɛ́: If she doesn't have a good character then she's an old basket. If
she does have a good character, she's not a basket.

71. H: Hm, well, then, why don't you fellows stop this business now. Let me tell
you a story. Just listen. How about it?

72. Pà-ɛ́: O.K.

