
98 Resurrection of a Barking Deer and Fulfillment of a Prophecy

[former title'' Barking Deer and Swallowed Berries, or He did it after all]

1. Once upon a time there was a married couple.\footnote{chɔ  tê  ca. The storyteller uses the singular 3p pronoun yɔ̂ in what follows when referring to the couple  Nqh Num  Clf}

2. They were very lazy.

3. They didn't work.

4. So day after day they just stayed home.

5. Well, one day came when they were getting hungry.

6. Since they were so hungry, they went off into the \{forest/jungle\} to dig for
yams to eat.

7. While they were off to dig for yams to eat, they found a barking deer \footnote{chɨ-pí-qwɛ̀ʔ: A small deer (\textit{Cervulus muntjac}; \textit{Muntiacus muntjac}) about 3 feet tall, known as the best eating in the jungle, but now hunted virtually to extinction in thIland. Its cry, conventionally imitated as \textbf{pé-pí} in Lahu, somewhat resembles the barking of a dog. See DL: 835.} that
had choked to death on a wild gooseberry!\footnote{qhâʔ-cÁ-šī: \textit{Sesbanea sesbans }or \textit{Phyllanthus emblica}. See DL:286.}

8. So then the two of them, husband and wife, carried the barking deer back home.

9. So they turned back and arrived home, and put it into a big basket for storing
\{rice/paddy\}, and set some water on the fire to boil.

10. Meanwhile their children were playing around stroking the barking deer.

11. The kids were saying, ``Are you the one that says \textbf{pé}? Are you the
one that says \textbf{pí}?,'' joking and playfully stroking it.

12. Then all of a sudden what was \{choking it/caught in its throat\} was stroked
back into \footnote{This complex verbal idea (stroke back into) is conveyed by a three-verb concatenation: šôʔ       qhɔ̂ʔ      kə.  Vn       Vv        Vv  stroke go back insert  Another way of saying this: g̈ô-pē  qhɔ  šôʔ  yàʔ e šē lɛ...   stomach inside stroke descend} it[s stomach], and it jumped up and ran away.

13. So then this couple went to a fortune-teller \footnote{mɔ́ `teacher; master; expert; possessor of occult knowledge' < Tai *cf, Si \textit{mɔ̌ɔ}). See DL:1029.} to ask a question.

14. He \footnote{The pronoun is not expressed, but the context makes it clear that it is the husband who is doing the talking to the fortune-teller.} said to the fortune-teller, ``Oh, master, will I be able to get my animal
back, or won't I be able to get it back?,'' he asked.

15a. The fortune-teller said, ``Ah, that animal of yours you cannot get back.

15b. ``If you try to do so, you will just have to lick your wife's `thing'.''\footnote{\textbf{mɔ̂ }`thing' is here used as a euphemism. Cf. also the expression \textbf{ŋà mɔ̂ lɛ̀ʔ} `lick my thing!', where it is not specified which thing is meant. See DL:1031.}

16. At this he \{retorted/answered back\}, ``Oh, you miserable fortune-telling
sonofabitch!'',\footnote{\textbf{mɔ́-khì-mɔ́-tələ̂ʔ} (ElabN): N1 + khɨ̀ + N1 + tələ̂ʔ is a template for forming abusive insults to N1. See DL:371. \textbf{khì} `shit' < Tai (cf. Si. \textit{khîi}); the origin of tələ̂ʔ (evidently a Tai loanword) is obscure.} and \{drove/chased\} him away.

17. After this, when they reached a certain point [on their way home], rain came
pouring down, and they got drenched and were freezing.

18. When they got back home, the husband and wife both sat down in front of the
tripod in the fireplace and warmed themselves by the fire.

19. Then he caught sight of his wife's thing, and stuck a finger right inside,\footnote{\textbf{g̈ɔ̀ jûʔ phɛ̂}: \textbf{jûʔ} `stick in'; the prehead versatile verb \textbf{g̈ɔ̀} and the posthead one \textbf{phɛ̂} both convey a lively flavor to the verbal event.  vV Vn Vv  drag      send forth}
and it disappeared in there. [When he took it out it was all wet, so] he wiped
it off on the tripod.

20. When he wiped it off on the tripod, he [his finger] got bured, so he licked
it vigorously.

21. Then all of a sudden he remembered.

22. ``What the fortune-teller told me just now was true!'', he realized.

