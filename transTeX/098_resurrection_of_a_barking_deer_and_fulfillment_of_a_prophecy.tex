\setcounter{footnote}{0}

1. Once upon a time there was a married couple.

2. They were very lazy.

3. They didn't want to work.

4. So day after day they just stayed home.

5. Well, one day came when they were getting hungry.

6. Since they were so hungry, they went off into the woods to dig for yams to eat.

7. While they were off to dig for yams to eat, they found a barking deer\footnote{\textbf{chɨ-pí-qwɛ̀ʔ}: A small deer (\textit{Cervulus muntjac}; \textit{Muntiacus muntjac}) about 3 feet tall, known as the best eating in the jungle, but now hunted virtually to extinction in Thailand. Its cry, conventionally imitated as \textbf{pé-pí} in Lahu, somewhat resembles the barking of a dog. See DL:835.} that
had choked to death on a

wild gooseberry!\footnote{\textbf{qhâʔ-cá-šī}: \textit{Sesbanea sesbans }or \textit{Phyllanthus emblica}. See DL:286.}

8. So then the two of them, husband and wife, carried the barking deer back home.

9. So they headed back and arrived home, and put it into a big basket for storing
rice, and set some water

on the fire to boil.

10. Meanwhile their children were playing around stroking the barking deer.

11. The kids were saying, ``Are you the one that says \textbf{pé}? Are you the
one that says \textbf{pí}?'', joking and

playfully stroking it.

12. Then all of a sudden what was caught in its throat was stroked back into its
stomach, and it jumped up

and ran away.

13. So then this couple went to a fortune-teller\footnote{\textbf{mɔ́} `teacher; master; expert; possessor of occult knowledge' < Tai (cf, Si \textit{\textbf{m}}ɔ̌ɔ). See DL:1029.} to ask a question.

14. He said to the fortune-teller, ``Oh, master, will I be able to get my animal
back, or won't I be able to get

it back?,'' he asked.

15. The fortune-teller said, ``Ah, that animal of yours you cannot get back.

16. ``If you try to do so, you will just have to lick your wife's `thing'.''\footnote{\textbf{mɔ̂} `thing' is here used as a euphemism. Cf. also the expression \textbf{ŋà} \textbf{mɔ̂} \textbf{lɛ̀ʔ} \textbf{ôʔ} `lick my thing!', where it is not specified which thing is meant. See DL:1031.}

17. At this he retorted, ``Oh, you miserable fortune-telling sonofabitch!'',\footnote{\textbf{mɔ́-khì-mɔ́-tələ̂ʔ}: N1 + \textbf{khɨ̀} + N1 +\textbf{ tələ̂ʔ} is a template for forming abusive insults to N1. See DL:371. \textbf{khɨ̀} `shit' < Tai (cf. Si. \textit{khîi}); the origin of \textbf{tələ̂ʔ }(evidently a Tai loanword) is obscure.}
and chased him away.

18. After this, when they reached a certain point [on their way home], rain came
pouring down, and they got drenched and were freezing.

19. When they got back home, the husband and wife both sat down in front of the
tripod in the fireplace and warmed themselves by the fire.

20. Then he caught sight of his wife's thing, and stuck a finger right inside,\footnote{\textbf{g̈ɔ̀} \textbf{jûʔ} \textbf{phɛ̂}: \textbf{jûʔ} `stick in'; the pre-head versatile verb \textbf{g̈ɔ̀} and the post-head one \textbf{phɛ̂} both convey a lively flavor to the verbal event.}
and it disappeared in

there. When he took it out it was all wet, so he tried wiping it off on the tripod.

20. When he wiped it off on the tripod, he [his finger] got burned, so he licked
it vigorously.

21. Then all of a sudden he remembered.

22. ``What the fortune-teller told me just now was true!'', he realized.

