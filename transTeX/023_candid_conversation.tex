
23 Candid Conversation: Warming up for one of the first recording sessions.

``H'' Headman: [explaining how playlets are acted out for recording---e.g., suppose
we pretend we're buying baskets]

\texttt{<}Playing role of buyer\texttt{>}

1a. We're recording.

1b. How much does one basket cost?

\. You carried them here on your shoulder \footnote{tâʔ ve: `carry on one's shoulder'}, right?

\. I'll act as if I'm your customer.

------------------------------------------------------------------------------------------------------

\texttt{<}Playing role of seller\texttt{>}

\. Oh, let's say one basket costs five baht. \footnote{ŋâ bàʔ qo ŋâ bàʔ: lit. ``if it's five baht it's five baht.''}

\. If you want to \{get/buy\} a basket it's up to you.

\. As far as the price goes.

------------------------------------------------------------------------------------------------------

\. Each person play a different part.

\. That's the way you speak, at the beginning [of the recording session].

\. Things---talk about the tools and utensils, all the kinds of things we use.

1\. All the kinds of baskets, right?---all the kinds of things we use, knives---

1\. Cà-bo ``(T)eacher'': This fellow now---[should take such-and-such a role]

1\. H: I'll speak first.

1\. T: Hey, blacksmith, forge a knife for me!

1\. H: [playing role of blacksmith] What kind of---what kind of knife?

14b. [whispers] Ask me another question?

1\. T: Oh, I'd like to have a portable \footnote{Á- thɔ pû tô  tù} knife.

1\.  ([whispers] Ask whether the head should be pointed or blunt.)

1\. H: Yeah, pointy-headed or blunt-headed?  Ask me \{another/something else\}.
Say you want one with a carved tip.

1\. Boy: Say you want a pointy one!

1\. T: Just ask something when you get an idea, and I'll answer you.

2\. Whatever you say, whatever you want to ask is up to you.

2\. Even if there's nothing to answer---

2\. H: You guys \footnote{Addressing those who aren't participating in the recording.} don't have other conversations, OK?

2\. T: In the same way, [let's talk about] everything you have to do when it's
time to build a house.

2\. We'll talk with each other so that each person will talk about a different
thing.

2\. There are several kinds of wood [involved].

2\. How we have to carry the house-posts. \footnote{From the forest to the village house-site}

2\. H: About killing pigs to eat.

2\. Do you want to \{get/record\} that first?

2\. Cà-lɔ̂: That too, after we've finished about house building, [then] all
this about killing pigs, OK?---and all about killing chickens to eat.

2\. All about going down to Chiang Mai to go shopping.

3\. How you go there, where you eat, where you spend the night how---

3\. OK, OK, OK, we'll help you, we'll help you.

3\. Cà-lɔ̂: How you ride [down there]---that sort of thing.

3\. That kind of thing isn't hard [to record extemporaneously].

3\. These are things that it's easy to think up.

3\. T: Talk about how much you have to pay for bus fare---oh, for one person it's
five and six baht \footnote{Five one way and six the other.}---damn expensive!

3\. Cà-lɔ̂: Whether that's a lot, or not a lot---

3\. T: Yes.

3\. Cà-lɔ̂: That business about quarreling with the bus driver---

3\. T: If it comes to five baht and six baht, altogether that's eleven baht---nobody
has ever had to pay that much [laughs].

4\. Cà-lɔ̂: The seats are no good either, you should say!

4\. T: Even so \footnote{qhe kàʔ: i.e., no matter how expensive it is} we don't even get to sit in a good seat, somebody should say.

4\. Cà-lɔ̂: Once you get there, how do you get off? \footnote{I.e., what do you do after you get off?}

4\. H: So then, I'll ask you things first, OK?

4\. I'll talk about all sorts of things---well---

4\. T: If one person can't answer, somebody else answer quickly!

4\. Think, and answer whatever you can manage to answer.

4\. H: About everything, right?  And no laughing!

4\. T: It doesn't matter if you laugh. Laughing---as kong as it's just a little
bit.

4\. I doubt we can do it without laughing. (laughter)

5\. H: Don't laugh, don't laugh, don't laugh!

5\. T: When it's time to laugh don't laugh all by yourself.

...

5\. Kachin: \texttt{<}getting into the mood\texttt{>}

Oh, say that the house-posts are all bent! (laughter)

5\. Cà-lɔ̂: The houseposts aren't driven in well!  They're not driven in properly!
That sort of thing.

5\. H: Don't laugh, be quiet!

