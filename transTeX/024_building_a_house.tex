\setcounter{footnote}{0}

T=Teacher (Pastor Cà-bo)

H=Headman (Cà-bí)

Issue: how much to vary translation of \textbf{âa}, which begins most of the turns. Usually
translated ``well.''

1. H: Well, we'll build a house.

2. T: Well, just how shall we build it, the house?

3. H: Well, once we've collected all the materials, we can do it!

4. T: Well, there's one thing we don't have enough of yet! What we call the roof-support
beams\footnote{\textbf{tó-làʔ}: see the diagram of a Lahu house under construction in DL, Plate 68.} aren't enough yet.

5. H: Well, we'll just go and cut\footnote{\textbf{bàʔ} \textbf{ve}: `cut into lengths of wood; cut into boards'.} some. We'll take one or two people along.

6. T: With only two people you won't be able to carry them back [from the forest
to the village].

7. H: Let four people carry them, four people.

8. T: Yeah, as long as four people carry it, they'll probably succeed.

9. H: Let everybody look for all the other stuff, all the things [we'll need].
You guys should select people, and go look for everything.

10. T: So who all will go with me---to go and cut some roof-support beams?

11. H: Well, Cà-qā, you should be one to go too! Will you go?

12. Cà-qā: I'll go.

13. H: Cà-ye, please go too.

14. Cà-ye: I'll go.

15. H: That makes two people, right? Cà-bo, are you going too?

16. T: Sure I am!

17. H: Thûyì, please go too.

18. Thûyì: OK.

19. H: That makes four people. That's enough, right?

20. T: If we've got four people, that's enough. Four people will carry the beams
for me, and when we've succeeded [in getting them to the building site] we'll just
put them in place!

21. H: Uh-huh.

22. T: What day shall we do it on?

23. H: I say we ought to do it on Tuesday.

24. T: Tuesday.

25. H: Yeah. As for food\footnote{\textbf{ɔ̄-chî}: `that which is eaten with rice', lit. ``rice-lifter'', often conventionally translated as `curry'. The person for whom the volunteers are working is normally expected to feed them.}, what they'll feed you guys probably won't be very
tasty.

26. T: Well, there's enough canned carp\footnote{\textbf{pa-mô-pa-kâʔ}: \textbf{pa-mô} `carp', \textbf{pa-kâʔ} `canned fish' (< Thai \textit{plaa-klàk}).} for the whole group, as far as food
goes. We Lahu we help each other! It's not that we just look at what food there
is to eat!

27. H: We should just kill one or two chickens to eat. In that case, a great big
cock and a hen, two nice fat animals, should be enough.

28. T: Oh, two chickens and a half should be enough!

29. H: \direct{laughs}\footnote{The laughter is because everybody knows that chickens are hard to come by in the impoverished village.} Well, as long as they're big it'll be enough.

30. T: How many people are there who would like to eat pork\footnote{Pork, the best of meats, is only eaten a few times a year, and is even less likely than chicken to be offered to the house-builders.} then, among all
of you house-builders?

31. Kachin: Well, just the whole village --

32. Somebody: Well, I'd like to eat it too.

33. Somebody else: Well, as for me, I'll take the pig intestines---I love to eat
them.

34. T: Pā-ɛ́\footnote{The name of \textbf{Yâ-pā-ɛ́} (lit. ``Little Son'') is often shortened to \textbf{Pā-ɛ́}.} , what about you?

35. Pā-ɛ́: The kidneys, the kidneys.

36. T: The kidneys. Thû-yì, how about you?

37. Ty: The stomach.

38. T: The stomach.

39. Kachin: Unless there's pig-meat it still won't be enough.

40. T: Well, I'd like to eat the [pig's] head boiled, the head boiled.

41. Somebody: I'd like to eat it too, your boiled head.

42. H: I'd like to eat the meat chopped up fine.

43. T: Sure, let everybody eat whatever he likes.

44. T: Well, then,\footnote{Returning to the topic after the food-fantasy is over.} is there enough thatching-grass already, you guys who are
building the house?

45. H: Well, eighteen sheaves of thatch have arrived, it should be enough.

46. T: Oh, eighteen sheaves.

47. Somebody: Mm-hm.

48. T: If there are eighteen sheaves, that'll probably be more than enough.\footnote{\textbf{lə́} \textbf{lɛ̀} \textbf{lə́} \textbf{tù}: lit. ``as for there being more than enough, there will be more than enough".}

49. H: Try to figure it out---on this house, how many tiers [of thatch] will there
be on one side [of the roof]?

50. T: In olden times, the way our ancestors used to do it, on one side there were
six tiers! On the other side they used to make seven tiers.

51. H: Thirteen tiers, right?

52. T: They used to make thirteen tiers indeed.

53. H: Well, if it's thirteen tiers, that's fine. Not reaching in even number
[on the two sides of the roof] is an advantage for us.\footnote{\textbf{ɔ̀-cɛ} 1. `a pair (including cosmological pairs like sun and moon)'. 2. `an even number'. Traditional Lahu roofs are pitched, with the two sides slanting down from the ridgepole.}

54. If we make six tiers on each side, that would make twelve tiers for both sides,
which is an even number.

55. T: That's not good. Building a house you can't let them reach an even number.
They say it would cause a lot of climbing up.\footnote{In order to fix problems with the roof.} That's what our ancestors say.

56. T: Well, we've just got to find somebody who's really good at it. Yeah, somebody
who's the best. The best one is Thû-swè! As a house-builder.

57. H: Let's us guys go and ask him.\footnote{There are two inversions in sentence (80). The normal order would be: \textbf{ŋà-hɨ} \textbf{yɔ̂} \textbf{thàʔ} \textbf{ca} \textbf{na-ni} \textbf{nē}.} Invite him and say ``Please
help us!"

58. Kachin: Yeah, it's Thu-shwe who's really good at climbing up a house!

59. H: Send somebody to fetch him, send somebody to fetch him!

60. T: Who will go?

61. Somebody: Let Cà-ye go, Cà-ye.

62. Kachin: hurry up and go, hurry up and go!

63. Boy: I'll go!

64. H: He's gone, he's gone! He'll be back in a minute.

65. T: That---

66. H: What is it, your ``that''?

67. T: That eggshell! \direct{Laughter}.\footnote{Everybody evidently felt that this was a snappy comeback. Perhaps some previous incident is referred to.} Well then, now let's keep
going. Once the house is built, we've got to install all the fireplace-props!
Have you almost installed them? Is the fireplace there already?

68. H: Down there above the river they've cut them\footnote{That is, the wood for the fireplace props.} into boards and left them.
Let's go up there and carry them back here. You guys find people [to do it].

69. T: You mean over there on the high ground near Ping River\footnote{\textbf{Na-bɛ̀} \textbf{lɔ̀-qá}: probably refers to the River Ping (also Lahuized as \textbf{nà-pèn} or \textbf{nà-pên}), a tributary of the Chao Phraya, having its source near Chiang Dao, quite close to Huey Tat.} ?

70. H: Yeah.

71. T: Oh, hell\footnote{By this mild expletive I translate a string of final exclamatory particles: \textbf{vɨ̂} \textbf{à} \textbf{yâ-o} \textbf{nē}.}, that's too far, where you [did it], at such a distance. Wasn't
there a closer place?

72. H: Oh, down there where the rice-mill is at that fork in the road there's one
[tree] already cut into boards. I don't know if it's any good now!\footnote{Perhaps the Headman is concerned that the wood for the fireplace was left sitting there too long.}

73. T: What kind of wood is it?

74. H: It's jupi wood, jupi wood.\footnote{\textit{Michelis champaca (Magnoliaceae)}.}

75. T: If it's jupi wood, that's fine. Somebody go and climb up -- Thû-šwè,
you go climb up. Are you going?

76. Thû-šwè: I'll go.

77. T: Yes, that's good!

78. Somebody: Cà-bo, you cook the rice and curry! Is the food all cooked yet?

79. Somebody else: Hey, I'm hungry too. It's time to eat.

80. T: It's not done yet, not done yet. It's not time yet. Just wait a while.

81. Kachin: Hurry up and spread out the bamboo slats!\footnote{\textbf{tha-phî}: split bamcoo slats used for walls and flooring.}

82. Somebody: And the sun is awful hot. It's time to have lunch! Aren't you hungry,
you guys?

83. T: The others have already finished with their house-posts. But you're taking
such a long time with yours, that I told you to work on. I've never seen the like!
\footnote{\textbf{chɔ (m)â mɔ̀ jɔ}: lit. ``people have never seen''.}

84. H: [What do you mean by saying] you've ``never seen the like''!
\textit{You're} taking so long cooking the food.

85. Cook: Come and eat, come and eat! It's ready, come eat!

86. T: He says ``Come eat''! Kachin, you get to say grace. Kachin\footnote{A long-time resident of Huey Tat, of Jingpho ethnicity. Everybody good-naturedly calls him \textbf{Khá-pā} `Kachin'.}, will
you say grace\footnote{\textbf{bo} \textbf{lɔ̀} \textbf{ve}: 1. `pray (in church)' 2. `say grace before meals', lit. ``beg for favor".}?

87. Kachin: Aw, I can't say grace properly yet. Pastor, you do it!

***

88. H: Well, since we've finished eating, my boys, let's hurry up and do everything!

89. T: Yeah, it's high time we thatched the roof, it's time to thatch, otherwise
it'll get dark! The sky will be black! And if it rains, we might not be able
to find a place to sleep.

90. H: Hey, hurry and drag up all the house-posts and set them upright! Each and
every one of them.

91. Kachin: Grab all the house-posts, grab the house-posts!

92. H: Here, here, take and stick one over there! Put another one underneath!

93. T: Do it properly! These house-posts might end up in even numbers. Don't
let them be paired up! Ah, all the house-posts have been stuck into place already!
So now, take the roof-sticks and tie the thatch to them! Come on, guys, the thatch-sticks!
\footnote{\textbf{kɛ̂ʔ-šɨ̄}: bamboo sticks forming the horizontal framework of a roof, to which the tiers of thatch are attached..}

94. H: Take them and pass them up, pass up the thatch-sticks!\footnote{\textbf{yɛ̀-kɛ̂ʔ}: a synonym of \textbf{kɛ̂ʔ-šɨ̄}.}

95. Somebody (on the roof): Bring the bamboo-splits\footnote{\textbf{vâ-ne}: bamboo fibers used for tying.} too!

96. Kachin: Pass `em up, pass `em up, pass up the thatch-sticks!

97. T: Bring the bamboo-splits! He says to take all the splits that anybody gave
you!

98. Kachin: Splits, splits, splits---try to find the splits! Hurry up and bring
them!

99. Somebody: Tie [them] up quick!\footnote{I.e., tie the thatch to the roof-sticks.}

100. T: O.K., then let the tie-ers tie, while the thatchers thatch! The ones who
carried [the stuff] up there to tie have already done it---but there are so many
people here now.\footnote{Here the Teacher ostensibly turns to the people who have been standing around watching.} If you just hang around, there's no point to it. It's no
use at all. You're just wasting your time.\footnote{\textbf{šá} \textbf{g̈ɔ̀} \textbf{bà} \textbf{ve} \textbf{cɛ} \textbf{tí} \textbf{yò}: lit. ``you're just throwing away your breath".}

101. H: Well, the thatching is done! On the lower part there are only five or
six tiers. Here on top re-tie it again!

102. T: The group working up top began working this morning, before the others
had started! But still they haven't finished even now!

103. H: It's almost done, almost done. Only two tiers are left.

104. Somebody: We already knew that bunch up top didn't work very hard.

105. T: It's not just a question of doing it fast. Whatever you're working at,
don't be concerned at the way others are doing it.\footnote{I.e., `Don't try to keep pace with the others, or compare your output to theirs.'} What you \textit{are}
very clever at is going to visit girls\footnote{\textbf{šu} \textbf{yâ}, lit. ``other people's children'', here referring to marriageable girls.} all the time, going to sneak looks
at them!

106. Somebody: Oh, that's not so! It's because they weren't quick passing us up
the thatch, they weren't fast enough handing it up to us.

107. Somebody else: When it gets dark you say they're good at sneaking around ``other
people's daughters''! If they're made to tie lots and lots of roof-sticks.\footnote{I.e., `You shouldn't begrudge them their little diversions after a hard day's work.'}

108. T: That's true. In this world nobody needs a teacher for that sort of thing.
Well, we're finished at this point. All we have to do is thatch the ridgepole,
the ridgepole. Go look for thatch to make the overhang too, the overhang.\footnote{\textbf{yɨ̂-vêʔ}: lit. ``thatch-flower''. A projection of thatch on a roof to carry away rain, here translated `overhang'.}

109. H: Hey, hurry it up, hurry it up! There's already a place for it up there.

110. H: Pass it up, pass it up!

111. T: Here, here! Make the overhang properly---people will laugh if it's not
good!

112. H: Aw, we've done it before! If that's all there is [left to do].

113. T: If you've done it before, that's fine. Well, at this point this house has
been completely built, right? Now that it's all finished, what will you do with
this place here? Will it be the place where you cook food, this area?

114. H: Over here is where the fireplace will be installed.

115. T: The fireplace, over there.

116. H: Over there is the bedroom.

117. T: The bedroom. Is that the living room\footnote{\textbf{chɔ} \textbf{mɨ} \textbf{gɨ̂} \textbf{kɨ̀}: lit. ``place where people sit for pleasure''.}over there?

118. H: Yep. The place where visitors will sit.

119. T: There are three rooms. Each kind of place has its own name. So, if you
say it's finished, that's all there is to it.

120. Kachin: Ah, the house is all built now. Since many people will live here
\footnote{\textbf{chɔ} \textbf{mâ} \textbf{ā} \textbf{lɛ}: `since there are many people'; lit. ``since people will keep being numerous''}, the house is also very big.

