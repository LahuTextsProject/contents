\setcounter{footnote}{0}

<told in Yellow Lahu>

1. Once upon a time there were two men.\footnote{One a Black Lahu, the other a Yellow Lahu.}

2. [This story is about how] one day the two of them picked some greens

to boil up for a curry.

3. They got up to the fields, and when noontime came the two of them

were looking for [something to] boil up and eat\footnote{Note the sequence of four syllables identical except for tone.} for a curry.

4. So the Yellow Lahu said to the Black Lahu, ``Why don't we go boil

up some cabbages?''

5. After a while they went and boiled the cabbage.

6. They cooked it, and it came to a boil.\footnote{\textbf{cá} is a transitive verb, `to cook by boiling'. \textbf{bî} is an intransitive verb `come to a boil'.}

7. After it was done, when he\footnote{I.e., the Black Lahu.} had mixed up some salt and hot

peppers [in a dish], he said ``Scoop out a little, will you, pal?''

8. Then, after a few seconds\footnote{Apparently the Yellow Lahu hadn't understood at all the first time. <Black Lahu translation follows in interlinear text>} he repeated, ``Scoop out a little salt,

will you, pal?''

9. Now the way the Black Lahu said [it was because] he didn't know

the Yellow Lahu dialect.

10. So that he [ended up] making the Yellow Lahu scoop out

some \textit{cabbage} [instead of the spice mixture].

11. Well, it was burning hot, and the Yellow Lahu jumped up and ran away.

