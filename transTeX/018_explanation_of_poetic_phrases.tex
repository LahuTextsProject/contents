\setcounter{footnote}{0}

1. This poem is ``Lordly a-yaw Tree\footnote{One of the four ``master trees'' (either \textit{Lagerstroemia cylindrica} or \textit{L. Macrocarpa}) in Lahu tradition. It is very tall and straight, and believed to be especially apt to be hit by lightening. It indicates good land for rice, chili, or opium. See DL p. 79.} a Lordly k'a-nyi Tree.\footnote{A huge tree indicating good land for all crops. See DL, p. 236.}''

2. ``á-lɔ́'' means ``the very latest/the very last''.

3. ``á-šɔ́'' means ``the most''.

4. So ``á-lɔ́-á-šɔ́'' means ``drive away the very last ones''.\footnote{The word \textbf{á-lé} is defined below.}

5. Then, there is ``mû kə̀-kə̀ â hêʔ'' [``the sky does not move''].

6. The ``kə̀'' in ``mì kə̀-kə̀ â hêʔ'' means ``move away/shift''.

7. That means ``move away/shift''.

8. As for ``á-lé'', it means ``drive away'' or ``protect by moving away''.

9. Then, it is said ``the sky does not move, the earth does not move.''

10. It says, ``the moon, the moon and the stars, the sun and the stars, sometimes
they can move.''

11. They say ``the stars are now above the mountains''.

12. But ``the sky doesn't move'', it is said.

13. ``Look how the full moon\footnote{\textbf{tɔ̄-ha-pa}: in ordinary language this is \textbf{ha-pa} \textbf{tɔ̄} `the moon is full'.} changes. But the sky does not change'', they say.

[The speaker dissolves in laughter]

