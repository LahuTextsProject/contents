
18 Explanation of poetic phrases

1 This poem is ``Lordly a-yaw Tree\footnote{One of the four ``master trees'' (either Lagerstroemia cylindrica or L. Macrocarpa)} a Lordly k'a-nyi Tree\footnote{A huge tree indicating good land for all crops. See DL, p. 236.}.''

2 ``Á-lɔ́'' means ``the very latest/the very last''.

3 ``Á-šɔ́'' means ``the most''.

4 So ``Á-lɔ́-Á-šɔ́'' means ``drive away the very last ones''.\footnote{The word Á-lé is defined below.}

5 Then, there is ``mû kə̀-kə̀ â hêʔ'' [``the sky does not move''].

6 The ``kə̀'' in ``mì kə̀-kə̀ â hêʔ'' means ``move away/shift''.

7 That means ``move away/shift''.

8 As for ``Á-lé'', it means ``drive away'' or ``protect by moving away''.

9 Then, it is said ``the sky does not move, the earth does not move.''

10 It says, ``the moon, the moon and the stars, the sun and the stars, sometimes
they can move.''

11 They say ``the stars are now above the mountains''.

12 But ``the sky doesn't move'', it is said.

13 ``Look how the full moon\footnote{tɔ̄-ha-pa: in ordinary language this is ha-pa tɔ̄ `the moon is full'.} changes. But the sky does not change'', they say.

[The speaker dissolves in laughter]

