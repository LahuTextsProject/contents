\setcounter{footnote}{0}

1. Once upon a time there was this old fellow.\footnote{\textbf{lɔ̂-pū}: this word ( < Chinese \TC{老夫} (Mand. \textit{laofū}), and perhaps influenced by Lahu \textbf{pū} `grandfather'), has several shades of meaning: (a) `an old man'; (b) `old boy; old fellow' (jocular term of address to any man); (c) `husband'; (d) `my old man' (wife referring to her husband). See DL:1400.}

2. He would go off to his wet-rice field to plow, spending the night in the paddy-field,
along with his son, but his wife wouldn't go along.

3. One day his son, who had gone home to fetch some polished rice [for his father's
meal], came back [to the field], and as he reached the edge of the paddy-field,
he looked at his father -- and there was his old man, screwing a female water-buffalo!

4. So the kid ran back home right away, and told his mother.

5. Then when the mother found out about this, she packed up some more rice to carry
out and went off to her husband, and there they were sitting and eating together
outdoors.

6. Now that day some rabbit-hunters happened to come to the village!

7. Well, as they came chasing the rabbit, making a great racket, the old fellow
asks his wife ``What's going on there?''

8. So the wife says, ``They've come to beat that buffalo-bugger to death!''

9. So all of a sudden, without even finishing his rice, he runs off fleeing for
his life.

10. Well, as he was running away, the rabbit happened to flee after him in the
same direction, and when they [the hunters] saw this they said, ``Come
on, let's chase him with all our might and catch him!" -- so he ran away
for dear life.

11. Finally when he had run [all he could] and he was at bay, he said, ``But
the gray buffalo wasn't yours! And the black one wasn't yours either!"
\footnote{The humor of this punch line consists in the fact that the old fellow now reveals that he hadn't confined his bestial propensities to a single partner, but had `played the field', as it were.}

