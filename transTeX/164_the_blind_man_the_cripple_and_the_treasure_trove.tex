\setcounter{footnote}{0}

1. Once upon a time there were two people, a cripple\footnote{\textbf{khɨ-qɔ̀ʔ-pā}. Later in the text he is referred to as \textbf{cɔ̀ʔ-qɔ̀ʔ-pā} `hunchback' or \textbf{khɨ-cɔ̀ʔ-qɔ̀ʔ-pā} `person with a crooked leg and back'.} and a blind man, and when
other people were celebrating New Year's\footnote{\textbf{qhɔ̀ʔ} \textbf{câ} \textbf{ve}, lit. ``eat the year''.}, the two of them did not get to celebrate.

2. They wanted to celebrate, but they had no money.

3. So the two of them talked it over.\footnote{At the end of many sentences in this text, \textbf{qôʔ-ve} `it is said' is equivalent to the quotative P.uf (final unrestricted particle \textbf{cê}, and is usually best left untranslated).}

4. As the two of them were talking it over, they said ``Let's go steal that plow
that they left down there!'', and so they went and stole the plow that had been
left, and the cripple made the blind man carry it\footnote{Nothing more is said about this plow.}, then he [the cripple] took
a walking-stick and walked on ahead. 5 So they went on and on this way, and got
to a strange village, where a person possessed by an evil spirit\footnote{\textbf{nê-cɔ̀-pā}. For more information on the animist concept of \textbf{nê} `spirit/demon', see DL: 775-778 and the numerous works of Anthony R. Walker.} was spending
the night in a house!

6. So the spirit-possessed person was spending the night in the house, and that
spirit then said to the two of them, ``I want to bite you and eat you all up.''

7. At this point, since the two of them had a gong and the spirit also had a gong,
\footnote{It is not clear why either side happened to have a gong with them. The story would work better if the two of them had stolen a gong instead of a plow!} he [the evil spirit] said, ``Let's try a [gong] striking match!\footnote{\textbf{dɔ̂ʔ} \textbf{dàʔ} \textbf{ve}: ``strike mutually''.}''

8. ``If the sound of you guys lasts longer,\footnote{\textbf{g̈ɔ̀} (V) `drag, draw, pull out' here refers to drawing out a sound.} I won't get to eat you,
but if that sound of yours doesn't last longer, I will get to eat you,'' he said,
and when they tried beating [the gongs] against each other,\footnote{\textbf{g̈ɔ̀} \textbf{dɔ̂ʔ} \textbf{dàʔ} \textbf{a-ni}: \textbf{g̈ɔ̀}, which means `drag/pull' as a main verb, is here grammaticalized as a \textsubscript{v}V with an intensifying meaning. See DL: 1139.} the sound of the
blind man and the cripple's gong lasted longer, but the sound of the possessed
guy's gong did not last, so the possessed man totally\footnote{\textbf{qha-pə̀-è} (AE): `completely'.} ran away.

9. So after he had run away, they went and looked into the room where the possessed
man had been staying, and they said ``There's gold and silver in here!'', so the
two of them raced to divide it up, and the cripple said he would make two portions.

10. [But] he made three portions in all, and told the blind man to take one portion
while he took two, and having divided it up, [the blind man] checked it over\footnote{\textbf{ni} \textbf{a} \textbf{lɛ-ɔ̄}: the literal meaning of \textbf{ni} `look at' is obviously inappropriate for the blind man, so the verb must be understood in its metaphorical sense of `try something; do something and see'.}
and said, ``Hey, you divided up the money into three portions!'', so he said ``No,
I didn't! You try checking it out!'', so he felt around to check and said, ``You
did make three portions!''

11. So then they started beating up on each other, and the cripple struck the blind
man right there on the eyes, and his eyes flew open [so he could see].

12. Then when the blind man hit the cripple back, the cripple's body straightened
up.

13. So then the two of them, who could now see and now had a straight back, divided
up [the riches] equally, and returned home.

14. That's the end.

