
164 The Blind Man, the Cripple, and the Treasure Trove

1 Once upon a time, people --- er --- there were two people, a cripple\footnote{khɨ-qɔ̀ʔ-pā. Later in the text he is referred to as cɔ̀ʔ-qɔ̀ʔ-pā `hunchback'.} and a
blind man, and when other people were celebrating New Year's,\footnote{qhɔ̀ʔ câ ve, lit. ``eat the year''.} the two of them\{did
not get to celebrate/were not included\}.

2 They wanted to celebrate, but they had no money.

3 So the two of them talked it over.\footnote{At the end of this sentence, qôʔ ve `it is said' is equivalent to the quotative P.uf cê in story-telling style, and is usually best left untranslated.}

4 After the two of them had talked it over, they said ``Let's go steal that plow
that they left down there!'', and the blind --- er, the cripple made the blind
man carry it, then he [the cripple] took a walking-stick and walked on ahead of
him [the blind man].\footnote{See footnote [3] and elsewhere in this text. Nothing more is said about this plow.}

5 So they went on and on this way, and got to \{a stranger/another\} village, where
a person possessed by a demon\footnote{nê-cɔ̀-pā. For more information on the animist concept of nê `spirit/demon', see DL: 775-778.} was spending the night in a house!

6 So the demon-possessed person was spending the night in the house, and that demon
then said to the two of them, ``I want to bite you and eat you all up.''

7 At this point, since the two of them had a gong and the demon also had a gong,\footnote{It is not clear why either side happened to have a gong with them.}
he [the demon] said, ``Let's try a [gong] striking match!\footnote{dɔ̂ʔ dàʔ ve: ``strike mutually''.}

8 If the sound of you guys \{lasts longer/keeps going longer\},\footnote{g̈ɔ̀ (V) `drag, draw, pull out' here refers to drawing out a sound.} I won't get
to eat you, but if that sound of yours doesn't last longer, I will get to eat you,''
he said, and when they tried beating [the gongs] against each other,\footnote{g̈ɔ̀ dɔ̂ʔ dàʔ a-no: g̈ɔ̀, which means `drag/pull' as a main verb, is here grammaticalized as a v.V with an intensifying meaning. See DL: 1139.} the sound
of the blind man and the cripple's gong lasted longer, but the sound of the possessed
guy's gong did not last, so the possessed man totally\footnote{qha-pə̀-è (AE): `completely'.} ran away.

9 So after he had run away, they went and looked into the room [of the house where
the possessed man had been staying], and they said ``There's gold and silver in
here!'', so the two of them quickly went to divide it up, and the cripple said
he would make two portions.

10 [But] he made three portions in all, and told the blind man to take one portion
while he took two, and having divided it up, [the blind man] checked it over\footnote{ni a lɛ-ɔ̄: the literal meaning of ni `look at' is obviously inappropriate for the blind man, so the verb must be understood in its metaphorical sense of `try sthg, do sthg and see'.}
and said, ``Hey, you divided up the money into three portions!'', so he said ``No,
I didn't! You try checking it out!'', so he felt around to check and said, ``You
did make three portions!''

11 So then they \{started/set to\} \{fighting with/beating up on\} each other,
and the cripple beat the blind man right there on the eyes, and his eyes flew open
[so he could see].

12 Then when the blind man struck the cripple --- yeah, when the blind man struck
him back, the cripple['s body] straightened up.

13 So then the two of them, who could now see ad now had a straight back, divided
up [the riches] equally, and returned home.

14 That's the end.

