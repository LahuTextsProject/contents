\setcounter{footnote}{0}

1. Well then, once upon a time there were two men, a Yellow Lahu and a Black Lahu.

2. They went along a road and reached the place where the Lahu fields were.\footnote{Note the \textbf{co-occurrence} of the Black Lahu nominalizer \textbf{ve} and the equivalent Yellow Lahu nominalizer \textbf{che} in the same clause.}

3. When they got there, the Black Lahu climbed up on top of a log [overlooking the
fields], to see how the Lahu paddy was ripening.

4. Then the Black Lahu went over and said,\footnote{Another case of the \textbf{co-occurrence} of \textbf{ve} and \textbf{che}.} ``Friend, down there our Lahu paddy
is getting ripe!

5. And there's paddy getting ripe up there too!'' he said to the Yellow Lahu.

6. Now when the Yellow Lahu say ``câ-mɛ'', they mean ``tiger''.\footnote{The misunderstanding involves a three-syllable \textbf{pun}. In YL, \textbf{câ-mɛ} means `tiger' and \textbf{lâ} [BL \textbf{là}] means `come'. In BL, \textbf{cà} means `paddy' [YL \textbf{câ}], \textbf{mɛ} means `ripe', and \textbf{la} is a verb-particle meaning `become/get to be'. Thus BL `the paddy is ripening' gets interpreted as YL `a tiger is coming'.}

7. Thinking that['s what was meant], the Yellow Lahu took to his heels.\footnote{Still another case of \textbf{ve} \textbf{che} in the same clause.}

8. So then this Black Lahu chased after him, and they got up onto the mountaintop,
and finally came to a halt on the other side of the mountain.

9. When they got there he asked, ``What did we see that made you go running away
like this?''\footnote{Literally, `our having seen what, are you going fleeing?'}

10. Then the Yellow Lahu answered, ``As we were talking just now didn't you say
`A tiger's coming!'?''

11. Afterwards the two of them had a good laugh over it.

