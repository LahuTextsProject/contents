
139 God Struggles with the Titan

1. I'd like to tell you brethren now about how long, long ago God \footnote{\textit{g̈}ɨ̀\textit{-ša}: [pre-Christian] `Great spirit; Creator' (standing} created mankind
and created everything else.

2. Please listen carefully.

3. Long, long ago when g̈ɨ̀-ša created mankind, he rubbed off the \{crud/caked-on
dirt\} from his  feet and his hands, and created human beings [with it], it is
said.

4. After he had finished creating all the humans, he took a left-over lump and
made a giant out of it, and so Cà-mû-cà(-qa)-pɛ\footnote{Sometimes translated below as `Titan'.} came into being.\footnote{The gloss for cà-mû-cà(-qa)-pɛ in DL:450 diverges somewhat from this story:}

5. Then g̈ɨ̀-ša said to all the human begins, ``When your crops are ripe, all
of you must offer to me the first-fruits,'' so since g̈ɨ̀-ša had said this
to the humans, when the crops were ripe they went and offered [the first-fruits]
to g̈ɨ̀-ša.

6. \{The Titan/cà-mû-cà-pɛ\} saw them from where he was at a point on the road
that the humans were \{traveling/going on\}, and he asked ``Where are you all going?'',
and the people said ``We're going to make offerings to g̈ɨ̀-ša, our first fruits,''
so the Titan said, ``Make the offerings to \textit{me}! What kind of God is there?
It's me, I'm g̈ɨ̀-ša,'' he said, scolding the humans.

7. [Addendum] These people were very afraid of the giant, so even though they didn't
want to make offerings to him, they offered everything to him and went back home.

8. So then one day, since g̈ɨ̀-ša didn't see any humans coming, he went to
check things out, and when he met the Titan, cà-mû-cà-pɛ said to him, ``What
have you come looking for?''

9, So g̈ɨ̀-ša said to the Titan, ``I've been thinking about how the human beings
are doing with their cultivation, so I've come to check it out,'' and cà-mû-cà-pɛ
said:

10. ``If you are really God, let's just have a contest and go at each other,''
he said, and they [decided to] have a race to pluck a certain flower growing on
a tree.

11. So when they raced with all their might\footnote{`with all their might' translates the vivid vV g̈ɔ̀}, when they went racing, g̈ɨ̀-ša
\{clung/hung on\} to the Titan's leg \footnote{\textit{g̈}ɨ̀\textit{-ša }had miniaturized himself for this purpose.}, and having arrived at the flower on the
tree, g̈ɨ̀-š managed to pluck it.

12. Therefore \{he/the Titan\} did not prevail over g̈ɨ̀-ša.

13. For the second time \{he/the Titan\} said, ``If you are really God let's now
play hide-and-seek,'' so g̈ɨ̀-ša made the Titan go and hide first.

14. But wherever he went to hide, g̈ɨ̀-ša found him.

15. So then, this time it was g̈ɨ̀-ša's turn to go and hide.

16. But when he went to look for g̈ɨ̀-ša, he couldn't find him.

17. So because \{he/the Titan\} vigorously searched for him all over this \{land/country\},
that's why there are these mountains and valleys, so our ancestors have said.\footnote{That is, the Titan's clumsy but awesomely powerful searching chewed up the}

18. Then, when the time was up, since he couldn't find him, g̈ɨ̀-ša said \{``I've
come down!''/``Here I am''\}, and he jumped down right before his eyes.

19. When he said this, the humans---er, the Titan, since he was very proud, he
was not satisfied, and when g̈ɨ̀-ša created seven suns, he put seven iron helmets
on his head.\footnote{In order not to be dazzled by so much light.}

20. So \{before long/soon\} g̈ɨ̀-ša gave him another punishment.

21. g̈ɨ̀-ša extinguished the suns, \footnote{Thus plunging the world into darkness.} so [the Titan] made a notch in a pitch-pine
tree \footnote{a-kɨ́: `pitch-pine tree'. Torches made from the vision of this tree were}, tied it to a buffalo born and set it on fire, so he could plow his field.

22. For this reason, \{[up to] today/to this very day\} buffalo horns \{have ridges/are
corrugated\}.

23. And, in another place--- [Hesitates]

24. Cà-lɔ̂ \texttt{<}prompting him\texttt{>} What about the dancing horses?\footnote{Cà-lɔ̂ is jumping the gun. See sentence [47], below.}

25. So then, because g̈ɨ̀-ša was angry, he \{took/made\} a dung-beetle and
smeared poison on its horns, and made it go to \{the Titan's dwelling place/where
the Titan lived\}.

26. When the dung-beetle arrived in the presence of the Titan, the Titan asked
him, ``Where are you going to?'', and the dung-beetle said, ``Oh, I'm going to
the Jewel City.''

27. ``I've never been to the Jewel City myself

28. Such a tiny thing like you, how could you get there?'' he said, and he slapped
it hard, so that the dung-beetle's horn \{pricked/pierced\} his hand.

29. So it hurt him a lot and he got mad, and so he gave a kick with his foot too!\footnote{Also impacting on the beetle's horn.}

30. So then his foot also swelled way up on him, and his hand also swelled way
up on him, and it hurt terribly, and this time he couldn't stand it any more.

31. So then he went to where g̈ɨ̀-ša was, and hugged and besought him, ``Please
cure me with some medicine!'' he said, and g̈ɨ̀-ša said, ``\{There is/I have\}
a medicine.''

32. So when the medicine is put on, when seven deep have passed---er, don't unwrap
[the bandage] for seven days.

33. Soon after the medicine has been \{applied/put on\}, your flesh will get all
tingly as it heals.\footnote{ɔ̀-šā cā la ve: lit. ``flesh will sprout.''}

34. That's just the way it's likely to be,'' he said, he said to him.

35. So then g̈ɨ̀-ša took seven basketsful \footnote{pɛ-chɨ̀ʔ ve `tie up with a cloth; wrap around (as a bandage)': DL 843.} of blowfly eggs, and put them
on the Titan's foot and hand, and tied them up nice and tight for him.\footnote{kÁ-hôʔ (N; clf.): a measurement equal to the amount of cooked rice that}

36. For seven days, \{Cà-mû-cà-pɛ/the Titan\} patiently suffered, no matter
how much it hurt \footnote{chèʔ-nà ve: lit. ``bite-hurt'', i.e., stung painfully.} him, and when the seventh day came and he opened up [the
bandages] to have a look, all that flesh had rotted away since [the larvae] had
eaten it up,\footnote{hɔ̂ʔ-câ pə̀ ve: hɔ̂ʔ-câ means `mix things together to eat'.} so he died.

37. So when he died, g̈ɨ̀-ša called together all the human beings and all the
animals, thinking he would have them bury him.

38. But the humans couldn't manage to bury him. since he was so huge, so g̈ɨ̀-ša
said to them,

38a. ``Make a big cannon, and put [his body] into a grind-stone \footnote{I.e., grind up his body to powder, so it can be shot out of a gun}, then shoot
it out,'' he said, and the people obeyed g̈ɨ̀-ša's words, and made a cannon
and shot him out of it, and to this very day there have come to be all these insects
that still chomp into people \footnote{chèʔ câ ve: lit., ``bite and eat''}, that still bite, that still drink our blood.\footnote{I.e., the powdery remains of the Titan's body turned into the insects that}

39. As for these \{bugs/mosquitoes\}, the reason why they don't have ears like
people is because they [their ears] were swallowed up\footnote{By the noise of the explosion when they were shot from the cannon?}, so it is said.

40. So then, since all the humans had obeyed g̈ɨ̀-ša's words, and since they
had dealt with the Titan just as he had \{commanded/said\}, g̈ɨ̀-ša then called
together all the humans and all the animals in order for them to sing and dance,
and he made them all dance.

41. So then while all the animals were dancing for all to see, the dog, who also
had horns, looked splendid as he danced.

42. So the barking deer said, ``Mr. Dog, \footnote{cà-phɨ̂: phɨ̂ `dog' is here preceded by cà, the traditional prefix to} please lend me your horns for a while!'',
and so he lent them to him, and after he had danced for a while he ran away.

43. To this day dogs can't stand the sight of barking deer, and they chase after
them to bite them and bark, ``Arf, arf!'' which means ``Gimme back my horns, gimme
back my horns!'', so they say.

44. After this, when the barking deer was fleeing the scene, he ran into somebody's
swidden down in a certain place, where he happened to step onto a \{plate/dish\}
where somebody was crushing chili-peppers in a field-hut, which is why the hooves
of a barking deer stink bitterly of onions, so they say.

45. So meanwhile, as the buffalo and the ox were dancing, since he [the buffalo]
had beautiful teeth, when he was smiling as he danced it was a fine sight to see.

46. So the horse said, ``Mr. Buffalo, please lend me your teeth for a while, so
I can just dance with them a little,'' he said, and the buffalo lent them to him.

47. So then the horse danced once around, then kicked up his heels and ran away
lickety-split.\footnote{`lickety-split' translates the vivid vV g̈ɔ̀, which means `drag, pull'}

48. To this very day, when a buffalo says ``aw-ehn, aw-ehn,'' when it's chasing
after a horse it means ``Gimme back my teeth!''

49. When a horse says ``î-î-î'' it's because he's laughing at the buffalo.

50. So to this day the reason why a buffalo can't \{abide/tolerate/stand the sight
of\} a horse is because of this, so our \{forefathers/ancestors\} tell us.

Footonotes:

