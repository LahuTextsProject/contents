\setcounter{footnote}{0}

1. I'd like to tell you brethren now about how long, long ago God\footnote{\textbf{g̈ɨ̀-ša}: 1. [pre-Christian] `Great Spirit; Creator' (standing far above the nature spirits (\textbf{nê}); 2.[Christian] reinterpreted as the God of the Bible. This word is left untranslated in the rest of the text. See DL 1148-9 and JAM 1985 ``God and the Sino-Tibetan copula''.} created mankind
and created everything else.

2. Please listen carefully.

3. Long, long ago when g̈ɨ̀-ša created mankind, he rubbed off the caked-on
dirt from his feet and his hands, and created human beings [with it], it is said.

4. After he had finished creating all the humans, he took a left-over lump and
made a giant out of it, and so Cà-mû-cà(-qa)-pɛ\footnote{Sometimes translated below as `Titan'.} came into being.

5. Then g̈ɨ̀-ša said to all the human beings, ``When your crops are ripe, all
of you must offer to me the first-fruits,'' so since g̈ɨ̀-ša had said this
to the humans, when the crops were ripe they went and offered [the first-fruits]
to g̈ɨ̀-ša.

6. At a point on the road that the humans were traveling, the Titan cà-mû-cà-pɛ
saw them, and asked ``Where are you all going?'', and the people said ``We're going
to make offerings to g̈ɨ̀-ša, our first fruits,'' so the Titan said, ``Make
the offerings to \textit{me}! What kind of God is there? It's me, I am g̈ɨ̀-ša,''
he said, scolding the humans.

7. \textit{[Addendum]} These people were very afraid of the giant, so even though
they didn't want to make offerings to him, they offered everything to him and went
back home.

8. So then one day, since g̈ɨ̀-ša didn't see any humans coming, he went to
check things out, and when he met the Titan, cà-mû-cà-pɛ said to him, ``What
have you come looking for?''

9. , So g̈ɨ̀-ša said to the Titan, ``I've been thinking about how the human beings
are doing with their cultivation, so I've come to check it out,'' and cà-mû-cà-pɛ
said:

10. ``If you are really God, let's just have a contest and go at each other to
see,'' he said, and they [decided to] have a race\footnote{The gloss for \textbf{cà-mû-cà(-qa)-pɛ} in DL:450 diverges somewhat from this story: ``the first human being, a titan who pits his strength against his creator, \textbf{g̈ɨ̀-ša}.''} to pluck a certain flower
growing on a tree in front of them.

11. So when they raced\footnote{\textbf{g̈ɨ̂} \textbf{yù} \textbf{dàʔ} \textbf{ve}: `have a race', lit. ``run together to get something''.} with all their might\footnote{`With all their might' translates the vivid pre-head versatile verb \textbf{g̈ɔ̀}.}, when they went racing, g̈ɨ̀-ša
clung to the Titan's leg, and having arrived at the flower on the tree, g̈ɨ̀-ša
managed to pluck it.

12. Therefore he did not prevail over g̈ɨ̀-ša.

13. For the second time the Titan said, ``If you are really God let's now play
hide-and-seek,'' so g̈ɨ̀-ša made the Titan go and hide first.

14. But wherever he went to hide, g̈ɨ̀-ša found him.

15. So then, this time it was g̈ɨ̀-ša's turn to go and hide.

16. But when he went to look for g̈ɨ̀-ša, he couldn't find him.

17. So because he vigorously searched for him all over this country, that's why
we still have these mountains and valleys, so our ancestors have said.\footnote{That is, the Titan's clumsy but awesomely powerful searching chewed up the landscape.}

18. Then, when the time was up, since he couldn't find him, g̈ɨ̀-ša said ``I've
come down!'', and he jumped down right before his eyes.

19. Even though he [g̈ɨ̀-ša] said this, the Titan, since he was very proud,
was still not satisfied, and when g̈ɨ̀-ša created seven suns, he put seven
iron helmets on his head.\footnote{In order not to be dazzled by so much light.}

20. So g̈ɨ̀-ša soon gave him another punishment.

21. g̈ɨ̀-ša extinguished the suns,\footnote{Thus plunging the world into darkness.} so he [the Titan] made a notch in a
pitch-pine tree\footnote{\textbf{a-kɨ́}: `pitch-pine tree'. Torches made from the resinous sap of this tree were the traditional Lahu means of lighting their way at night.}, tied it to a buffalo born and set it on fire, so he could
plow his field.

22. For this reason, to this very day buffalo horns have ridges on them.

23. And, in another place--- [Hesitates]

24. Cà-lɔ̂ \direct{prompting him} What about the dancing horses?\footnote{\textbf{Cà-lɔ̂}, already familiar with the broad outlines of the story, is jumping the gun. See below.}

25. So then, because g̈ɨ̀-ša was angry, he made a dung-beetle and smeared poison
on its horns, and made it go to where the Titan lived\}.

26. When the dung-beetle arrived in the presence of the Titan, the Titan asked
him, ``Where are you going to?'', and the dung-beetle replied, ``Oh, I'm going
to the Jewel City.''

27. ``I've never been to the Jewel City myself.

28. ``Such a tiny thing like you, how could you get there?'' he said, and
he slapped it hard, so that the dung-beetle's horn pricked his hand.

29. So it hurt him a lot and he got mad, and so he gave it a kick with his foot
too!\footnote{Also impacting on the beetle's horn.}

30. So then his foot also swelled way up on him, and his hand also swelled way
up on him, and it hurt terribly, and this time he couldn't stand it any more.

31. So then he went to where g̈ɨ̀-ša was, and begged and besought him, ``Please
cure me with some medicine!'' he said, and g̈ɨ̀-ša said, ``I have a medicine.''

32. ``So when the medicine is put on, don't unwrap [the bandage] for seven
days.

33. ``Soon after the medicine has been applied, your flesh will get all
tingly as it heals.\footnote{\textbf{ɔ̀-šā} \textbf{cā} \textbf{la} \textbf{ve}: lit. ``flesh will sprout.''}

34. ``That's just the way it's likely to be,'' he said, he said to him.

35. So then g̈ɨ̀-ša took seven basketsful\footnote{\textbf{ká-hôʔ} (N; Cl\textsubscript{f}.): a measurement equal to the amount of cooked rice that fits into a bamboo storage basket.} of blowfly eggs, and put them
on the Titan's foot and hand, and tied them up nice and tight for him.\footnote{\textbf{pɛ-chɨ̀ʔ} \textbf{ve} `tie up with a cloth; wrap around (as a bandage)': DL 843.}

36. For seven days, Cà-mû-cà-pɛ patiently suffered, no matter how much it
hurt\footnote{\textbf{chèʔ-nà} \textbf{ve}: lit. ``bite-hurt'', i.e., stung painfully.} him, and when the seventh day came and he opened up [the bandages] to
have a look, all that flesh had rotted away since the maggots had eaten it all
up, so he died.

37. So when he died, g̈ɨ̀-ša called together all the human beings and all the
animals, thinking he would have them bury him.

38. But the humans couldn't manage to bury him. since he was so huge, so g̈ɨ̀-ša
said to them,

38a. ``Make a big cannon, and put what's left of his body into a grindstone\footnote{I.e., grind up his body to powder, so it can be shot out of a gun.},
then shoot it out,'' he said, and the people obeyed g̈ɨ̀-ša's words, and made
a cannon and shot him out of it, and to this very day there have come to be all
these insects that still chomp into people\footnote{\textbf{chèʔ} \textbf{câ} \textbf{ve}: lit. ``bite and eat''.}, that still bite, that still drink
our blood.\footnote{I.e., the powdery remains of the Titan's body turned into the insects that still plague us today.}

39. As for these insects, the reason why they don't have ears like people is because
they [their ears] were swallowed up\footnote{By the noise of the explosion when they were shot from the cannon.}, so it is said.

40. So then, since all the humans had obeyed g̈ɨ̀-ša's words, and since they
had dealt with the Titan just as he had commanded, g̈ɨ̀-ša then called together
all the humans and all the animals in order for them to sing and dance,\footnote{Note the unusual use of \textbf{lɛ} `and' to join two verbs, `sing' and `dance' (here translated `singing' and `dancing'), no doubt because they are both in constituency with the purposive verb-particle \textbf{tù}, which can also function as a purposive nominalizer in other contexts.} and
he made them all dance.

41. So when you watched all the animals dancing, the dog, who also had horns, looked
splendid as he danced.

42. So the barking deer said, ``Mr. Dog,\footnote{\textbf{cà-phɨ̂}: \textbf{phɨ̂} `dog' is here preceded by \textbf{cà}, the traditional prefix to male names. Actually boys born on the Day of the Dog were also traditionally given this name. See DL 448-451.} please lend me your horns for a while!'',
and so he lent them to him, and after he had danced for a while he ran away.

43. To this day dogs can't stand the sight of barking deer, and they chase after
them to bite them and bark, ``Woof, woof!'' which means ``Gimme back my horns,
gimme back my horns!'', so they say.

44. After this, when the barking deer was fleeing the scene, he ran into somebody's
swidden down in a certain place, where he happened to step onto a mortar with which
somebody was crushing chili-peppers in a field-hut, which is why the cloven hooves
of a barking deer stink bitterly of scallions, so they say.

45. So meanwhile, as the buffalo and the ox were dancing, since he [the buffalo]
had beautiful teeth, when he was smiling as he danced it was a fine sight to see.

46. So the horse said, ``Mr. Buffalo, please lend me your teeth for a while, so
I can just dance with them a little,'' he said, and the buffalo lent them to him.

47. So then the horse danced once around, then kicked up his heels and ran away
lickety-split.\footnote{`Lickety-split' translates the vivid vV \textbf{g̈ɔ̀}, which means `drag, pull' as a main verb.}

48. To this very day, when a buffalo says ``aw-ehn, aw-ehn,'' when it's chasing
after a horse it means ``Gimme back my teeth!''

49. When a horse says ``î-î-î'' it's because he's laughing at the buffalo.

50. So to this day the reason why a buffalo can't tolerate a horse is because of
this, so our ancestors have told us.

