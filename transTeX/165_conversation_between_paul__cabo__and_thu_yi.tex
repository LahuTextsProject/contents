
\textbf{165/166 Conversation between Paul, Càbo, and Thû-yì}

[XIV 2 \#165-166]

1: (P) Well, how many years has it been now since you Lahu who are living in Huey
Tad arrived in Thailand?

2: (T-Y) Well, I guess this year makes about twelve years that we've been here
in Thailand.

3: (P) When you came down [from Burma] what month was it? Was it the cold season
that you came down, or the hot season?

4: (T-Y) Well, I can't really be sure of the (exact) time. Since I was a little
boy then, you know.

5: (P) How old were you then--when you came down?

6: (T-Y) Oh, I must've been just about then.

7: (P) \{On your way down here / When you came down then\} how many days did you
have to spend the night by the roadside? [=How many days and nights did you have
to spend on the road?]

8: (T-Y) Well now, from the time we began to run away up there to when we came--by
the time we got here to Thailand, it must've taken about a month I guess.

9: (P) When you got to Thailand what did you--what did you start doing for a living?
To support your \{wives and children / families\}? And to support yourselves? To
feed your bodies?

10: (T-Y) The first year we got here we worked for hire for the Big Boss.[1] For
five and a half baht a day.[2] Besides that the Big Boss helped us out with all
of our food.

11: (P) What sort of work did you do for the Big Boss?

12: (T-Y) Oh, we did all kinds of work. We would weed the tea \{plantation / gardens\},
loosen the earth out the base of the plants with a hoe. That sort of thing.[3]

13: (P) Didn't you start to till the fields that year?

14: (T-Y) No, the first year we \{stayed / came to live\} here we didn't start
cultivating the land yet. We just supported ourselves by working for hire.

15: (T) That year we didn't till the fields yet. The year we got here we hired
ourselves out to the Big Boss and just kept working like that for a whole year,
from \{day to day / hand to mouth\}.

16: (P) So you worked like that, and with the money the Big Boss paid you you bought
your rice to eat, right?

17: (T-Y) No, as far as rice went there was no need for us to buy any then. He
fed us. Curry,[4] rice, and everything.

18: (P) How many years did you say it was that you hired yourselves out to work
for the Big Boss? I mean how long did you have to work like that?

19: (C) That business of working for the Big Boss lasted for the whole first year
that we got here, but the second year we cleared \{land / fields\} and earned our
living by cultivating [our own] upland fields.[5] \{Then / And\} after we'd gotten
so up could feed ourselves, we'd earn some money by hiring ourselves out to work
on the side.

20: (P) Besides working your upland fields what other things do you do? Do you
raise pigs and chickens and all that too?

21: (T-Y) Oh, we do all sorts of things. We raise pigs and chickens, and when we
sell a pig or a chicken, you see, we get a little money for it; so we go on doing
it and we have a little bit [of money][6] [to show for it].

22: (P) What I'd just like to know now is, [with respect to] your cultivating the
fields here, clearing fields and all that, what sort of rules and regulations does
the Thai government, the Thai regime, have--how have they set them up?

23: (T) Right now as far as our being allowed to cultivate the land here in Thailand
is concerned, according to the Thais' orders, the rulers' orders, we're not supposed
to clear fields. However, since we're living on the land of a \textit{nîkhom7,
we're permitted to work it a little bit. But even though we're allowed to cultivate
the land, there's one thing that gives us a lot of trouble. What I mean is, groups
of police are always coming up here, and when they go and inspect the places where
we've made our fields they \{want to / threaten to\} arrest[8] us. But they haven't
been able to ruin us yet, not even once. }

\textit{24: (P) Why is it that they don't want to let you clear your fields--what
do they say the reason is, then?}

\textit{25: (T) What they say is, if we clear fields, the water [supply of the
region] will dry up. They say that if a lot of field-clearing goes on up in the
mountains,[9] there won't be any water in the towns. That's what they say. But
for us Lahu, if we're not allowed to clear the land to earn our living, there's
no way for us to exist. And there's no way for us to go and get \{cash / money\}
either. And since we also don't have wet-rice lowland fields, we've got to go on
suffering like this. }

\textit{26: (P) This business about your clearing fields--how many years have you
been managing to make a living from a field--that is, from each single field [that
you clear]? After how many years do you have to begin clearing a new one again?[10]}

\textit{27: (T-Y) Well, a field, one piece of land, is fine for one year. The second
year isn't quite up to the first year. }

\textit{28: (T) That's right. That's how it is when you 'hold onto the field-bones.'[11]
Sometimes, when you have a place where the soil is no good, even for the first
year--even for the space of a single year--it's not adequate; and when the second
year comes, even if the plants should be all right they don't have any ears. }

\textit{29: (P) In that case you have to begin clearing a new one, right? Well
now, according to what I've heard, when a field is cleared a great many trees are
cut down, they say. Now the lifetime of a tree is as long as the lifetime of a
person! They say that a tree lives for something like fifty or sixty years.[12]
And these trees are very important to Thailand, they say. Thailand's economy[13]
depends on these trees. So that's why they say they don't want to let you clear
them away.... But for the Lahu people, upland fields--er, trees--uh, if you aren't
allowed to clear fields, then how do they suggest that you earn a living? How have
they figured that out? Since they don't want to let you clear any fields.... }

\textit{30: (T) This is what they say. So that [we] in the mountains will be able
to make a living from a single piece of land, this is what they want to make us
do. They say that we [should] plant all kinds of fruits, plant tea, plant coffee,
and if we want to till fields, we must have terraced} fields. \{So / But\}, as
far as we Lahu are concerned, there is no way for \{this sort of thing to come
about / us to accomplish any such thing\}. And yet, when it comes to the question
of how we are to get our food and drink in the future, the Thai officials are of
no help to us--and so it's a tremendous worry for us.

For one thing, if we Lahu weren't permitted to cultivate upland fields, we wouldn't
have lowland fields [to fall back on] either. And neither would we have any \{cash
/ money\}. If we had a few pigs and chickens, we'd have to sell them all in order
to be able to buy rice to eat. Besides that, we couldn't get clothes to wear, and
we would be disgraced in the eyes of others.

31: (P) When you till a field, after a year or two when the field is played out[14],
do you abandon it? Once it's abandoned this way it becomes an \texttt{"}old-field,\texttt{"}
right? If it's an 'old-field,' how many years have to pass before you can earn
a living from it again--from this 'old-field'?

32: (T) After we've gotten a crop out of a piece of land, if we had only worked
it for \textit{one} year we can cultivate it again about fire years later. But
if we've 'held onto the field-bones,' if we've held onto the field-bones' for a
second year, it takes about five years longer--in about ten years we can get a
crop from it again.

33: (P) What you call 'holding onto the field-bones,' do you mean 'tilling the
same field a second time?' I don't quite understand what you mean! How do you 'hold
onto' it??

34: (T) Yes. What we call 'holding onto the field-bones' means tilling it a second
time, planting the same field two years in a row. That's what we mean by 'holding
onto the field-bones.' And so, if we hold onto the same piece of land for an extra
year, we have to wait about ten years. After we've waited for ten years, we can
get a meager crop from it. But it's still not very good yet.

35: (P) So now the Thai government is saying, isn't it, that you won't be permitted
to clear new fields in all kinds of different places, and it's a saying that you'll
have to do your cultivation in one place, and that fertilizer--and that you should
use fertilizer to grow crops on your old fields year after year. What's your opinion
on all that? Will it be possible or not?

36: (T) Well, when we think it over, [we use that] this business of using fertilizer
up here in the mountains is extremely difficult. The reason is that the ground
isn't level, so that when we put on fertilizer, as soon as it rains all the fertilizer
is carried away--so it wouldn't do us too much good, I daresay.

37: (P) Well then, supposing you made terraces, couldn't you do it then?

38: (T) \texttt{"}Make terraces!\texttt{"} They say--\texttt{"}since the ground
is so very steep, if you just did that you'd be able to manage.\texttt{"} But it
would take an enormous amount of \textit{time}, and we would have to suffer and
be miserable for a long time.[15] Besides, this sort of terracing, even under the
best of conditions, would be a very hard job for us Lahu, since it's not level
ground. The thing is, there's just no place to make terraces. Because the ground
is very steep. And even if the time should come when we will have succeeded in
making terraces, how much--how long it will have taken us to do it we have no idea.

39: (P) Take a mountain that you've already cleared for fields--about how many
years would it take you to terrace it? So that it would be properly terraced?

40: (T) I would \{estimate / guess\} that on a place where a field has already
been cleared, to go on to terrace it, on a mountaintop--to make a terraced field
out of a single stretch of land--if it's a field that took two months to plow,
you couldn't terrace it in less than two or three years, that single stretch of
land.

41: (P) If that's so, during the time when you're still tilling the terraced fields--I
mean, during the time when you're still \textit{making} terraced fields, during
the one or two years it would take you to make them, how do you intend to earn
your living? What [sort of work] would you look for to live on?

42: (T) That's just what we Lahu are discussing at the present time. We're stymied.[16]
What sort of thing, what sort of way we can feed ourselves, we just can't figure
out. You see, even when we simply make our fields and plant our rice [in the ordinary
way], even without making terraces,[17] some of us don't get enough to \{eat /
live on\}. Some years it's enough, and some years it's not enough.... During the
time when we'd have to be making terraces on top of everything else, we wouldn't
be able to raise pigs and chickens. We couldn't manage to sell any. There'd be
no time. And the fodder [we'd need] to feed the pigs and chickens we'd be raising,
so they'd get big and we could then sell them to live on--[this fodder] we wouldn't
have either.

43: (P) Hm, well, as you say, Teacher--uh--making terraces and putting fertilizer
on your land would probably be impossible. (But) it \textit{is} a perfectly possible
thing to do, as far as that goes. That is, if they would really help you do it,
right? If they would use that kind of machine that butts the earth,[18] you know,
if they came themselves and did it, if they pushed [the earth away] for you, then
it could be done.

But I tend to think that they couldn't be bothered to move [the earth] just for
the sake of these one or two villages, for this number of people. What conclusions
have \textit{you} come to about all this?

44: (T) That's what I think too. As far as their helping us goes, they can only
be a tiny help. For instance, even if we get them to move earth for our houses,
just to level the ground to make a place to build a house[19], they don't want
to do a proper job for us. So, as far as their using the government's machines
and moving the earth for us goes, as far as the time ever coming when they'd actually
do it[20] for one or two villages--I just don't think they'd ever do it. In my
opinion, they'd never do it for a single village. And yet as far as money is concerned,
they've got plenty of it. And they have several of these machines too. But for
them to really help the hill-tribes isn't yet in their hearts.

45: (P) Well, as I was saying a while ago,21 if you--er--wh--since they don't want
to let you clear fields in the mountains--wh--in several districts there are many
stretches of [unoccupied] plainsland, you know, and when foreigners[22] have come
to look they've said \texttt{"}This Thailand has vast plains, there is an abundance
of cropland, of arable land, so as no one dies \{of hunger / for lack of rice\}.\texttt{"}
If that's the case, supposing you were to ask for some level land[23] like that,
or for a large wooded plain,[24] wouldn't you get it?

46: (T) Sure. This is the way the matter stands. We have supervisors at the \textit{nikhom}.
There's a stretch of plainsland [in a certain place] over there, but when we went
and asked them for it, they said that the trees [on it] were very big, so they
couldn't give it to us. Then, to take a second example, in another direction there
are several stretches of forested wilderness.[25] But even though there are many
level places there, they tell us that some of the area belongs to the Big Boss,[26]
some belongs to the government, some belongs to the Forestry Department--so it's
all laid claim to [by somebody else], and there's nothing for us Lahu to get.

47: (P) What they've been saying here, this business about the Forestry Department,
and one bit of plainsland being that fellow's, and the other piece belonging to
the Big Boss, and all that--still no one has ever seen any of them actually \textit{cultivating}
the land, isn't that right? It's just left abandoned that way, isn't it?

48: (T) That's right. People like them have laid claim to it [all]. And we've never
seen them \textit{do }anything with it. Yet, since they have already claimed it,
we hill-people can't get it away from them.

\begin{center}
\textbf{\#166 }

* * *
\end{center}

\leftskip=0pt
49: (P) Well, now that I'd like to know is this--wh--how do we Lahu go about \texttt{"}cultivating
a field,\texttt{"} from the day that we begin to work the field until the time
we harvest the rice? Tell me [about it] one stage after the other, all right? That's
the sort of thing I want to know. In the beginning we--wh--look for a [good] place
to clear; then when we've gotten a place how do we go about clearing away the trees?
How many days do we need to work at clearing them away?--that sort of thing, right?

50: (T) This is the way we Lahu cultivate our upland fields: First of all we must
cut down the undergrowth.[26] After the undergrowth is cut down, we \{clear away
/ slash down\} the trees.[27] The time it takes us to slash down the trees is about
two months. Then, after the slashing is over, we do the first burning.[28] When
the first burning is completed, we have the second burning.[29] It doesn't take
very much time for the second burning. Just a week or two. Then, during the period
of time when we have to get on with the [actual] planting, we must spend about
a month and a half planting the rice. After we finish the planting, this is how
long it takes us to harvest it--[30] After the planting, we must do the weeding.[31]
And this is how long we weed: Some years we find as we work the land that the weeds
are very thick, so we can't get the weeding over with right away. There are times
when we have to weed for about a month. There are also times when we have to weed
for about two months. If they're \textit{very} thick, we have to keep on weeding
right up until harvest-time. Now the harvesting time itself doesn't last very long.
Those who have large fields harvest--have to spend about a month harvesting; for
those above fields are not large it takes two or three weeks.

51: (P) Well, now, this 'cutting down the undergrowth' that you just mentioned,
how is it done? And what you call 'the second burning'--how do you do that? I don't
really understand this 'undergrowth-cutting' and 'second-burning.' \textit{How
}do you re-burn and \textit{how} do you cut the undergrowth?

52: (T) What we call 'cutting the undergrowth' is this way: We have to do the cutting
in the sunny season. If we get rid of these plants by cutting them away, \{the
weeds decrease greatly / the weed-seeds don't...\}--they become fewer, so the fields
don't get overgrown. That's what we Lahu say, [anyway]. That's why we have to cut
the undergrowth early [in the growing cycle]. Now the so-called 'second burning'
in this way: After the fields have been burnt over for the first time, we take
and gather together all the things that \textit{didn't} get burned up and set fire
to them--this is what we call 're-burning the fields.'

53: (P) What you're saying is, the 'undergrowth-cutting' is done before the big
trees and things are cleared away, right? Before these are cleared away you finish
pulling out the [other] plants little by little, is that it? This kind of weeding
away is what you mean by the term, eh?

54: (T) Mm-hm. That's what we mean. It doesn't mean that we clear away the big
trees at the same time. [Rather] we get rid of all the various plants that grow
under the trees, all the various plants to be found in the area to be cultivated.
[But] this weeding-away isn't just a simple matter of getting rid of the stuff.
It becomes fertiliz-- when it rots it gets to be fertilizer, and it's very good
for the rice.[32]

55: (P) Well, when you harvest the rice how do you do it? Do you help each other?
Or does everybody do his harvesting separately?

56: (T) When we Lahu harvest our rice, we don't help each other the way the plains-people
do. Once in a while somebody or other can't harvest his rice, because he doesn't
feel well, when he's sick or something--when we see this happen the entire village
must go and help him out. [But] if he's not sick every person cultivates his own
field; we don't all work together on the same piece of land. So it never happens
that[33] one person goes [to work] with another. [But] in case of illness, we go
and keep them out.

57: (P) Now I'd like to know a little something about the harvesting process. The
harvesting--I mean--how does the rice have to get for you to know the time has
come to reap it? How can you tell by looking at the rice that it's ripe? When you
do reap it, tell me how many days you have to spend at it. When it's all harvested
how do you bundle it up? How do you thresh[34], tell me?

58: (T) This is the way we reap the rice. When we look at it we can tell this way:
After the rice has come out, when the grains are getting hard, we look at it and
we can tell. When the grains are getting hard and turning yellow we know that it's
time to harvest them--so we must reap. When the reaping is over, we have to pile
it all up [in sheaves].[35] The purpose behind this 'piling-up' is this: if you
don't pile up the rice-plants [and leave them to dry], the paddy is no good. And
the rice won't \texttt{"}thresh off.\texttt{"}[36]

59: (P) What do you mean 'the rice won't thresh off'?

60: (T) 'Won't thresh off'--er--'won't thresh off' means means that when you thresh
it it doesn't all fall off.

61: (P) It doesn't fall off.

62: (T) Yeah, some of it stays hanging [on the stalks]. That's what I mean.

63: (P) About how many days do you have to leave it piled up before you thresh
it?

64: (T) Well, in principle, even if you only leave it piled up for two or three
nights, you can really thresh it. But when we don't have enough time, when we have
a great deal to sheave, some years it happens that we have to leave it stacked
up for a month. Some years it turns out that we've got to leave it for two or three
weeks. Once it's sitting there, after we've managed to finish sheaving it, we don't
need to worry about whether it'll rain or whether the sun will shine. It won't
get wet.

65: (P) Well, when you thresh the rice--after it's been threshed out, do you carry
it home to put away? How do you go about transporting it?

66: (T) This is how we thresh the rice. If the field is far away [from the village]
we leave the grain in the field. We build a house in the field, make a great big
storage-basket,[37] and do the threshing in the field. On the other hand, if the
field is not far off but is right close-by, we carry it home at once.

67: (P) So, once the grains of rice are all carried home and stored, that's the
end of it, right? Then what do you do with that stuff--you know, after you thresh
the rice, the--wh--what do you call it--the kind of plant[38]--how do you say it
in Lahu?

68: (T) Aha.

69: (P) How do you keep that stuff?

70: (T) When the threshing is finished, the Lahu don't keep it[39] and prepare
it [for other uses][40] the way the people in the towns do. Those \textit{rice-stalks}
we simply thresh away, and stack them up into bundles. Then, if we find ourselves
in the position of having to be able to re-till an old field, we carry all of the
stalks back and scatter them about in the four corners of the field and set fire
to them--so that it will be all right to plant the field with rice again.

71: (P) Well, then, now I've found out all kinds of things about the way the Lahu
clear and cultivate their rice-fields, haven't I. Now, while you're working your
rice-fields, what other things do you grow for food besides? The various things
you plant, your cornfields, whatever[41] other fields you may have--that sort of
thing.

72: (T) We don't confine ourselves to cultivating rice-fields. At the same time
that we're \texttt{"}living from our upland paddy-fields,\texttt{"}[42] we Lahu
plant within these fields all kinds of beaus, quantities of bananas, lots of \{taro
/ manioc[43]\}, everything that's good to eat, everything that's fit to be eaten.
Besides this, we plant cornfields too, but we have to make the cornfields on a
separate plot of land. If there were no cornfields--seeing that we Lahu live by
raising pigs--if we didn't have fodder for our pigs, there'd be no way for us to
earn any money.

73: (P) Well, that's very good indeed. [But] what's rather distressing now is the
Thai authorities' saying they don't want to let you clear [new] fields--that's
quite a worry, isn't it? If this Lahu people of ours can't clear [new] fields,
we won't have anything to eat, right? Money we can get along without.[44] As long
as we just have rice we can live, isn't that so? These things they say about not
being allowed to clear [new] fields, and just relying n our old fields and putting
on fertilizer to get a crop, all this is very hard to bring about. Furthermore,
if someone should suggest that you go find new places,[45] it's as you paid before:
wherever [the places] are, they all have an owner--even though nobody ever sees
these so-called owners working the land. For as much as ten years the land is left
fallow. You even see land that has been abandoned like that for twenty years. So
you just don't know what's to be done now, right?... Maybe if you get together
in a group, those [of you] who know a little of their language, and went down there
to the government office and tried discussing the matter with them--with the officials--it
might do some good, don't you think so?

74: (T) Yes. Not all of us understand their ways and customs, nor are we all able
to speak their language properly. But what we had thought was this: until [it was
absolutely certain that] we wouldn't get what we wanted and that it was impossible,
we would beg them for it time and again. That's what we planned to do. For one
has nothing to lose until he tries.[46] So the government officials said to us,
\texttt{"}Go look for a place where the ground is level!\texttt{"} But we have
looked and we can't find any. So therefore, unless they in the government take
pity on us and give us some help, and if we can't clear new upland fields and don't
have any lowland fields either, the situation for us Lahu will continue to get
even worse in the future than it is now already. We shall very likely never succeed
in making anything of ourselves now.

75: (P) They say to you, \texttt{"}Make coffee plantations and fruit orchards and
things, and you'll make a living.\texttt{"} Couldn't you do as they say? How would
you do that sort of thing?

76: (T) It's this way. If we Lahu were only able to do as they say, we \textit{could}
make a living. But even if we wanted to make a fruit orchard, we couldn't afford
to buy the fruit-trees.[47] We don't have the money to buy them. There's no money
to do it.

77: ( A man's voice) It takes a very long time.[48] You can't make a living from
them right away.

78: (Someone else or T?) So in the meantime,[49] food is a very difficult problem
for us Lahu. And the second thing is the \textit{money} that it would take to buy
the fruit-trees--if we don't have the money, we can't buy them. Since we must use
money for everything, feeding ourselves is a great worry for us.

