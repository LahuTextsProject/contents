\setcounter{footnote}{0}

1. P: Hurry up and do it!

2. Headman's wife [HW]: I'll bring my dog and have him go at it for you!

3. Somebody: Bring your dog and come put him down here\footnote{I.e., in front of the tape recorder.} for us!

4. HW: The dogs are right outside, a big pack of them.

5a. H: Come on, come on, you guys have nothing to be bashful about!

5b. They\footnote{I.e., my chief consultant \textbf{Cà-lɔ̂} (= Paul = Tcalo) and I.} say they want to get stuff.

5c. When it's time for him to go back to America---our Lahu way of life.

5d. Anything and everything,\footnote{\textbf{à-thɔ̀ʔ-mû-thòʔ-ma}: elaborate expression expanded from \textbf{à-thòʔ-ma} `what?'; \textbf{mû} (Belab) conveys a nuance of intensity. See DL:1003-4.} all kinds of things.

5e. Like, we could pretend to be bachelors\footnote{\textbf{chɔ-há-pā}: young man of marriageable age.}: ``I wonder if we should go down
there to pitch a little woo''\footnote{\textbf{mâʔ} (V): `go courting; woo; flirt with the opposite sex (of men or women)'.}, ``If we go down there there's not going
to be any place to flirt,'' one guy can say, and the other one can say, ``Over
there you've got to wait! You'd have to stay down there a long time.''

6. P: Tell a little bit about the Kachin\footnote{A Jingpho resident in Huey Tat, popular with his Lahu fellow villagers, who enjoy teasing him gently. He is called either \textbf{Khá-pā} `Kachin' or \textbf{Cùnphɔ̀ʔ }`Jingpho'} looking for a wife! Why is he looking
for one, I wonder?

\direct{Woman laughs}

7a. H: I hear the Jingpho is coming back from looking for a wife.\footnote{The Headman is addressing someone presumably living in the Jingpho's house while he is on his quest.}

7b. You\footnote{Still talking to that person.} stay at home and pound lots of rice and store it, and when he gets
back you can eat

it up---say something like that.

8. P: Will he get one, or won't he---think about that, and talk nicely about it!

9a. H: Hey, hurry up and do it, hurry up, you guys! About everything---

9b. Well then, I'm going to roll me up one of those great big Klèt-Thawng\footnote{The Headman is ironically referring to his home-made cheroots wrapped in corn-husks as \textbf{Klèt-Thɔɔŋ} (lit. ``Gold Flakes''), a popular brand of Thai cigarettes. It's not clear who the ``lunatics'' are.}
cigars like you've seen

the lunatics smoke---

9c. In the old days in Shan State we used to smoke great big cheroots that would
fill your mouth

right up!

10. P: Were they what was called\textbf{hsêi-pyîn-leiʔ}\footnote{WB \textbf{chê-pyâŋ-lip}: a Burmese cigar or ``cheroot''.}? Or were they
called \textbf{hsêi-pɔ̂'-leiʔ}?\footnote{WB \textbf{chê-paw'-lip}: ``a cheroot-like smoking article consisting of a filler of roasted tobacco plus other ingredients and a filter encased in a cylinder of banap-leaf'' (\textit{Myanmar-English Dictionary}, p. 134). A cheroot (ult. Tamil \textit{curruṭṭu} < \textit{curi} `be spiral') is a cigar with square-cut ends.}

11. H: Cigars---big cigars. Now I'm going to roll me a big one!

12. P: Go for it!

13. Woman: They're doing it, I tell you! It's already turned on\footnote{\textbf{phɔ̀}: lit. ``open''. She is referring to the tape recorder.}! Hurry up
and put something in

it!\footnote{I.e., record something!}

14. Kid: Great big Klèt-Thawng cigars.

15a. H: Hmm. Come on, hurry up and come do it! About all kinds of stuff.

15b. Everything about courting the girls---say ``That girl is waiting for you there!''

15c. Say ``She's hiding all your shoulderbags!''

15d. Say ``She's hiding your towels''!\footnote{\textbf{phá-cèʔ}: < Shan (cf. Si. \textbf{phâa-chét-tua}, lit. ``cloth-wipe-body'').}

16. P: Is that what goes on when the young women and young men\footnote{\textbf{yâ-mî-há} `marriageable young woman'.} court each
other? They

hide things from each other?

17. H: They hide things from each other. They take things and hide them. They hide
towels from

each other.

18. P: They don't return them, do they, the things that they've hidden?

19. H: They don't give them back. They buy very good gold rings to hand over\footnote{Two phonologically similar verbs occur in this passage: \textbf{fá} \textasciitilde{} \textbf{phá} (< PLB \textbf{*ʔwak}) `hide stag', and \textbf{fâʔ} (< Tai) `hand over to, entrust to'.}
[to their betrothed].

20. P: Uh-huh.

21. H: The girls' things they hand over to us, and our things we hand over to them.
Later on they

have discussions about getting married.

22a. HW: Hurry up and record, they say. They say they're going to take pictures.

22b. The next time they come, they'll give them to us, they say.

23. P: Hurry up and do it. Can't you do it?

24a. H: Come and go at it, come and go at it, you guys!

24b. You could even talk Karen!\footnote{\textbf{yân-khɔ̂-mû-khɔ̂}: \textbf{mû} is an intensive morpheme occurring in elaborate expressions. See note [3].}

24c. Hey, Cà-g̈âʔ, why don't you record something in Karen-Shmaren\footnote{The Headman uses the slightly pejorative term \textbf{Yân-qə̂ʔ-lə̂ʔ} for Karen. \textbf{Cà-g̈âʔ} evidently can speak some Karen.}!

25. P: You can ask things like ``How old are you?'', right?

26a. H: Hey, Cà-g̈âʔ, shall we do it in \textit{Thai}?!\footnote{\textbf{Thây-khɔ̂-mû-khɔ̂}: See notes 3 and 17.}

26b. Sit down here! We'll talk White Man's language\footnote{\textbf{Kâlâ-phu-khɔ̂}: lit. ``white people's' language,'' usually referring to English.}!

\direct{laughter}

27. P: No need to be bashful! Just do it any old way---

28. Boy: Stop it!\footnote{Another kid has just poked him.}

29. P: It's like playing.

30. H: \direct{sings} Mmm---\textit{so la so mi do mi re do re mi re
so la so mi do so la }

\textit{do ti la so} \textit{\textbf{\direct{get the rhythm from the tape}
}}

Well, put in everything about the cocks crowing when women give birth!

31. Somebody: Let a woman do that then!

32. H: Put in how people long ago, how the Chinese for no reason at all drove us
Lahu along

by force\footnote{\textbf{g̈àʔ-bɔ̂ʔ}: lit. ``chase by shooting''.}, how we fled down below that high plateau, and how afterwards we
then

stayed down there in the valley where they attacked us again---how year after year
we

were exhausted [by all that].

33. P: Yes, do it just like that!

34. H: Let us hear about how when we're clear-cutting a swidden and a branch breaks
off a

tree and knocks somebody over ---

That's happened to everybody.

Cà-qā, come sit here! Come record here!

34d. For ten liters of hulled rice figure out how many baht they ought to pay and
sell it to them

at that price---say that!

34e. These kids don't want to help us---

\direct{The young men keep fiddling with the guitar that JAM brought as a gift.}

34f. Do it right, do it right!\footnote{He says this to the guitar players.}

34g. Did you hear,\footnote{\textbf{chi} \textbf{ni-ʔ}: lit. ``look at this''.} a while ago that was a pretty tune! \direct{sings}\textit{la
sol mi do do ↑}

[octave higher] \textit{la do ti la sol}.

35. P: Hurry up and do it! Don't you have anything to record finally?

36. H: Oh this bunch of kids are ``going\textbf{pu-lî-khɔ̂-lî}''
\footnote{\textbf{pu-lî-khɔ̂-lî} \textbf{te} \textbf{ve} `have a moronic face; act like an idiot'. This expression was inadvertently left out of DL.}, so probably none of them will come up

with an idea.

\direct{laughter}

37. P: What does ``going\textbf{pu-lî-khɔ̂-lî}'' mean?

\direct{much laughter}

38. H: So about\textbf{pu-lî-khɔ̂-lî}, right?

39. P: Yeah.

40a. H: \direct{making grotesque gestures} When your shins are like
this, and your head, your face---like

this---that's what we mean by \textbf{pu-lî-khɔ̂-lî} in Lahu.

\direct{laughter}

40b. \textbf{Pu-lî-khɔ̂-lî} means ``all bulging and pursed''---a pursed mouth---

41. Cà-g̈âʔ: You ask me things and I'll answer you back---`Today where did
you go?,'

`I went there', `I went here'---

42. P: Yes, that's great, great!

43. H: Yes, okay, I'll ask.

44. Cà-g̈âʔ: Is it already turned on? Has it been going like this the whole
time? Can [our voices] enter

it already?

45a. P: Let this [tape] finish first. Then we'd like to put in a new one.

45b. [To me, in English] Do you want to start another one?

46. JAM: [In English] Ask them if they want to hear what they just said.

* * *

47. P: Wait a second. I'll just put in a new one.

48. H: What'll we do? Hunting? We've already recorded about hunting. Shall we do


``hunting squirrels,''\footnote{\textbf{fâʔ} `rodent' is here used to refer specifically to \textbf{fâʔ-thɔ̂ʔ} `squirrel', the only rodent the Lahu habitually hunt to eat.} Cà-g̈âʔ?

49a. P: Let's also do it about fighting wars!

49b. You've fought alongside\footnote{Note different translations of the commutative noun-particle, \textbf{gɛ} , which must be understood as `alongside, on the same side as' in this sentence, but as the opposite, `against' in the next sentence.} the Chinese, haven't you?

49c. Like that time when we were fighting with Japan.

49d. Can't you [talk about] the Japanese coming up [to Shan State] and the Thais
coming up there,

and their fighting each other?

49e. The white men chased them out, and the Japanese fled, you can say.

49f. If it hadn't been for so many airplanes coming like that---

50. H: \direct{talking to the boys playing the guitar}

You tap on the upper part like this, there on the upper part [of the neck of the
guitar] you press

upwards [i.e., higher up on the strings] or you press downwards.

51a. P: Yeah, like this?

51b. The thing is you have to use all five [fingers], five of them.

51c. If you can do this you can earn a lot of money.

51d. People who do this in White Man's country, in one night they get a thousand...
\footnote{Presumably 1000 baht, then an unimaginably large sum of money for the villagers.}]

>END OF TAPE<

\begin{figure}
\includegraphics{Lahu_tune}\centering
\caption{Tune sung by Headman}
\label{fig:score1}
\end{figure}
