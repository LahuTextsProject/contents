\setcounter{footnote}{0}

\textbf{<Retelling by Paul Tcalo>}

1. I am retelling in November, 1965, this story which was told by a girl who lives
in the Lahu village of Pa-yeh-shu in Thailand.

2. The story is about a widow and her eldest and youngest daughters.

3. Once upon a time there was a certain widow who had seven very beautiful daughters.

4. One day this widow got a great craving for pine-nuts, so she went into the forest
and found a pine-tree.

5. Well, she couldn't manage to climb up the pine-tree.

6. So she couldn't pick them to eat.

7. After a while she said, ``Ah, if there were only someone to pick those
pine-nuts for me, I would give him one of these seven daughters of mine to marry!''

8. Just as she said this, a big snake came out and picked them for her.

9. Then, when she turned back for home, the big snake also came following after.

10. They got near the widow's house, but the big snake would not enter.

11. So she said, ``If indeed and in truth you are my son-in-law, enter
my home!''

12. And the big snake came into the house.

13. When the snake had come into the house, she had him choose from among her seven
daughters.

14. The girl that the big snake wanted to have was the first-born, the eldest sister.

15. The eldest sister, however, did not want the snake.

16. But finally the very youngest of all the seven daughters did accept the snake,
in obedience to her mother.

17. Well, after they were married, towards the morning of the next day, the big
snake had a lantern brightly lit in the room where his wife, the youngest daughter
slept.

18. Then, when the other six girls and their mother came sneaking up to have a
look, they found there behind the partition a very handsome man.

19. The big snake was no more to be seen.

20. After this had happened, that eldest daughter who hadn't accepted him the night
before --who had said she wouldn't have him--she was most distressed at heart.

21. For she was very jealous that her youngest sister had fared thus, and had won
such a fine young man.

22. So she decided she would try to cause her younger sister's death and then marry
her husband.

23. Well, after the younger sister had married the snake, there came a day when
she gave birth to a child.

24. So one day the eldest sister got an idea and put it into action like this:

25. She chopped through a certain pine-tree in the forest just far enough so that
it wouldn't fall over by itself but would remain attached by a thread--and left
it there.

26. Then one day she took her youngest sister along and went off to pick some pine-nuts
to eat.

27. When they reached that pine-tree, when they got to the pine-tree she had cut
half-way through, she had her younger sister climb up to pick the nuts.

28. And then, when her sister had climbed up and reached the top of the tree, she
pushed the tree over.

29. The younger sister died when the tree was pushed over.

30. After her death the elder sister removed the tunic, the headdress, and all
the other clothes from her younger sister and put them on herself.

31. Then she took her sister's child and led him back home.

32. When they had returned home, she went in to live with her younger sister's
husband.

33. When the younger sister died, she turned into a dove.

34. Then, on a certain day, this dove came to sit on a tree-top near their house--near
her husband's house--and called out to her son.

35. Her little son's name was Sheh-khao.

36. The words she called were these: ``Sheh-khao! O Sheh-khao!''

37. ``Don't you remember your mother anymore?''

38. ``Look, I am your mother!''

39. ``Don't you remember the place where you would eat?''\footnote{The dove is referring to the breasts she had when she was human.}

40. Then she called out to her husband as well: ``Serpo, Serpo!\footnote{There is a bit of delicious humor in this name. Many traditional Lahu male names consist of the prefix \textbf{cà}- plus the day of the 12-day animal cycle (\textbf{jɔ}) on which the child was born. Although there is no ``\textbf{vɨ̀-ni}'' or ``Day of the Snake'' as such, there is another day in the cycle ``\textbf{šɨ̄-ni}'' with that interpretation, where the first syllable is a borrowing from Chinese. (See DL, pp. 576-7, 1239.) It is thus entirely appropriate to call the erstwhile snake ``\textbf{Cà-vɨ̀}'' in his human guise. In that spirit, we henceforth translate his name as ``Serpo''.}''

41. ``It is I who am your wife, it is I!''

42. At this Serpo came out and said, ``Hey, what is going on here with
this dove!

43. ``It can talk like a human being!

44. ``Wife, oh Wife, hurry and come look!

45. ``This dove is calling out and saying `I am your wife!'

46. ``It's amazing!''

47. Thus he called to his ``wife'' (it wasn't his real wife, you
know, it was the elder sister), and she came to look on.

48. Then the dove conversed with her husband, saying, ``I am your wife.

49. ``I am your true wife.''

50. She told him the whole story of how she had died on that day.

51. She told him many things:

52. ``This person who is now called your wife is not your true wife.

53. ``She is my eldest sister.''

54. If we are to explain how this could have happened to Serpo, it is because all
of the widow's seven daughters had the same face.

55. Their bodies and their figures were identical.

56. So you couldn't tell one from the other.

57. Then Serpo asked the dove, ``How did she lie to you that day to make
you go into the woods with her?''

58. ``Well, my sister said to me, `Let's go pay a visit to our relatives
in yonder village for a few days!

59. ``But that's not what happened at all.

60. ``On the way, at the side of the road we saw a pine-tree and she made
me climb it to pick the nut-bearing cones, and she pushed me over.

61. ``The tree fell. She had chopped into it. So I died.

62. ``Now, having been killed that way, I have become a dove,''
she said.

63. ``I am your true wife,

64. ``But I am no longer a human being. I have become a dove. I have become
an animal.

65. `` I have become a bird. I am no more a human being,'' she
said.

66. ``So I am very sad, you see, because of you,'' she said.

67. ``Today I wanted to come and visit you.

68. ``And I wanted to see my son Sheh-khao too, so I came to visit,''
she added.

69. When she had finished saying this, the elder sister said, ``Look you,
this is impossible--that filthy, miserable bird!

70. ``Don't listen to what it says!

71. ``Anything a bird like this comes and says is just stuff and nonsense!
\footnote{\textbf{chɔ-khɔ̂} \textbf{mâ} \textbf{hêʔ}: lit. ``is not human speech'', i.e. `is utter nonsense'.}

72. ``It's a great bird of lies!

73. ``\textit{I} am your real wife.

74. ``Just look at my face!

75. ``And look at my tunic and skirt!

76. ``Am I not your wife?

77. ``This is my child.

78. ``Don't let your thoughts be led astray by a wretched bird talking
to you like this!'' she said.

79. But in Serpo's heart something had changed.

80. In his heart he did not really want to believe the elder sister who called
herself his wife.

81. Day after day he thought about it. He pondered all sorts of things.

82. Then at last he got an idea.

83. He carefully brought seven very sharp knives, and placed them on top of a board,
arranging them so that they were spaced out each at a little distance from one
another.

84. When he had finished arranging them this way, he summoned his wife.

85. ``Wife, come here a moment,'' he said.

86. ``If you are indeed my wife, you just climb up onto these seven knives
and try to walk.

87. ``Try walking up there over them all, one after the other.

88. ``When you're walking there, if the knives don't cut into your feet,
I shall believe that it is you who are really my wife,'' he said.

89. ``The reason for this is that I have sworn an oath before God.

90. ``'If she is really my wife, do not let these seven knives cut her!'

91. ``But if you are not my wife, they will cut you. This is the oath
I have sworn.''

92. Thereupon his wife--that is, the elder sister who had been playing the part
of his wife--carefully lifted up [her feet in a pair of] excellent shoes, great
shoes that nothing could pierce, and climbed up [onto the board].

93. But, when she stepped up onto the first of the knives, it cut into her foot,
and she suddenly fell over forward.

94. Falling forward, she sprawled flat onto the seven knives, and the seven of
them, one after the other, cut her body up into seven pieces.

95. And there she died.

96. Then Serpo said, ``Aha! Now I know that she was not really my wife.

97. ``That dove was really my wife.

98. ``The widow's eldest daughter was no good.

99. ``She was not a good daughter.

100. `` She acted out of envy. Now I know all.''

101. That's the end of the story.

102. If we ask what we ought to learn from this story, it is this:

103. If we are envious of others, if we try to bring about someone's ruin and downfall,
it does not profit us, and it does not profit him.

104. Thus, thinking it will profit us, if we try to ruin someone else, that person
may indeed be brought to destruction.

105. But we also must thereby be destroyed, you see.

106. It does us no good, and it does him no good. That's the way it is.

107. So, let no man envy anyone else, and let no one bring his fellow-man to ruin.

108. This story is a bit like the story of the potter and the laundryman.\footnote{See ``The Potter and the Laundryman'', section~\ref{sec:29}.}

109. Do not be envious of others, it means.

110. If we envy other people there is no advantage for us. This is what we have
said.

111. The story is finished.

