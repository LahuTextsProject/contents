
\textbf{133 The Snake and the Widow's Daughters}

Once upon a time there was a certain widow who had seven very beautiful daughters.
One day this widow got a great craving for pine-nuts, so she went into the forest
and found a pine-tree. Well, she couldn't manage to climb up the pine-tree. So
she couldn't pick them to eat. After a while she said, \texttt{"}Ah, if there were
only someone to pick those pine-nuts for me, I would give him one of my seven daughters
to marry!\texttt{"} Just as she said this, a big snake came out and picked them
for her.

Then, when she turned back for home, the big snake also came following after. They
got near the widow's house, but the big snake would not enter. So she said, \texttt{"}If
indeed and in truth thou art my son-in-law, enter my home!\texttt{"} And the big
snake came into the house.

When the snake had come into the house, she had him choose from among her seven
daughters. The girl that the big snake wanted to have was the first-born, the eldest
sister. The eldest sister, however, did not want the snake. But finally the very
youngest of all the seven daughters did accept the snake, in obedience to her mother.

Well, after they were married, towards the morning of the next day, the big snake
had a lantern brightly lit in the room where the youngest daughter slept. Then,
when the other six girls and their mother came sneaking up to have a look, they
found there behind the partition a very handsome man. The big snake was no more
to be seen.

After this had happened, that eldest daughter who hadn't accepted him--who had
said she wouldn't have him--she was most distressed at heart. For she was very
jealous that her youngest sister had fared thus, and had won such a fine young
man. So she decided she would try to cause her younger sister's death and then
marry her husband.

Well, there came a day when the younger sister gave birth to a child. So one day
the eldest sister got an idea and put it into action. She chopped through a certain
pine-tree in the forest just far enough so that it wouldn't fall over by itself
but would remain attached by a thread--and left it there. Then one day she took
her youngest sister along and went off to pick some pine-nuts to eat. When they
reached that pine-tree, when they got to the pine-tree she had cut half-way through,
she had her younger sister climb up to pick the nuts. And then, when her sister
had climbed up and reached the top of the tree, she pushed the tree over.

The younger sister died when the tree was pushed over. After her death the elder
sister removed the tunic, the headdress, and all the other clothes from her younger
sister and put them on herself. Then she took her sister's child and led him back
home. When they had returned home, she went in to live with her younger sister's
husband.

When the younger sister died, she turned into a dove. Then, on a certain day, this
dove came to sit on a tree-top near their house--near her husband's house--and
called out to her son. The words she called were these: \texttt{"}Sheh-khao! O
Sheh-khao! Don't you remember your mother anymore? Look, I am your mother! Don't
you remember the place where you would eat?\texttt{"}\footnote{The dove is referring to the breasts she had when she was human.} Then she called out to
her husband as well: \texttt{"}Serpo, Serpo!\footnote{There is a bit of delicious humor in this name. Many traditional Lahu male} It is I who am your wife, it is
I!\texttt{"}

At this Serpo came out and said, \texttt{"}Why what is going on here with this
dove! It can talk like a human being! Wife, O Wife, hurry and come look! This dove
is calling out and saying 'I am your wife!' It's amazing!\texttt{"} Thus he called
to his \texttt{"}wife\texttt{"} (it wasn't his real wife, you know, it was the
elder sister), and she came to look on. Then the dove conversed with her husband,
saying, \texttt{"}I am your wife. I am your true wife.\texttt{"} She told him the
whole story of how she had died on that day. She told him many things: \texttt{"}This
person who is now called your wife is not your true wife. She is my eldest sister....\texttt{"}

(If we are to explain how this could have happened to Serpo, it is because all
of the widow's seven daughters had the same face. Their bodies and their figures
were identical. So you couldn't tell one from the other.)

The dove told how her elder sister had lured her into the woods by a lie. \texttt{"}But
how did she lie to you?\texttt{"} Serpo asked her. \texttt{"}Well, my sister said
to me, 'Let's go pay a visit to our relatives in yonder village for a few days!
But we never got that far. On the way, at the side of the road we saw a pine-tree
and she made me climb it to pick the nut-bearing cones, and she pushed me over.
The tree fell. She had chopped into it. So I died. How, having met my death that
way, I have become a dove. I am your true wife, but I am no longer a human being.
I have become a dove. I have become an animal. I have become a bird. I am no more
a human being,\texttt{"} she said. \texttt{"}So I am very sad, you see, because
of you. Today I wanted to come and visit you. And I wanted to see my son Sheh-khao
too, so I came to visit,\texttt{"} she added.

When she had finished saying this, the elder sister said, \texttt{"}Look you, this
is impossible--that filthy, miserable bird! Don't listen to what it says! Anything
a bird like this comes and says is just stuff and nonsense! It's a great bird of
lies! \textit{I} am your real wife. Just look at my face! And look at my tunic
and skirt! Am I not your wife? This is my child. Don't let your thoughts be led
astray by a wretched bird talking to you like this!\texttt{"}

But in Serpo's heart something had changed. In his heart he did not really want
to believe the elder sister who called herself his wife. Day after day he thought
about it. He pondered all sorts of things. Then at last he got an idea. He carefully
brought seven very sharp knives, and placed them on top of a board, arranging them
so that they were spaced out each at a little distance from one another. When he
had finished arranging them this way, he summoned his wife. \texttt{"}Wife, come
here a moment,\texttt{"} he said. \texttt{"}If you are indeed my wife, you just
climb up onto these seven knives and try to walk. Try walking up there over them
all, one after the other. When you're walking there, if the knives don't cut into
your feet, I shall believe that it is you who are really my wife,\texttt{"} he
said. \texttt{"}The reason for this is that I have sworn an oath before God. 'If
she is really my wife, do not let these seven knives cut her!' But if you are not
my wife, they will cut you. This is the oath I have sworn.\texttt{"}

Thereupon his wife--that is, the elder sister who had been playing the part of
his wife--carefully put on a pair of excellent shoes, great shoes that nothing
could pierce, and climbed onto the board. But, when she stepped up onto the first
of the knives, it cut into her foot, and she suddenly fell over forward. Falling
forward, she sprawled flat onto the seven knives, and the seven of them, one after
the other, cut her body up into seven pieces. And there she died.

Then Serpo said, \texttt{"}Aha! Now I know that she was not really my wife. That
dove was really my wife. The widow's eldest daughter was no good. She acted out
of envy. Now I know all.\texttt{"}

If we ask what we ought to learn from this story, it is this: If we are envious
of others, if we try to bring about someone's ruin and downfall, it does not profit
us, and it does not profit him. Thus, thinking it will profit us, if we try to
ruin someone else, that person may indeed be brought to destruction. But we also
must thereby be destroyed, you see. It does us no good, and it does him no good.
So, let no man envy anyone else, and let no one bring his fellow-man to ruin. This
story is a bit like the story of the potter and the laundryman.\footnote{See text \#29.} If we try to
destroy others, we too shall be destroyed. This is what we have said.

The story is finished.

