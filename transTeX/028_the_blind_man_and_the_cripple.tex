
28 The Blind Man and the Cripple

or: The Advantages of Faith in Fortune-tellers

or: The Power of Positive Thinking

Huey Tat 3/9/65

1. Once upon a time there were two people who were friends.

2. One of them was blind, and the other was a cripple \.\footnote{khɨ-qɔ̀ʔ-pā: lit. `one with a crooked leg'}

3. While the two of them were friends, one day they had an idea, and they both
went together to a fortune-teller \footnote{mɔ́-pā: `medicine man; fortune-teller; seer; magician; spirit-doctor; shaman'} to ask how his eyes could come to see and
how his leg(s) could become straight.

4. When the two of them went to the fortune-teller, the fortune-teller said to
them, ``On your way going [back], above you on the road when you see a bird's nest
in the crutch of a tree, make the blind man climb up and take it,'' he said.

5. Just exactly as the fortune-teller had said, as the two of them were going on
their way, when the cripple looked upwards, he saw a bird's nest.

6. Since he caught side of the bird's nest, he told the blind man to climb up.

7. The blind man said, ``Since my eyes \{are blind/ don't see\}, I can't see all
these things.

8. ``I can't climb up,'' he said.

9. But the cripple said, ``You can climb up.

10. ``I'll show you how,'' he said.

11. As the blind man was climbing, at a certain moment he said, ``Have I \{got
there/reached it\}?'' he asked.

12. But his friend said, ``You're not there yet.  Keep on climbing a little.''

13. After he had climbed some more, when he [the blind man] asked again, ``You
got there!'' he [the cripple] said.

14. When he [the cripple] said ``You got there!'' since he [the blind man] really
wanted to see, he opened his eyes wide with all his might, and all of a sudden
his eyes could see.

15. In just the same [magical] way, when he took the bird's eggs in his hand, these
eggs turned into a snake, as because he was as terrified he dropped it right towards
the place where his friend was sitting, at which point his friend also became terrified,
so he suddenly jumped up and his [crippled] leg straightened out.

16. So then, we Lahu of this generation also ought to learn from these two people
and take it to heart \.\footnote{Lit., ``If we absorb the lesson of these two people it is good.'' hên-yù:}

17. Since they heeded the words of the fortune-teller, we've learned that even
the blind man got to see again.

18. We've learned also that the cripple's leg straightened out.

19. Everybody who \{is heedful/ takes good advice\}, their descendants and all
of their relatives as well, if they take all this to heart it will be good.

20. Thus our Lahu ancestors told us long ago.

