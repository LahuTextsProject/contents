\setcounter{footnote}{0}

1. (H) Ahem, well, I have a little announcement\footnote{\textbf{kâ} \textbf{lâ} \textbf{tù} \textbf{ɔ̀-lɔ}: lit. ``a matter to let you hear''.} to make to you, brothers and
sisters. On Monday morning\footnote{\textbf{ší} \textbf{pə̀} \textbf{mû-šɔ́} \textbf{qo} \textbf{ɔ̄}: lit. ``on the morning when the week is finished''. Sunday is \textbf{ší-ni}.} Java and Naphui are planning to get married! Now
don't any of you go off anyplace!\footnote{The expression \textbf{tê} \textbf{kà} \textbf{qay} \textbf{ve}, lit. ``go to a place'', means ``go off someplace else''.} We're going to celebrate the wedding, you
see. We'll have a feast in joy and gladness, I tell you.

2. (T) Well, my brothers, getting married -- the joining of hands in marriage --
is a very important thing according to our Christian customs. Therefore let everyone
remain [in the village] in joy and gladness. For this is no laughing matter for
us. Since they are to take one another in marriage in the sight of God, we Christians
must all observe the occasion with the two of them in joy and gladness.

3. (H) Well, will they slaughter a pig [to eat]?

4. (T) Well, I daresay they'll have to slaughter a pig, all right. Even in times
past --from time immemorial\footnote{\textbf{tê} \textbf{co} \textbf{pə̀} \textbf{tê} \textbf{cá}: lit. ``having finished one generation, another generation''.} -- there has been a wise saying to the effect that
when people get married, if they don't celebrate it with a wedding feast, there
will be no luck and no success with the crops, either.

5. (H) Well, in that case, it ought to be done early in the morning, the whole
business of killing the pig.\footnote{\textbf{tê} \textbf{g̈ɨ̀} is a group classifier: ``all the various things connected with the pig-killing." Eating a pig is a very special event for the impoverished Lahu, something that was possible only a few times a year.}

6. (T) Yes, we ought to do it rather early. According to my plans we'll kill the
pig at six-thirty, finishing with it at seven, then at eight o' clock we'll join
their hands [in marriage]. In the church. Who all will be in the pig-slaughtering
party? Make the arrangements properly, each side by itself.\footnote{That is, the bride's side and the groom's side each designate several men for the job or honor of serving on the team of slaughterers.}

7. (H) Well, on my side, Ja-ui is one, and Jaha is another, and Jageh is another.
Three people ought to be enough, for my side.

8. (T) Now on my side, Jaqa is one, and Ehkhapaw\footnote{A Shan name.} is another, and Jaqunyi is
another -- with these three people it should be enough, as far as my side goes.

9. (H) It comes to six people. That's plenty. We have enough then, with that many.

10. (T) Well, this girl now, I don't really know her. Whose child is she anyway?

11. (H) Oh, don't they say that she's the daughter of Jaqeh and Nada?

12. (T) Jaqeh and Nada's daughter, eh?

13. (H) Yep.

14. (T) Aha, so that's the one. But I doubt if she can take a husband yet! She's
not big [enough] yet!

15. (H) She's big enough, she's big enough now.

16. (T) How old has she gotten to be, do they say?

17. (H) She's eight.

18. (T) Eight! Good grief, if she's only eight, how can she sleep with a man!

19. (H) \direct{laughing} She's thirteen, thirteen and a half!

20. (Boy) I bet you can't tempt her into it yet.

21. (H) She's thirteen and a half already.

22. (T) Thirteen, huh? And how old do they say the man is?

23. (H) I heard them say the man is twenty-five.

24. (T) Twenty-five. Wow, a man that strong can't make it with a girl who's only
eight! I mean, a girl who's only \textit{thirteen}.

25. (H) She's big, I tell you. I'll bet she can. She's big [enough] already. Her
body is a good size.\footnote{\textbf{ɔ̀-šɛ̄-ma} `a female body.' Cf. \textbf{ɔ̀-šɛ̄-phâ} `a male body.' \textbf{ɔ̀-to} `body' is a Shan borrowing.}

26. (T) I wonder just how big her body is. Her build, I mean.

27. (H) It's plenty big, plenty big.

28. (T) The guy is big, you know! Because he's from great stock.

29. (H) Oh, she's not afraid, they say. Since she's big already.

30. (Kachin) Could anybody make it with her now?

\direct{laughter}

31. (H) They say she can, I tell you!

32. (Boy) I still don't really know who she is.

33. (H) Naphui, Naphui! Nada's daughter.

34. (boy) Oh.

35. (T) I see. She's a north-country person, isn't she?

36. (H) Right. A north-country girl, from up there. She lives south of Vanna,\footnote{A village in Shan State, Burma.}
they say.

37. (Boy) Yeah, for sure!\footnote{The boy is imitating the way of saying `yes' in north-country dialect.}

38. Is she a Christian?

39. (H) I hear she became one not long ago -- she's a Christian.

40. Has she learned a little about our ways and customs already?

41. (T) Well then, about the man -- whose son do they say he is?

42. (H) You mean the groom?

43. (T) Yes.

44. (H) Didn't you hear me say he was Jaqeh's son?\footnote{The Headman is confused (see sentence 11); either he forgot that Jaqeh was the \textit{bride}'s father or else both fathers-in-law have the same name.}

45. (T) Jaqeh's son?

46. (H) Jaqeh's son.

47. (T) I bet they call him Jaqeh because he's an oddball.\footnote{\textbf{qɛ́} \textbf{ɛ̀} \textbf{ve} means `to be peculiar, strange.' It is common to assign nicknames by using the prefix \textbf{cà}- (for men) or \textbf{na}- (for women) to nouns or adjectives describing salient aspects of the individual's character or appearance.}

48. (H) Exactly. He grows tea with the Thais. He's peculiar.

49. Yeah.

50. Because they say he acts like some kind of nut,\footnote{\textbf{kâʔ-ku} \textbf{ve} (slang) ``an eccentric, a nut.'' Lit: `something cold and dried out,' as in \textbf{ŋâ} \textbf{kâʔ-ku} `dried fish,' \textbf{ɔ̄} \textbf{kâʔ-ku} `dried boiled rice.'} I bet.

51. He's a nut, all right, all right.

53. (T) He's what you [call] a nut.

54. He's probably the farm laborer who went and stole those people's things yesterday
to feed his face and who the Thais chased and beat up.

55. (H) Of course. He's the guy they were talking about who stole their elephant-chains
at Mehthawmaw\footnote{A place not far from the village where the Thai keep elephants to work in the teak-forests.} -- the elephant-ropes and the neck-chains and went and sold them
to fill his belly!

56. (T) Ah, ah, ah, ah.

57. (Boy) The one who the dogs were chasing to bite!

58. (T) Now I know, now I know. Well then, the time has now come to join their
hands in marriage. Haven't they beaten the gong yet?\footnote{To summon the villagers to the church.} Have they beat it already?

59. (H) They've beat it, they've beat it. Let's go, let's go.

60. (T) You've heard it? Everybody hears it.\footnote{l.e., `everybody except me.'} I suppose everyone has heard it,
eh?

61. (H) We've heard it.

62. (T) Ah, now I hear it [too]!

63. (Boy) I hear it now!

64. We ought to make it snappy and go.

\begin{center}
* * *

\direct{In the church}
\end{center}

65. (T) Well, since our two friends are getting married today, this morning, this
magnifies and elevates the very great laws and ways of God. If all of us have indeed
arrived, He\footnote{We know that God is the subject because of the 3p > 3p benefactive verb \textbf{pî}. If the pastor were the subject, the verb particle \textbf{lâ} (1p > 3p benefaction) would be used instead.} will join their hands in marriage. Now, do you, Java, swear in
the sight of God and in the sight of all these people that you will not separate
or stray from this your wife Naphui as long as you shall live, that you shall not
abandon or reject her, even should she become crippled or blind, even should she
become sick or unwell; that until you must be separated by death you will not separate
or stray from her, neither abandoning nor rejecting her?

66. (Java) I swear it.

67. (T) He swears it. It is well.

Now then, Naphui, [do you swear that] your husband who is now holding your hand
-- er --

68. [prompting] Java!

69. (T) This man Java -- do you swear in the sight of God and in the sight of all
these people that as long as you live, unless God should take you away and cause
you to separate from one another, whether he should become ill and diseased, or
crippled and blind, or poor and wretched, that you will look after him, and help
him, and take care of him, and that until you must be separated by death he will
be your husband forever?

70. (Naphui) I swear it.

71. (T) She swears it. It is well.

Well, then, [now] we just have to divide up the pork! Has the pork been divided
yet?

72. (H) It's [all] cut up and stacked, it's [all] cut up and stacked. Look over
there!

73. (T) Well, how shall we do it?

74. (H) Shall we cook and eat it [all together]? Or shall we divide it up, among
the various houses?

75. (Jingpho) Divide it, divide it!

76. (T) Divide it!

77. (Boy) Divide it!

78. (T) How shall we do it? Shall we make it one share per couple? Or else shall
we make it one share per household?\footnote{A household may include grandparents and/or collateral relations.}

79. Make it one share per household. It's too much trouble [otherwise].

80. (H) But look, I have many people [in my house], two or three couples, and if
there are lots of them they want a lot to eat!

81. (T) Even if you do have lots of people, when it's time to work, they don't
do it! Only one person [from your house] did any work with this business.\footnote{I.e., helping with the preparations for the wedding.}

82. Right! When it's time to work, only one person does it.

83. (T) [But] when it's time to eat, they want to eat a lot! That's what [I call]
being greedy!

84. (H) Shouldn't this be what you call ``acting out of voluntary charity''?\footnote{\textbf{bo} \textbf{te} \textbf{ve}: `do out of charity; perform a Christian religious ceremony'.}
If you were being charitable, [everyone] ought to eat!

85. (T) To receive such charity, one should also do some work! There are always
some of us who don't do the work!

86. (H) We \textit{do }work!

87. (T) Only one person from your whole house, \textit{one person}. And when he
does happen to come once in a while, he goes home before noon! When it comes to
working.

88. Would you swear that they don't work?

89. (T) I would swear to it. I never once saw them working.

90. (H) Well then, how much pork will we eat, in all?

91. (Boy) Don't give those guys anything to eat!

92. (H) Thirty-five kilos, we'll eat 35 kilos.

93. (T) Are you setting up a rule\footnote{\textbf{ɔ̀-lî} \textbf{tɛ} \textbf{ve}. It is the headman's prerogative, subject to approval by the village at large, to lay down general guidelines of this sort.} [that it should be] 35 kilos?

94. (H) I \textit{am} making it a rule. Starting from today and in the future everyone
will have to observe\footnote{\textbf{ɔ̀-lî} \textbf{yɛ̂} \textbf{ve}: lit. ``use the rule''.} it, and it alone.

95. (T) If that is the only way it may be done, I'm afraid\footnote{\textbf{à-mù} ... (\textbf{pɨ́} \textbf{à}) serves to introduce warnings or unpleasant predictions.} that only those
who are well-off will be able to set up housekeeping. If you're poor -- the poor
and penniless won't be able to set up housekeeping like them.

96. (H) As for that, that's something that both sides -- the groom's parents and
the bride's parents -- should agree on. There are people who join hands [in marriage]
even without having a wedding-feast. There are also those who just kill a little
tree partridge\footnote{\textbf{g̈âʔ-mɛ̄-cwi} is said to be a Red Lahu word. \textbf{g̈âʔ-mɛ̄-wí} is the `quail-tailed chicken'. See DL:1122.} or two and get married. If your word is good.\footnote{The Headman leaves the rest of his thought unsaid: ``As long as you are a man of your word, any arrangement you might make will be all right." The Headman has retreated from his previous hasty statement which was interpreted as laying down a blanket rule of 35 kilograms of pork per wedding. He only meant to say, he now claims, that if a pig is slaughtered by a wealthy bridal pair, it should be of a standard size.}

97. (T) Oh, well in that case it's all right. So, how shall we divide up this pork?
If no one objects,\footnote{\textbf{mâ} \textbf{hêʔ} \textbf{qo} `if not; otherwise.' Here elliptical for something like \textbf{ni-ma} \textbf{mâ} \textbf{cɔ̂} \textbf{pā} \textbf{cɔ̀} \textbf{ve} \textbf{mâ} \textbf{hêʔ} \textbf{qo} `if there is no one who disagrees.'} distribute it to the [various] houses, headman -- give it
out to the houses!

98. (H) Here, give it out to the houses, to the houses. Give it out to the houses!
Try dividing it up so each house gets eight hundred grams.\footnote{A \textbf{mɛ̂ʔ} is 1/10 of a kilogram.} Take it and put
it in [the scales]! Spear\footnote{\textbf{thɔ} \textbf{ve}: `pierce a hole in a piece of meat and put a bamboo strip through the hole for a handle'.} [the meat]!

99. Spear it, spear it, spear it!

100. (H) Put it in, into that scale.

101. (T) Ah, everybody must lend a hand\footnote{The four-verb sequence \textbf{g̈a} \textbf{ga} \textbf{ni} \textbf{pî} literally means ``must try helping''.} in the proper spirit. Otherwise, if
we're all arguing and squabbling about it, if they should overhear us, if the others
overhear us, it would be very embarrassing. We'd be ashamed in front of them.

102. (H) Ah, here come the people! Give out the pork. It's already divided up.

103. (T) Announce it, announce it!

104. Come and get the pork, come and get the pork! It's for the whole village!

105. We're coming to get it, we're coming to get it.

106. (T) I don't see them coming from over there yet, from Jabvuh's part of the
village.\footnote{A \textbf{qhô} is a subdivision of a village, including two or three houses connected by location or familial ties.}

107. (H) Go call them, go call them!

108. (T) He says to go call Jabvuh's part of the village.

109. Jabvuh! Come and get your pork, I tell you!

110. (Jabvuh) \direct{lazily} Let me make some tea first.

111. (T) There was plenty of time to make tea this morning! You didn't even come
to the wedding ceremony. People like you are hard to teach to and hard to preach
to, that's all there is to it! It never fails.

112. (Boy) He's just smoking his opium!

113. (T) He has no meat on his bones,\footnote{\textbf{nā-qā-pɨ} \textbf{pâ-nê} \textbf{ɛ̀} \textbf{te} \textbf{ve}: lit. ``to have it near the forehead/to have one's forehead flat`` (\textbf{pâ} `flat,' (\textbf{pâ-)nê} `near'), i.e., ``be scrawny/emaciated'' (e.g., of an opium addict).} you know.

\direct{laughter}

114. (Jabvuh) Fault-finding bastards! I wouldn't eat anything of yours now anyway!

115. (T) It's all the same to us if you don't eat! Since we've tried to offer it
to you properly.

116. Don't eat it then! You're the only one who doesn't get along with the others.

117. (H) Don't quarrel, don't quarrel. Let's do this good deed, let's do this good
deed, gaining grace and gaining blessings.

118. (T) I don't hear the sexton's part of the village coming either!\footnote{The intensifying particle \textbf{qha-pâʔ} (or simply \textbf{pâʔ}) is used especially when auditory perception is at issue.} Carrying
on this way, you people are hard to reason with. Even when we say ``Come
and eat!" you still don't come.

119. (H) Look up there, look up there, I see his son coming, look up there! Give
it to him. Give him a skewer [of meat]. One bundle.

120. (T) Give it to him, one bundle.

121. Here, here, here you are. Take\footnote{\textbf{yù} \textbf{ŋɔ́} \textbf{və-ʔ}: \textbf{ŋɔ́} is Yellow Lahu for the post-head versatile verb \textbf{ni} (`V and see'). The dialect mixture is intended to be humorous.} it away!

121A. (Boy playing Jabvuh's son) It's not even as big as my dick. \direct{laughter}

122. (H) Well, it's all over then, this wedding-feast. We've finished giving out
the pork, each house's share. If anybody hasn't come, it's dark out already.\footnote{I.e., if somebody hasn't come to pick up his meat yet, it's too late now.}

\begin{center}
\direct{A few weeks or months later}
\end{center}

123. (T) Well, this time we've got to discuss my son's running back home [and abandoning
his wife]. Ah, my son has gone away from you!\footnote{The speaker is addressing the bride's father.} And you haven't seen him come
back to your house yet?

124. (H) I don't see him coming back!

124A. He can't be reasoned with, that son of yours!

125. (Boy) He ought to be thrashed, right?

126. (T) That boy says that I scolded him, that's what he's saying! But I have
never scolded or rebuked him, not once!

127. I told you so! ``They're not old enough yet, don't marry them off,
don't marry them off!" I told you.

128. (T) Nobody married them off. They went and fell in love on us all by themselves.\footnote{\textbf{cɔ̂} \textbf{və} \textbf{lâ} \textbf{ve}: `to fall in love', lit. ``be transported by harmonious feelings''.}
There was nothing anybody could do about it. A thing like that.

129. I'll bet he ran away [from her] because he couldn't stand sleeping with her.

130. That's probably not true.

131. Sure. If that wasn't it, he shouldn't have run away.

132. (T) I wonder if he left because he was tired of the work. Since he was working
for other people.

133. There wasn't all that much work, as far as I could see.

134. (H) He's always sleeping, always sleeping. He's as lazy as they come, that
one.

135. That's not true. He just couldn't stand the woman.

136. (T) Oh, good God! Our children are hard to teach to and hard to preach to!
I doubt if they'll ever come up to the standard of other people's children.\footnote{\textbf{šu} \textbf{yâ} \textbf{lɔ} \textbf{šū} \textbf{ve}: `to be the same as others' children's matters.' Usually simply \textbf{šu} \textbf{lɔ} \textbf{šū} \textbf{ve} `to be up to other people's standards.'}
If you tell them [to come] \textit{here}, they'll go \textit{there}. If you tell
them [to go] \textit{there}, they'll come \textit{here}. Ah, me, me! I'm tired
of living. I'm ashamed [to compare them to] other people's children.

137. (H) If you mean that son [of yours], he's a lazy good-for-nothing puppy. He
doesn't earn his living by working and toiling.

138. (T) That's just it, that's just it.

139. Wow, speaking of that son of yours, he's got a prick the size of somebody's
forearm, so...

\direct{laughter}

140. (T) Come now.

141. (Boy) Well, her big pussy, her big pussy was large enough, wasn't it?

142. (T) Well, I can't do anything about that son of mine anymore. I'll just have
to go to the headman. I'll just have to go to the headman. Headman! Call all these
people together, please. I can't cope.

143. (H) Hey there! Let everyone come to my house tonight! Here's a matter that
has come up. You may not stay away!

\begin{center}
* * *
\end{center}

144. Say, I wonder why there are people shouting up there tonight.\footnote{See footnote 32.}

145. (T) \direct{playing the role of a bystander} Oh, they say it's
about somebody's son. Somebody's son can't stay with his in-laws.\footnote{\textbf{šu} \textbf{yâ} \textbf{šu} \textbf{gɛ} \textbf{mâ} \textbf{chɛ̂} \textbf{phɛ̀ʔ}: lit. ``somebody's son can't stay with somebody."}

146. Always, always, it's about somebody's son!

147. God, I'm tired to death of listening to stories about somebody's son!

148. (Boy) No, sir, I'm not going. I'm not going.

149. (T) You can't stay away, boys. That's been pounded into your heads!\footnote{Lit: `pounded into your foreheads'. \textbf{ó-qō} `head' is also usable in this idiom. See below.} You've
really got to go. Like it's been drummed into everybody's head: you cannot absent
yourselves. Somebody if \textit{you} have a case for discussion you won't have
anybody to listen to you. You ought to go and listen, even if you don't say anything
or want to do anything.

150. (H) Well, has everybody arrived?

151. (T) Yes, we're all that there will be, I guess.

152. (H) Come in, come in!

153. Let's discuss it till we get to the bottom of it.

154. (H) Our business this evening concerns this pair of young people. These young
people have behaved thus-and-so. All of you consider what is to be done.

155. (T) \direct{in role of pastor} Dear Lord! After swearing and vowing
in the sight of God, and taking each other and marrying one another, how could
it be good for them to squabble like this!\footnote{This rhetorical question translates the final unrestricted particle \textbf{nɛ̀-ɔ̄} `supposition or wonderment'.} How could it be in accordance with
the will of God, acting in this manner! I am your Pastor, so let me give you some
words of instruction. If you wish to listen, that's up to you. If you don't want
to listen, that's up to you. I say it is not in accordance with the will of God!

156. (H) \direct{addressing the young man's father} Well, as far as
this son [of yours] running back to you goes, it's because of \textit{your} upbringing
and \textit{your} teaching and \textit{your} instruction that he acts this way.
You are to blame for not sending him back [to his wife].

157. (T) \direct{playing role of groom's father} Ah, this son of mine
is unteachable and incorrigible! All of you have known long since that he's hard
to teach to and hard to preach to. There's nothing to be done. Since you've got
the son\textit{ }here you've just got to put him on trial and tell \textit{him}
what to do.\footnote{Without wasting time trying to blame the father for his son's behavior.}

158. (H) Well, it has been several days now since you led\footnote{The speaker uses the expression \textbf{šɛ} \textbf{qòʔ} \textbf{e} \textbf{ve} `to lead back home,' implying that the son left his wife on the father's instigation.} your son back to
your house. Whatever is going on, you ought to know all about it. It's your son,
after all.

159. (T) Well, then, you're telling me [that I've got] to pay compensation, aren't
you!

160. If you don't stay married, you've got to pay compensation.

161. (H) Decide the case! Decide the case!

162. (The groom) Whether I live or die I don't want to stay married [to that woman]!

163. (H) Well, my brothers, let's all try to bear in mind the principles which
our ancestors used to observe of old, and act accordingly, shall we? Like when
a husband and wife say they will get divorced, whichever one it is who doesn't
want to stay married and [who wants to] cast the other off, he is the one who must
pay for it. Our ancestors have had such a rule and such a precept.

164. (T) Yes, although some say that ``people are greater than the law,''
we are all under the law. Everybody is.

165. That's right, that's right. However you slice\textbf{ }it.\footnote{Lit: ``however you do.''}

166. (T) Come on and give the verdict!

167. There's nothing he can do now.\footnote{Except pay the compensation.} His obligation\footnote{\textbf{ɔ̀-bo} is a very abstract word with several distinct senses that are interrelated in complex ways. See DL:186-187.} is too great.

168. (H?) Well, let us act according to the traditions and principles which were
observed in the days of our ancestors, shall we not, my brothers? According to
the practice of our ancestors, whenever anyone wanted to cast off [a spouse], he
had to pay fifteen rupees.\footnote{\textbf{thɛ̀ʔ}: a Burmese unit of currency. One \textbf{thɛ̀ʔ} equalled 5 Thai baht. The total fine at the time was thus approximately 75 baht, or US\$3.75.} And in case there were children, it was thirty rupees.\footnote{Three \textbf{khá}. One \textbf{khá} (< Shan) is worth 10 \textbf{thɛ̀ʔ}, or 50 baht. The total fine in this case was thus equal to 150 baht, or US\$7.50.}
Tonight you all [should settle] this matter according to the rule observed by our
forefathers that ``He who abandons [his spouse] must pay for it.''

169. Pay the money, pay the money!

170. (The husband) Here, here, here [it is].

171. Count it, Headman, count it.

172. (T) If it comes to 150 baht, it's customary [for the headman] to deduct 15
baht, you know. The ``listening-cost.'' Here, you take 135 baht with you.
These 15 baht are taken out to cover the ``sitting expenses'' for
all these people -- the tea money and the tobacco money.

173. So here, take it, take it, take it already!\footnote{The speaker is carried away in a plethora of particle-repetitions.}

174. \direct{chuckling} Since you've paid me my compensation, that's
all there is to it.

175. (T) Well, this business is over and done with.

