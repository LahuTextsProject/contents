\setcounter{footnote}{0}


1. Once there was a certain man.

2. He went off to clear land for a road.\footnote{\textbf{qɔ̂} `hoe; work land.' \textbf{yàʔ-qɔ} \textbf{qɔ̂} \textbf{ve} (OV) `clear land for a road.'}

3. Since they were making [people] go clear land for a road.

4. He worked and worked on the road, and when the time came to go back home, he
was very hungry.

5. Since he was very hungry, he went to beg something to eat from a Shan.

6. So he said to the Shan: ``Madam,\footnote{Polite form of address to a middle-aged woman in Shan. (Shan \textit{mɛ-lêŋ}, Thai \textit{mɛ̂ɛ-líaŋ}, lit. ``nourishing mother'').} Wouldn't you have some cooked rice?''
\footnote{In Shan in the original.}

7. The Shan said,

7. ``I have, I have.

8. I don't have any curry\footnote{Shan \textit{phák} `that which is eaten with rice' (cf. Thai \textit{kàp-khâaw}), here conventionally translated as `curry'.} to go with it, though."\footnote{In Shan in the original.}

9. So he said, ``Even if there's nothing to eat with it, I'll eat it.''
\footnote{The Lahu is now talking Lahu. though the narrative conventions require us to assume that he is still really talking Shan.}

10. So the Shan [lady] ground up some peppers\footnote{Faute de mieux, the Lahu and other impoverished hill-people will choke down tier rice with hot peppers when they have nothing else to eat.} and gave them to him to eat.

11. When she had ground them up and fed them to him, he ate and was about to leave.
\footnote{\textbf{phɛ̂} `to release, let go; leave, depart.'}

12. As he was about to leave, the Shan said: ``[Sorry] you didn't even get
any curry to eat."

13. At this he [tried to] make [polite] conversation with the Shan.\footnote{He is touched by her kindness, and wants to say something nice in return.}

14. ``Though I didn't get to \textit{eat} any curry, it was still delicious!

15. May madam feast on my gratitude!", he said.\footnote{The humor here is of a gentle sort. The Lahu is trying to use his most flowery Shan for the occasion, but gets tangled up in awkward phraseology to the point where it is hard to see what he means at all. The word \textbf{wá} `sweet' is quite inappropriate in this context. With foods of pleasantly mild flavor it is often used by extension to mean simply `good-tasting'; but no Shan would ever use it to describe a fiery mess of rice with hot peppers. The next sentence is still more original. The Lahu is trying to thank and compliment the lady at the same time. As the translation implies, the word \textbf{bwé} seems to be a conflation of two words: Lahu \textbf{ɔ̀-bo} `thanks' and Shan pɔ́j \textasciitilde{} pwáj (< Burmese \textit{pwê}) `feast'. It is certainly not the normal Shan way of expressing gratitude. If the Shans explicitly say `thank you' at all, it is usually via the Pali/Burmese word \textit{cêi-zû} (Shan \textit{kjé-šú}).}

