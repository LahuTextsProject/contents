\setcounter{footnote}{0}

*For more on Lahu proverbs, see JAM 2006:449-50 and 2011.

Speaker A:

1. OK, come on\footnote{This interjection, meaning 'come on!; let's go!' is a borrowing from Burmese (DL:341).} ---once upon a time sweet peppers---whoops!\footnote{The speaker starts out as if he's going to tell a story, then catches himself and launches into an explanation of his first proverb.}---hot peppers,
sweet sugar-cane, there are these two kinds of things.

2. That is, sugar-cane on the one hand is very sweet.

3. Peppers, on the other hand, are very hot.

4. So, as they say\footnote{``As they say'' translates the quotative particle \textbf{cê}.}, ``People don't die under pepper-bushes; they die under
clumps of sugar-cane.''\footnote{I.e., misfortune often strikes when one is least expecting it.}

5. Then they say, ``Dry branches make good tinder; the young make good servants
\footnote{\textbf{cɨ} \textbf{lɛ̀ʔ}: lit. ``eat by using others to work for one''. \textbf{lɛ̀ʔ} , lit. ``lick'', is used as an informal synonym for \textbf{câ} 'eat', both as a main verb and as a post-head versatile verb meaning 'V for a living'.} for their elders.''

6. When you say, ``If you don't know what you're doing, don't go into an abandoned
old field,\footnote{\textbf{hɛ-šā}: an old field now overgrown and reverting to jungle, where animals are likely to hang out.}'' it means that bears

might come and bite you.

7. When you say ``Don't knock out [the ashes from] a clay pipe on a stone,'' it
means a piece might break off.\footnote{Cf. Eng. ``Don't build a fire in a wooden stove.''}

8. ``A fair maid should not sit next to a youth; a fair youth should not sit next
to a maid'', they say.

Speaker B (correcting him):

9. You mean ``should not sit next to a \textit{bear}!''\footnote{This is indeed the correct version of the proverb.}

Speaker A:

10. So then everybody should take these words of wisdom to heart!\footnote{Lit. ``everybody listen duratively to this.''}

