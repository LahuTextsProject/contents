\setcounter{footnote}{0}

1. P: What I'd just like to know is, in a Lahu village how old do a young man and
a girl have to be before they can get married, would you say? Is there a rule about
that sort of thing? In other tribes' villages, everywhere, if you haven't reached
a certain age you can't get married, I hear. In some places you can't take a wife
until you're twenty years old. In other places you can't take a husband until you're
sixteen. Do you have rules like that? Or else can you even get married when you're
ten years old? Can you marry at whatever [age] you please, would you say?

2. T: Well, as far as that goes, if you're ten years old you certainly can't take
a husband or a wife. Our Lahu custom is, that a woman who is fourteen or fifteen,
or a man who is seventeen or eighteen years old may marry. Besides that, there's
a rule that we Lahu have that goes like this. A man who is younger may even marry
a woman who is older, according to us. [But] if the man is older and the woman
is younger, they can't get married. That's what we say. And there's a Lahu proverb
to that effect.

3. P: \direct{changing the subject} Well, then, there isn't anybody who
would be higher-up than the headman in our Lahu villages, is there? Other people,
like the Shans -- right? -- when we were in Shan State, [there were] the so-called
\textit{pukeh}, right? The ones above the \textit{pukeh} were called \textit{pumyn},\footnote{An over-headman whose authority extends over several villages. Cf. Thai \textit{kamnan}.}
right? Those above the \textit{pumyn} were called \textit{pusheh},\footnote{A super-headman who controls the administrative districts of about five \textit{pumyn}.} isn't that
so? And there were still several kinds [of officials] above the \textit{pusheh}.
Are there such things among the Lahu?

4. T: Well, we Lahu don't really have that sort of thing. There is one thing that
we do have, though. There's one person who is higher up than all the headmen --
er, than the headman and all the villagers put together. He is not the headman.
What we call him in Lahu is ``the committee of one for our physical well-being.''\footnote{\textbf{ɔ̀-šɨ̄} \textbf{ɔ̀-šā} \textbf{phɔ̂}: lit. ``blood and flesh aspect,'' i.e. the material, physical side. \textbf{kɔ̄mītī} is a loanword from English used in a special sense. It is clear from the context that a single person is referred to. For the Lahu, `committee' can be simply a fancy word for a prestigious official.}
That person now, if there should be strife and disharmony in the village, if people
are quarreling and bickering with each other, he's the fellow that people have
to go to first. If this fellow can't decide the matter, one must go in to the headman.

5. P: So that fellow is higher-up than the headman, is that it?

6. T: If we're talking about the letter of the law,\footnote{Lit: ``if we say the rules and ways straightly.'' Actually the headman does seem to have the last word. It's just that the ``committee'' has to be consulted \textit{first}. In that sense alone he is ``higher.''} he's the person who's higher-up
than the headman.

7. P: But he doesn't have to work like the headman does.

8. T: No, he doesn't.

9. P: He's just appointed to be an important person, right?

10. T: Yes, and he must only do good things.\footnote{The implication is that he will be replaced if found to be unsatisfactory.}

11. P: How do you choose this fellow? Do you pick him the way you pick a headman?

12. T: Yes, both the headman and the villagers all have to [vote to] pick him.

13. P: \direct{changing the subject} Well, today we\footnote{Paul and the author.} have come from
the city here to this Lahu village to celebrate the New Rice Festival together,
and since we've been privileged to celebrate with you, we are really and truly
happy. We will never forget the way we have been able to celebrate [with you].
And we're very grateful indeed. But, Pastor, could you please tell us a bit about
the customs and practices connected with the Lahu New Rice Festival?

14. T: Oh, I certainly can tell you about it. The way we Christian Lahu celebrate
New Rice it's not at all the same as those pagan\footnote{\textbf{lɔ̂kì}: `pagan/heathen/gentile,' ultimately from Pali \textit{lōka-} `world', i.e. ``of this world, worldly.''} Red\footnote{\textbf{Lâhu-ní} `Red Lahu' is used somewhat disparagingly by the Black Lahu Christian villages to designate animist Lahu. It thus seems to be largely a cultural, rather than a linguistic label.} Lahu do it out there.
This is how the pagans do it. When they celebrate New Rice, they bring all their
hoes and knives and put them together in one place, and prepare food and drink,
and offer it to the hoes and knives. The reason is, that it is thanks to these
tools that we get our food and drink, so they say. But it's different with all
us Christians. We get our food and drink because God takes pity on all of us human
beings on earth, and cares for us day after day, and helps us. Furthermore, while
we live upon the earth, God blesses the land\footnote{\textbf{G̈ɨ̀-ša} \textbf{yɔ̂} \textbf{mì-gɨ̀}: \textbf{yɔ̂} may be taken either in apposition with \textbf{G̈ɨ̀-ša} or as the possessor of a genitive construction with \textbf{mì-gɨ̀}, meaning ``God blesses his earth with fertility.''} with fertility in order that our
physical natures may be nourished. And since we know it is because He moistens
the earth with rain that the rice-seeds we human beings have planted send forth
their sprouts, and because He has put His powers inside the seeds and seedpods,
we praise the grace of God. For this reason, it is a very great joyous occasion
for us all.

15. P: Oh, so when the pagans -- the ``beeswax-burners''\footnote{\textbf{pɛ̂-tú-pā} ``those who burn beeswax.'' This is a rather patronizing term for the animist Lahu, who burn beeswax candles while worshiping their spirits. The animists themselves often refer to themselves this way.} --
celebrate the New Rice, they say it's thanks to the various hoes and knives that
we get rice to eat, so they offer thanks to their tools and praise them, is that
it? But we say that we are nourished because of God -- the grace of God -- so we
praise God and give thanks to Him, and that's the difference between us, right?

16. T: Yes, that's how we're different.

17. P: Well, don't other people like the Northern Thai or the Shan have one too,
a New-Rice celebration? Haven't you ever heard whether they do or not?

18. T: Well, I myself haven't been in Thailand -- in Northern Thai country -- for
very long yet, so I'm really not sure. But my guess is that they probably don't
have the custom of celebrating the New Rice. What we do know about is the Red Lahu
and the Karen, and that's all we know. The people called ``Karen''
have similar customs. When the pagans among them\footnote{There are many thousands of Christian Karen in Thailand, many of whom have close and friendly relations with the Lahu Christians.} celebrate the New Rice, they
prepare food and drink for their hoes and knives and for the spirits and false
gods\footnote{\textbf{šá-hòʔ-šá-ná} refers to representations of animist deities. In the Bible translation this expression is used to render `idols' or `false gods'.} that they depend on, and after they've offered it to them they celebrate
the New Rice. They say that those idols of theirs help them!

19. P: Hm, I don't really understand what \textit{šá-hòʔ-šá-ná} means in
Lahu, so could you please explain its meaning to me? What do you mean by \textit{šá-hòʔ-šá-ná}?

20. T: The thing we call \textit{šá-hòʔ-šá-ná }is like this. Way back at
the time when our Lord Jesus was born on earth, he too --- that is, the man who
preached about the \textit{šá-hòʔ-šá-ná} --- was also born at that time.
This man went off to the east. He went to preach his doctrine of the \textit{šá-hòʔ-šá-ná},
while Jesus went preaching to the west. So as he went around preaching and preaching
at that time, the words this fellow taught were quite good. He would say, ``If
you want to reach the kingdom of God, do upright and righteous deeds even as I
do,'' thus he would preach. Well, after he had done his preaching, this fellow
died. When he had died, his successors, a group of people, set the fellow up as
a god, and made images and pictures\footnote{The elaborate expression \textbf{ɔ̀-hòʔ-ɔ̀-ha} `images and pictures' shares a morpheme with \textbf{šá-hòʔ-šá-ná}.} of him which they bowed down to.\footnote{\textbf{ó-qō} \textbf{pɨ} \textbf{ve}: ``incline the head (in prayer)''. Baptist Lahu use the expression \textbf{bo} \textbf{lɔ̀} \textbf{ve} "ask for grace''.} We
call them \textit{šá-hòʔ-šá-ná} because they were made by human beings.\footnote{It is hard to say where the Pastor picked up this bit of theological history. Perhaps he is influenced by the story of John the Baptist.}

21. P: Oh, I see, I see. Well, since you arrived in Thailand how many times has
it come to now that you've been celebrating the New Rice Festival, if you count
this time too?

22. T: Well, if you count this time, we've been celebrating the New Rice in Thailand
now for eleven\footnote{\textbf{tê} \textbf{chi} \textbf{lɛ} \textbf{tê} is an unusual way of saying ``eleven'', which is usually expressed simply as \textbf{tê} \textbf{chi} \textbf{tê}.} years! Eleven years.

23. P: Rather a long time now, isn't it?

24. T: Mm, quite a long time now!

25. P: Ah, it's a tremendous joy for us Lahu, isn't it, that through the grace of
God we've been able to celebrate the New Rice Festival properly year after year!

26. T: Yes, indeed.

27. P: How is it that you do this ``celebration of the New Rice''?
Is it as we saw today, that you slaughter pigs and chickens and in joy and gladness
pray to God together and eat? Is that right, do you celebrate that way every year?

28. T: Yes, that's right. Some years we have plenty to eat and drink, and we must
praise the grace of God the more.\footnote{Lit: ``some years, more than the having plenty to eat and drink, we must praise the grace of God,'' i.e., the grace of God is more to be praised than the abundance of food is.} Other years, since we Lahu hill-tribesmen
are poor and needy folk, we don't have anything to eat, no curry\footnote{\textbf{ɔ̄-chî} `that which uplifts rice'; cf. Thai \textit{kàpkhâaw}.} or meat to
go with our rice, but even so everybody contributes whatever he may have and we
kill a few chickens, we kill them and eat them together in joy and gladness and
praise the goodness of God.

29. P: So do the ``beeswax-burners'' also kill chickens and pigs
like that and cook them happily, the way we do? If we just consider the matter
of what they eat, their eating.\footnote{As opposed to their religious practices on this occasion.}

30. T: As far as what they eat goes, it's the same. But it's just that they make
their offerings differently. As for their eating, after they worship their idols,
their spirits, their hoes and knives, they all eat together.

31. P: Well, talking here with you now, I've learned what I wanted to know about
the celebration of the New Rice Festival, the way the Lahu celebrate the New Rice
Festival, and I'm really grateful to you. Thank you very much indeed. Now I still
have a little bit more that I'd like to know about the Lahu, so could I ask some
more, Pastor?

32. T: You can ask.

33. P: Pastor, as you now, people who live in the mountains like the Lahu, the Akha,
the Lisu, the Wa, the Hmong, they're called ``hill-people,'' right?
While people like the Shans, the Burmans, the Northern Thai are called ``plains-people,''
right? Well, how is it that it's turned out that there are people who live in the
plains and people who live in the mountains?

34. T: Well, it's this way. Long ago our Lahu race, in the days of our forefathers,
a time long ago, we have a Lahu legend\footnote{\textbf{chɔ-mɔ̂-khɔ̂}: ``words of the elders''.} about this. This Lahu legend of ours
goes like this. It is said that one day one of our Lahu forefathers was climbing
up a mountain. He was climbing a certain very high mountain. He packed a water-tube\footnote{\textbf{í-kâʔ-dɛ̄}: a large tube of bamboo, carried in a basket on the back, used for transporting water.}
and carried it on his back as he climbed. Well, he went on and on, and climbed
and climbed, and before he had wound his way up to the middle of the mountain,
when he reached the middle, his water gave out. For this reason, all of us Lahu
go to live on mountaintops, or so this Lahu legend has it.\footnote{The brevity of this story, as well as Càbo's hesitation in telling it, lead one to believe that it is merely a half-remembered shadow of an interesting and elaborate legend.}

35. P: You mean he got up onto the mountain and his water gave out, so he wasn't
able to come back down again?

36. T: Yes.

37. P: Oh. Well then, how is it that all these plains-dwellers like the Shans and
the Northern Thai live in the flatlands, would you say?

38. T: Well, those people now, about the Northern Thai we --- we Lahu --- don't
really know very much about such matters. Do \textit{you} know?

39. P: All that I know, you see, about the Shans and the Wa, you know\footnote{These English pause fillers ``you see'' and ``you know'' are used here to translate the Lahu final unrestricted particle \textbf{lê} `request for assent.'} -- I once
did hear something about the Wa! They say that once upon a time the Wa were plainsmen!
The Lahu, the Wa, the Akha were all people who lived in the plains, they say. I
don't really know whether it's true or not though, you see. I heard some old men
talking about it, and they said that the Wa once--that the Wa and the Lahu and
all the hill-people once lived in the plains. The Shans used to live in ``the-mountains-and-the-valleys,''\footnote{\textbf{qhɔ-qhô-lɔ̀-qhɔ̂}: ``in the mountains and valleys.'' This elaborate expression simply means `in the hills'/`in the mountains.'}
they said. But the Shans were a lot smarter! They realized that they were having
a terrible hard time living off their dry-rice fields up in the hills, so they
said to the plains-living Wa--to some Wa and Lahu, ``Say, come on up and
live in the mountains! You can earn a living with opium fields in the mountains,
you can get a living from all sorts of things, and there's a lot of silver, too.
And there are plenty of game-animals and lots of room for dry-rice fields besides.
But in the plains there's no game, there's nothing!`` they said. ``So
you plains-people have no wealth, no means of livelihood, none of these various
ways to make a living." Well, the Lahu and [the other present-day] hill-people
[like] the Wa thought this was the truth, so they changed places. They switched
with the others: ``So we shall go up and live in the mountains, and you people
come live in the plains," they're supposed to have said. So they went
and switched around! That's how the people who now live up in the hills came to
live there.\footnote{The original repeats \textbf{qhɔ-qhô-lɔ̀-qhɔ̂} instead of substituting `there.'} And the Shans and the Northern Thai and all the rest of them came
to live in the plains, and they turned out the way they did!

40. T: Oh, is that what they said?

41. P: That's right.

42. T: It was really very good of you to retell this story to us.

43. P: Somehow or other, nowadays many of the plains-people are making a fine living
from their wet-rice fields, and are doing very well in trading, so they're coming
up in the world\footnote{ɨ \textbf{la} \textbf{mu} \textbf{la}: lit. ``become big become high.''} day by day. The hill-folk cultivate swiddens. But the hill-people
are very badly off, whether from their dry-rice fields or anything else they try
to earn a living from. So no matter how many different things they think up, no
matter what plans they may think up to get ahead in the world, they have no time.\footnote{I.e., they must work constantly just to eke out their subsistence, so they have no time for experimentation in trying to improve their lot.}
And so day after day they suffer in poverty. There is no way for them to improve
themselves like other people. That's how it goes. So now one can't see how these
so-called ``hill-tribesmen'' will ever be able to find a way to
work themselves up.

44. T: Oh, that's absolutely right! These Lahu words, the things those Lahu elders
said, are absolutely right. Ah, it's disgusting to think of those Lahu ancestors
of ours!\footnote{I.e., we are revolted to think of the stupidity of our ancestors in coming to live up in the mountains.}

45. P: Yes, yes, yes. These Shans, they used to live up in the mountains, and they
had a wretched living from their dry-rice fields and from everything else. But
these people like the Lahu and the Wa and the Akha had no education. The Shans
and the Northern Thai had oily tongues,\footnote{\textbf{hē-câ-ve}: `to earn a living by lying; live by one's oily tongue.'} so the others\footnote{The original just has \textbf{yɔ̂-hɨ} `they.' The speaker is taking the deictic viewpoint of an outsider, speaking of his ancestors as `they'. It would have been clearer if he had said \textbf{ŋà-hɨ} `we'.} went flying up into
the mountains! They switched around, they exchanged their lands. They changed their
habitat. Now they're very clever, more than we are. Oh, in all respects! They're
coming up in the world, I tell you.

46. T: Well, then, I for one don't know what has to be done now for our Lahu people
to find a way to improve themselves. Do you know the answer to that, too?\footnote{``do you also know this matter?''}

47. P: Well, as far as I can see, in order for the hill-people to raise their standard
of living I don't even say they've got to go live in the plains. Even up in the
mountains there are certain places where you can make wet-rice fields--live off
wet-rice fields--or off fruit-orchards of various kinds. So if you go about it
properly I think you could probably find a way to prosper like other people, like
the people living in the plains. Then too the Lahu just\footnote{\textbf{a-cí} `a little, just' is used to soften these hard truths.} don't have any education,
they just haven't managed to get schooling like other people. So since they don't
understand the ways and means of getting ahead, why, I think there's probably no
way they can actually do it. But in the future if the Lahu also come to have education
and training, if in their hearts they come to desire it, to want to seek it, then
someday I suppose it will be possible. However, if I were to say what would really
be best, I think that the best thing would be to go live in the plains and live
off wet-rice fields or anything else. It would certainly be easier!

48. T: Well, so we should probably go look at the places where there is flat land
for wet-rice fields.

49. P: Yes, yes, everybody ought to go and look, each one of you. As we've come
to realize, even if we live and work for a living in the hills for a hundred years,
or for a thousand years or more, there will never be a year when we'll get to see
any improvement in our lot, right? It's that way the first year. It's that way
the second year. It's that way for a hundred years. That's just the way it is!
The plains-people over there from one year to the next get to ride in cars, and
after the cars come airplanes, and when the airplanes come so do their big beautiful
houses--all kinds of things happen for them. That's just the way it is! On the
intellectual side too they're making progress! So if we think it over now, and
manage to live in the plains as we should, I really think we'll probably begin
to see the path to progress.

50. T: Ah, they're absolutely right, these words of yours.

