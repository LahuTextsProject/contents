
102 The Stupid Son-in-law Beats a Benevolent Bonze

1. Once upon a time there was a certain person.

2. He used to go and set traps for animals.

3. One day when he was setting traps, an old Buddhist monk \footnote{phàʔ-tū-mɔ̂: 1st syllable < Tai (cf. Si. \textbf{phrÁʔ} `lord; sacred thing')} who had gone there
to pick sticky berries \footnote{qhɔ-mêʔ-šī: lit. ``mountain eye'', I kind of sticky tart whitish berry. See DL:302.} to eat caught sight of a barking deer \footnote{chɨ-kɛ-mɛ, lit. ``dwarf deer.'' This name for the barking deer is especially characteristic of YL. The usual BL form is chɨ-pí-qwɛ̀ʔ. Later in the text the variant chɨ-qwɛ̀ʔ occurs.} that had been
trapped .\footnote{This sentence was somewhat garbled by the speaker: phàʔ-tū-mɔ̂ tê g̈â qhɔ-mɛ̂ʔ-šī ca g̈ɔ̂ʔ lɛ̀ʔ qay lɛ, [phàʔ-tū-mɔ̂-ló tê mà---èe---chɨ-kɛ-nɛ yɔ àʔ pî ve yò]. The portion in brackets has been edited to: chɨ-kɛ-nɛ tê khɛ yɔ àʔ pî ve thàʔ g̈a mɔ̀ ve yò.}

4. When he saw the trapped barking deer, he set it free.

5. After this, he [the trap-setter] went home.

6. When he got home he said to his father-in-law, ``Granddad \footnote{ɔ̀-pū (voc. à-pū) means either `grandfather' or `father-in-law', according to context. Vocatively it seems best to translate it as ``grand-dad'' or ``gramps,'' even when `father-in-law' is understood.}, we must have
trapped somebody else's calf!'' \footnote{The stupid guy is making two mistakes: (a) he takes the barking deer for a calf; (b) he let the monk free it because he thought it must have belonged to somebody.}

7. So then he went off again [to set traps].

8. Once again he trapped a barking deer.

9. And again it was set free.

10. ``Granddad, I just keep trapping somebody else's calf, and it's been set free!'',
he said.

11a. This time the father-in-law said, ``Granddad \footnote{Here the father-in-law uses the kin term as a 1p pronoun.} tells you [to say], `No matter
what it is, it's my grand-dad's kitty-cat!'''\footnote{The father-in-law, realizing that his son-in-law is not the sharpest knife in the drawer, uses the baby-talk word \textbf{mɛ́-mɛ́} `kitty; pussy' (cf. \textbf{mɛ́-ni} `cat') as a general term for `animal'.}

11b. ``Hurry and bring it [home]!'', he said.

12. So the next time that the senior monk went to pick sticky berries to eat, he
was waiting for him.

13. So when he saw him, he cut down a cudgel [from a tree] and beat him with it.

14. While he was beating him, [the monk] said ``\textbf{khɔ́ tə̄, khɔ́ tə̄}!''\footnote{Shan or N. Thai \textbf{khɔ́ }`beg, supplicate'; \textbf{tə̄} is\textbf{ }apparently a verb particle expressing an urgent request.}

15. ``\textbf{Khɔ́ tə̄} or no \textbf{Khɔ́ tə̄},\textbf{ }it's my old man's
kitty-cat!'' he said. \footnote{The son-in-law treats the foreign phrase \textbf{khɔ́ tə̄ }as a verb compound, negating it in the frame V1 + mâ + V1 + thɔ̂ `whether V1 or not'. This sentence conveys a highly ridiculous impression (which the translation attempts to convey), with its juxtaposition of a foreign phrase and a baby-talk word.\textbf{ }}

