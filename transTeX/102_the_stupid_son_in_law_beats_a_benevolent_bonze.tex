\setcounter{footnote}{0}

1. Once upon a time there was a certain stupid person.

2. He used to go and set traps for animals.

3. One day when he was setting traps, an old Buddhist monk\footnote{\textbf{phàʔ-tō-mɔ̂}: 1st syllable < Tai (cf. Si. \textbf{phráʔ} `lord; sacred thing'). DL:888.} who had gone there
to pick

sticky berries\footnote{\textbf{qhɔ-mêʔ-šī}: lit. ``mountain eye'', a kind of sticky tart whitish berry. See DL:302.} to eat caught sight of a barking deer\footnote{\textbf{chɨ-kɛ-nɛ}, lit. ``dwarf deer.'' This name for the barking deer is especially characteristic of Yellow Lahu. The usual BL form is \textbf{chɨ-pí-qwɛ̀ʔ}. Later in the text the variant \textbf{chɨ-qwɛ̀ʔ} occurs.} that had been trapped.

4. When he saw the trapped barking deer, he set it free.

5. After this, he [the trap-setter] went home.

6. When he got home he said to his father-in-law, ``Granddad\footnote{\textbf{ɔ̀-pū} (voc. \textbf{à-pū}) means either `grandfather' or `father-in-law', according to context. Vocatively it often seems best to translate it as 'grand-dad', even when `father-in-law' is understood.}, we must have
trapped

somebody else's calf!''\footnote{The stupid guy is making two mistakes: (a) he takes the barking deer for a calf; (b) he was not angry that the monk freed it, because he thought it must have belonged to somebody.}

7. So then he went back again [to set traps].

8. Once again he trapped a barking deer.

9. And again it was set free.

10. ``Granddad, I just keep trapping somebody else's calf, and it's been set free!'',
he said.

11a. This time the father-in-law said, ``Granddad\footnote{Here the father-in-law uses the kin term as a 1st person pronoun.} tells you [to say], `Whatever
it is,

it's my grand-dad's pussy-cat!'''\footnote{The father-in-law, realizing that his son-in-law is not the sharpest knife in the drawer, uses the baby-talk word \textbf{mɛ́-mɛ́} `kitty-cat; pussycat' as a general term for `animal'. The usual BL word for 'cat' is \textbf{mɛ́-ni}.}

11b. ``Hurry and bring it [home]!'', he said.

12. So the stupid guy went back again, and when that senior monk was picking sticky
berries to

eat, this time he was waiting for him.

13. So when he saw him, he cut down a cudgel [from a tree] and beat him with it.

14. While he was beating him, [the monk] said ``\textbf{khɔ́ tə̄, khɔ́ tə̄}!''\footnote{Shan or N. Thai \textbf{khɔ́} `beg, supplicate'; \textbf{tə̄} is apparently a verb particle expressing an urgent request.}

15. ``\textbf{Khɔ́ tə̄} or no \textbf{Khɔ́ tə̄},\textbf{ }it's my old man's
kitty-cat!'' he said.\footnote{The son-in-law treats the foreign phrase \textbf{khɔ́} \textbf{tə̄} as a verb compound, negating it in the frame V1 + \textbf{mâ} + V1 + \textbf{thɔ̂} `whether V1 or not'. This sentence conveys a highly ridiculous impression with its juxtaposition of a foreign phrase and a baby-talk word.}

