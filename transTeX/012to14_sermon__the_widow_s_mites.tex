\setcounter{footnote}{0}


\textbf{12}

1. My brethren, the story I'd like to teach you and have us reflect upon together,
is about a group of rich men who

made offerings to God and of a certain widow who offered money to God.

2. This story which I would like to impart to you, my brothers, is to be found in
Mark 12:41-44.\footnote{Lit: ``Mark big-number 12, small-number from 41 to 44.''}

3. A second passage is to be found in Luke 6:38.

4. And a third passage is contained in II Corinthians 9:6-7.

5. This book of the Bible\footnote{I.e., \textit{Mark}.} has many passages which are full of meaning.\footnote{\textbf{ɔ̀-ti} \textbf{cɨ́-kɨ̀} `meaningful passages': \textbf{cɨ́-kɨ̀} lit: ``sticking place": `meaning, interest, value, worth'.}

6. So please listen carefully to this story about offering money in the temple.

7. When Jesus was living on the earth, there was a certain very poor widow in the
city of Jerusalem.

8. Her husband had died long before.

9. She had to seek her own living, and though she did her utmost, [her] money was
never enough to gain her food and drink [or] to clothe and garb herself, so she
would go hungry and suffer.

10. This woman, despite the fact that she was so wretched, had not forgotten God.

11. She would go up to the temple\footnote{\textbf{bo-yɛ̀} lit: ``merit-house; grace-house'': in Christian villages used to mean `church' or `temple'; in animist villages it referred to a temple to G'uisha, a deity conceived to be far above the nature-spirits (\textbf{nê}). See DL:943, 1148-9. G'uisha (\textbf{g̈ɨ̀-ša}) was reinterpreted as the God of the Bible by Christian missionaries. \textbf{bo-yɛ̀} \textbf{tâʔ} \textbf{ve} lit: ``go up to church'': `go to church'.} as was fitting, and not only did she praise
Him, but every day she would pray\footnote{\textbf{bo} \textbf{lɔ̀} \textbf{ve} lit: ``beg for grace'': `pray'} fervently\footnote{\textbf{qha-dɛ̀ʔ} An all-purpose adverbial expression signifying `action performed in the approved manner': ``nicely, properly, as is fitting, very well, fervently, etc.''} to God.

12. One day this widow went to the great Temple to contribute and donate an offering
of money.

13. For her offering she could not give a great deal of money, but she took all
the money she did have and went to offer

it to Him.

14. When she arrived at the Temple, she saw a group of rich men\footnote{\textbf{pɔ-ša-pa} lit: ``those who are born easily.'' Cf. English \textit{well-born}.} putting great
sums of money into the offering-box, and her heart was downcast.

15. She wanted to offer lots of money just like them, but she didn't have anything
to give.

16. She only had two small coins.\footnote{\textbf{pɛ̀ʔ-šā} < Burmese \textit{pai'-hsaN} (WB \textit{puik-cham}) < Bengali.' See DL:856.} Those two small coins were only worth a farthing
apiece.\footnote{\textbf{tê} \textbf{pɛ̀ʔ}. A \textbf{pɛ̀ʔ} is 1/100 of a Burmese rupee.}

17. When the woman put the two small coins in the offering-box--into the offering-box--Jesus
saw it.

18. He also saw the group of rich men offering great sums of money.

19. Then he said to his disciples:

20. ``More than all the people who have put [money] into the offering-box--this
poor and wretched widow has put in the most of all."

[\textbf{13}/ Hymn. After the hymn is sung, the pastor continues: ]

\textbf{14 }

21. When Jesus was living upon the earth, in the city of Jerusalem, in the great
Temple, all the Pharisees who had

much earthly wealth\footnote{\textbf{mì-gɨ̀} \textbf{qhô} \textbf{phu-ši} \textbf{cɔ̀} \textbf{mâ}: ``had much gold and silver on the earth."}, and all the High Priestly Pharisees, and all the rich
people and tycoons\footnote{\textbf{šathê} `rich man; boss; tycoon' < Shan < Burmese < Pali \textit{sathi/sethi} < Skt. \textit{shre(ṣṭha)} `most splendid; preeminent'. See \#69 ``Picking Tea'' and DL:1157.} were offering

great sums of money to God.

22. At this time there was a very poor and wretched woman.

23. Since her husband had died, she was obliged to earn her living all by herself:
everything, whether it was her money

and belongings, or her clothing, or her food and drink.

24. So when the moment for putting money into the offering-place came, she had nothing
to offer.

25. She wanted to make an offering to God, but she had nothing to offer.

26. The other people, all the tycoons coming up [to the temple] in their horse-drawn
wagons pouring money into the

offering-box until it overflowed--this woman saw them.

27. When she saw this, how could she feel?

28. ``These others can offer so very, very much to God. I cannot offer God
anything.

29. ``I can only give this tiny amount, and my heart is sore indeed.''

30. This is what the woman was thinking.

31. As she thought this she was filled with embarrassment.

32. She had no idea\footnote{\textbf{dɔ̂} \textbf{ma} \textbf{tɔ̂ʔ}: ``think not emerge'', i.e., be unable to think something through to a solution.} how she could go and put anything into the offering-place.

33. She had no idea how she could go put anything into the offering-box.

34. There was a great throng there, thousands and tens of thousands of people.

35. Slowly she carried her money, hiding it [in her hand], and went to put it into
the offering-place.

36. How much was it that she put in?

37. Only two farthings.

38. She only had two farthings. For herself. To put in there.

39. As Jesus was sitting there he saw what was going on.

40. Then he said to his disciples, ``As far as you can see\footnote{\textbf{nɔ̀-hɨ} \textbf{ni} \textbf{ve} \textbf{tê} \textbf{yâ} \textbf{thâ}: ``while you are looking''.}, who is
the single person who has contributed the very most

money among all those who made offerings to God today?

41. ``Who has put in more than anybody else?''

42. What did the disciples say?

43. This is how the disciples saw things:

44. All the disciples had their attention fixed on a certain person who had taken
an enormous amount of money and put it

into God's offering-box. They thought he had put in a tremendous amount.

45. ``[But] what is that woman doing giving a couple of farthings to God
-- putting them into His offering-place?"

46. This is what the disciples were thinking.

47. What did Jesus say?

48. ``So\footnote{'So' translates the quotative final unrestricted particle \textbf{cê}: ``you say that's what you think''.} that's what you think,'' he said.

49. ``Those people put in a lot of money. That woman put in a little bit
of money.

50. ``As far as her working for God's sake\footnote{\textbf{g̈ɨ̀-ša} \textbf{ɔ̀-po} \textbf{ká} \textbf{te} \textbf{ve} ``to work for God's sake.'' An expression used by the Christian Lahu to mean `to give money in church.'} goes, she has offered God
very little: that's what you all think.

51. ``But this woman does not have the slightest bit of substance\footnote{\textbf{phu-ši-mɔ̂-jɛ̀} ``wealth and property.''} to
keep herself alive.

52. ``She only had two farthings.

53. ``She took everything she had and offered it to God.

54. ``The one who is the greatest of all these people, the one who has put
in the largest offering, is this woman" , thus

did Jesus teach his disciples.

55. This means that among you Christians, when it is time to make an offering to
God, there should be no

embarrassment [such that you say to yourselves]: ``I have so little that
I cannot offer God anything."

56. We ought to make offerings to God in accordance with our means.\footnote{\textbf{ca} \textbf{ve} \textbf{g̈àʔ} \textbf{ve} \textbf{g̈a} \textbf{ve} \textbf{qha-šu-šu}: ~"just as we have earned through our seeking and pursuing". See DL:433.}

57. Oh, my brothers! Let everyone give to God to the best of his ability!

58. Because, you see, your body, your intelligence, your wealth and property, your
pigs and chickens, your

livestock, all of these things belong to God.

59. My body does not belong to me.

60. Everything belongs to God.

61. For this reason we should make offerings to God.

62. For this reason, when we spend our money, when we give it away on the outside
\footnote{\textbf{ɔ̀-bà-phɔ̂}: i.e., outside our village.} to those who go and preach to non-Christians\footnote{\textbf{lɔ̂-kì-yâ}: `people in the present mortal world' (\textbf{lɔ̂-kì} < Shan < Pali < Skt. \textit{loka} `world'), i.e., `gentiles' in the Jewish or Mormon sense.}, we should make sure\footnote{\textbf{g̈a} \textbf{ši} \textbf{ve} `must know.'} that this money is doing God's work.

63. Therefore, my Christian brethren, may the doing of the Lord's work ever remain
in our hearts.

64. This is how I encourage you all.

