
77a Wooing the Maidens (Part II)

1. Somebody:  ao⁴⁴ meɛ⁵⁵ yauʔ⁵⁵ nɛʔ\ensuremath{^2}\ensuremath{^1}  maɨ⁵⁵\footnote{This sentence was uttered in Shan with a Lahu accent by someone fooling around. The sentence is transcribed and glossed as in my field notebook (1966). [Check: is the last word really `you' or is it a yes-no qst. part. like Thai \textit{mǎi}?]}

take wife already PRT/or? you?

Are you married yet?

2. T (Cà-bo): Well, we can get started now, we can do it now.

3. Yâ-pā-ɛ́\footnote{Yâ-pā-ɛ́ (``Sonny'') was an unmarried teenager when this was recorded. He was the chief consultant during my 1977 fieldtrip.}: <Selections on the jewsharp>

4. T: Ah, that sounds really great! Hmm --- maybe it wasn't recorded.\footnote{\textit{kh}ɔ̂ `voice, sound, noise, words' is used in two senses in \#4: ɔ̀\textit{-kh}ɔ̂\textit{ dà}ʔ\textit{ jâ} `the sound is very good', then \textit{â c}ɔ̀\textit{ ve kh}ɔ̂ `probably doesn't have/not there' (i.e., probably not recorded), where \textit{kh}ɔ̂ means `the probable reason why'. See DL pp. 380-383.}

5. P (Cà-lɔ̂): [IN ENGLISH] You have rubbed out ---

6. Thû-yì\footnote{Thû-yì, another young man who was a good friend in 1965-6, died soon after, reputedly because of black magic. See DL: 681.}: <Jewsharp selection>

7. T: Wow, that sounds great!

8. H: Well, the way we Lahu use the jewsharp \footnote{Now often euphemistically called ``jawharp'' in English. For details on this Lahu musical instrument see DL: 85 and Plate 2.}, it's for the young men and marriageable
girls \footnote{chɔ-hÁ `young man, bachelor'; yâ-mî-hÁ `marriageable young woman'} to use!

9. This blowing of the jewsharp is a language for courting the girls!

10. When we call this Lahu custom of ours \textit{nā-kû-qÁ nā-kû-qō-lò}ʔ
that's just \{the sound/onomatopoeia\}.\footnote{These seven syllables are meant to imitate the twanging of the jewsharp, as the Headman says. But some people etymologize the syllables \textit{nā-kû }as \textit{na kù} (\textit{na} Bn `morpheme in female names'), \textit{kù} (V) `call', i.e. ``girl-calling''.}

11. If you just say \textit{nā-kû-qÁ} it means telling the girls `Come!'

12. The meaning is, ``Oh, Lahu maid,\footnote{Lâhū-ma òʔ: a formulaic term of address in courtship.  VOC} whether I'm worthy or not,\footnote{qhâ mâ qhâ thɔ̂: ``worthy-not-worthy-even''. qhâ (Vadj) `have good qualities'.   Vadj NEG Vadj Punf} do come!
Let's try to have a conversation!''

13. Otherwise, unless we go around blowing the jewsharp, unless we go and do that,
no matter what our guys say, they <the girls> won't listen.

14. All around the village, at night\footnote{tâ-khɨ̂ `at night' < Shan (cf. Si. thÁŋ khyyn) is sometimes used instead of native Lahu mû phə̀ʔ thâ (lit. ``when the heavens are revealed''). See DL: 998.} when you can't see anything, even if we
just go walking around,\footnote{I.e., without going into any girl's house.} when the sound of the jewsharp is heard \{twanging/buzzing\}
in the distance,\footnote{The intensifying particle qha-pâʔ is used in situations where hearing or not hearing is the issue. (See DL: 272).} all the girls say, ``Oh, the guys \{are coming/are here/have
come\}!'', and they want to come <out of their houses.>

15. When a girl comes, they can \{talk/hold a conversation\} with each other.

16. This custom, this blowing the jewsharp, if an older person should hear it,
would be embarrassing.

17. But one doesn't blow it in \{the/a\} house.

18. At night, when it's dark, when you're outside the house sitting on the ground
chatting with the girl, that's when you blow for her.

19. A girl --- when a girl hears it, she knows ``Ah, I hear\footnote{ni-qhâ šɨ ve: lit. ``heart's path is stable''.} the guys coming!'',
<that's what> she'll think in her heart.

20. If she wants to flirt with the guy, even if she had been doing some work she'll
put it aside, and say ``I'll just go outside for a while, I'll just take a poop.''

21. So, since she wants to flirt with the guy, well, she just goes off to flirt
with him.

22. In the deep woods, at night even though people are sleeping, they're not sleeping
yet.

23. So when they hear the sound of a guy blowing the jewsharp, it's very melancholy,
and they converse with each other to \{his/her/their\} heart's content.\footnote{ni-qhâ šɨ ve: lit. ``heart's path is stable''.}

24. There are even some people who marry each other because of this jewsharp blowing.\footnote{Á-thâ mêʔ dàʔ lɛ: The reciprocal Pv dàʔ here indicates that the jewsharp blowing is a mutual activity, involving the listener as well as the blower.   N  V  Pv  Punf}

25. Among us Lahu young men and women, the sound of some of them blowing the jewsharp
is very pretty, they can blow pretty well.

26. <But> some others, seeing other people doing it, just fool
around when they do it.

27. They don't know the meanings or the sounds.

28. Among us Lahu, there are even some people, some girls, who even if the guy
is quite ugly, if he can do various things like play the sideways flute,\footnote{lɛ́-tû(-qâ): single-tubed wind instrument, held sideways like a Western transverse flute.} sing,
blow the gourd-flute,\footnote{nɔ̄: musical instrument with a gourd as resonating chamber, to which five bamboo tubes are attached with beeswax. See DL: 794 and PLATE \#42.} make songs, blow the jewsharp, then that's just who they
want to marry.

29. Some of them <girls>, even if the guy has nothing, even if
he's ugly, if he has just one thing, a possession that she would like to have,
a thing that the girl wants, then she'll marry him.

30. There are also people who get married even if they don't want to have the <guy's>
body, but just want the thing <he has>.

31. We Lahu have customs like this.

