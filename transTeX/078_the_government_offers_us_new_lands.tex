x\setcounter{footnote}{0}

1. H: Well, a message from the government has come to us\footnote{Lit. ``has arrived on top of us''.}, to all of us Lahu.
We're going to discuss it. Let everybody come here for a while, please. The matter
which we're all going to be discussing this evening is this. There's talk that
they're planning to give us all--all of us [Lahu] who are living in Thailand--lands,
fields, places where we can cultivate the soil, beginning right now and forever
after.\footnote{\textbf{yàʔ-ni} \textbf{tà} \textbf{lɛ} \textbf{ɔ̀-g̈û-šɨ} \textbf{qay} \textbf{qo} \textbf{ɔ̄}: ``beginning today and going on ahead''.} But how is this going to come about?

2. T: I'm tired of hearing this all the time. It's always the same old story.

3. H: Well that's where you're wrong! I tell you they're saying that they're going
to give us land that we can live off forever in the future, for our own adult generation
as well as our children's generation! They say they'll give each person about twenty
rai.\footnote{A Thai measure of area, equal to about 1600 square meters (2.53 \textit{rai } equals 1 acre). The government had been making sporadic attempts to stamp out slash-and-burn agriculture on the mountain sides and to encourage sedentary cultivation of wet-rice.} What shall we do [about it]? Shall we go take a look at the places?

4. T: Where the hell are these places anyway?

5. H: Well,

5. -A T: It's not just in one place.

5. -B H: I've heard that it's someplace over at Pha-pu-jawn,\footnote{A Hmong village near Huey Tat.} near where the Hmong
are. Go over there and have a look, they said to me!

6. T: Didn't somebody say that they've parceled that out to the Hmong already?

7. H: The Hmong won't get all of it. They say they'll give us Lahu our share, too!

8. T: With the business of giving some to the Hmong and some to the Lahu there's
only going to be quarreling back and forth.

9. H: Oh, they're going to separate off each person's land for him, they say. They'll
measure it off for us. So who all is going to go, to have a look at it?

10. T: Well, I for one am not going. From as far back as I can remember until
today I have never lived off a wet-rice field.\footnote{\textbf{ti-mi}: an irrigated paddy-field, cultivated in the lowlands. Opposed to \textbf{hɛ} (꞊ \textbf{yɛ̂-hɛ}, \textbf{hɛ-ŋə̂}): a dry-rice field or swidden in the mountains.} I've grown up working in dry
mountain-fields, and that's all. There's malaria [down there in the plains]! I'm
afraid of it.\footnote{This was a very real fear before the Thai government, working together with UNESCO, instituted an extremely effective anti-malaria campaign in the 1950's.}

11. H: Well, since they're giving out the land, let's go take a look at it. That's
the least we can do.\footnote{This idea is expressed by the verb particle \textbf{thɔ̂} `even so/anyway,' used here as an afterthought sentence all by itself.} Even if we don't cultivate it right away, we still have
no way of knowing whether we might not gradually have to use it in the future.
They won't be making this offer forever.\footnote{Lit. ``They're only giving at the time they're giving.''} If there comes a time when they're
not giving it out anymore, even if we want it we won't get it. We really ought
to go out and have a look at it, brothers!

12. T: You lead them there then. Take Ja-keh and his group and Ja-pho and his
group.

13. H: Oh, they'll go, they'll go. One or two of them will go. You should go early
tomorrow morning, after you have something to eat.\footnote{This point is stressed repeatedly, since the usual Lahu practice is not to have the morning meal until about 11 A.M.} The official\footnote{Lit: \textbf{chɔ-ɨ̄} ``big person''.} said that
he'd go with us too tomorrow morning, after we eat. He's going to wait for us.
So after we eat--all of you who want to go, everybody who's going, have something
to eat and go up\footnote{The Thai officials nominally charged with administering Huey Tat live in the \textit{nikhom } offices about a 20-minute walk up the mountain from the village. A \textit{nikhom } (Thai \textit{níkhom, }Lahu \textbf{nîʔkhô}) is a hill-tribe resettlement center established by the government.} and fetch the leader, then go off together and have a look.
Personally I don't think I'll go along yet. Because I don't have time to go. Right
now I'm just swamped with work,\footnote{\textbf{ká} \textbf{kàʔ} \textbf{ká} \textbf{kì} \textbf{jâ}: `as for work, [I'm] very busy working'.} you see.

14. Ty: For crying out loud, you're the headman. You have to go! Otherwise
how can you look after the villagers' interests?

15. H: Well, I'm not one who's getting any [of that land] yet. It's not there that
I'm getting any [land], so I won't go yet. I'm not going there to look. But all
of you who do want to get some, you go and look it over!

16. T: Well, when the headman informs the villagers about it this way but doesn't
go himself, I daresay nobody's going to pay any attention to him, in the future.

17. H: For my part, I don't want to get\footnote{\textbf{hə̂ʔ} `get' is sometimes used for \textbf{hə̂ʔ} \textbf{gâ} `want to get.'} any [land] over there because it's too
far away. I'm going to try to get some that's near here. But all of you who want
to get some, go. If you don't want any ---

18. T: What you're saying is really very selfish. If it's a place nearby you'd
like to get it for yourself. But other people you'd like to send off to places
far away. If you act like that how can this village prosper?\footnote{\textbf{qhâʔ} \textbf{phɛ̀ʔ} \textbf{ve}: `to be a village', i.e., to be a unified and prosperous community.}

19. H: No, whether it's far or near, if it suits you they'll give it to you--no
matter where it is and no matter how much you want. I'm not saying I'll take it
all for myself! Whether it's good land or bad land they'll share it out equally,
the same to everybody. That's how it's done, when the authorities give things out.

20. Ty: Well, in that case you certainly ought to go look at it--tomorrow,
after you have something to eat.

21. H: But I don't want any of it ever at Pha-pu-jaw--really I don't!

22. T: Just look at him, just look at him! With his own mouth he says he doesn't
want any.

23. H: Well then, where would \textit{you } like to get land, eh? Someplace nice
and close, right on the \textit{nikhom }'s land, not more than a furlong away
from Huey Tat, I suppose, right? Well, I [also] would like to get [some land] right
around here. If it's too far away I just can't afford to go and work there. There's
nobody [to help me].\footnote{\textbf{chɔ} \textbf{â} \textbf{cɔ̀}: lit. ``there are no people'', i.e., there are no other men in my family.} I don't have any children, you know.

24. Ty: You won't find any place to make a paddy-field around these parts!
Except for Mɛ-thàʔ-lây\footnote{A relatively flat area at the base of the mountain where Huey Tat is situated. It was already mostly under cultivation by Thai farmers.}; that's the only place.

25. H: No, if you go over around Mì-g̈â-tù, on the way to the cattle and elephant
grazing place,\footnote{A place near Huey Tat, to which animals used to return after a day's grazing. Perhaps this was once a locale for a livestock market.} there's a great big plain! If we can only get water it'll be
a perfect place. Go look around there.

26. T: Cultivating paddy fields in a desert where there's no water and earning
a living! Maybe your father could, I suppose!\footnote{\textbf{nɔ̀} \textbf{pa} \textbf{tí} \textbf{yò}: ``your father's the only one.'' A sarcastic expression meaning that something is impossible.}

27. H: Come now! Up above the valley there's a stream called Stony Brook\footnote{\textbf{Há-cú-pɨ}: lit. ``a place strewn with rocks''.} --
now if you could just go and dig a channel [from it to the fields] there'd be all
[the water] that you need for cultivation! I guess you haven't been over there
to take a proper look yet.

28. T: Without spending about ten thousand\footnote{This figure is in baht, equivalent to about \$500 at the time, which was an inconceivably large sum of money for a Lahu.}, you wouldn't be able to dig
the channel.

29. H: No, as long as we've found\footnote{\textbf{ca} \textbf{ā}: lit. ``have sought'', i.e., have found.} a place that we want to have, whatever has
to be spent or whatever has to be done the government will help us. It's a trifling
matter for them.\footnote{\textbf{šu} \textbf{àʔ} \textbf{â} \textbf{khə̀ʔ} \textbf{nē}: `it doesn't trouble them.'}

30. T: Yes, a long time ago, the government was already telling us ``We'll
help you, we'll help you!'', but up to today I don't even see a mouthful of tea
[that we've gotten out of them].

31. H: The government says that they can do whatever they set their minds to do.

32. T: Two or three pigs, fifty or sixty big coffee-plants--how can anybody
earn a living from that sort of thing!\footnote{Càbo is referring to the government's occasional attempts to enrich and diversify the village's agriculture by making little presents of plants or livestock.}

33. H: Anyhow, whether we do it or not is up to us --- since they're offering it.
We are backwoods people.\footnote{\textbf{hɛ̂ʔ-pɨ́-qhɔ} \textbf{chɔ}: `jungle people; forest people.'} We're a tiny\footnote{\textbf{i}: `small', used to mean both `few, not numerous' and `weak, unimportant.'} race of people. We ought to listen
carefully to what the government says. Whatever they tell us we should do. As we
go forward in the future, if each of us doesn't have his own piece of land here
in Thailand, our children and our posterity will suffer for it.\footnote{Lit: ``when the era of our children and posterity arrives, it will be difficult."} Those of us
who are grown-ups really ought to ponder this very carefully now. In the past too,
in this Lahu nation of ours, for generation after generation people have been thoughtless,\footnote{\textbf{mâ} \textbf{dɔ̂-ni} \textbf{pɨ́}: lit. ``unable to think things through.''}
so that today whatever we do we are beset with troubles and difficulties.\footnote{This long speech is in elevated, sermonizing style, as the headman speaks \textit{ex cathedra }.} We
have no education, none of the various skills that other people have. It is because
we don't listen to the advice of others that we have come to this pass. So at this
point we should all think things over very carefully, my brothers! Liši, where
would you like to get land, if they offer you some?

33a. Liši: Wherever they give it to me, I will accept it.

33b. H: Well, that's it, I feel the same as you do. [But] there isn't anybody who
wants to go to Pha-pu-jaw that we were talking about just now! And you're telling
me, ``Go! Since you're the headman!'' So what shall I do?

34. T: Since you're the headman, you really must go.

35. H: I don't want to get any [land there], I tell you! Even if I go there I'm
not taking any.

36. T: But you must go and look for [land] on the villagers' behalf even though
you don't want any for yourself.

37. H: If the villagers want it, they'll get it.

38. T: You've got to go look after things yourself. If you don't go and watch
how they're measuring it off, how will the villagers know whether it's being done
fairly or not?

39. H: Well, once they're all through measuring it properly, I do plan to go, you
know.

40. T: When it comes to doing a job, this kind of piecemeal, half-hearted\footnote{\textbf{tê} \textbf{tɔn-nɛ́} \textbf{tê} \textbf{tɔn-nɛ́} \textbf{te} \textbf{ve}: ``to do just a little at a time''.}
way of doing it is really annoying! Year after year [this trait of yours] hasn't
changed,\footnote{\textbf{â} \textbf{mɛ́} \textbf{šē}: ``hasn't disappeared yet''.} right up to today. It's time for you to realize this.

41. H: Even if you say you want [a certain piece of land] it doesn't mean that you
have to cultivate it right away. If there isn't anybody to work it at the moment,\footnote{E.g, if your sons are still too young.}
you can just keep it in reserve\footnote{\textbf{fá} \textbf{ve}: undoubtedly < Thai \textit{fàak} `deposit for future use'..} for the future! We'd better take advantage
of their offer while it still holds.\footnote{``Because their offering it is only offering it for a time.'' \textbf{ɔ̀-yâ} \textbf{thâ} \textbf{cɛ} \textbf{tí} `only temporarily.' A certain inconsistency in the headman's attitude is obvious, amusing, and intentional!} So all of you who have the time ought
to go tomorrow! Otherwise you might say someday,

``You didn't tell me about this.'' ``Even though the government said they'd give
us land, you didn't tell me.'' ``If you had only told me when they said they'd
give it to us!'' ``If we had only known about it, we would have gone!'', you might
say, so now I'm telling you about it while there's still time, [I'm telling] you
all.\footnote{This passage, which has been slightly edited, was difficult because of the constant shifting in the headman's deictic point of view vis-à-vis the villagers. Sometimes he called himself \textbf{ŋà} `I', sometimes he referred to himself as \textbf{nɔ̀} `you', through the villagers' eyes. Sometimes the villagers are \textbf{nɔ̀-hɨ} `you all', sometimes \textbf{ŋà-hɨ} `we'.}

42. T: Yes, well, we've known about this for a long time, this whole business.
However, I don't see any advantage or benefit\footnote{\textbf{g̈a-câ-g̈a-lɛ̀ʔ} \textbf{tù}: lit. ``something which will enable us to eat''.} for us if we do work [these new
fields]. I can only see difficulties. Rather than listening to that kind of advice,
it wouldn't hurt\footnote{\textbf{qhɛ} is a verb meaning `be objectionable, harmful', perhaps related to the verb particle \textbf{qhɛ} `V to excess'.} if we paid attention to what the Boss\footnote{\textbf{šathê}: < Burmese \textit{$\theta$əthêi } `tycoon' (WB \textit{su-ṭhê, }with first syllable apparently remodeled under the influence of su `person' [cf. Lahu \textbf{šu} `3p pronoun'], ultimately from Sanskrit \textit{śre(ṣṭha) } `most splendid, preeminent'). This refers to the rich Thai landowner who had adopted a semi-benevolent paternalistic attitude toward the Lahu in the Huey Tat \textit{nikhom }. See \textit{How we came from Burma}.} says.

43. H: In the future, when Thailand, this Thailand of ours gets to be packed chock-full
of people, whoever doesn't have his own land is going to be in trouble, I bet.
Now, while they're telling us this in good time, we all ought to think it over
carefully. They're only going to be offering it to us for a while. If we don't
watch out ``By the time our thoughts catch up, the prey can't be caught!''\footnote{\textbf{dɔ̂} \textbf{mi} \textbf{qo} \textbf{g̈àʔ} \textbf{â} \textbf{mi}: lit. ``when think-overtake, chase not overtake.'' A proverb meaning `By the time we realize what's happening it will be too late to do anything about it.'}

44. T: Well, as far as I'm concerned, I'm still not going.\footnote{\textbf{mâ} + V + \textbf{šɔ̄}: `still not V'. But \textbf{mâ} + V + \textbf{šē}: `not V yet'.} You all can
discuss it and go if you want to.

45. H: Well, then, this is where the matter stands. If there's anybody who wants
to go, let him go tomorrow. That's all there is to it. As for me, I won't be going
yet. I don't have the time yet. Because I'm very busy working, seeing that I don't
have anybody [to help me]. Among all of you who want to go, [why don't] some of
you get together and go off hunting or something\footnote{The morpheme \textbf{mû} appears in elaborate expressions with intensifying or approximative force. See DL: 1003-4.} from the cattle-tending huts
to Porcupine\footnote{\textbf{fâʔ-pɛ́}: `brush-tailed porcupine' (\textit{Atherurus macrourus}).} Mountain, and cook yourselves a good roast and continue on from
there!\footnote{I.e., the Headman is suggesting they go hunting on their way to inspect the new fields the government is offering. This is by way of tempting more people to go, but is hardly a serious suggestion. A hunting expedition is an all-day affair, not something to be sandwiched in between other activities.}

46. Ty: Well, but did they say they'd dig out the paddy-fields for us? What
did they say they'd do?

47. H: No, they're not going to dig out the field for us. We've got to dig ours
out all by ourselves. We'll have a go at it, then if it turns out that we can't,
in case we're not able to do it,\footnote{Lit. ``if it's a not-being-able thing''.} we'll just tell the government officials.
The government will take care of us and help us, they said. Also, in the matter
of bringing water [to the fields], even if it's a long way, if the regime--er,
if the government\footnote{The headman first uses the Burmese word for `government' \textbf{a-šô-yàʔ}, then corrects himself and uses the Thai word \textbf{lâthâbâ}.} helps us, we'll make it all right. Little people that we\footnote{\textbf{nɔ̀-ŋà-hɨ}: `you and we', i.e., inclusive we.}
all are, we've been thinking ``This is impossible and that is impossible''
-- that's how we think. But, look, if we don't have fields, land to till, whatever
we might try to do in the future, it will really be impossible for us to succeed.
Everybody had better think this over mighty fast. Elephant and cattle grazing grounds,
now, [that's a place] I also would like to get.\footnote{I.e., ``a place that I too would like to get''. Unlike the lands at \textbf{Pha-pu-jaw} in which the Headman has no interest.} They say the trees there are
big, but we could chop them down, [in our] fields. In a year or two we could have
the stumps burned out and be earning a living from our paddy-fields. Oh, I'm sure
we could! So, as far as I'm concerned, I want the big stretch of flat land above
the cattle-tending hut -- the whole plain, above everybody else's [fields].

48. T: You're so greedy! That's just it.\footnote{The accusative particle here indicates that the underlying meaning is ``this is what you have''.} That's why nobody wants to go
[with you], why nobody wants to follow your example!

49. H: Ha, ha! That's nonsense, that is!

50. : He's right, he's right!

51. H: Okay, okay, I'll take [land] at the tail-end of everybody instead,\footnote{The prehead versatile verb \textbf{qɔ̀ʔ} here translates as `instead'.} if
that's the way you feel. Way below everybody else, down at the rocky ridge ---

\direct{everybody talks at once}

52. Ty: Hold on! The land down there is mine!

53. H: Down there at Há-tó-cē in the plains near Kɛ̀-ma-lón.

54. T: Just look at him, just look at him. Now even though they aren't going
they've already been fighting over it --- acting like this, there's no way we can
earn a living.

55. Pà-ɛ́: It's mine, it's mine! The land down there is mine!

56. H: Look, when the authorities give it to us, we'll carefully measure it off,
the good places as well as the bad, and divide it up. Everybody will get an equal
share.\footnote{\textbf{qha-šwí} \textbf{hə̂ʔ} a: ``Let us get equally.''}

57. Ty: Me, if it's bad land I'm not taking it!

58. Pà-ɛ́: I won't take it either -- I won't!

59. H: What our forefathers said ... what you've been saying is, ``When I said I
wanted to take the upper part, you said you wouldn't give it to me.'' But according
to what our forefathers said long ago, it's the lower part which is to be taken.
Because when you get land below other people's, you can\footnote{The conjunction \textbf{à-mù} expresses `future probability'.} gradually move a little
bit upwards one year, and a little bit further upwards the second year, until someday
the one who got the lower part will get all the upper fields as well! There's an
old saying to that effect --- it's a good one. So if I say I want the upper part,
why don't you give it to me!

60. T: Just look at him, just look at him! Now he's even teaching you the way
to rob other people of their possessions!

61. H: I'm not robbing anybody, I'm not robbing anybody! I [told] you I'd take the
upper part even if there wasn't enough water. [But when I said] I'd take the upper
part, what you said was ``If it's a good place you take it for yourself!''
So now I'm saying I'll take [land] down there, way downhill from everybody else
in the rocky plains down there!

62. T: Well, then, what about what you were saying there about digging your
way upwards little by little into other peoples' [land]?

63. H: \direct{Laughs} Oh, that! You know there's an old saying like
that, don't you? ``When you stake out\footnote{\textbf{vɛ} (V) `lay claim to; stake out'.} a paddy-field, stake it out
low! If you get a high field, you may get water, but cold water makes for bad paddy
.\footnote{The higher up the mountain the colder the water presumably.} If you get land below the others, one day you will get to be greater than
they`` -- isn't that what they say? As our ancestors have said, ``The
first year while you hoe your weeds, undermine the paddy-field boundaries\footnote{\textbf{ti-mi-tɛ̂}. The \textbf{ɔ̀-tɛ̂} are earthen dikes forming the boundaries of a paddy-field.} a
little and push up a ways. The next year dig under again and advance upwards another
length. [That is how you become] great." So I'm just trying to get other
people's [property] for myself!

64. T: And here's where your great greed comes in!

65. H: It's not greed at all! You guys don't want it. It should be given to me!

66. T: Then let each one of us go down and work only in his own place!

67. H: Well--

68. T: This business of knocking over and breaking up other people's boundaries,
you would always have to make new ones and there'd be no end to it.

69. H: [I'm not saying] ``Knock them over and break them up''--when the
water weeds are thick around there, if you just dig at them a teeny-weeny bit,
the earth comes tumbling down [into your field]--if it's soaked with water.

70. T: Terrific, I'll keep scraping yours away too, then!

71. H: \direct{Laughing} I'm only kidding!\footnote{Lit: ``It's not that way.''} You're talking nonsense!\footnote{Lit: ``Yours isn't even human speech.''}

72. Pà-ɛ́: Don't cultivate the upper part. Don't cultivate the [whole] middle
part. Dividing up the middle part, each person will get half. I'll tear them [the
boundary markers] down.

73. H: Do it honestly for us. Well, you know what people say, if they're narrow
ones--if all the strips [of land] are narrow, they'll say, ``It's too long!
I don't want it!"\footnote{I.e., if a field is narrow it is apt to be long. Rectangular fields involve more footwork than squarer ones.} So let everybody measure off his own section squarely.

74. Ty: Well, if you keep doing it that way...

75. T: Don't give any to people who don't want to accept it.

76. Ty: It doesn't look nice, it doesn't look nice.

77. H: When I've made the boundaries between the sections, I'll certainly try to
do it in such a way that each and every person gets enough.

78. Ty: Well--

79. Pà-ɛ́: I'll bet they'll tear them down.

80. H: Oh, you people are talking at cross-purposes here\footnote{\textbf{jɛ̀} \textbf{dàʔ} \textbf{mâ} \textbf{hɔ̄}: lit. ``discuss mutually not harmonize''.}! Just talking at cross-purposes.
When the government is good enough to give us something, let everybody at least
go and have a proper look at it! No matter what land falls to any individual's
lot.

81. Ty: It depends entirely on the government.

82. H: Look, when the government gives us something there's nothing for us to go
complain about, is there? Shaw-hpa has suggested that we draw lots! You'd write
down the paddy-field [locations] and put them in [the pool], then you'd draw the
lots, and whichever piece you happen to get, it would be that land which you'd--that
area there that you would get.

83. T: That would never work. When you draw lots [and you lose] it's not a
happy business.

84. H: It's because nobody agrees on how to divide it up! So what can we do about
it?

85. T: What'll happen is, when the lots are drawn a loser\footnote{\textbf{â-cɔ̂-pā}: lit. ``one who doesn't happen to [get a good place]''.} will--

86. H: The hell with losers! they just won't get it, that's all.

87. Ty: Well, if we draw lots and I don't get a good place, I simply won't
take it! Even if it falls to my lot.

88. Pà-ɛ́: I'd take it, I'd take it!

89. H: Well, if that's the way it is, there's nothing to be done with you people.
Well, if you won't even do what they advise you to do, let somebody else tell you
what to do, let somebody else tell you. We're all just quarreling with each other
and not getting anyplace.

90. T: Let me decide, let me decide! I'll solve them for you, all these problems
of yours. If there is anybody now who doesn't want to get any [land] as the headman
said to, let him just give his share to the Lisu\footnote{A member of the Lisu tribe, resident for several years in Huey Tat, and well liked by everybody.}, all of it.

91. H: Which place do you mean?

92. T: Even the [lands] at Pha-pu-jawn.

93. H: The ones at Pha-pu-jawn, now--

94. T: And let [the lands] at the ``cattle and elephant grazing grounds'' be
given away, too! You people are so stubborn\footnote{\textbf{tɔ̂} \textbf{yɔ} \textbf{hā}: lit. ``hard to talk to''.} right now!

95. Lisu: I for one will take them!

96. H: So how are you guys going to go and earn a living in the future?

97. Lisu: No matter where we work for our living\footnote{\textbf{qhɔ̀} \textbf{te} \textbf{câ} \textbf{â} \textbf{te} \textbf{câ}: lit. ``where we \textbf{work-for-a-living} not \textbf{work-for-a-living}'', i.e. `no matter where we work for a living'.} we'll just be scratching out
a bare existence anyway.

98. Ty: Well, if we haven't got places to live off, someday we'll be in trouble.

99. H: As for me, I don't care whether I get any [new land] or not.

100. T: That's why you're not going, eh?

101. H: If I don't get any I won't be angry. If I do get some, I won't be happy.

102. T: You're just saying that with your mouth!\footnote{I.e., `you don't really mean it'.} You do that with everything.

103. H: See here, now, I've grown old working old fields\footnote{\textbf{hɛ-(g̈ɔ̂)-šā}: a field that has already yielded a crop, an `old field'. The headman professes to prefer working other people's old fields to clearing new ones of his own.} these many years! Among
you there's nobody who earns his living [that way]. Me, each year I work somebody
else's old field, a different one each time. I go roaming around and earn my living.\footnote{\textbf{ca} \textbf{te} \textbf{câ} \textbf{tô} \textbf{qay}: an impressive 5-verb concatenation!}

104. Lisu: Okay, keep scrabbling for a living then! All by yourself.

105. H: Scrabble I will. I'm not going.

106. Lisu: Stay sitting all forlorn\footnote{phɨ́ \textbf{ɛ̀} \textbf{mɨ} \textbf{ve}: lit. ``sit grayly''. This refers to a former Lahu custom of removing the shirt when sad or angry, thus exposing one's ``grayish'' skin. This expression is now usable of anyone who is in a funk or a huff, whether or not he has actually removed his shirt.} and keep scratching for a living!

107. H: When people give you fellows advice, you don't pay any attention. Even when
those important people advise you and tell you what to do, you still don't want
to listen.

108. Lisu: The one who doesn't want to listen is you!

109. T: You just sit under an eggplant-berry bush\footnote{\textbf{má-hɛ-pɨ̄}: Thai \textit{məkhy̌a phuaŋ }. A small shrub of the eggplant family ( \textit{Solanaceae torvum }), whose bitter berries are eaten in curries. The English translation is my own coinage.} and scratch for a living!

110. H: There's a saying about the Lahu people\footnote{Lâhu \textbf{yâ} \textbf{qôʔ} \textbf{ve} \textbf{qôʔ} \textbf{qo}: ``if/when we say about the ones called Lahu people''.}, that ``when one of us
is working, the other turns away.'' He is obstructed. ``When one of us is carrying
on his work, the other pulls him back.'' We can't go on like this. We Lahu, no
matter what we do--

111. T: ``While one man is climbing a tree, the other is pulling at his legs''
--- that saying is to the point here.

112. H: Because we don't want to obey the laws and customs of our superiors or even
listen to what they're saying.

113. T: It's all because you didn't listen from the very beginning when people
were telling you ``Don't do this, don't do that, it's impossible!''
So today, you see, it's nothing but troubles, whatever we do. No matter who teaches
you, you won't listen!

114. H: Well, unless we let those government officials handle it for us I don't
see what we can do.

115. T: Then let's leave it to the government. You people don't quarrel about
it anymore!

