
82a The Smoke and the Non-Smoker

83 H: Well, once upon a time there were two men.

84 One of them was a tobacco-smoker, he used to smoke all the time. He was a great
smoker.

85 The other one didn't smoke.

86 His pipe --- the smoker's pipe was a very fine one.

87 The inside\footnote{ɔ̀-kɛ 1. `the innermost, hardest part of a log; the pith' 2. `the inside of a tube'} of it was red and smooth, and it was streamlined\footnote{phi ɛ̀ʔ `flat'. Here probably refers to its elegant shininess. CHECK.} and shiny,
so that the non-smoker wanted to have it, and stole it away from him.

88 When he had gone and stolen it, the two of them then [a few words are inaudible]
and they make a big count-case\footnote{à-mù-ló phɛ̀ʔ dàʔ ve ``to have a great case happen mutually''} out of it.

89 They made a big case of it, they were fighting with each other, and (the time
came that) they reached the great official --- the lord --- the king\footnote{The headman decides to make his story impressive by making the judge a king.} --- and
this king was a very shrewd man.\footnote{Lit. ``as for this king, his cleverness was very great.''}

90 He made the non-smoker \{make a /wad of\} tobacco, to put in the pipe.

91 And he had the smoker crumple some tobacco too.

92 ``Both of you crumple it and put it down here,'' he said.

93 He didn't have them wad it up one tobacco --- er, I mean one \textit{man} after
the other\footnote{The headman's slip of the tongue occurred at the beginning of the sentence, but the English word-order requires that it be translated late in the sentence.} [while they were there] together.

94 [Rather] he summoned one man at a time and had him \{crumple it / wad it up\}
--- the tobacco.

95 Well, then, after the tobacco had been wadded and place [before him], he summoned
the smoker.

96 ``Come here, you!'' he said to the owner of the pipe. He came.

97 Then he summoned the one who had stolen the pipe too.

98 ``Come here!'' he said.

99 ``If you smoke a pipe --- er, you, if the pipe is really yours, try putting
the tobacco you wadded into it,'' he said.

100 When he said ``Try putting it in,'' they went at it with a will\footnote{g̈ɔ̀ kə a lɛ: g̈ɔ̀ is a lively v.V.} --- and
the one the smoker had made\footnote{``The smoker's one'' (šú dɔ̀ pā tê mà).} filled the bowl of the pipe perfectly.

101 The non-smoker's one was a big lump like this... \footnote{The inaudible part of the sentence must mean something like ``and wouldn't go into the bowl, \textit{since he didn't know }its exact size.''} ... because he didn't
know...

102 He was jealous of someone else's pipe, and stole it, and even made a court
case out of it and went before the king, but he couldn't win against the pipe-smoker!

103 That fellow went and robbed someone else's property, went and stole someone
else's possession ---

104 Then, after he had coveted it and stolen it, when it came to putting it to
proper use,\footnote{qɔ̀ʔ the ve qɔ̀ʔ: the first qɔ̀ʔ implies the action takes place \textit{later}. the is a pro-verb for `wad up' --- i.e., `do an action.' The second qɔ̀ʔ is similar to the Punf kàʔ `even'.} he had no idea of how to do it.

105 When someone forced him to make a wad for it, he crumpled up enough tobacco
for \{anyone / a person\} to put into \textit{two }pipes --- he didn't know [\{how
to do it / any better\}].

106 He wanted to get a hold of his pipe so he brought the fellow\footnote{qɔ̀ʔ the ve qɔ̀ʔ: the first qɔ̀ʔ implies the action takes place \textit{later}. the is a pro-verb for `wad up' --- i.e., `do an action.' The second qɔ̀ʔ is similar to the Punf kàʔ `even'.} to court,
thinking all the time\footnote{``all the time'' translates the reduplicated yɔ̂ g̈â yɔ̂ g̈â `he'd win, he'd win [he thought].'} that if a serious issue were made of it\footnote{ɔ̀-tɛ̀ ɔ̀-na the ve.} he would
win.

107 [But] it was in vain.

108 He went and stole another's property, and behaved like a miserable wretch.\footnote{chɔ lù pā `a ruined person, a person who has gone bad'}

109 But the real pipe-smoker, in joy and gladness, got [back] his possession.\footnote{In case one desires to impress the moral more strongly, one may add 109a: hɔ́ khi chi chɔ-lù-pā àʔ šu mɔ̂ qhɔ̂ ve thàʔ patɔ thɔ̄ qhɔ mɔ-chwɛ̂ yù kə šē ve gò cê. ``And the king threw the wretch into jail for a long time, since he had stolen what belonged to another.''}

110 That's right, my boys! ``Don't go sitting on someone else's stool,'' as the
saying goes. If it's not your own stool ---

111 T-y: (Teasing) Did you say ``don't go shitting'' or ``don't go sitting''?\footnote{The original says ``did you say tâ mi (= don't catch) or tâ mɨ (don't sit).'' The humorous impertinence is, if anything, enhanced by the translation.}

112 Don't \textit{sit}, I said! That man who stole the pipe couldn't best the other
when it came right down to it, in the end.\footnote{Lit: ``when the doing-thus time came, when the being-the end/last time came.''}

113 H: Mm-hm.

