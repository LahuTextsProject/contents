
\textbf{10 Agriculture and the Need for Unity}

[TAPE II, Side 1]

66: Well, since wild boars have eaten rice from the Sexton's field,  tomorrow we've
got to help hunt them down and beat them [dead] for him, right?

67: Yeah, you've got to help me hunt them down and beat them [dead]. The rice itself
is very good [this year in my field].

68: Otherwise you won't have anything to eat. If \textit{they} eat it all up.

69: Whether we catch them or not, let's go drive them away for him. If we drive
them away, they won't come back for a while, at least.[43] We'll just go and drive
them away for him. Whether or not we catch them we'll go give them a chase. Tomorrow
we'll go....

70: Right....

\begin{center}
* * *
\end{center}

\leftskip=0pt
Well, now the time has come for harvesting our rice!

71: Yep, rice-harvest time has come. Tomorrow I plan to go start trying to reap.
Since the wild pigs are eating away at it.

72: Kheh-ki, have you managed to try harvesting yet?

73: Yes, I plan to try to harvest. My field has just begun to ripen a little, too,
now. Tomorrow I'll go reap,[44] I guess.

74: Jali, what about you?

75: Well, mine is being harvested.[44] Even today.

76: I see. Well, how many bushels[45] have you gotten this year--I mean \emph{you}?[46]

77: Five bushels.

78: Five bushels, eh.

79: Uh-huh.

80: Five bushels probably isn't enough to live on.

81: It doesn't even come to five bushels, unfortunately.[47] In my field this year
the paddy is no good, since they set the damn thing on fire for me, so I'll really
only get \textit{three} bushels, Pastor. I have no idea how I'll ever be able to
feed myself now.

82: Ah, you can't think your way out [of a situation like this]![48]

83: How about you, Jali?

84: Well, I've got ten bushels, ten bushels for me.

85: Hm, you're a little better off, then.[49] Shaw-lu [The Kalin], how many bushels
did you get?

86: At the moment, I've only got five bushels. But I haven't finished my reaping
yet.

87: I see.

88: For my part, I've only got five bushels. Because the wild pigs were eating
it, too. Half of it.

89: Hm... well....

90: And that's why we kept saying before \texttt{"}Work\textit{ together}, work
\textit{together}! If everybody works all by himself, the rats eat it, the birds
eat it, the wild pigs eat it!\texttt{"} It's because you don't listen to what people
tell you, you guys!

91: There's an old saying, \texttt{"}If you would eat, smell.\texttt{"}[50] That
is, if you're going to eat something, try smelling it first. Oh, my children![51]
Cultivating the land is also like that. If you plan to do it, you've got to think
about it [first]. Furthermore, if you want to \textit{speak}, you have to think
about it first, I tell you. Even if we just intend to speak, we must carefully
think over [what we want to say]. When we don't listen to people's advice sometimes,
and disobey, that's the reason why we have to suffer for it this way....

\begin{center}
* * *
\end{center}

\leftskip=0pt
Well, have you tried threshing your paddy already this year?

92: Yeah, I've threshed it out[52] already.

93: How much grain did you get?

94: I got ten baskets[53] per bushel [of paddy].

95: If you really get ten baskets per bushel from your field, that's not bad at
all. Since you've gotten five bushels. From mine I only[54] get six baskets or
[sometimes] as much as seven. Sometimes I only get two basketsful, or one and a
half!

96: Jali, what about yours?

97: Well, mine now, sometimes I get ten basketsful and sometimes fifteen.

98: I see. Well, [now] I'll teach you children one \{point / thing\}! Let everybody
carry back his grain and \{put / store\} it inside his house! In this country the
Thais are very mean, and they're apt to set it on fire. If the Thais burn it up,
we'll have nothing to eat.

99: No matter how much of it there is, even if I have to spend all day tomorrow
carrying mine, I'll probably get [the job] finished. I'll carry it home and store
it, all of it. It's [a] hard [thing] to go and do. However, since these Thais are
such irrational people[55]....

100: Will everybody carry it back to store?

101: They'll carry it back to store [if they know what's good for them]. What [this
fellow][56]\textbf{ }carried back came to ten basketsful. But all together [his
crop] had come to fifteen baskets. So that there were still five basketsful [lying
around in the fields]. And since you people don't listen to your elders' advice,
some elephants came and trampled [the rest]!

102: Ah, me! That's why we said on that day long ago: \texttt{"}When people instruct
you, listen to what they say! Listen to what they say! When you cultivate the land,
work where the others are working,\texttt{"}[57] we said. But today even the elephants
have again trampled [on somebody's crop]! That's no way to earn a living! When
people instruct you, you don't heed their words.

\begin{center}
* * *
\end{center}

\leftskip=0pt
103: Has everybody finished carrying his grain back to store in his own hearth
and home[58]?

104: Well, I haven't managed to store mine yet. This weekend I plan to have some
Thais carry the rice [to my home] [on their shoulders].[59]

105: Yes, do [have them] carry it all up and store it away! Immediately.

106: Well, I'm finished already.. Because mine didn't amount to very much. [But]
I've thought it over, and next year, after we've celebrated New Year's, when it's
time to work again, from then on I'll work together [with the others].

107: Okay.

