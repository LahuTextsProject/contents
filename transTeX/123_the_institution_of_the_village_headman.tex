\setcounter{footnote}{0}

1. P: Well, what I'd like to know a bit about---is about the \textit{headman}.

2. Throughout this country if a village is established you've got to have a headman,
right? All over the country.

3. T: Right.

4. P: So the headman in a Lahu village, how do you---er, choosing a headman, what
kind of person---what kind of person is is worthy of being a headman?

5. Because I'd just like to find out about this matter.

6. Couldn't the Pastor please tell me such things?

7. T: Well, I certainly can.

8. P: So then, please try explaining it.

9. T: The matter of how we Lahu and the other hill-folk choose a headman is something
that everybody---women and young people included---ought to be made aware of, ought
to be made to listen to.

10. The second thing is, a person who has been made headman must constantly be
mindful of what goes on in his village.

11. If we were to specify all the things that he should be mindful of, he must
always carefully teach his villagers about how to earn their living,\footnote{\textbf{g̈a-câ-g̈a-dɔ̀} \textbf{tù}: `in order to get to eat and drink'.} how to
become the equal of other people's villages, all these things.

12. And in the third place, a person who has become headman must be just like a
tree.

13. Fourthly, all those who are his villagers are just like his branches.

14. And fifthly, those who are on the Committee\footnote{\textbf{kɔ̄mītī}: a group of respected men who were consulted on important matters by the headman of a Christian village.} are like all his leaves.

15. If there is a tree there are branches.

16. There are leaves as well.

17. Therefore when a person becomes headman, there must also be several other kinds
of villagers to work with him.

18. P: How old must a person be before he is fit to be headman?

19. Does it matter whether he is young or old?\footnote{Lit. ``If he is young, can he become it, if he is old can he become it?''. The reduplication of the verb \textbf{phɛ̀ʔ} `become' conveys the notion of `even if; if still.'}

20. T: Well, to be a headman we must choose someone who is a mature person rather
advanced in years, a person who is wise and intelligent.

21. As for the matter of his age, if a person is around twenty or thirty\footnote{\textbf{nî-šɛ̂ʔ} \textbf{chi}: short for \textbf{nî} \textbf{chi} \textbf{šɛ̂ʔ} \textbf{chi}. Consecutive numerals may be used approximatively, as in Japanese \textit{go-roku nen mae ni} `about 5 or 6 years ago'. The Teacher means that the minimum age should be 20 or 30, but people 35 or over would also be fine. In general the Lahu are rather vague about their ages according to the calendar.}, or
thirty-five and above, he can become headman.

22. P: He should also have a little more education than other people, right?

23. T: Yes.

24. P: He should also know a bit more about other peoples' languages and customs,
right.?

25. T: Yes.

26. P: Hmm, so if somebody is a bachelor\footnote{\textbf{chɔ-há-pā}: a marriageable but still unmarried youth (ca. 17-20 years old).} could he be headman?

27. T: Well, as for bachelors, they can't be headmen!

28. The reason is, they\footnote{The teacher uses the singular 3p pronoun \textbf{yɔ̂} instead of \textbf{yɔ̂-hɨ} `they', but since he is speaking of bachelors in general, \textit{they} is more appropriate as a translation.} don't have wives, so when they see a pretty girl and
they are tempted to\footnote{\textbf{ca} \textbf{te} \textbf{yàʔ} \textbf{qo}: the pre-head versatile verb \textbf{ca} `go and V' in this context seems to mean `have the intention of V'ing'.} go and commit an offense they don't listen to people's advice.

29. P: Oh, so what you mean is, as for [being] headman a bachelor can't do it,
is what you're saying, right?

30. They all must be people who have wives,\footnote{Lit. ``they are all people who must have wives.''} you're saying.

31. T: Right.

32. P: So then, a headman---er---

33. JAM: Could a woman be a headman?

34. P: Yeah, could a woman do it too, being a headman?

35. T: Well, a woman probably couldn't.

36. P: Why couldn't she?

37. T: We Lahu don't have such a custom.

38. Our ancestors for generation after generation have never used to do that, so
we can't do it.

38. P: So then Lahu headmen are all men.

39. T: They must just be men only.

40. P: So as for a person who gets to be headman, do you also have to consider
how much money he has, how much property he has?

41. Could anybody be a headman whether he has money or not?

42. T: Well, to be headman, as long as one knows proper customs and behavior, and
always follows the laws of the country honestly, he can do it even if he doesn't
have money.

43. Sometimes, even if a headman is struggling with poverty, if the villagers find
out that he really has no money they could help out their headman, those villagers
could.\footnote{This sentence, syntactically garbled in the original, has been lightly edited.}

44. P: If it's a headman like that, a Lahu headman, does he get a monthly salary
to live on?

45. T: Well, a Lahu headman doesn't get a monthly salary.

46. But there is one thing.

47. He gets to eat the special portions of killed game animals.\footnote{\textbf{šā-qɔ̄-šā-šɛ}: \textbf{ɔ̀-šɛ} `special portion of killed game'; possibly the same morpheme as the 2nd syllable of \textbf{qhâʔ-šɛ} `headman'. The choicest parts of the animal were the muscles of the neck (\textbf{qɔ̄}) and back.}

48. And sometimes when the villagers don't get along with each other and are in
the midst of fighting and squabbling, he will levy a fine.

49. That money can go to the headman.\footnote{Lit. ``that money the headman can eat''.}

50. P: When you said that he gets to eat special parts of the game, is it like
when you shoot a barking deer he gets the special parts?

51. T: Yes. Barking deer, wild boar, sambhur deer---

52. Lisu: Wild cattle.

53. T: Wild cattle---all these he can eat.

54. There are two kinds which he cannot eat.

55. One kind is bear. The other kind is goat-antelope. These he doesn't eat, the
headman.

56. P: How come you say that he can't eat those two things?

57. T: The reason is like this. It is said that they don't taste good, those things.

58. P: Since they don't taste good one can't eat them.

59. T: Since they don't taste good one can't eat them.

60. -A. P: I see.

60. -B. T: They say one doesn't want to eat them.

61. P: As for eating them, they're permitted to eat them, aren't they? It's just
that they don't want to eat them because they don't taste good.

62. T: To eat them is permitted. But they don't eat them. They don't like them.\footnote{This whole part of the discussion is quite academic. Even back in the 1960's large game animals in Northern Thailand had been hunted almost to extinction. Few villagers had ever had the chance to taste the body parts of such animals, unless they had been lucky enough to have done so in Burma.}

63. P: Oh, I see, I see, I see. Well then, a headman---when somebody has become
headman, after how many years do you choose a new one?

64. Or else---if he's a very good one, could he keep being headman until he dies?

65. T: Well, sometimes a certain person, if he's smart, and takes care of the villagers
properly, and leads them honestly, he can be headman until he dies.

66. The thing about being a headman, as long as he hasn't done anything wrong,
we have no procedure for dismissing him.

67. We Lahu have a proverb like this: ``If the tiger's not dead, you can't strip
off its skin.''

68. P: Yeah. Very true, that is.

69. T: If the tiger's not dead and you scrape its skin, it'll bite you, they say.

70. If a headman hasn't done anything wrong and you bring him down, it could cause
a lot of trouble, people say.

71. P: So when a headman dies, how do you go about choosing a new one?

72. T: After a headman dies, when you want to get a new one, all the people in
the village must select him.

73. One must choose the sort of person who has a good heart, who is patient, who
doesn't want to listen to gossip spoken behind people's backs, a person who knows
others' languages and customs.

74. All the people must choose him.\footnote{I.e., there should be unanimous consent to his selection.}

75. Speaking of other people's customs, if there's a king, and the king dies, his
son can then become king after he dies, right?

76. In the same way, if it's a headman, when the headman dies can his son then
become headman after his death?

77. Is there this kind of custom among the Lahu?

78. T: Well, there is---we have that too.

79. Sometimes it's like that.

80. Our current headman,\footnote{In Huey Tat village.} he doesn't have any sons.

81. He only has a daughter.

82. So after the father dies, we'll have to choose some other person.

83. And even if this son---er, even if he does have a son, if he [the son] doesn't
understand how things are we can't use him.

84. P: You can't use him.

85. T: He can't be used.

86. P: As for the way we establish a headman, are the Lahu just the same as other
people---for example, the Northern Thai, the Central Thai, the Akha---in the way
a headman is chosen?

87. How are they different?

88. T: Well---

89. P: Well, if you try comparing them---how do they do it, how do we Lahu do it?

90. Which aspects are the same, and which aspects are not?

91. T: Well, they're not quite the same.

92. The way the animist Lahu\footnote{\textbf{pɛ̂-tú-pā}: lit. ``beeswax burners,'' so called because they use beeswax candles in many religious ceremonies.} choose a headman is that if everybody likes someone
they just make him headman on the spot.

93. There is no kind of oath.

94. Whereas we Christian Lahu have an oath when we choose a headman.

95. It's this way: once a headman has been chosen, if everybody is satisfied, in
the church the young men and women and the adults all shake his hand and testify
that they are making him headman.

96. P: They take an oath---

97. T: Yes, they take an oath, that's how it is.

98. P: So when you pick a headman, how do you do it?

99. Do you have some kind of celebration?

100. You probably don't, I guess.

101. T: Well, when we pick a headman we don't do it that way.

102. We don't make a celebration, we Lahu.

103. When we choose a headman, after he's been made headman, once we have shaken
his hand, everybody must listen to what he tells us.

104. P: ``From today onward the leader of our village is you!''\footnote{Paul is pretending to quote the oath.}

105. T: Yes.

106. P: That's what you say, eh?

107. Then, once we've chosen a headman, other people, government officials and
such, recognize him as headman, right?

108. T: Right.

109. P: They have to call him headman.

110. T: Yes. Once we've chosen him, when an official comes and asks who the headman
really is, we have to point out the headman to him.

111. And when the officials need the headman they come to the headman's house.\footnote{The comitative noun particle \textbf{gɛ} here has a meaning similar to French \textit{chez}.}

112. P: That's how they come.

113. They probably don't come and choose somebody on their own---don't we choose
a headman all by ourselves?

114. T: Yes, we choose all by ourselves.

115. And it's not the same as the Thai.

116. When Thai people pick a headman, an official appoints him.

117. And they pay him a monthly salary.

118. But it's not like that for us Lahu.

119. When we Lahu choose a headman, we get to choose him all by ourselves.

120. If even a single person is not happy with the choice, he can't be made headman.

121. When everybody is in agreement, he can be made headman.

122. P: So then the way the Lahu make their headmen\footnote{\textbf{qhâʔ-šɛ} \textbf{te} \textbf{ve}: lit. ``do headman'', can mean either `select to be headman' or `act/function as a headman'.} is not the same as the
way the Northern Thai do it, right?

123. The way a Lahu acts as a headman, he takes on the work of a headman with no
remuneration, right?\footnote{\textbf{bo} \textbf{te} \textbf{ve}: `to make merit; do something \textit{pro bono}'.}

124. T: Yes.

125. P: If they're Northern Thai headmen, they would like to make merit.

126. T: Strictly speaking, if we observe the custom properly, in a single village
there ought to be three headmen.

127. The first---the Chief Headman; the second---an assistant under him.

128. He's a headman, this one too---the second person.\footnote{A sort of Vice Headman.}

129. The third---somebody who is fairly intelligent and obedient.\footnote{This is a sort of messenger or liaison person with the world outside the village.}

130. Sometimes if we had a letter to go send, we could write the letter and give
it to him, a person who can be made to take it and send it off.

131. We should have somebody like that too. A messenger.

132. P: Oh, a messenger.

133. T: He's called a `messenger', yes. A messenger who sends letters. We must
have somebody like that too.

134. P: There's also a person you said that helps the headman, right?

135. T: There is.

136. P: The second headman.

137. T: Yes.

138. P: How do you choose him, that one?

139. T: It's like this:

140. When he's being chosen, it's the Chief Headman who has to select him.

141. After the Chief Headman has chosen him, he has to tell all the villagers that
``This person I want to be the one who serves under me to help me.''

142. If this meets with the approval of all the villagers, this guy also his assistant\footnote{Paul finishes the Teacher's sentence for him.}
---

143. P: Can be.

144. T: He can be that. Just the same [as the chief Headman].\footnote{Since the rest of this text was badly recorded, it was re-recorded by Paul and me on Nov. 5, 1965.}

145. P: So is there anything that can protect our Lahu villages from thieves and
bandits and evil people who might come to make trouble?

146. T: Well, that's something we don't have.

147. However, there is one thing.

148. In this village, when thieves or suchlike try to come, the whole village has
to keep watch.

149. Sometimes when we hear the words ``A thief has come!'', all the men, old and
young,\footnote{\textbf{ɨ̄-yâ-i-yâ}: lit. ``big guys and little guys''.} every man jack must protect our village properly, must keep watch properly.

150. When it gets dark we can't sleep.

151. We have to keep watch carefully.

152. P: Yeah, in our country of Burma don't the headmen have ``self-defense groups''\footnote{\textbf{kā-kwɛ̄-yê}: < Bs. \textbf{ka-kwe-yê} (WB \textbf{ka-kway-rê}) < \textbf{kway-ka} `protect by interposing; shield', \textbf{rê} `be brave; soldier'. See DL: 328.}
in each and every village?

153. It's called the ``village protection pā-lêʔ'',\footnote{He uses the Burmese word \textit{pu'lei}ʔ (WB \textit{pu'lip}) `police'.} like the police.\footnote{The speaker clarifies the meaning of the Burmese word by glossing it with the Lahu version (\textbf{tàʔ-nòʔ}) of the equivalent Thai word \textit{tamrùat }`police'.}

154. What they do is, they pick two or three villagers to buy guns for the people
in each village.

155. That's how it has been done.

156. Don't you have that in Thailand?

157. T: Well---

158. P: When it gets dark you have to wait in two or three places around the village
perimeter, since they have to fear that bandits might come visiting and steal from
them or set the village on fire.

159. Don't you have that here?

160. T: Well, that's something we don't have!

161. P: So then when somebody commits a crime in the village, how do you deal with
it?\footnote{\textbf{a-mòʔ} \textbf{te} \textbf{dàʔ} \textbf{ve}: `try a case' < Bs. ʔ\textit{əhmu' }`court trial'.}

162. T: Well, when somebody commits an offense against someone else, we summon
everybody to the headman's house and we have to discuss the matter.

163. P: For example if somebody in the village beats or shoots someone to death,
how do you discuss this at the trial?

164. T: Well, if a person should kill somebody, he must pay 150 in silver rupees.\footnote{\textbf{phu-pī}, lit. ``old money'', silver coins dating from the British raj in Burma. Paper money will not do to compensate for the crime.}

165. If the parents are not satisfied, that's not the end of it.

166. Whatever they ask for they must be given.

167. It is not possible not to give them what they ask for.

168. You'd have to take him [the culprit] to the authorities.

169. P: Oh, would the headman have to take him there?

170. T: The headman must take him.

171. P: Oh, so that's what you do!

172. Is the job of the headman in animist villages the same as the headmen in our
Christian\footnote{Paul here uses \textbf{šú-tɔ̂(n)-pā}, the animists' epithet for the Christian Lahu (< Shan \textit{suu-toŋ} < Bs. \textit{hsu'tâuN} (WB \textit{chu'tôŋ}), lit. ``those who beg for grace/favor'').} villages?

173. T: It's the same.

174. P: I'd like to know the reason for that.

