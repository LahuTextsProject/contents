
{123 The Institution of the Village Headman}

{\texttt{<}Dialogue between Pastor Cà-bo and Cà-lɔ̂\texttt{>}[1]}

{\. P: Well, what I'd like to know a bit about---is about the }{\textit{headman}}{.}

{\. Throughout this country if a village is established you've got to have
a headman, right? All over the country.}

{\. T: Right.}

{\. P: So the headman in a Lahu village, how do you---er, choosing a headman,
what kind of person---what kind of person is is worthy of being a headman?}

{\. Because I'd just like to find out about this matter.}

{\. Couldn't the Pastor please tell me such things?}

{\. T: Well, I certainly can.}

{\. P: So then, please try explaining it.}

{\. T: The matter of how we Lahu and the other hill-folk choose a headman
is something that everybody---women and young people included---ought to be made
aware of, ought to be made to listen to.}

{1\. The second thing is, a person who has been made headman must constantly
be mindful of what goes on in his village.}

{1\. If we were to specify all the things that he should be mindful of,
he must always carefully teach his villagers about how to earn their living,[2]
how to become the equal of other people's villages, all these things.}

{1\. And in the third place, a person who has become headman must be just
like a tree.}

{1\. Fourthly, all those who are his villagers are just like his branches.}

{1\. And fifthly, those who are on the Committee[3] are like all his leaves.}

{1\. If there is a tree there are branches.}

{1\. There are leaves as well.}

{1\. Therefore when a person becomes headman, there must also be several
other kinds of villagers to work with him.}

{1\. P: How old must a person be before he is fit to be headman?}

{1\. Does it matter whether he is young or old?[4]}

{2\. T: Well, to be a headman we must choose someone who is a mature person
rather advanced in years, a person who is wise and intelligent.}

{2\. As for the matter of his age, if a person is around twenty or thirty[5],
or thirty-five and above, can become headman.}

{2\. P: He should also have a little more education than other people, right?}

{2\. T: Yes.}

{2\. P: He should also know a bit more about other peoples' languages and
customs, right.}

{2\. T: Yes.}

{2\. P: Hmm, so if somebody is a bachelor[6] could he be headman?}

{2\. T: Well, as for bachelors, they can't be headmen!}

{2\. The reason is, they[7] don't have wives, so when they see a pretty
girl and they are tempted to[8] go and commit an offense they don't listen to people's
advice.}

{2\. P: Oh, so what you mean is, as for [being] headman a bachelor can't
do it, is what you're saying, right?}

{3\. They all must be people who have wives,[9] you're saying.}

{3\. T: Right.}

{3\. P: So then, a headman---er---}

{3\. JAM: Could a woman be a headman?}

{3\. P: Yeah, could a woman do it too, being a headman?}

{3\. T: Well, a woman probably couldn't.}

{3\. P: Why couldn't she?}

{3\. T: We Lahu don't have such a custom.}

{3\. Our ancestors for generation after generation have never used to do
that, so we can't do it.}

{3\. P: So then Lahu headmen are all men.}

{3\. T: They must just be men only.}

{4\. P: So as for a person who gets to be headman, do you also have to consider
how much money he has, how much property he has?}

{4\. Could anybody be a headman whether he has money or not?}

{4\. T: Well, to be headman, as long as one knows proper customs and behavior,
and always follows the laws of the country honestly, he can do it even if he doesn't
have money.}

{4\. Sometimes, even if a headman is struggling with poverty, if the villagers
find out that he has no money they could help out their headman, those villagers
could.[10]}

{4\. P: If it's a headman like that, a Lahu headman, does he get a monthly
salary to live on?}

{4\. T: Well, a Lahu headman doesn't get a monthly salary.}

{4\. But there is one thing.}

{4\. He gets to eat the special portions of killed game animals.[11]}

{4\. And sometimes when the villagers don't get along with each other and
are in the midst of fighting and squabbling, he will levy a fine.}

{4\. That money can go to the headman.[12]}

{5\. P: When you said that he gets to eat special parts of the game, is
it like when you shoot a barking deer he gets the special parts?}

{5\. T: Yes. Barking deer, wild boar, sambhur deer---}

{5\. Lisu: Wild cattle.}

{5\. T: Wild cattle---all these he can eat.}

{5\. There are two kinds which he cannot eat.}

{5\. One kind is bear. The other kind is goat-antelope. These he doesn't
eat, the headman.}

{5\. P: How come you say that he can't eat those two things?}

{5\. T: The reason is like this. It is said that they don't taste good,
those things.}

{5\. P: Since they don't taste good one can't eat them.}

{5\. T: Since they don't taste good one can't eat them.}

{60-A. P: I see.}

{60-B. T: They say one doesn't want to eat them.}

{6\. P: As for eating them, they're permitted to eat them, aren't they?
It's just that they don't want to eat them because they don't taste good.}

{6\. T: To eat them is permitted. But they don't eat them. They don't like
them.[13]}

{6\. P: Oh, I see, I see, I see. Well then, a headman---when somebody has
become headman, after how many years do you choose a new one?}

{6\. Or else---if he's a very good one, could he keep being headman until
he dies?}

{6\. T: Well, sometimes a certain person, if he's smart, and takes care
of the villagers properly, and leads them honestly, he can be headman until he
dies.}

{6\. The thing about being a headman, as long as he hasn't done anything
wrong, we have no procedure for dismissing him.}

{6\. We Lahu have a proverb like this: ``If the tiger's not dead, you can't
strip off its skin.''}

{6\. P: Yeah. Very true, that is.}

{6\. T: If the tiger's not dead and you scrape its skin, it'll bite you,
they say.}

{7\. If a headman hasn't done anything wrong and you bring him down, it
could cause a lot of trouble, people say.}

{7\. P: So when a headman dies, how do you go about choosing a new one?}

{7\. T: After a headman dies, when you want to get a new one, all the people
in the village must select him.}

{7\. One must choose the sort of person who has a good heart, who is patient,
who doesn't want to listen to gossip spoken behind people's backs, a person who
knows others' languages and customs.}

{7\. All the people must choose him.[14]}

{7\. Speaking of other people's customs, if there's a king, and the king
dies, his son can then become king after he dies, right?}

{7\. In the same way, if it's a headman, when the headman dies can his son
then become headman after his death?}

{7\. Is there this kind of custom among the Lahu?}

{7\. T: Well, there is---we have that too.}

{7\. Sometimes it's like this:}

{8\. Our current headman,[15] he doesn't have any sons.}

{8\. He only has a daughter.}

{8\. So after the father dies, we'll have to choose some other person.}

{8\. And even if this son---er, even if a headman does have a son, if he
doesn't understand how things are we can't use him.}

{8\. P: You can't use him.}

{8\. T: He can't be used.}

{8\. P: As for the way we establish a headman, are the Lahu just the same
as other people---for example, the Northern Thai, the Central Thai, the Akha---in
the way a headman is chosen?}

{8\. How are they different?}

{8\. T: Well---}

{8\. P: Well, if you try comparing them---how do they do it, how do we Lahu
do it?}

{9\. Which aspects are the same, and which aspects are not?}

{9\. T: Well, they're a little bit different.}

{9\. The way the animist Lahu[16] choose a headman is that if everybody
likes someone they just make him headman on the spot.}

{9\. There is no kind of oath.}

{9\. While we Christian Lahu have an oath when we choose a headman.}

{9\. It's this way: once a headman has been chosen, if everybody is satisfied,
in the church the young men and women and the adults all shake his hand and testify
that they are making him headman.}

{9\. P: They take an oath---}

{9\. T: Yes, they take an oath, that's how it is.}

{9\. P: So when you pick a headman, how do you do it?}

{9\. Do you have some kind of celebration?}

{10\. You probably don't, I guess.}

{10\. T: Well, when we pick a headman we don't do it that way.}

{10\. We don't make a celebration, we Lahu.}

{10\. When we choose a headman, after he's been made headman, once we have
shaken his hand, everybody must listen to what he tells us.}

{10\. P: ``From today onward the leader of our village is you!''[17]}

{10\. T: Yes.}

{10\. P: That's what you say, eh?}

{10\. Then, once we've chosen a headman, other people, government officials
and such, recognize him as headman, right?}

{10\. T: Right.}

{10\. P: They have to call him headman.}

{11\. T: Yes. Once we've chosen him, when an official comes and asks for
the headman, we have to point out the headman to him.}

{11\. And when the officials need the headman they come to the headman's
house.[18]}

{11\. P: That's how they come.}

{11\. They probably don't come and choose somebody on their own---don't
we choose a headman all by ourselves?}

{11\. T: Yes, we choose all by ourselves.}

{11\. And it's not the same as the Thai.}

{11\. When Thai people pick a headman, an official appoints him.}

{11\. And they pay him a monthly salary.}

{11\. But it's not like that for us Lahu.}

{11\. When we Lahu choose a headman, we get to choose him all by ourselves.}

{12\. If even a single person is not happy with the choice, he can't be
made headman.}

{12\. When everybody is in agreement, he can be made headman.}

{12\. P: So then the way the Lahu make their headmen[19] is not the same
as the way the Northern Thai do it, right?}

{12\. The way a Lahu acts as a headman, he takes on the work of a headman
with no remuneration, right?[20]}

{12\. T: Yes.}

{12\. P: If they're Northern Thai headmen, they would like to make merit.}

{12\. T: Strictly speaking, if we observe the custom properly, in a single
village there ought to be three headmen.}

{12\. The first---the Chief Headman; the second---an assistant under him.}

{12\. He's a headman, this one too---the second person.[21]}

{12\. The third---somebody who is fairly intelligent and obedient.[22]}

{13\. Sometimes if we had a letter to go send, we could write the letter
and give it to him, a person to go take it and send it off.}

{13\. We should have somebody like that too. A messenger.}

{13\. P: Oh, a messenger.}

{13\. T: He's called a `messenger', yes. A messenger who sends letters.
We must have somebody like that too.}

{13\. P: There's also a person you said that helps the headman, right?}

{13\. T: There is.}

{13\. P: The second headman.}

{13\. T: Yes.}

{13\. P: How do you choose him, that one?}

{13\. T: It's like this:}

{140: When he's being chosen, it's the Chief Headman who has to select him.}

{141: After the Chief Headman has chosen him, he has to tell all the villagers
that ``This person I want to be the one who serves under me to help me.''}

{142: If this meets with the approval of all the villagers, this guy also
his assistant[24] ---}

{14\. P: Can be.}

{14\. T: He can be that. Just the same [as the chief Headman].[2\.5]}

{***}

{14\. P: So is there anything that can protect our Lahu villages from thieves
and bandits and evil people who might come to make trouble?}

{14\. T: Well, that's something we don't have.}

{14\. However, there is one thing.}

{14\. In this village, when thieves or suchlike try to come, the whole village
has to keep watch.}

{14\. Sometimes when we hear the words ``A thief has come!'', all the men,
old and young,[25]  every man jack must protect our village properly, must keep
watch properly.}

{15\. When it gets dark we can't sleep.}

{15\. We have to keep watch carefully.}

{15\. P: Yeah, in our country of Burma don't the headmen have ``self-defense
groups''[26] in each and every village?}

{15\. It's called the ``village protection pā-lêʔ'',[27] like the police.[28]}

{15\. What they do is, they pick two or three villagers to buy guns for
the people in each village.}

{15\. That's how it has been done.}

{15\. Don't you have that in Thailand?}

{15\. T: Well---}

{15\. P: When it gets dark you have to wait in two or three places around
the village perimeter, since they have to fear that bandits might come visiting
and steal from them or set the village on fire.}

{15\. Don't you have that here?}

{16\. T: Well, that's something we don't have!}

{16\. P: So then when somebody commits a crime in the village, how do you
deal with it?[29]}

{16\. T: Well, when somebody commits an offense against someone else, we
summon everybody to the headman's house and we have to discuss the matter.}

{16\. P: For example if somebody in the village beats or shoots someone
to death, how do you discuss this at the trial?}

{16\. T: Well, if a person should kill somebody, he must pay 150 in silver
rupees.[30]}

{16\. If the parents are not satisfied, that's not the end of it.}

{16\. Whatever they ask for they must be given.}

{16\. It is not possible not to give them what they ask for.}

{16\. You'd have to take him [the culprit] to the authorities.}

{16\. P: Oh, would the headman have to take him there?}

{17\. T: The headman must take him.}

{17\. P: Oh, so that's what you do!}

{17\. Is the job of the headman in animist villages the same as the headman
in our Christian[31] villages?}

{17\. T: It's the same.}

{17\. P: I'd like to know the reason for that.}

{FOOTNOTES:}

{[1] In what follows, T (= Teacher) is used for Pastor Cà-bo, and P (=
Paul) is used for Paul Tcalo, chief consultant on the 1965-66 fieldtrip.}

{[2] g̈a-câ-g̈a-dɔ̀ tù: `in order to get to eat and drink'.}

{[3] kɔ̄mītī: a group of respected men who were consulted on important
matters by the headman of a Christian village.}

{[4] Lit. ``If he is young, can he become it, if he is old can he become
it?''. The reduplication of the verb phɛ̀ʔ `become' conveys the notion of `even
if; if still.'}

{[5] nî šɛ̂ʔ chi: short for nî chi šɛ̂ʔ chi. Consecutive numerals
may be used approximatively, as in Japanese }{\textit{go-roku nen mae ni}}{
`about 5 or 6 years ago'. The Teacher means that the minimum age should be 20 or
30, but people 35 or over would also be fine. In general the Lahu are rather vague
about their ages according to the calendar.}

{[6] chɔ-hÁ-pā: a marriageable but still unmarried youth (ca. 17-20
years old). }

{[7] The teacher uses the singular 3p pronoun yɔ̂ instead of yɔ̂-hɨ
`they', but since he is speaking of bachelors in general, Eng. }{\textit{they}}{
is more appropriate as a translation.}

{[8] ca te yàʔ qo: the prehead versatile verb }{\textit{ca}}{
`go and V' in this context seems to mean `have the intention of V-ing'.}

{[9] Lit. ``they are all people who must have wives.''}

{[10] This sentence, syntactically garbled in the original, has been lightly
edited.}

{[11] šā-qɔ̄-šā-šɛ: ɔ̀-šɛ `special portion of killed game';
possibly the same morpheme as the 2nd syllable of qhâʔ-šɛ `headman'. The choicest
parts of the animal were the muscles of the neck (qɔ̄) and back.}

{[12] Lit. ``that money the headman can eat''.}

{[13] This whole part of the discussion is quite academic. Even back in
the 1960's large game animals in Northern Thailand had been hunted almost to extinction.
Few villages had ever had the chance to taste the body parts of such animals, unless
they had been lucky enough to have done so in Burma.}

{[14] I.e., there should be unanimous consent to his selection.}

{[15] In Huey Tat village.}

{[16] pɛ̂-tú-pā: lit. ``beeswax burners,'' so called because they use
beeswax candles in many religious ceremonies.}

{[17] Paul is pretending to quote the oath.}

{[18] The comitative noun particle gɛ here has a meaning similar to French
}{\textit{chez}}{.}

{[19] qhâʔ-šɛ the ve: lit. ``do headman'', can mean either `select to
be headman' or `act/function as a headman'. }

{[20] bo te ve: `to make merit; do something }{\textit{pro bono}}{'.}

{[21] A sort of Vice Headman.}

{[22] This is a sort of messenger or liaison person with the world outside
the village.}

{[24] Paul finishes the Teacher's sentence for him.}

{[2\.5] Since the rest of this text was badly recorded, it was re-recorded
by Paul and me on Nov. 5, 196\.}

{[25] ɨ̄-yâ-i-yâ: lit. ``big guys and little guys''.}

{[26] kā-kwɛ̄-yê: \texttt{<} Bs. }{\textit{ka-kwe-yê }}{(WB
}{\textit{ka-kway-rê}}{) \texttt{<} }{\textit{kway-ka}}{
`protect by interposing, shield', }{\textit{rê }}{`be brave; soldier'.
See DL: 32\.}

{[27] He uses the Burmese word }{\textit{pu'leiʔ }}{(WB }{\textit{pu'lip}}{)
`police'.}

{[28] The speaker clarifies the meaning of the Burmese word by glossing
it with the Lahu version (tàʔ-nòʔ) of the equivalent Thai word }{\textit{tamrùat
}}{`police'.}

{[29] a-mòʔ te dàʔ ve: `try a case' \texttt{<} Bs. }{\textit{ʔəhmu'
}}{`court trial'.}

{[30] phu-pī, lit. ``old money'', silver coins dating from the British
}{\textit{raj}}{ in Burma. Paper money will not do to compensate
for the crime.}

{[31] Paul here uses šú-tɔ̂(n)-pā, the animists' epithet for the Christian
Lahu ( \texttt{<} Shan }{\textit{suu-toŋ}}{ \texttt{<} Bs. }{\textit{hsu'
tâuN}}{ (WB }{\textit{chu'tôŋ}}{), lit. ``those who beg
for grace/favor''.}


