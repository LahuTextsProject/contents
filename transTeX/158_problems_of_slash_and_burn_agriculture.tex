
158 Problems of Slash-and-Burn Agriculture

Conversation among Paul Tcalo (Cà-lɔ̂) ``P'', Thû-yì (T-y), Pastor Cà-bo
(T for ``teacher'')

Répliques 1-48; 49-78 ``The Lahu Agricultural Cycle''

1. P: Well, this group of Lahu of yours living in Huey Tat, how many years has
it been since you arrived here in Thailand?

2. T-y: Well, this year makes about 12 years since we arrived here in Thailand,
I guess.

3. P: When you came down\footnote{\textit{yàʔ la ve}: the villagers' previous home in Shan State was evidently at a higher elevation than Huey Tat.} here what month was it? Did you come down in the cold
season or in the hot season?

4. T-y: Well, as for the time, I don't really know! I was still a kid then, you
know.

5. P: How old were you then? When you came down here.

6. T-y: Hmm, I must have been only about ten years old.

7. P: When you came down then, how many days did you have to spend the night on
the road?

8. T-y: Wow, from when we started to flee from up there until we arrived here in
Thailand, it took about a month, more or less.

9. P: When you arrived in Thailand, what did you do at first to earn a living?
To support your wives and children? And to support yourselves? To keep body and
soul together?\footnote{\textit{ɔ-to hu tù}: ``in order to nourish the body.''}

10. T: The year that we got here, we worked for hire\footnote{\textit{tha-ŋâ ve} `work for hire' [not a traditional Lahu activity] < Thai \textit{tham ŋaan} (do work). Treated as an OV construction in Lahu, so that it is negated as \textit{tha mâ ŋâ}.} for the ``Big Boss.''\footnote{\textit{šathê ló}: owner of a large tea plantation near Huey Tat. \textit{šathê }ult. from Pali \textit{saṭhī \textasciitilde{} seṭhī} < Skt. śre(ṣṭha) `most splendid; preminent'.}
For one day's work five and a half baht. And as for food and drink, the Big Boss
helped us out with everything.

11. P: What sort of work did you do to earn a living with the Big Boss?

12. T-y: Oh, we did all kinds of work. We'd weed the tea gardens and loosen [the
earth at] the base of the plants with a hoe.\footnote{\textit{qɔ̂ g̈âʔ ve}: ``hoe-scratch'', here translated ``loosen with a hoe''.} We'd do all that kind of stuff.

13. P: That year you hadn't yet begun to cultivate swiddens?

14. T-y: Ah, the year that we came here to live we didn't cultivate yet. All we
did was work for hire.

15. T: That year we didn't cultivate swiddens yet. The year we got here we worked
for the Big Boss, a day's worth at a time\footnote{\textit{tê ni pu tê ni}, i.e., from hand to mouth.}---for a whole year we only did that
kind of work to survive.

16. P: So working for him that way, with the money the Big Boss gave you, you bought
rice to eat?

17. T-y: Well, as far as rice goes, we didn't have to buy it to eat then! He fed
us. Both curries and rice, everything.

18. P: So how may years did you work that way for the Big Boss? In other words,
how long did you have to earn your living that way?

19. T: We worked for the Big Boss like that for the whole year that we arrived.
By the second year we had cleared fields, and were living by cultivating swiddens.
Besides, even after we had managed to feed ourselves, working for hire was still
a way for us to get some money.

20. P: Besides cultivating the swiddens, what other things did you do? Were there
all kinds of other things like raising pigs and chickens?

21. T-y: Oh, we did all kinds of things. By raising pigs and chickens, and selling
the pigs and chickens, we got a little extra money, so we kept on doing it, and
we earned a little bit for it.

22. P: Now what I'd just like to know, your cultivating swiddens, clearing land
for swiddens---what rules and regulations are there by the Thai government, the
Thai authorities,\footnote{The speaker uses both the Burmese (\textit{acúyàʔ}) and Thai (\textit{lâthâbân}) words for `government'.} how have they laid them down?

23. T: As for our cultivating fields here in Thailand nowadays, the Thai --- the
authorities have issued a command that land cannot be cleared for swiddens. However,
since we live on \textit{nikhom}\footnote{A Thai government hill-tribe resettlement center. Lahu \textit{nîʔkho} < Thai \textit{níkhom} `settlement, colony' < Skt./Pali \textit{nigama-} `market town'} land, we are allowed to cultivate a little
bit. But even though we can manage to earn a living that way, there are still very
troublesome things. For example, the police come up here all the time, and when
they come to inspect the places where we made our fields, they would like to arrest
us. But so far they haven't been able to ruin us.

24. P: Why is it that they say they don't want you to clear fields then?

25. T: They say that clearing fields makes the water dry up. When hillfolk clear
many swiddens, they say there is no water left for the towns. That's what they
say anyway. But for us Lahu, if we can't clear fields to earn a living, there's
no way we can exist. And there is no other way for us to be able to earn money
either. Since we don't have [irrigated] paddy fields, we'd have to live in misery
if it came to that.

26. P: When you clear a swidden, how many years are you usually able to keep cultivating
a given one in the same place? After how many years do you have to begin clearing
a new one?

27. T-y: Well, for swiddens one piece of land is good for a single year. The second
year is not quite up to the first\footnote{\textit{ɔ̀-šɨ́}, lit. ``the new one'', i.e. when the field was fresh.} year.

28. T: That's right. That's the way it is with ``hanging onto old-fields.''\footnote{\textit{hɛ-g̈ɔ̂ g̈ɔ̂ʔ câ ve}: lit. ``holding onto the field-bones''. \textit{hɛ-g̈ɔ̂} `old field', perhaps literally ``swidden-bones''; \textit{g̈ɔ̂ʔ} `grasp, hold onto'; \textit{câ} `eat; earn a living'. Contrast with \textit{hɛ-šā }`an old field reverting to jungle'.}
Sometimes if we get a place where the soil is no good, if it's not that good for
one year, not even for a single year, and if we go on to a second year, even if
the plants look good,\footnote{\textit{dàʔ} `be good; look good, be pretty'.} they don't have seedpods.

29. P: That means that you have to begin to clear new ones. So what I've heard
is, when you clear land for swiddens you cut down and discard lots of trees, they
say. The lifespan of a tree is about as long as a human being's, they say. A given
tree lives about fifty or sixty years! These trees have a great value for Thailand,
they say. The earning power of Thailand\footnote{\textit{Thây mû-mì phu g̈a ve kàʔ}: ``Thailand's getting money.''} depends only on those trees, they say.
So they don't want to let you chop them down. So then, for the Lahu people, if
--- swiddens ---trees --- fields can't be cleared, how will they be able to survive,
that's the problem. What have you been thinking about this? If they don't want
to let you clear land.

30. T: This is what they say. In order for us to manage to cultivate any piece
of land, they want to make us Lahu do it the way they say. What they say is, they
want us to buy various kinds of fruit, tea, coffee to plant --- and they say that
if we want to cultivate mountain rice fields, we should make them terraced. So
for this reason, for us Lahu, if that's what [we have to] do, there's no way for
us to go on.\footnote{\textit{phɛ̀ʔ la kɨ̀ lɛ̀ mâ cɔ̀}: ``there's no way to become.''} But in the future, how are we going to keep putting food in our
mouths\footnote{\textit{g̈a-câ-g̈a-dɔ̀}: ``get-eat-get-drink''}, if the Thai authorities don't help us, for us it will cause a great
deal of suffering. Another thing is, if we Lahu can't cultivate swiddens in the
mountains, we don't have paddy-fields\footnote{\textit{ti-mi}: irrigated rice (paddy) field. These are only possible in the plains, or on terraced fields (like the Shan and Chinese have made).} either. And we don't have money either.
If you have a few pigs and chickens, and sell them all off, you'd be able to buy
rice to eat. Besides all that, since we wouldn't have any clothes to wear, we'd
be ashamed before other people.

31. P: When you've made a swidden, after a year or two, if the field is no good,
do you abandon it? When you've abandoned it like that, it becomes a \textit{hɛ-šā}\footnote{We could translate this conventionally as ``oldfield.'' \textit{šā} is a bound morpheme meaning `old'.},
right? Once it's become an oldfield how many years does it take before you can
cultivate it again, that oldfield?

32. T: After cultivating a certain field, if we've only worked it for one year,
we can cultivate it again after about five years. But if we ``hold onto the field-bones,''
and cultivate it for another year, we should wait five more years, so after about
ten years we can cultivate it again.

33. P: What you call ``holding onto the field-bones,'' does it mean ``cultivating
twice [in the same place]''? I haven't understood very well. How do you ``hold
onto'' it?

34. T: Yes. What we call ``holding onto the field-bones'' means cultivating twice,
planting two years in a row in the same field. That's the meaning of ``holding
onto the field-bones.'' So if you hold onto a single field for two years, you have
to wait about ten years. After you've waited for ten years, you can get a fair
crop [out of it]. But it's still not really good yet.

35. P: Now what the Thai government is saying, is that you can't clear fields for
cultivation in lots of new places, right? What they say is, that if you've got
to cultivate in just one place you should just add fertilizer to the old field
once every year. What do you think of that? Can you do it or not?

36. T: As for that, when we think it over, this business of adding fertilizer up
in the hills is extremely difficult. The thing is, since the ground isn't level,
when you add fertilizer and it rains all of a sudden all the fertilizer is carried
off, so it's probably of no use to us, we think.

37. P: In that case, can't you make terraces?

38. T: ``Make terraces, terraces!'' they say, ``since the land is very steep, if
you just make them you'll succeed.'' But it takes a huge amount of time, so for
a very long time we would have to suffer and be miserable. Besides, even though
making terraces is the best way, if the ground isn't flat it's a very tough problem
for us Lahu. The thing is, there are no places to make terraces. Even if we did
finally manage to make terraces, how long it would have taken us, we don't know.

39. P: If you were to make terraces on a mountain where you've cleared a field,
about how many years would it take? In order to make terraces the right way.

40. T: I guess that to go on to make terraces in a field that we've already cleared
in the mountains, making field-terraces, in a single stretch of field, if it would
take two months to hoe, then unless [you work at it for] two or three years, you
couldn't finish making a terrace for that one field.\footnote{That is, a field of a size that it takes two months to plow would take two or three years to terrace.}

41. P: So then if you guys were just to make terraces, making them for one or two
years, how would you plan to eat? What would you look for to eat?

42. T: Right now that's what we Lahu are talking about. We can't figure it out.
We can't figure out what we'd be able to eat. The thing is, just cultivating our
swiddens, just planting rice, even without making terraces, some people don't get
enough to eat. Some years there's enough, other years it's not enough. So during
the time that we would have to make terraces, we wouldn't be able to raise pigs
and chickens. We wouldn't manage to sell them for a living. There wouldn't be enough
time. So raising pigs and chickens, feeding them until they get big, and then being
able to sell them for money\footnote{\textit{g̈a qɔ̀ʔ hɔ̂ câ}: a 4-verb concatenation, lit. ``get to-again-sell-eat''.} --- there'd be nothing to feed them with.

43. P: Oh, well then, according to what you, Teacher\footnote{Paul addresses Pastor Cà-bo as \textit{šālā} `teacher' (ult. < Skt. \textit{ācārya-}).}, are saying, this making
of terraces and adding fertilizer to cultivate is probably impossible. But maybe
you actually could do it.\footnote{\textit{phɛ̀ʔ lɛ̀ phɛ̀ʔ ve yò}: lit. ``as for being able, [you could] be able.'' V\_1 + \textit{lɛ̀} + V\_2 is a productive syntactic pattern. See DL 1392.} If they really do help you, right? They've got those
bulldozers\footnote{\textit{mì-gɨ̀ gûʔ càʔ}: ``machine for butting earth.'' \textit{gâʔ} `collide; butt (as goats); crash (of cars)'.}, so if they themselves would come and do it for you, move the earth
for you, it could be done. But moving the earth just for one or two of those villages,
for such a small number of people, they probably couldn't be bothered\footnote{\textit{te bɔ̀}: `tired of doing; lazy to do'.}, I think.
What do you guys think about this?

44. T: I think so too. As for their\footnote{``They'' here probably refers to the Thai officials of the \textit{nikhom}.} helping us, they can only help us a little
bit. The thing is, they don't even help us properly when we get them to bulldoze
a place as a site for a house. So as far as their using the government's machines
to bulldoze for us, when the time to really do it comes, they probably wouldn't
do it just for one or two villages. What I think is, they wouldn't do it even for
a single village. But the thing is, they've got a lot of money. And they also have
several machines. However, it has never yet entered their heads to properly help
us hillfolk.

45. P: Well, it's just like I was saying before, isn't it, that if they're not
willing to let you clear land in the hills for a living, there are many stretches
of flat land in several provinces\footnote{\textit{mə̂} `country; district; province; town, city' < Tai (Si. \textit{myaŋ}).}, so that when foreigners come to look things
over they say ``There are plenty of plains in this country of Thailand, there are
lots and lots of places to till and cultivate, so people don't starve.'' So then
couldn't you manage to go ask for such a flat stretch, or a wooded plain\footnote{\textit{thə̄-ló}: large forested expanse of land}?

46. T: Yes. This is the way matters stand. Our \textit{nikhom} has officials. When
we go ask them for a certain stretch of flat land, they say that since the trees
are very big, they can't give it to us. And the second thing is, there are forested
wildernesses in many districts, there are plains in several places, but they say
that some of the places belong to the Big Boss, some are the government's, others
are the Forestry Department's. They all have made their claims, so there's no way
for us Lahu to get anything.

47. P: So when they say, this is the Forestry Department's, that plain is the other
guy's, this one belongs to the Big Boss --- they say this, but you never see them
using that land to cultivate, isn't that right? They've just abandoned [that land],
right?

48. T: Yes. Those guys have just laid their claims like that. And I don't see them
ever doing anything [with that land]. But since they've claimed it, we hillfolk
can't win over them.

