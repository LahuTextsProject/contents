\setcounter{footnote}{0}


1. Once upon a time there was a certain man.\footnote{I.e., a Lahu.}

2. He saw someone\footnote{I.e., a Shan.} taking care of a grazing elephant\footnote{\textbf{hɔ́} `to lead to pasture; look after a grazing animal', \textbf{šɛ̄-phâ} `agentive nominalizer'.} someplace,\footnote{\textbf{ô} \textbf{hɔ꞊} \textbf{ô} \textbf{ɔ̄} `there; someplace.' The narrator uses the variant \textbf{hɔ} of the locative noun-particle \textbf{ɔ̄}, perhaps for the sake of a triple \textbf{pun} with \textbf{hɔ} `elephant' and \textbf{hɔ́} `care for an animal.'} [and
said] ``Friend, is

your elephant dangerous--is he dangerous?"\footnote{These sentences are in Shan in the original.}

3. ``Sure he's an elephant! Sure he's an elephant!''\footnote{These sentences are in Shan in the original.}

4. ``Is your elephant dangerous? Is he dangerous, friend?''\footnote{These sentences are in Shan in the original.}

5. ``Sure he's an elephant! Sure he's an elephant!''\footnote{These sentences are in Shan in the original.}

6. ``Boys,\footnote{This is the first time the narrator mentions that the Lahu has some boys with him, though this seems to be an important convention in stories of this kind. See\textit{ The empty Coconuts} (\#110) and \textit{A Lahu gives a Shan a good cursing (\#119).}} he says it's dangerous. Let's get out of here!''
he said. ``Vamoose, vamoose!''\footnote{The Lahu here uses the Shan word \textbf{pâj} `go' to his companions. The jocular but urgent tone this adds is conveyed by the pseudo-Spanish (''Spanglish) `vamoose' (< Spanish vamós `let's go!).}

7. ``Where are you going?''\footnote{These sentences are in Shan in the original.} [said the Shan].

8. ``Vamoose, vamoose!''

\begin{center}
Explanation
\end{center}

\direct{Another Lahu explains the story we've just heard.}

9. When he said ``šaháy'' just now, you see, he meant ``friend.''

10. Then ``sá can nɛ̀ʔ cà tɛ̀ hâ means ``Is your elephant
dangerous\footnote{\textbf{cà} `dangerous/ferocious' is an old loan into Lahu from Shan kyam.}?"

11. But to the Shan, ``cà(n) meant ``elephant.''\footnote{\textbf{cà(n)} is the Lahu version of Shan \textit{sâaŋ} `elephant' (cf. Siamese \textit{cháaŋ}).}

12. So it was as if he'd said ``[Is your elephant an] elephant''
to the Shan.

13. So then, he was very frightened and ran away and reached a coral tree\footnote{\textbf{qá-chû-cɛ̀} `coral tree' [\textit{Erythrina variegata]}: a thorny tree with white bark and red flowers, beloved by birds'. See DL:233.}
which he climbed up.

14. Once he got up to the top of the coral tree, he couldn't manage to get back
down.

15. So somebody had to put a ladder against\footnote{\textbf{cɨ́} ``affix to/put against.''} [the tree] to take him down.\footnote{The additional details supplied by the explainer in sentences 13-15 are by way of showing that he could have told the story better than the original teller!}

