
\textbf{103 The Ferocious Elephant }

[Tape X, Side 1]

1 Once upon a time there was a certain man.[1] 2 He saw someone[2] taking care
of a grazing elephant[3] someplace,[4] [and said] \texttt{"}Friend, is your elephant
dangerous--is he dangerous?\texttt{"}[5]

3 \texttt{"}Sure he's an elephant! Sure he's an elephant!\texttt{"}[5]

4 \texttt{"}Is your elephant dangerous? Is he dangerous, friend?\texttt{"}[5]

5 \texttt{"}Sure he's an elephant! Sure he's an elephant!\texttt{"}[5]

6 \texttt{"}Boys,[6] he says it's dangerous. Let's run!\texttt{"} he said. 7 \texttt{"}Vamoose,
vamoose!\texttt{"}[7]

7 \texttt{"}Where are you going?\texttt{"}[5] [said the Shan].

8 \texttt{"}Vamoose, vamoose!\texttt{"}[6]

\begin{center}
Exegesis
\end{center}

\leftskip=0pt
[Another Lahu tries to explain the story we've just heard:]

1 What he was saying just now, see, [was this]: \texttt{"}s\textit{a}-hai\texttt{"}
means \texttt{"}friend.\texttt{"} 2 Then \texttt{"}sÁ can nɛ̀ʔ cà tɛ̀ hâ[8]
means \texttt{"}In your elephant dangerous (cà)?\texttt{"} \. But to the Shan,
\texttt{"}cà(n) meant \texttt{"}elephant.\texttt{"} 4 So it was as if he'd said
\texttt{"}[In your elephant on] elephant\texttt{"} to the Shan.

5 So then, he was very frightened and ran away and reached a thorn-tree which he
climbed up. 6 Once he got up to the top of the thorn-tree, he couldn't manage to
get down again. \. So somebody had to put a ladder against[9] [the tree] and take
him [down].[10]

Note: Since the Lahu word cà(n) 'ferocious, dangerous' is a yes \texttt{<}Kyam
from Shan, anyway, it is hard to see how the Shan could have misunderstood the
Lahu's intent. CHECK pronunciation of Shan 'elephant' as. 'fierce.' Maybe cà is
native Lahu.

\. I.e., a Lahu.

\. I.e., a Shan.

\. hɔ́ 'to lead to pasture; look after a grazing animal.'

\. ô hɔ ꞊ ô ɔ 'there, someplace.' The narrator uses the variant hɔ of the
topic P_{} \emph{ɔ} for the sake of a triple pun with hɔ 'elephant' and hɔ́
'care for an animal.'

\. These sentences are in Shan in the original.

\. This is the first time the narrator mentions that the Lahu has some boys with
him, though this seems to be an important convention in stories of this kind. See\textit{
Empty Cremuts} and \textit{A Good Cursing.}

\. The Lahu here uses the Shan word pâj 'go' to his companions. The jocular but
urgent tone this adds is conveyed by the pseudo-Spanish 'vamoose.'

\. The wording here is slightly different from that of Sentence 2 above.

\. cɨ́ \texttt{"}affix to, strike to, put against.\texttt{"}

1\. The additional details supplied by the exegete in sentences 5-7 are by way
of showing that he could have told the story better than the original teller.



